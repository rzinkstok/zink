
\chapter{Note on citation and translation}


All page references are to the original, unless indicated otherwise. Citations are taken from the original in all cases except in the case of the \textit{Beytr\"{a}ge}, which is cited from the editions by Fabbianelli, in which case a reference to the pagination of these editions is included in the reference. The references to the Introduction of the edition of the \textit{Versuch} by Onnasch were taken from the proofs; the pages therefore appear in brackets.

 From Reinhold's works cited in the present study, only the second and third installments of `Gedanken \"{u}ber Aufkl\"{a}rung' and the \textit{Merkur}{-}`Briefe' are currently available in an English translation. Translations other than from these two works (or form Kant) are my own. I have aimed for functional, rather than for literal or literary translations. For the translations from the \textit{Versuch} I was able to compare my translations with those of Professor Tim Mehigan of the University of Otago, New Zealand, of whose forthcoming translation of the \textit{Versuch} I consulted a preliminary version. 

 In citing I have maintained the emphasis of the original. In the \textit{Versuch}, emphasis is indicated by italics, which is maintained in citation and translation. In Reinhold's other works, which are printed in \textit{Fraktur}, emphasis is indicated either by using a slightly different font or by spacing, both of which are rendered bold in the citation, and italic in the translation. 

mainmatter

