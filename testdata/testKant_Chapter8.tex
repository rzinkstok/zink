
\chapter{`Practical reason' in the \textit{Merkur}{-}`Briefe'}


In the previous chapter we have sought to establish the most plausible story behind the interest Reinhold took in the Kantian philosophy from 1785 onwards and we have dealt with the complex of philosophical, personal and political reasons that must have played a role in his decision to study Kant's first \textit{Critique}. The present chapter will focus on the initial results of this study, as they emerged in the `Briefe \"{u}ber die Kantische Philosophie'. This series of articles published in \textit{Der Teutsche Merkur} between August 1786 and September 1787 not only formed the first stage of the plan Reinhold had communicated to Voigt, but also produced some very welcome side effects: he became a celebrity overnight and an extraordinary professor at the University of Jena. The popularity of his `Briefe' can be gathered from the fact that two pirate editions of them appeared not long after their publication in the \textit{Merkur}.\footnote{ \textit{Briefe \"{u}ber die Kantische Philosophie von Hn. Karl Leonhard Reinhold Rath, und Professor der Philosophie zu Jena. Zum Gebrauch und Nuzen f\"{u}r Freunde der Kantischen Philosophie gesammelt} (Mannheim: Bender, 1789); \textit{Auswahl der besten Aufs\"{a}zze \"{u}ber die Kantische Philosophie} (Frankfurt and Leipzig; de facto Marburg: Krieger, 1790). } Through Reinhold's efforts the Kantian philosophy was no longer just an incomprehensible curiosity, but became a hot topic. With regard to this stunning success, Karl Ameriks has termed the `Briefe' ``arguably the most influential work ever written concerning Kant.''\footnote{ In his introduction to the English translation of the `Briefe': Reinhold, \textit{Letters on the Kantian Philosophy}, ix. Citations from the `Briefe' will be taken from this translation, with reference to the pagination of the original article. } Apparently, then, Reinhold had found exactly the right perspective from which to promote the Kantian philosophy to the German public in the 1780's. 

 The present chapter of this study is dedicated to analyzing this perspective and showing how Reinhold's pre{-}Kantian interests determined the way in which his reception of the Kantian philosophy took shape. The first section will elucidate the context in which the `Briefe' could make such an impact, the fertile soil, as it were, in which Reinhold's Kantian seeds landed. The single most important factor determining the fertility of this soil is the so{-}called pantheism controversy which had erupted in 1785 between Jacobi and Mendelssohn, but which soon involved others as well. On the surface it concerned the question as to the extent of Lessing's alleged Spinozism, but the real issue at stake was the compatibility of philosophy and religion. The first two `Briefe' place the project firmly into the framework of this controversy with the claim that the Kantian philosophy provides the solution to that pressing issue. The second section of the present chapter will discuss the remainder of the `Briefe', analyzing the way in which Reinhold argues for his bold claim. From this analysis it will be clear that there are strong continuities with his pre{-}Kantian writings. Finally, the third section will focus on Reinhold's use of the Kantian concept `practical reason' in the `Briefe', where it plays an important role as the foundation of fundamental religious tenets, such as the existence of God. As the term `practical reason' does not occur in Reinhold's previous writings, its employment is clearly associated with his interests in Kantian philosophy. Likewise, Reinhold employs the Kantianizing concept `pure sensibility' in a way that is closely related to his use of `practical reason'. As shown in the previous chapters, however, Reinhold had a conception of reason of his own, before he became acquainted with the Kantian view. The evaluation will show that his use of the terms `practical reason' and `pure sensibility' is to be regarded as a continuation of his pre{-}Kantian views on reason. 


\section{The `Briefe \"{u}ber die Kantische Philosophie' and the pantheism controversy: the first two `Briefe' (August 1786)}


In order to understand the extraordinary success of the `Briefe' we must regard the context in which they were published. This context is formed by the so{-}called pantheism controversy or Spinoza controversy.\footnote{ For a detailed account of the pantheism controversy, cf. Beiser, \textit{The Fate of Reason}, Chapters 2{-}4, 44{-}126. } This debate erupted when, in 1785, Jacobi published his correspondence with Mendelssohn without the latter's consent.\footnote{ Friedrich Heinrich Jacobi, \textit{\"{U}ber die Lehre des Spinoza in Briefen an den Herrn Moses Mendelssohn} (Breslau: L\"{o}we 1785). } The particular subject of the letters Jacobi and Mendelssohn had exchanged in the two years prior to publication was Jacobi's claim that the late Lessing, German enlightener \textit{par excellence}, had revealed to him, Jacobi, that he, Lessing, was a Spinozist. Jacobi's cautious disclosure of this information to Mendelssohn, who was working on Lessing's biography, implied that he believed that this entailed that Lessing had been an atheist. In his \textit{Morgenstunden }(1785) and especially his \textit{An die Freunde Lessings }(1786, posthumously) Mendelssohn, who could boast a long friendship with Lessing, sought to defend him against the more serious suspicion of atheism, by claiming that even if Lessing was committed to Spinozism, it would have been a harmless version, which would not automatically have made him an atheist.\footnote{ Moses Mendelssohn, \textit{Morgenstunden oder Vorlesungen \"{u}ber das Daseyn Gottes} (Berlin: Vo\ss{} 1785); Mendelssohn, \textit{An die Freunde Lessings. Ein Anhang zu Herrn Jacobis Briefwechsel \"{u}ber die Lehre des Spinoza} (Berlin: Vo\ss{} 1786). } Thus, apart from the surface question regarding Lessing's alleged Spinozism, the underlying issue concerned the consequences of Spinozist metaphysics. In the second chapter, we have seen that Reinhold, in his article on Enlightenment and elsewhere stated that true Enlightenment would pose no threat to religion at all. Jacobi, in contrast, claimed that the consistent use of reason, that is, rationalist metaphysics, resulted in nihilism, a term coined by him to denote a total lack of values that goes with the atheism that inevitably follows from rational metaphysics. This result and the ensuing claim that, apart from empiricist skepticism, the only option left was Jacobi's own fideism shook the German intellectual world. Apart from the implications for Enlightenment that could be drawn from the controversy, the debate also generated a lot of public interest because of the air of scandal that surrounded it. After all, Jacobi had published private correspondence without his opponent's consent and Mendelssohn's death led to speculations pointing to Jacobi as the culprit. Not everyone took Mendelssohn's side, however. In May 1786, Thomas Wizenmann (1759{-}1787) published \textit{Die Resultate der Jacobischen und Mendelssohnschen Philosophie kritisch untersucht von einem Freywilligen},\footnote{ Thomas Wizenmann, \textit{Die Resultate der Jacobischer und Mendelssohnischer Philosophie kritisch untersucht von einem Freywilligen} (Leipzig: G\"{o}schen 1786). } in which he sided with Jacobi on many points. In his `Was hei\ss{}t: sich im Denken orientiren?' Kant assumed his stance in the controversy, criticizing both standpoints.\footnote{ Kant, `Was hei\ss{}t: sich im Denken orientiren?,' \textit{BM}, October, 1786, 306; \textit{AA} 8: 134. According to Beiser it was Wizenmann ``who convinced Kant that Jacobi and Mendelssohn were both heading in the dangerous direction of irrationalism and that something had to be done about it.'' Beiser, \textit{Fate of Reason}, 110. } Kant's standpoint was foreshadowed by Sch\"{u}tz's reviews of the works of Mendelssohn, Jacobi and Wizenmann, which employed the Kantian philosophy as a possible way to overcome this controversy. 

 Undoubtedly, Reinhold was seriously interested in this debate as well. He had, after all, been a supporter of Enlightenment for years, with a special interest in religion. Jacobi's attack, which was, through Mendelssohn, directed at all the Berlin \textit{Aufkl\"{a}rer}, must have touched Reinhold as well, who would not have been too pleased with the anti{-}rational fideism that Jacobi presented as the only viable alternative. On the other hand, Mendelssohn's metaphysics, firmly based on Wolff, was not what he was looking for either. We have seen in the second chapter that Reinhold's ideal of Enlightenment sought to mediate between the abstract concepts of reason and the concrete world of the senses. Jacobi's claim that it was `either\ldots , or\ldots ' with no possible middle course would have captured his attention. The pantheism controversy sharply showed that there was a tension inherent in Enlightenment, between the pretensions of reason and the concrete results it could produce. Reinhold, in his works on Enlightenment, had shown awareness of this tension before the debate between Jacobi and Mendelssohn actually erupted. It is no wonder, then, that he reacted to the radical standpoints assumed in the discussion. The context of the pantheism controversy undoubtedly played a role in his understanding of Kant and his subsequent presentation of the Kantian philosophy in the `Briefe'. The first two `Briefe', both published in August 1786, explore the possible role the Kantian philosophy could play in solving the debate on the status of reason with regard to religion, following from the controversy between Jacobi and Mendelssohn. Reinhold would have found an example of how to employ the Kantian philosophy in this context from the reviews of the main works in the controversy (by Mendelssohn, Jacobi and Wizenmann) that were published in the \textit{ALZ} and were probably the work of Sch\"{u}tz. In his review of Mendelssohn's \textit{Morgenstunden}, Sch\"{u}tz uses the Kantian criticism of the ontological proof of God's existence to criticize Mendelssohn.\footnote{ Sch\"{u}tz, review of \textit{Morgenstunden}, by Mendelssohn, \textit{ALZ}, January 2 (nr.1) and January 9 (nr. 7), 1786; Landau, \textit{Rezensionen}, 249{-}261. Cf. Schr\"{o}pfer, \textit{Kant's Weg in die \"{O}ffentlichkeit}, 222. } In reaction to Jacobi, Sch\"{u}tz aimed to disarm the attempt to identify the Kantian position on space with Spinozism and further criticized Jacobi for creating confusion by the way he employs the term `faith'.\footnote{ Sch\"{u}tz (?), review of \textit{Ueber die Lehre des Spinoza}, by Jacobi \textit{ALZ}, February 11 (nr. 36), 1786; Landau, \textit{Rezensionen}, 271{-}276. Cf. Schr\"{o}pfer, \textit{Kant's Weg in die \"{O}ffentlichkeit}, 226{-}227. } Both these reviews were published in the first months of 1786, so Reinhold would have had enough time to take notice of them before his `Briefe' were published. In the following analysis of the `Briefe' we shall see how much he took those reviews to heart. 

 As we have seen in the letter to Voigt, discussed in the previous chapter, the aim of the `Briefe' is to provide Kant's \textit{Critique }with a wider readership. The first installment is therefore fittingly titled `The need for a Critique of Reason'. Clearly, Reinhold's aim is to reverse the reception of Kant up until that point, by convincing the readers of the \textit{Merkur} that, although the \textit{Critique }may be difficult to read and to understand, it not only makes sense, but even fulfils a philosophical need. As the general title indicates, the articles have the form of letters to a fictitious correspondent, whom Reinhold addresses as `friend' and whom he seeks to convince of the usefulness of Kant's new philosophy. This correspondent represents the general reader of the \textit{Merkur}: educated, aware of contemporary debates on philosophy, but not a professional philosopher who would professionally and/or spontaneously indulge in Kant's first \textit{Critique}. So far, the correspondent has heard nothing but bad news regarding Kant: his philosophy is incomprehensible, and as far as it is comprehensible, it is old news. From the opening of the `Briefe' it is also clear that the correspondent favors Enlightenment and that he has become seriously worried about its success in Protestant Germany. The focus on Enlightenment in relation to Protestantism is something we have encountered earlier in Reinhold's writings on Enlightenment, in which he presented the Reformation as a crucial first step towards Enlightenment. By 1784 Reinhold had started to distance himself from the Austrian (Catholic) version of Enlightenment. Now, in 1786 he, through the worries of his fictitious correspondent, addresses the question whether Protestant Enlightenment has become complacent and is starting to lose momentum. In a way, this can be regarded as a continuation of the fictitious correspondence presented in the \textit{Herzenserleichterung}, which in a more specific context also concerned the difficulty of judging the progress made by Enlightenment so far.\footnote{ Cf. Chapter 2, section 4. } In summarizing his correspondent's worries, Reinhold provides a picture of the decline of the status of reason after its initial rise associated with the Reformation. 

Seit dem (schreiben Sie) der freye Vernunftgebrauch in Religionssachen f\"{u}r seine alten Vertheidiger den Reiz einer verbothener Frucht zu verlieren anf\"{a}ngt, tritt an der Stelle des vorigen Eifers f\"{u}r die Rechte der Vernunft eine Gleichg\"{u}ltigkeit ein. (\ldots ) Das Ausschliessende Recht der Vernunft \"{u}ber den Bibelsinn zu entscheiden, dieses Recht mit dessen Anerkennung der ganze Protestantismus steht oder fallt, wird von von protestantische Theologen (\ldots  ) angefochten (\ldots ). (Erster Brief, 100{-}101)

[You write: Ever since the free use of reason in religious matters began to lose for its old defenders the charm of a forbidden fruit, the former zeal for the rights of reason has been displaced by indifference (\ldots ). The exclusive right of reason to decide on the meaning of the Bible {--} that right with whose recognition the whole of Protestantism either stands or falls {--} is being attacked even by Protestant theologians (\ldots ). ](\textit{Letters}, 2)

As in `Gedanken \"{u}ber Aufkl\"{a}rung' and the essay on the sciences, the Reformation is presented by Reinhold as the necessary precondition of Enlightenment, because of the freedom to apply reason to matters of religion it advocates.\footnote{ Cf. Chapter 2, section 2.2. } His main worry is that the advances made during that historical period on behalf of the rights of reason are currently being nullified because these rights are no longer zealously defended. This justifies serious worries regarding the status of the science that is most closely associated with reason, metaphysics. 

Die Wissenschaft, von welcher alle \"{u}brigen ihre Grunds\"{a}tze entlehnen, die von jeher das eigenth\"{u}mlichste und angelegenste Gesch\"{a}ft der Vernunft ausmachte, und durch deren Bearbeitung sich die \textbf{Leibnize}, die \textbf{Wolfe} und \textbf{Baumgarten}, um die wahren Vorz\"{u}ge unsres Zeitalters so sehr verdient gemacht haben, mit einem Worte, die \textbf{Metaphysik}, wird auf eine Art vernachl\"{a}ssiget, die mit den Anspr\"{u}che unsres Jahrhundertes aur den Ehrentitel des \textbf{Philosophischen} den seltsamsten Contrast macht. (\ldots ) Warmk\"{o}pfige Schw\"{a}rmer, und kaltherzigen Sophisten sind gegenwartig mehr als jemals gesch\"{a}ftig, durch die Tr\"{u}mmer dieser Wissenschaft die alten Systeme des Aberglaubens und Unglaubens neu aufzustutzen (\ldots ). (Erster Brief, 101{-}102)

[The science from which all the other sciences borrow their principles, the science that from time immemorial constituted the most distinctive and important employment of reason, and through whose development Leibniz, Wolff, and Baumgarten have rendered such a great service to the true priorities of our age {--} in a word, metaphysics {--} is being neglected in a way that contrasts oddly with the claims of our century to the honorary title of `the philosophical'. (\ldots ) Out of the ruins of this science, hot{-}headed enthusiasts and cold{-}hearted sophists are at present busier than ever propping up anew the old systems of superstition and nonbelief (\ldots ).] (2{-}3)

Together with the status of reason, the status of metaphysics as a science which is very useful use with respect to ``the true priorities of our age,'' has declined. It is currently being attacked simultaneously from two sides. On the one hand, there are the ``hot{-}headed enthusiasts'' who seek to re{-}establish the old system of superstition, whereas the ``cold{-}hearted sophists'' on the other hand aim for a system of non{-}belief. The useful and important metaphysics of the good old days of Enlightenment is in severe trouble. According to Reinhold's correspondent, this is a cause for serious worries regarding the success of Enlightenment. The situation as sketched in Reinhold's summary is clearly related to the controversy between Mendelssohn and Jacobi, and especially to the dilemma that Jacobi had posed. It appeared that the only options left were to become a hot{-}headed enthusiast, who allows his religious feelings to determine his intellectual position, or a cold{-}hearted sophist, whose rationalizations fail to have a beneficial effect on his conduct. The two horns of this dilemma, however, are not only inspired by the controversy. We have already encountered them in Reinhold's earlier works (cf. Chapter 2, section 3), which warned against the danger in one{-}sidedly stressing either our sensible capacities (which leads to superstition or misguided religion) or our rational capacities (which leads to ineffective, abstract speculation). That the proper use of reason, identified by his correspondent with Leibnizian{-}Wolffian metaphysics, avoids this dilemma is therefore no news for Reinhold, even if he would not agree that the philosophy of the schools represents this proper use of reason. 

 Although subscribing to the general analysis of his fictitious correspondent, Reinhold refuses to draw the same pessimistic conclusion. Instead, he interprets the signs optimistically, ``as reliable harbingers of one of the most far{-}reaching and beneficent revolutions that has ever occurred at one and the same time in the scholarly and moral world'' (5; Erster Brief, 105). The reason for Reinhold's optimism is his conviction that the different phenomena adduced by his correspondent to prove that Enlightenment is going downhill are not isolated but rather are symptoms of a common underlying ground. This ground, so Reinhold, is ``none other than the old and still persistent misunderstanding, (\ldots ) regarding the right and power of reason in matters of religion'' (5; Erster Brief, 105).\footnote{ Cf. Sch\"{u}tz's review of Kant's \textit{Grundlegung zur Metaphysik der Sitten}, which opens: ``Mit Hn. \textit{Kant's Critik der Vernunft}, welche vor einigen Jahren erschien, ist eine neue Epoche der Philosophie angegangen.'' Sch\"{u}tz, review of Review of \textit{Grundlegung zur Metaphysik der Sitten}, by Immanuel Kant. \textit{ALZ}, April 7 (nr. 80), 1785; Landau, \textit{Rezensionen}, 135. } Refraining from immediately discussing this misunderstanding, he proceeds by analyzing the symptoms in order to present the following diagnosis that fits them all. 

Da\ss{} es den Antworten, welche die Vernuft, oder vielmehr welche man im Namen der Vernunft auf manche der allerwichtigsten Fragen bisher gegeben hat an Evidenz oder Allgemeing\"{u}ltigkeit gefehlt habe, davon ist selbst der uralte immerfortw\"{a}hrende Streit \"{u}ber diese Fragen der \"{u}berzeugendste Beweis, und die Frage \"{u}ber das \textbf{Daseyn Gottes} das auffallendste Beyspiel. (Erster Brief, 106)

[The age{-}old and never{-}ending dispute over many all{-}important questions is itself the most convincing proof that the answers reason has so far given to these questions {--} or rather, the answers that have been given in the name of reason {--} lack evidence and universal validity]. (5)

In connection with the subject of Mendelssohn's \textit{Morgenstunden}, the question of God's existence is taken as the prime example of a question that metaphysicians have not answered in a universally valid manner, and neither have those who do not rely on reason but rather on supernatural revelation {--} even if the adherents of both parties themselves believe their solutions are universally valid. However, as the validity of these solutions is questioned by the opposite party, the members of each party can only reiterate the same claims over and over again, drawing attention to their respective weaknesses, for they are not able to convince their opponents. Fortunately, it is precisely this stalemate which opens up the possibility of progress, as it naturally leads to doubts whether a universally valid answer to the problem of God's existence is even possible. In one of his more metaphorical passages Reinhold describes the current situation of philosophy with regard to this question in the following way. 

Die\ss{} Problem ist gleichsam der Punkt, wo sich de beyden Wegen der Metaphysik und Hyperphysik endigen, die sich r\"{u}kw\"{a}rts ins Unendliche verlieren und immer weiter vom Ziele abf\"{u}hren {--} der Punkt, von welchem der einzigen Weg \textbf{vorw\"{a}rts} angeht. Wir haben die beyden Abwegen zur\"{u}ckgelassen: wenn wir uns einmal bey diesem Punkte befinden, und da wir nich stehen bleiben k\"{o}nnen, so m\"{u}ssen wir den Weg vor uns antreten, oder welches eines ist, wir m\"{u}ssen das Problem aufl\"{o}sen. Die Bedingungen dieser Aufl\"{o}sung \textbf{ausserhalb des Gebiethes de Vernunft} aufsuchen, oder dieses Gebieth mit unsrer \textbf{bisherigen Metaphysik} verwechseln, w\"{u}rde eben so viel seyn, als r\"{u}kw\"{a}rts gehen. Und sich wieder auf einem der vorigen Wege verirren. Es ist also nichts anderes \"{u}brig als das \textbf{noch unbekannte Gebieth der Vernunft}, auf welche die gedachte Bedingungen liegen m\"{u}ssen, vor allem kennen zu lernen, und der neu betretene Weg, f\"{u}hrt zu einem neuen und zweyten Problem: \textbf{Was ist durch eigentliche Vernunft m\"{o}glich}? (Erster Brief, 115{-}116)

[This problem lies, as it were, at the point where the two paths of metaphysics and hyperphysics come to an end, where both paths trail off backwards into the infinite and lead further and further away from the goal {--} the point from which the only remaining path is the one that moves forward. Once we find ourselves at this point, we have left both stray paths behind. And since we cannot remain at a standstill, we must take up the path before us {--} which is to say, we must solve the problem. Seeking the conditions of this solution outside the domain of reason or confusing this domain with our previous metaphysics would be tantamount to moving backwards and losing our way again on one of the previous paths. Thus, there is nothing left to do than to become acquainted, above all, with that still unknown domain of reason in which these conditions must lie. And the newly entered path leads to a new and second problem: What is possible through reason proper?] (11)

In a few steps the problem of the possibility of universally valid grounds for the conviction that God exists has become the problem of the nature and capacities of reason. Of course, precisely these capacities are at stake in the pantheism controversy. It now also becomes clear that the misunderstanding underlying the phenomena collected by Reinhold's correspondent is one regarding the nature of reason. If we seek to answer the question: How is an answer regarding the question of the existence of God possible? either by looking outside the domain of reason, or by confusing the domain of reason with previous metaphysics, we would be going backwards, on the paths that have been taken before with such little success. In order to move forward, then, a solution has to be found that answers the question on the basis of reason, without, however, confusing this reason with previous metaphysics. At the end of the first installment, Kant is introduced for the first time when Reinhold claims that his new philosophy, despite being condemned as ``absolutely incomprehensible'' (15; Erster Brief, 124), nevertheless contains the solution to the problem posed, as well as solutions to many problems that follow from the misunderstanding of reason. He intends to make his correspondent familiar with a work

das meiner innigsten Ueberzeugung nach den dringendsten philosophischen Bed\"{u}rfnissen unsrer Zeit wie gerufen k\"{o}mmt, und unsren Nachkommen in so vielen R\"{u}ksichten eine bessere Zukunft zusichert. (Erster Brief, 125)

[that, according to my innermost conviction, is just what the most pressing philosophical needs of our time call for {--} a work that secures in so many ways a better future for our descendants.] (16)

Although Reinhold's enthusiasm for Kant may seem an exaggeration, it is striking to see the neat correspondence between this first installment of the `Briefe' and Kant's own Preface to the first edition of the \textit{Critique of Pure Reason}. Kant laments the state of metaphysics as well, especially pointing out the lack of a plan that could count on unanimous support from the metaphysicians and the indifference regarding reason that resulted from the disarray of metaphysics.\footnote{ Cf. \textit{KrV}, A ix{-}x. } He also stressed the need to come to a solution and describes his own venture as the only path forward after having found the point where the misunderstanding of reason had started.\footnote{ Cf. \textit{KrV}, A x{-}xii. } Reinhold is thus trying to sell Kant to his correspondent on the basis of Kant's own presentation of his project. However, the terms in which he describes the debate regarding reason differ from those Kant preferred in his Preface. Where Kant primarily describes a conflict between dogmatists and skeptics when it comes to the alleged capacity of reason to judge with regard to objects that cannot be found in experience, Reinhold explicitly relates the conflict regarding reason to the rights of reason to judge in matters of religion. The opponents of the claims of reason are not skeptics, but rather religious supernaturalists, who seek to ground religion in supernatural revelation. Reinhold's perspective on the debate regarding reason naturally follows from his ideas on Enlightenment, which, as we have seen, stresses the importance of the Reformation, which entails the independent use of reason in matters of religion. Reinhold would have found support for the importance of the freedom of reason to judge in matters of religion in Kant's essay on Enlightenment, which states that the political authorities should abstain from prescribing the private religious convictions of the citizens.\footnote{ Cf. Kant, `Beantwortung der Frage: Was ist Aufkl\"{a}rung?' \textit{AA} 8:41.} Apart from being related to his own pre{-}Kantian ideas, Reinhold's description is also implicitly related to the controversy between Jacobi and Mendelssohn; his description of this problem is soon cast in the familiar terms of the insufficiency of two one{-}sided approaches, which can be overcome by taking their common ground into account. 

 The first installment of the `Briefe' can hardly be considered separately from the second, which appeared in the same issue of the \textit{Merkur}. Here Reinhold specifies the claim made earlier, that the Kantian \textit{Critique }contains the solution to the controversy on the role of reason with regard to religion. This specification is already clear from the title of the second installment: `The result of the Kantian philosophy on the question of God's existence'. The problem is described more precisely in the opening paragraph. It is not the conviction that God exists that is the subject of controversy, since this ``is pronounced by such universal agreement and confirmed by the equally universal interest of humanity'' (18; Zweyter Brief, 127). Rather, the controversy concerns the grounds for this conviction. Given the universal acceptance of the conviction, its grounds must be ``irrefutable and universally evident [\textit{allgemeineinleuchtend}]'' (18; Zweyter Brief, 127). Since the grounds given until now by metaphysicians and supernaturalists are far from irrefutable and universally evident, these cannot be the true grounds of this conviction. The true, universally evident grounds must have operated in the background, as it were, without having been specified. It is precisely because they have not been specified before, that the role of reason in providing the grounds for the conviction of God's existence has been contested. It is here that the fundamental contribution of the Kantian investigation of reason is to be found according to Reinhold. Because of this investigation, Kant, in Reinhold's description, has been able to overthrow the grounds that had previously been put forward by the two rival parties and had thus already decided the pantheism controversy before it had even erupted. This controversy is then cited as exemplifying the misunderstanding of reason by equating reason with traditional metaphysics. Thus, although it needed not have taken place, the controversy has still been useful as it brought to light the long existing tensions in metaphysics (Zweyter Brief, 140).

As we have seen before, Reinhold himself was already aware of the limitations of traditional metaphysics, without at the same time becoming indifferent to reason. For him, as pointed out in our second chapter, the key lay in acknowledging that reason, in order to function properly, must maintain a healthy relation to sensibility. Thus, the domain of reason, conceived of properly, must be wider than the scope of traditional metaphysics which, at least in Reinhold's description, dealt with abstract, a priori concepts alone. From Reinhold's early works it is clear that reason has to be understood in this broader sense in order to ensure the progress of Enlightenment in practice. In his brief indication of Kant's solution to the question regarding the grounds for the conviction that God exists we find a similar move. Reason is presented as including something beyond the demonstrations of traditional speculative metaphysicians. It is in practical reason that the proper grounds for the conviction of God's existence are to be found. 

Indem sie [the new answer] den von der praktischen Vernunft gebothenen Glauben festsetzt, st\"{u}rzet sie die Lehrgeb\"{a}ude der \textbf{apodiktischen Beweise} und des \textbf{blinden Glaubens} um, und stiftet durch die gl\"{u}ckliche Vereinigung der gel\"{a}uterten Hauptgr\"{u}nde ven beyden Lehrgeb\"{a}uden ein neues System, in welchem die Vernunft anmassend, und der Glaube blind zu seyn aufh\"{o}ren, und anstatt sich, wie bisher, zu widersprechen, in ewiger Eintrach sich wechselseitig unterst\"{u}tzen. (Zweyter Brief, 134{-}135)

[In so far as the new answer is founded on a faith that is commanded by practical reason, this answer topples the doctrinal structures of both apodictic proofs and blind faith and establishes a new system through a most successful union of the clarified principal arguments of both doctrinal structures. In the new system, reason ceases to be presumptuous and faith ceases to be blind, and instead of opposing one another as before, they mutually support one another in perpetual harmony.] (22;

Although Reinhold does not as yet supply more details regarding this answer and system, it is clear that the general tendency is similar to his own view of the proper use of reason, developed with respect to Enlightenment. The term `practical reason' is new to his vocabulary and associated with Kant, but, again, Reinhold is as yet not very clear on how exactly to understand it. It is obvious, though, that practical reason fulfils a crucial function in Reinhold's efforts to sell the Kantian philosophy. Only if reason must be understood as involving something that traditional metaphysics has neglected and only if this extra feature contains something that will satisfy those who seek to base religion on reason as well as those who rightly mistrust metaphysics, only in that case has Kant solved the pressing problem that had become common knowledge with the pantheism controversy. Considering the way in which this controversy figures in the final pages of the second `Brief', it is very likely that Reinhold was aware of the reviews that Sch\"{u}tz had produced relating the Kantian philosophy to the pantheism controversy. Regarding Jacobi, one of Sch\"{u}tz's main points, as we have seen, was that he misunderstood Kant. Reinhold states that Jacobi's way of dealing with Kant's \textit{Critique} obviously shows ``that he has not thoroughly grasped it'' (24; Zweyter Brief, 138). Further, Reinhold follows Sch\"{u}tz in criticizing Jacobi's lack of clarity when it comes to his conception of faith (Zweyter Brief, 140). Regarding Mendelssohn, Reinhold appears to be inspired by the excerpt from a letter from Kant himself, included in Sch\"{u}tz's review. In it Kant says that it is in the interest of philosophy that the arguments are stated in their strongest and clearest form, and, although he believes Mendelssohn's argumentation is faulty, he praises him for showing the best that rational metaphysics is capable of.\footnote{ Sch\"{u}tz, review of \textit{Morgenstunden}, by Mendelssohn, 56. Landau, \textit{Rezensionen}, 260.} Reinhold describes Mendelssohn's \textit{Morgenstunden }as a work ``that with rare clarity expounds ontological pseudo{-}arguments on the basis of their fundamental concepts, presents these arguments in their strongest possible forms, and seeks to supplement them with new ones'' (24; Zweyter Brief, 138). Reinhold's brief discussion here of the positions in the pantheism controversy show that Sch\"{u}tz's reviews were a source of inspiration for starting the `Briefe \"{u}ber die Kantische Philosophie'.


\section{Reinhold's Kantian solution to the problem of the rational foundation of religion: the remainder of the Merkur{-}`Briefe'}


Having stated his enthusiasm for the Kantian philosophy in the first two `Briefe', while situating the Kantian system in relation to current debates on the nature of reason, Reinhold took a break before continuing the series of `Briefe'. Wieland's announcement in December that the `Briefe' were to be resumed in January suggests that the discontinuity was due to external and editorial factors, rather than to the author's choice.\footnote{ \textit{TM}, December, 1786, 294. Wieland writes: ``Die im August angefangenen Briefe \"{u}ber die Kantische Philosophie, welche durch zuf\"{a}llige Ursachen unterbrochen, und zum Theil durch andere Artikel, die man nicht zur\"{u}cksetzen konnte, verdr\"{a}ngt worden sind, sollen in dem bevorstehenden Jahrgang von Monat zu Monat fortgesetzt werden.''} It appears, however, that this pause was not unwelcome to Reinhold. As mentioned earlier, in his letter to Voigt he claims that the birth of his first child, in October 1786, had robbed him temporarily of ``the peace of heart and the leisure of mind,''\footnote{ \textit{RK} 1:145, Letter 35, beginning of November 1786, to Voigt; cf. Chapter 3, section 2.} which implies that he had personal reasons to take it easy. At first sight, this appears to contradict a letter he wrote later, probably to Sch\"{u}tz, in which a remark concerning the continuation of the `Briefe' in January 1787 is followed by the statement that during the break, he had had ample time and fancy ``to penetrate more deeply into the mind of my great master.''\footnote{ \textit{RK} 1:206, Letter 45, end of March/before April 4, 1787. } There is only a seeming contradiction here, however. It will not be contested that understanding Kant in more depth requires a lot of time and motivation. And while the letter to Voigt opens with the image of an overwrought new father, the letter also makes it clear that Voigt's request and his possible promise are highly motivating for Reinhold to continue his work on Kant (cf. Chapter 3). The motivation to study Kant found during the break may thus well be related to encouragement received from Voigt and others. The letter to Sch\"{u}tz cited above indicates the importance of external encouragement as well, as Reinhold speaks of ``encouragement that I have received from others to continue this project.'' Some of these others are explicitly mentioned, namely Johann Christian Friedrich von Schiller (1759{-}1805) and Heinrich Christian Boie (1744{-}1806).\footnote{ \textit{RK} 1:206. } This picture is also supported by Wieland's announcement in \textit{Der Teutsche Merkur} of December 1787, promising that the `Briefe' will be continued, 

da die, wie wir h\"{o}ren, die Aufmerksamkeit unserer ernsthaften Leser erregt haben, und der Wunsch, sie fortgesetzt zu sehen, uns von vielen Orten her zu erkennen gegeben worden ist.\footnote{ \textit{TM}, December 1786, 294.} 

[since these, as we have heard, have aroused the interest of our serious readers and since the wish to see them continued has been communicated to us from many sides.]

In combination, Reinhold's letters to Voigt and Sch\"{u}tz and Wieland's announcement concerning the continuation of the `Briefe' yield the following picture. It is well possible that Reinhold originally intended to write only the two `Briefe' that were published in August 1786, forming a neat unity. They follow the line on presenting Kant as the solution in the pantheism controversy that was initiated in Sch\"{u}tz's reviews. Having received encouragement from esteemed figures like Voigt and Schiller, Reinhold was motivated to make more of the `Briefe' and wrote a plan that he communicated to Voigt. However, in order to continue the `Briefe' Reinhold needed to study the Kantian philosophy in more depth. As shown in the previous section, the first two `Briefe' hardly focus on Kant at all and where they do, they merely follow the master's own Preface to the first \textit{Critique}. Since Reinhold admits that he used the break to deepen his knowledge of the Kantian philosophy, he apparently deemed his knowledge up to that point insufficient for continuing the `Briefe'. As we shall see in the following, Reinhold's familiarity with Kantian themes and terminology would continue to grow throughout the `Briefe'.

 Another factor that may have delayed the continuation of the `Briefe' was the publication of Kant's `Was hei\ss{}t: sich im Denken orientiren?' in the \textit{Berlinische Monatsschrift }in October 1786.\footnote{ Kant, `Was hei\ss{}t: sich im Denken orientiren?,' \textit{BM }October 1786, 304{-}330; \textit{AA} 8: 131{-}147. Hereafter referred to as `Orientation'{-}essay. } After all, Reinhold had claimed that Kant had solved the problem of the pantheism controversy, and now the master came forward with a contribution to that debate. This could not be overlooked by Reinhold, who indeed cited a long passage from the `Orientation'{-}essay in his third `Brief'. All in all, it is clear that the pause in the publication of the `Briefe' also involved a pause in Reinhold's production, as he used the four months between August and January to gain a deeper understanding of Kant's philosophy. From his own statements we can infer that he was encouraged to do so by people like Voigt and Schiller. 

 This second section will show how Reinhold's claim in the first two `Briefe' {--} that Kant has solved the problem of the role of reason regarding religion by means of practical reason {--} takes shape in the remainder of the series. Thematically, a division can be made between the third and fourth `Briefe' on the one hand and the fifth through eighth on the other. Correspondingly, section 2.1 will deal with the third and fourth `Briefe', which mainly elaborate Reinhold's claim regarding the existence of God in a systematic and historical manner, whereas section 2.2 will discuss the remainder of the `Briefe', which deal with the rational grounds for the conviction that the human soul continues to exist after the death of the body. 


\subsection{The third and fourth `Briefe': Systematic and historical backgrounds to Reinhold's claim concerning Kant}


In his first two `Briefe' Reinhold had described the Kantian project in relation to its external grounds, that is the historical context, rather than on the basis of an overview of the actual contents of the first \textit{Critique}. His argument to introduce the Kantian project to his fictitious correspondent involved the following bold claim: Kant has solved the most pressing philosophical problem of the time and has thus been able to provide by means of practical reason a rational ground for the conviction that God exists. These first two `Briefe' were clearly of an introductory nature and the solution was only proposed in general terms without much specification of the manner in which this extraordinary result was obtained. 

 With regard to a more in{-}depth discussion of the Kantian solution to the problem posed in the first two `Briefe' the title of the third installment (January 1787) sounds promising enough: `The result of the Critique of Reason concerning the necessary connection between morality and religion'. By way of introduction, Reinhold considers his correspondent's doubts concerning the desirability of overcoming the traditional metaphysical proofs for God's existence. If metaphysics did provide proofs that rationally grounded the conviction that God exists, why would one want to supersede them? Clearly, Reinhold is aware of some of the difficulties his correspondent, that is, the general reader of the \textit{Merkur}, will have with the Kantian project. By means of the didactical move of discussing these difficulties, Reinhold provides himself with an opportunity to defend the positive results of the Kantian project with more depth, introducing new (external) grounds in order to reassure his friend. The claim with which he seeks to do so is already presented at the beginning of the article. 

Die Religion gewinnt durch die Hinwegr\"{a}umung dieser Beweise, so wie sie durch die Kritik der Vernunft aus ausgef\"{u}hrt wird, nichts geringeres: als einen einzigen unersch\"{u}tterlichen und allgemeingiltigen Erkenntni\ss{}grund, der auf dem Wege der Vernunft die Vereinigung zwischen Religion und Moral vollendet, welche durch das Christenthum auf dem Wege des Herzens eingeleitet worden ist. (Dritter Brief, 5)

[By the clearing away of these proofs in the manner accomplished by the Critique of Reason, religion gains nothing less than a single, unshakeable, and universally valid ground of cognition, one which completes by means of reason the unification of religion and morality that was introduced through Christianity by means of heart.] (29) 

Bearing in mind that Reinhold did not want to lead his readers ``into the depths of speculation from which Kant has unearthed so many previously undiscovered treasures of the human spirit'' (16; Erster Brief, 125), it would be misguided to expect him to provide a detailed Kantian account of the reasons why the `proofs of previous metaphysics' can be cleared away without detriment to religion. However, the claim that he does advance on behalf of the positive results of the Kantian project is surprising and certainly goes beyond anything Kant claimed in his first \textit{Critique}. Moreover, this claim requires substantial argumentation as it is far from obvious that `Christianity unified religion and morality by means of the heart' and that `Kant completes this unification by means of reason'. Finally, these claims need to be related to the main claim that through this unification, `religion gains a single, unshakeable, and universally valid ground of cognition'.

 In order to explain his claim, Reinhold starts by putting forward his account of Christianity, and the way in which this religion, according to him, succeeded in unifying religion and morality. The starting point for Christianity, so Reinhold argues, was a situation in which the general masses had religion without morality, while a few philosophical sects had morality without religion (Dritter Brief, 5). From this brief description of the situation it is not at all clear who these general masses are (Jews, Romans, Greeks?) and which philosophical sects Reinhold has in mind. However, it may well be the case that he does not have any concrete masses and philosophers at all in mind, but is simply using a scheme very similar to the one employed in `Gedanken \"{u}ber Aufkl\"{a}rung', where the philosopher and the common man were opposed as well (cf. Chapter 2). What emerges from the description is that there is not only a disconnection of religion and morality, but that the disconnection is associated with a social division between the general masses and the intellectual elite. However, since these groups and the key terms `religion' and `morality' are not clearly specified, it is not at all obvious that this would be a problematic situation. The description of the changes brought about by Christ's teachings yields some more insight into what the problem was in the first place.

Seine [Christ's] Lehre setzte also den \textbf{Mittelbegriff} fest, an den sich die feinste Spekulation und die sinnlichste Vorstellungsart der Menschen mit gleicher Leichtigkeit anschliessen konnte; und allenthalben, wo man sich dieser Lehre gem\"{a}\ss{} das h\"{o}chste Wesen als \textbf{Vater}, und das menschliche Geschlecht als dessen \textbf{Familie} dachte, wurde die Moral auch f\"{u}r den gemeinsten Verstand einleuchtend, und die Religion f\"{u}r den kaltbl\"{u}tigsten Philosophen r\"{u}hrend. (Dritter Brief, 6)

[Christ's teaching thus established a mediating concept for human beings by which the most subtle speculation and the most sensory manner of representation could be connected with equal ease. And wherever, according to this teaching, the highest being was thought of as a father whose family was the human race, morality became illuminating for even the most elementary understanding, and religion became moving for the most cold{-}blooded philosopher.] (29{-}30)

Although the exact problem remains unspecified, the main aim of the unification effort appears to lie on the social level. Christianity provides something that makes the former division between masses and intellectual elite disappear and is accessible to both these classes of people. By now, Reinhold's connection of the philosophers with rational efforts (subtle speculations) and of the masses with a sensory manner of representation should come as no surprise. We have already encountered this association in his essay on Enlightenment, along with a call upon philosophers to employ concepts like `father' that can serve as bridges of communication between them and the general masses, in order to make the abstract philosophical concepts accessible to them as well.\footnote{ Cf. Chapter 2, section 1.} Here, in the `Briefe', Christianity is credited with establishing a connection that puts an end to the division between the rational speculation of the philosophers and the sensible world in which the general masses live. It is, of course, no coincidence that the concept `father' is used in both instances as an example, as this is precisely the sort of concept that can serve as a mediator: it is common to all human beings, regardless of their education, and it does not have to be learned in order to be understood. The everyday concept `father' at once acquires both a religious and a moral dimension. As the teaching of Christ presents God as the father of all mankind, the love people naturally feel for their own family can be extended to all of humanity, and morality will become just as easy to practice {--} on the assumption of course that everyone naturally loves their family and treats their siblings well. The second chapter has demonstrated that this interpretation of Christianity is strongly related to Reinhold's Illuminatist background. According to him, the new religious and moral use of the concept `father' is advantageous for both the social classes which, until then, had suffered from a one{-}sided development of their mental capacities: the general masses relying too much on their sensory capacities, and the philosophers attaching too much value to their rationality. The philosopher's rationality was exclusively focused on morality, but with the Christian concept of God this morality became easy to follow as it had found a way to influence the heart. The sensory capacities of the general masses were focused on religion, which with the new concept of God became closely connected to moral behavior. Or, as Reinhold puts it, morality and religion became united ``by an internal relation according to which morality depended upon religion, at least in so far as it was indebted to religion for its dissemination and effectiveness'' (30; Dritter Brief, 6). In this way Reinhold briefly attempts to clarify his earlier statement that `Christianity has united religion and morality by means of the heart'. Although he has not specified his understanding of `by means of the heart', the phrase must refer to the concept of God introduced by Christ's teaching, that of a father of all humanity. This concept represents the way of the heart because, it is a concept that is natural, simple and common to everyone. This means that it is equally accessible to everyone, regardless of the education that they may or may not have had. Further, this concept is associated with the `heart' as it expresses a close relationship that naturally involves a mutual affection. It is expressly in this capacity that Christianity's employment of the concept `father' can perform the unification of morality and religion for both the general masses and the philosophers. 

Seine [Christianity's] eigentliche Bestimmung war also, und wird es zu allen Zeiten seyn: \quotedblbase die moralischen Ausspr\"{u}che \textbf{der Vernunft} theils f\"{u}r den Verstand des gemeinen Mannes zu versinnlichen, theils dem Denker ans Herz zu legen, und folglich \textbf{der Vernunft} bey der sittliche Bildung der Menschheit wohlth\"{a}tig and die Hand zu gehen.`` (Dritter Brief, 7)

[Thus, its [Christianity's] actual purpose was, and will be for all times, partly to make the moral claims of reason tangible for the understanding of the common man, partly to gain them a place in the thinker's heart, and consequently to be the benefactor of reason in the moral cultivation of humanity.] (30)

As mentioned earlier, for Reinhold the term `heart' relates to a complex of concepts and feelings that is best described as relating to the real, actual world in which human beings lead their lives, the world of the senses, the world of action, the world of contacts with other human beings. It is in this world that the rational, abstract and universal rules of morality have to be concretely applied. The actions of general masses for the sake of religion had been unrelated to morality, or at least so Reinhold claims, because the rules of morality were not yet ``tangible'' for their ``sensory manner of representation.'' On the other side of the intellectual spectrum, the philosophers had established these rules rationally, but were not motivated by their rationality to act upon them, as the rules remained universal and abstract, without influence on the heart. Christianity, with its innovative concept of God is credited with providing, through this concept, both classes of people with what they needed most in order to act morally. The common man gained insight into the rules of morality, while the philosopher found an easy motivation to act upon them. 

 In this analysis of part of Reinhold's third `Brief' the connection with his earlier work stands out much more clearly than the relation to Kant. In some form or other, many elements of this presentation were already there in his pre{-}Kantian work. He had not only used the concept of `father' as an example before, he had also intended it to perform a similar function. In `Gedanken \"{u}ber Aufkl\"{a}rung' this concept facilitated the communication between the philosopher with his abstract rational concepts and the common man, whose concept of God is corrected by means of his natural understanding of the concept `father', which is not only concrete concept but also close to his heart. We have also encountered the opposite problem, that of the philosopher, before. In his speech on the `Wehrt einer Gesellschaft' Reinhold already made clear that without a healthy relation to the heart, the lofty plans of the \textit{Aufkl\"{a}rer} would come to nothing.\footnote{ Cf. Chapter 2, section 3.2. } 

 Kant, on the other hand, only gains prominence in the second half of the article, after a brief account of the historical fate of Christianity, which, as in the first `Briefe', attaches great importance to the Reformation. As a result of the historical developments Reinhold sees in his own time once again a disconnection of religion and morality. He describes the current circumstances and the different perspectives of the `orthodox' and the `moralists'. 

Die einen wollen die Moral h\"{o}chstens nur als ein Kapitel ihrer Theologie, und die andern die Theologie nich einmal f\"{u}r ein Kapitel ihrer Moral gelten lassen. Diese bestreben sich ihrer Vernunft alle Religion entbehrlich zu machen; und jene, ihre Religion gegen allle Vernunft zu verwahren. (Dritter Brief, 11)

[The orthodox will admit morality at most as a chapter of their theology, and the moralists will not admit theology even as a chapter of their morality. The moralists strive to make all religion dispensable to their reason, and the orthodox strive to secure their religion against reason.] (32{-}33) 

As with the philosophers and the general masses at the time of the introduction of Christianity, Reinhold fails to specify the exact positions of the `orthodox' and the `moralists'. On the basis of his context we can make an educated guess, however. The orthodox may be the Pietist \textit{Glaubenstheologen}, mentioned in the earlier `Briefe' in relation to their tendency to rely almost exclusively on ``proofs derived from supernatural sources'' (6; Erster Brief, 107). The moralists, who seek to make morality independent of religion, may be thinkers in the field of natural rights or adherents of the moral sense theories of men like Shaftesbury (1671{-}1713) and Hutcheson (1694{-}1746). Reinhold himself was not averse to theories that sought to establish morality on an independent ground, as is testified by his \textit{Herzenserleichterung}.\footnote{ In \textit{Herzenserleichterung} Lichtfreund claims ``Wir haben nun wieder eine Moral, die von allen Religionssystemen und Glaubensmeinungen unabh\"{a}ngig ist'' and he continues ``Man wei\ss{}, da\ss{} man Religion auf Moral, nicht diese auf jene gr\"{u}nden m\"{u}sse.'' Reinhold, \textit{Herzenserleichterung zweyer Menschenfreunde}, 13{-}14. } In contrast to the situation at the time of the foundation of Christianity, the opposing groups in Reinhold's own time are not identified as social or intellectual classes. Reinhold only remarks that currently there is a ``more universal disposition toward morality'' that can be used in the same way for a unification as Christ had used the ``more universal disposition towards religion'' in his time (33; Dritter Brief, 12), but this does not appear to imply that the current group of `moralists' is associated with the general masses. As in the time of Christ, morality and religion need to be reunited, but the process through which this can happen will be different this time, because the circumstances have changed. 

Soll nun die Philosophie nach ihrer Art an der Religion thun, was das Christenthum nach der Seinigen an der Moral gethan hat, indem es von Religion zur Moral durch den \textbf{Weg des Herzens} f\"{u}hrte, so mu\ss{} sie von der Moral zur Religion durch den \textbf{Weg der Vernunft} zur\"{u}ckf\"{u}hren. (Dritter Brief, 13)

[If philosophy, in its own way, is to do to religion what Christianity in its way did to morality, then as Christianity led from religion to morality by means of the heart, so philosophy must lead from morality back to religion by means of reason.] (34)

The connection is currently to be established in the opposite direction and by the opposite means in comparison to the connection introduced in early Christianity. This means that religion has to be founded upon morality by reason. This latter part is of course in line with Kant's claims, even if the preceding analysis of Christianity is not. It is only here, fourteen pages into the article that Reinhold can start to address the worries of his correspondent, namely that the clearing away of any proofs for God's existence may damage religion rather than support it. According to Reinhold, the establishment of the much needed foundation of religion upon morality is hindered by the claims that religion must either be founded upon ``hyperphysical events'' or upon ``metaphysical speculations'' (34; Dritter Brief, 14). As in his first two `Briefe' Reinhold refers to the debate between Jacobi and Mendelssohn and to the general confusion regarding the possibility of providing universally valid grounds for religion that resulted from this debate. In order to disarm the metaphysical and hyperphysical competition for good, however, reasons must be given ``that themselves nullify all counterproofs as well as proofs'' (37; Dritter Brief, 19). With this Reinhold can satisfy his fictitious correspondent that the toppling of metaphysical proofs by Kant is not only undertaken without harming religion, but even for the sake of providing religion with a more appropriate foundation. He offers a further insight into the manner in which the old foundations have to be overcome. 

Soll dem moralischen Erkenntni\ss{}grunde sein \textbf{Vorzug der Einheit} auf immer zugesichert, und der Vernunft ihr endloses Bestreben nach neuen Beweisen, (\ldots ) auf immer eingestellt werden; so m\"{u}ssen die Gr\"{u}nde, welche die Nichtigkeit der metaphysichen Beweise f\"{u}r und wider das Daseyn Gottes aufdecken, nich nur die bisher vorgebrachten, sondern alle m\"{o}glichen Beweise dieser Art, oder vielmehr, ihre \textbf{M\"{o}glichkeit }selbst treffen; eine Sache an die sich nicht denken l\"{a}\ss{}t, bevor es nicht apodiktisch erwiesen ist, \quotedblbase da\ss{} die Vernunft kein Verm\"{o}gen besitze, das Daseyn, oder Nichtseyn von Gegenst\"{a}nden zu erkennen, die ausser der Sph\"{a}re der Sinnenwelt liegen.`` (Dritter Brief, 19{-}20)

[If the moral ground of cognition is to be guaranteed its singular pre{-}eminence, and reason is to be forever suspended from its endless striving for new proofs (\ldots ), then the arguments that uncover the emptiness of metaphysical proofs for and against God's existence must count not only against previous proofs that have been brought forward but also against all possible proofs of this kind {--} or rather, against their very possibility. Such a state of affairs cannot be conceived until it is apodictically proven `that reason does not possess any faculty for recognizing the existence or non{-}existence of objects that lie outside of the world of sense'.] (37{-}38)

Now, of course, Reinhold claims that Kant has undertaken an investigation of reason to the effect that exactly this last claim is proven, and that thereby all metaphysical proofs for or against God's existence are shown to be illegitimate. Apart from this, Reinhold also wants to establish that Kant effectively criticized hyperphysical arguments for the existence of God. In order to do this, Reinhold cites about a page and a half from Kant's `Orientation'{-}essay. The upshot of the cited passage is that the ``concept of God'' can only originate from reason, since no intuition can be adequate to it.\footnote{ \textit{Letters}, 39{-}40; Dritter Brief, 22{-}23. The cited text corresponds to \textit{AA} 8: 142{-}143. Kant is not sticking to his own terminology of the first \textit{Critique} here. He uses the term `concept of God' to argue that this concept is in fact an idea in the sense of the first \textit{Critique}, that is, that its content cannot be intuited. } This confirms Kant's criticism of the hyperphysical as well as the metaphysical arguments by establishing, on one hand, that reason must have the right to judge ``in matters concerning supersensible objects such as the existence of God and the future world'' (40; Dritter Brief, 23)\footnote{ Citing Kant, `Was hei\ss{}t: sich im Denken orientiren?' \textit{BM }October 1786, 322; \textit{AA} 8:143.} and on the other hand showing that the idea God as an idea of reason does not receive its material from sensibility. Thus, Reinhold's citation of Kant here confirms his reading of the achievement of Kantian philosophy vis{-}\`{a}{-}vis both hyperphysical and metaphysical arguments for the existence of God. As a bonus, the cited passage reflects Reinhold's previously expressed opinion that the Reformation had been a crucial preparatory development for Enlightenment, by establishing reason's right to judge in matters of religion. Both the indispensability (of reason in general) and the limitations of (speculative) reason with regard to the fundamental truths of religion are being confirmed as Kantian tenets by Reinhold's citing from the `Orientation'{-}essay. 

 Having confirmed his authority as a spokesman on behalf of the Kantian philosophy, Reinhold has no need to linger over the texts of the master, but swiftly returns to the issue at hand, that is, the way in which the moral `ground of cognition' is able to supersede the previous (metaphysical and hyperphysical) grounds of cognition for God's existence. Since the previous grounds on their own were ``neither completely true, nor completely false,'' they were in need of determination by ``a more fundamental concept'' (40; Dritter Brief, 24). This is indeed what the moral ground of cognition is credited with: it ``imparts determination and internal coherence to all the metaphysical doctrinal principles that belong to rational theology'' (42; Dritter Brief, 27). While there is no doubt about the enthusiasm expressed here regarding Kant's achievements in the first \textit{Critique}, Reinhold has yet to present the manner in which the moral ground of cognition can do this. Even in what he announces as an example of how this ground of cognition can actually secure the achievements of metaphysics with regard to theology, he remains rather vague.

Denn so wie der moralische Erkenntni\ss{}grund als der einzige Probehaltige fest steht, erhalten die \textbf{Notionen}, welche von der \textbf{Ontologie}, \textbf{Kosmologie} und \textbf{Physikotheologie} zum Lehrgeb\"{a}ude der reinen \textbf{Theologie} geliefert werden, auf einmal Inhalt, Zusammenhang und durchg\"{a}ngigen Bestimming. So bald das sonst \textbf{unerweisliche Daseyn }des Wesens, dessen \textbf{Idee} sie der \textbf{spekulativen Vernunft} festsetzen, und vollenden helfen, auf das nothwendige und unwiderstehliche Geboth der \textbf{praktischen Vernunft }angenommen ist, empfangen sie gewisser massen iher wirkliches ausser der Idee befindliches Object. (Dritter Brief, 28)

[For, just as the moral ground of cognition stands firm as the only one that survives testing, it gives at once content, coherence, and thoroughgoing determination to the notions that are supplied by ontology, cosmology, and physico{-}theology for the doctrinal structure of pure theology. As soon as the otherwise indemonstrable existence of a being whose idea these notions fix and help to complete for speculative reason is accepted on the necessary and irresistible demand of practical reason, these notions receive, in a certain way, their actual object {--} an object that lies outside the idea.] (43)

From this `example' we can gather that Reinhold believes that, although the existence of God is indemonstrable by means of the ideas of speculative theology, these ideas can and do have an external object once the existence of God is accepted from the moral ground of cognition. The process of this determination of the idea of God by means of the moral ground of cognition would have been a lot clearer if Reinhold had provided a clear statement of his understanding of Kant's so{-}called `moral argument', that is, the argument that the conviction that there is a God is justified because it is a necessary assumption of moral action. Note that, according to Kant, this justified conviction does not imply that we know that God exists. Although Kant was to present the moral argument in detail only later, in his second \textit{Critique},\footnote{ Kant's \textit{Prolegomena }(1783) and \textit{Grundlegung zur Metaphysik der Sitten} (1785) would have been available to Reinhold, but do not focus on the moral argument. From the letter to Voigt we know that Reinhold was at least aware of both of these works. Cf. \textit{RK} 1:148, 157. } it figures in abbreviated forms in both the first \textit{Critique }and the `Orientation'{-}essay.\footnote{ \textit{KrV}, A 808{-}810; \textit{AA} 8:139. } At the point where Reinhold comes closest to giving an argument his starting point is the thought that speculative reason fails to provide a rational ground or proof, while practical reason succeeds, not in providing an apodictic proof of God's existence, but in providing a rational ground for the conviction that God exists. The reason why speculative reason fails is that its demonstrations require that the concepts or ideas have an object in possible experience, which is not the case with the idea of God. As we have seen, Reinhold had already cited Kant himself authorizing this claim. In his understanding the moral argument works because, contrary to speculative reason, the ideas of practical reason \textit{do} have their objects in experience, that is, in moral actions.

Inde\ss{} alle Ideen der \textbf{spekulativen} Vernunft ohne Ausnahme von aller Anschauung leer sind, das hei\ss{}t, keinen Gegenstand habe, der in einer wirklichen oder m\"{o}glichen Erfahrung, den einzigen Erkenntni\ss{}gr\"{u}nde alles Daseyns, vorkommen k\"{o}nnte: sind die Ideen der \textbf{reinen praktischen }Vernunft, durchaus bestimmt, in einer wirklichen Erfahrung ( in den moralischen Handlungen der Menschen) ihre Gegenst\"{a}nde zu erhalten. (Dritter Brief, 30)

[Whereas all the ideas of speculative reason are without exception void of all intuition {--} that is, they have no object that could occur in an actual or possible experience, which are the only grounds of cognition of anything existent {--} the ideas of pure practical reason are definitely destined to be given their objects in an actual experience (in the moral actions of human beings).] (amended translation KJM, cf. \textit{Letters}, 44) 

Note that the argument does not actually conclude that practical reason can provide a rational ground for believing that God exists. It presents the relation of ideas (speculative or practical) to experience as the crucial difference between speculative and practical reason. This difference might then be considered the reason why speculative reason fails to provide a rational ground for the conviction that God exists, while practical reason succeeds. Still, there is no straightforward argument here. First of all, Reinhold would have to argue that the idea of God is among the ideas of pure practical reason. Further, the claim that this idea will be given its object in the moral actions of human beings would require more argumentation. Especially if the idea of God is an idea of practical reason, this claim would be hard to understand. 

Although there is no straight argument here, Reinhold's perspective does appear to be related to the version of the moral argument in Kant's first \textit{Critique}, which also figured in the `Orientation'{-}essay. In both cases, Kant had argued that in order for the practical ideal of a moral world (to be realized by our moral actions) to have objective reality, the idea of the highest good must be accepted by us as having objective reality and with it the idea of God, who would make this highest good possible. In his emphasis on moral action Reinhold may have taken Kant's reply to his counter{-}review to heart, which also stressed that human action yields a relevant kind of experience.\footnote{ Cf. Chapter 3, section 1.3.} There is also an important parallel between Kant stressing the role of the idea of a moral world as an ideal to be brought about in the sensible world and Reinhold's phrase that ideas of practical reason are ``destined to be given their objects'' in moral action.\footnote{ It is for this reason that I have given my own translation here, instead of Hebbeler's, which reads: ``the ideas of pure practical reason are thoroughly determined, for they have objects in an actual experience.'' \textit{Letters}, 44. I believe this is a serious misinterpretation of the phrase ``durchaus bestimmt (\ldots ) ihre Gegenst\"{a}nde zu erhalten.''} Although Reinhold may be hinting at Kant's moral argument as it was available at the time, his version of it is at best partial and omits crucial steps. Reinhold appears to believe that it is the connection to experience that allows practical reason to provide a rational foundation where speculative reason fails to do so. If this is the case, he overlooks the circumstance that the failure to connect to intuition is only a problem when one tries to \textit{prove} speculatively that something exists. By stressing the connection to actual experience Reinhold suggests that practical reason can achieve something equivalent to apodictically proving God's existence. In the end, Reinhold does not put forward an interpretation of Kant's moral argument, or an explanation of the actual nature and role of practical reason, but simply asserts that there is such an argument.\footnote{ In one of his letters to Kant (January 19, 1788) Reinhold showed awareness of the incompleteness of his account. ``Wie lieb ist mirs nun da\ss{} ich mich in meinen \textit{Briefen \"{u}ber die kantische Philosophie }bis itzt noch nicht auf die eigentliche Er\"{o}rterung des \textit{moralischen Erkenntni\ss{}grundes der Grundwahrheiten der Religion} eingelassen habe. Ich h\"{a}tte da ein schwaches L\"{a}mpchen aufgesteckt wo \textit{Sie} durch die Kr. d. pr. V. eine Sonne hervorgerufen haben.'' \textit{RK} 1:313.} His interpretation of the moral argument as focused upon providing the lacking connection between reason and sensibility, however, is very understandable when we take into account that his pre{-}Kantian philosophy centered on the thought that in order for Enlightenment to succeed reason needs to be connected to sensibility. His messy understanding of Kant's moral argument is a clear sign that he is working from his own, not Kant's perspective. 

Towards the end of the third installment of his `Briefe' Reinhold proceeds to relate the insights regarding the grounds of cognition for God's existence to religion in general. This yields important information on the background of his understanding of the moral argument and practical reason. 

\textbf{Wie der Erkennti\ss{}grund f\"{u}r das Daseyn und die Eigenschaften der Gottheit: so die Religion}. Isolierte Sinnlichkeit, vernunftloses Gef\"{u}hl, blindes Glauben reissen unaufhaltsam zum Fanatismus dahin; Isolierte Vernunft, kalte Spekulation, ungeregelte Wi\ss{}begierde f\"{u}hren, wenns hoch k\"{o}mmt, zum frostigen, gr\"{u}belnden, unth\"{a}tigen Deismus. \textbf{Vernunft und Gef\"{u}hl} hingegen in \textbf{ihrer Vereinigung}, {--} die Elemente der Sittlichkeit, {--}bringen den \textbf{moralischen Glauben }hervor, und machen (\ldots ), den einzigen, reinen und lebendigen Sinn aus, den wir f\"{u}r die Gottheit haben. (Dritter Brief, 33)

[What is true of the ground of cognition of the existence and properties of the deity is also true of religion. Isolated sensibility, feeling without reason, and blind faith pull inexorably toward fanaticism; isolated reason, cold speculation, and the unrestricted desire to know lead at best to icy, carping, inactive deism. Yet when they are unified, reason and feeling {--} the elements of morality {--} give rise to moral faith and constitute (\ldots ) the only pure and living meaning that we have for the deity.] (46)

Immediately following this passage, Reinhold identifies religion based on sensibility alone with hyperphysical religion, and religion based on reason alone with metaphysical religion. With this he has returned to the pantheism controversy. As hyperphysicists base their conviction of God's existence on (accounts of) supernatural experience, and metaphysicians base theirs on proofs of reason alone, we can extrapolate that the moral faith, based on practical reason, must rely on the unity of sensibility and reason to ground the conviction that God exists. Thus we arrive at the unity of head and heart supposedly established by Kant's first \textit{Critique}, upon which religion is to be founded and which completes the earlier unity established by Christianity. His comparison of the achievements of Kant and Christ is developed to the point where he can refer to the \textit{Critique }as a ``gospel of pure reason,'' following the ``gospel of the pure heart,'' that is, Christ's teachings (49; Dritter Brief, 39).

 All in all, the third installment of Reinhold's `Briefe' does not lead the reader into the speculations of the first \textit{Critique} but rather continues to relate its claimed results to external circumstances. Although Reinhold does cite Kant's `Orientation'{-}essay literally and at length, the account he gives of Kant's moral argument is at best an abbreviated and elliptic version of Kant's expressions of it up to that point. Reinhold's portrayal of the manner in which, according to him, the Kantian philosophy fulfils the current need for a rational religion is clearly related to his previously expressed views on Enlightenment. Both the vocabulary and the theme of the unification of our rational and sensory capacities are almost directly taken from the works written in the period prior to his acquaintance with Kant. The Kantian philosophy appears to fulfill a role here similar to that of the Enlightenment in `Gedanken \"{u}ber Aufkl\"{a}rung', namely the role of providing the unification between the concrete world of the masses and the abstract reasoning of the philosophers. 

 Like the third, the fourth installment of the `Briefe' (February 1787) can be seen as a continuation of Reinhold's previous interests in philosophy, for it presents an overview of the ``elements and the previous course of conviction in the basic truths of religion'' (50; Vierter Brief 117). As this title indicates, a substantial part of the fourth installment is dedicated to an historical overview of the course of development of the moral ground of cognition concerning the conviction that God exists. In the second chapter, we have seen a similar strategy on Reinhold's part with regard to the concept of Enlightenment. If his systematic elucidation did not suffice to convince his audience that Enlightenment was desirable, his historical account of the development of reason would have to show that it was inevitable. Moreover it would show how the previous phases of the development of reason were but phases, characterized by the one{-}sided development of human capacities. With regard to the moral ground of cognition for the existence of God, Reinhold now proceeds similarly. Having proposed, in the third `Brief', the systematic claim that the moral ground of cognition supersedes both the metaphysical and the hyperphysical grounds, he now seeks out the elements of this moral ground and shows their development through history, in order to strengthen his claim that the current philosophical circumstances require that religion be based on morality by means of reason. 

 Reinhold opens the article by claiming that the result of Kant's \textit{Critique} corresponds to the basic truths common sense has cherished with regard to religion, namely that there is a God and that the soul continues to exist when the body dies. Now the ground upon which common sense is committed to these claims consists, in Reinhold's opinion, of nothing other than 

dem \textbf{Gef\"{u}hle des moralischen Bed\"{u}rfnisses}, welches durch die Kritik der Vernunft in deutliche Begriffe aufgel\"{o}set, und zum einzigen und h\"{o}chsten philosophischen Erkenntni\ss{}grunde der Religion erhoben worden ist. (Vierter Brief, 120)

[the feeling of the moral need, which the Critique of Reason has resolved into distinct concepts and elevated to the single and highest philosophical ground for cognition of religion.] (51)

This means that, according to Reinhold, Kant's \textit{Critique} has first expressed and elucidated a ground of cognition for religion that had already been the basis of religion in a pre{-}reflected, common{-}sense way. In one of the previous `Briefe' Reinhold had already likened Kant's achievement to Newton's discovery of the spectrum of colors making up white light. It was not as if, before that discovery, light was not comprised of seven visible colors, only that nobody knew that it was (Zweyter Brief, 128). The identification of a basic feature of common sense with the results of the subtle philosophy in the \textit{Critique} almost automatically leads to the historical question: Why has it taken so long and has it been so hard to establish this result by reason? Reinhold's answer to this question presupposes in a way his views on the development of human reason throughout history as expressed in his pre{-}Kantian works, because the general line of his answer is that the uneven development of human reason has hindered a thorough understanding of this moral need. 

The two fundamental truths of religion {--} as Reinhold calls the convictions that there is a God and an afterlife {--} are characterized as ``just as incapable of being intuited, as they are necessary according to concepts'' (53; Vierter Brief, 123). The circumstance that these truths cannot be intuited forms the first element of faith, the second element being their necessity. In the early days of the development of reason, people had to support these incomprehensible, yet necessary convictions by means of ``intuitions that likewise had to contain something incomprehensible so that they could testify to the existence of incomprehensible objects'' (54; Vierter Brief, 125).\footnote{ Reinhold still appears to be a naturalist when it comes to explaining historical religion. Cf. Chapter 2, section 4.} This, according to Reinhold, explains the development of the so{-}called historical ground of cognition of these religious convictions, that is, a ground that is taken from experience. In comparison to his previous `Briefe' Reinhold appears to switch terminology here. Earlier he had used the dichotomy `hyperphysical{-}metaphysical' with regard to religion and the grounds of cognition for its fundamental truths. This dichotomy refers to the sources of these grounds, that is, they originate either from supernatural revelation or from metaphysics. The dichotomy in use here, `historical{-}philosophical' refers to the same distinction but from the perspective of the way in which we come to know these grounds, that is, either by experience and historical tradition or by using reason. The `historical{-}philosophical' dichotomy is reminiscent of his `Schreiben des Pfarrers' in which he had used this distinction in relation to the two `poles of the totality of human knowledge', experience and metaphysics.\footnote{ Cf. Reinhold, `Schreiben des Pfarrers,' 174; cf. Chapter 3, section 1.2.} Although he is no enthusiast for historical religion himself, Reinhold stresses its necessity for the development of reason. 

So wie er auf der eine Seite allen Kr\"{a}ften der Sinnlichkeit und Phantasie f\"{u}r das Interesse der Religion aufbieten konnte und mu\ss{}te, wenn bey dem damaligen Zustande der h\"{o}heren Geisteskr\"{a}fte die Aufmerksamkeit der Menschen vom Sichtbaren auf das Unsichtbaren gelenket werden sollte: so war auf der andere Seite nichts nat\"{u}rlicher, als da\ss{} er die Religion, eben durch das Uebergewicht, welches er dem Unsichtbaren \"{u}ber das Sichtbare gab, zum ersten und \"{a}ltesten Gegestande der Nachforschungen denkender K\"{o}pfe machen mu\ss{}te. So war der historische Erkenntni\ss{}grund als Vorbereitung zum Philosophischen unentbehrlich. (Vierter Brief, 126)

[On the one hand, it could and had to summon all the powers of sensibility and fantasy for the interest of religion so that, given the state of the higher mental powers at the time, the attention of human beings could be steered from the visible to the invisible. On the other hand, however, there was nothing more natural than the fact that, precisely by giving priority to the invisible over the visible, the historical ground of cognition had to make religion into the first and oldest object of investigation for thinking minds. In this way, the historical ground of cognition was indispensable as preparation for the philosophical.] (54{-}55)

This two{-}sided picture of the development of the historical ground of cognition fits perfectly with Reinhold's pre{-}Kantian way of dealing with history. First of all, he stresses that this ground of cognition was a perfectly acceptable, even necessary answer to the needs of reason at the time when it was developed. We have seen a similar regard for the historical context in his Vienna reviews, in which it was argued for instance that Latin had indeed served a rational purpose as language of Christianity, even if it had later come to hinder people's understanding of the Mass.\footnote{ Cf. Reinhold's review of \textit{Abhandlung von der Einf\"{u}hrung der Volkssprache in den \"{o}ffentlichen Gottesdienst (\ldots )}, \textit{RZ}, 1783, March 4 [176]. } Secondly, the historical ground of cognition is placed in the wider context of the development of reason, as it is described as the necessary preparation for the stage of philosophical religion, even containing the germs, as it were, for the onset of that stage. However, with the beginning of philosophical investigation into the objects of religion, the impossibility of intuiting these objects remained. Again, Reinhold places the development of philosophical religion based on metaphysical grounds of cognition firmly within the context of the development of human reason. 

Die Entdeckungen, die man auf dem Wege der Vernunft \"{u}ber den \textbf{Gegenstand} des Glaubens gemacht hatte, wurden auf den \textbf{Grund} des Glaubens in eben dem Verh\"{a}ltnisse \"{u}bertragen, als mit der Kultur des Geistes einerseits die \textbf{Evidenz} jener Entdeckungen, andererseits aber das \textbf{Bed\"{u}rfni\ss{}} zugenommen hatte, sich \"{u}ber den \textbf{Grund des Glaubens} Rechenschaft zu geben. Von diesem Bed\"{u}rfnisse gedrungen, und von jener Evidenz geblendet, schlo\ss{} man von den Attributen der Gottheit in der Idea, auf das Daseyn derselben im Objekte, unterschob den Gegenstand des Wissens dem Gegenstande des Glaubens und glaubte von dem letztern bewiesen zu haben, was eigentlich nur von dem ersteren gelten konnte. Der philosophische Erkenntni\ss{}grund des leeren Wissens, der sich hierauf immer mehr und mehr festsetzte, war also der Vernunft auf ihrem Wege zurm Moralischen eben so unvermeidlich als der historische. (Vierter Brief, 128{-}129)

[The discoveries regarding the object of faith that were made on the path of reason were projected onto the ground of faith at the same rate that the evidence of these discoveries, on the one hand and the need to give oneself an account of the ground of faith, on the other, increased with the cultivation of spirit. Compelled by this need, and blinded by that evidence, one inferred from the attributes of the deity in the idea to their existence in an object, presented as an object of knowledge what was an object of faith, and took oneself to have proven something of the object that could only properly hold true of the idea. The philosophical ground of cognition of empty knowledge, which established itself ever more firmly in this way, was thus as inevitable for reason on the way to the moral ground of cognition as the historical ground was.] (56) 

The rationale behind the development of the philosophical ground of cognition is again provided by ``the cultivation of the spirit,'' or the degree of development of reason at a certain time in human history. It is clear that for Reinhold the philosophical ground of cognition is closely related to the so{-}called ontological argument for God's existence. Like historical religion's recourse to miracles, this argument can be understood as a way of dealing with the problem of the lack of intuition for fundamental religious convictions. Like historical religion, philosophical religion is a necessary stage in the development towards moral religion. 

Even though philosophical religion developed from historical religion, the two kinds of ground they offer for the fundamental truths of religion are not compatible. This has only become clear, however, since the revival of the sciences and has culminated in the clash of metaphysical and hyperphysical arguments in the pantheism controversy which in this manner is likewise placed in a wider historical context (cf. 59{-}61). It is clear that, according to Reinhold, the main problem with the fundamental truths of religion is the combination of inevitability and incomprehensibility. On the one hand, the concept of an infinite being is a necessary concept of the human mind, which, on the other, can provide no intuition to support it. These two elements of religion have, in combination with the growing development of human reason, determined the outlook of both historical and philosophical religion. Now that these forms of religion stand in opposition to one another it is time to come to a rational faith and introduce a third element binding the two together, the ``command of practical reason, which makes moral faith necessary'' (62; Vierter Brief, 139{-}140). Needless to say that Reinhold believes that Kant was the first to present this third element with the required clarity. His explanation of the way in which the third element can unite the other two again provides some insight into his understanding of Kant's project. 

Die \textbf{Unbegreiflichkeit des g\"{o}ttlichen Daseyns} vertr\"{a}gt sich so lange nicht in einer und eben derselben Vorstellung mit dem \textbf{nothwendigen Vernunftbegriffe }von der Gottheit, als nicht das \textbf{unwiderstehliche Geboth der praktischen Vernunft }anerkannt wird, welches den Begriff des Daseyns, der sich hier nicht beweisen l\"{a}\ss{}t, mit dem Vernunftbegriffe, der durchaus bewiesen werden mu\ss{}, zu verbinden n\"{o}thiget, und den \textbf{Ueberzeugungsgrund der Religion} vollendet, in dem es dasjenige, was an demselben ewig unerweislich bleiben mu\ss{}, zu unsrer Befriedigung ersetzet. (Vierter Brief, 141)

[As long as the irresistible command of practical reason is not recognized the incomprehensibility of divine existence is not compatible in one and the same representation with reason's necessary conception of the deity. This command requires connecting the concept of existence, which cannot be proven in this case, with the concept of reason, which certainly must be proven, and it completes the ground of conviction in religion by making up, in a satisfying way for that which must forever remain indemonstrable in this ground.] (63{-}64)  As in the third installment of the `Briefe', Reinhold does not actually provide anything like a moral argument here, but is content with presenting the problem in some detail and then claiming that it has now been solved because of Kant's \textit{Critique}. His `explanation' of the Kantian \textit{solution} definitely lacks clarity and does therefore not allow for conclusions concerning the depth of his knowledge of Kant at this time. His explicit formulation of the \textit{problem}, however, in terms of a necessary concept of reason, which lacks the connection to intuition that would enable the existence of its object to be proven, shows that he was well aware of the problem of the theological idea as presented by Kant.\footnote{ Cf. \textit{KrV}, A 636{-}637. } His main achievement in the third and fourth `Briefe' is therefore not his exposition of the Kantian solution to the problem concerning the provability of God's existence, for the exposition provided is formulated in a vague and superficial manner. Rather, his merit lies in the formulation of the problem, which is clear and true to Kant. If Reinhold had done nothing more than present the problem to the rather broad audience of the \textit{Merkur}, he would already have taken a major step towards providing the \textit{Critique} with a wider readership. 

The unique selling point of these `Briefe' is, however, that Reinhold presents the problem in the context of the development of human reason throughout history. In both a systematic and a historical way this leads to the conclusion that the problem is inevitable, but that the solution is close at hand.\footnote{ Karl Ameriks has already pointed out the novelty of Reinhold's approach to the history of philosophy and the importance of this approach for the subsequent developments in philosophy. Marion Heinz has discussed the intimate relation between the systematic and the historical approach in Reinhold's later \textit{Fundamentschrift}. Cf. Ameriks, \textit{Kant and the Historical Turn}; Ameriks, `Reinhold \"{u}ber Systematik, Popularit\"{a}t und die `historische Wende''; Heinz, `Untersuchungen zum Verh\"{a}ltnis von Geschichte und System der Philosophie in Reinholds \textit{Fundamentschrift}.' } Systematically, the problem is a result of the structure of the human mind, in which some ideas present themselves as necessary, without there being a possibility to prove the existence of the objects of those ideas. Historically, the problem results from the uneven development of human rationality, which led to the development of competing and incompatible solutions to the problem. By presenting the problem as stemming from the structure of reason and having developed because of the lack of clarity regarding the nature of reason, Reinhold provides more plausibility for his claim that Kant, by investigating the very structure of reason, has solved the problem. Moreover, this way of presenting the problem enables Reinhold to claim that the Kantian solution is not really new at all, for religion had been founded upon the moral need all along. Kant is only credited with unearthing this fact regarding the foundation of religion. His investigation of reason has achieved a rational insight into this foundation, thereby completing the development of reason. For Reinhold this way of presenting Kant is crucial to his efforts to counter the important negative comments and reviews, the main objection of which was that the \textit{Critique} amounted to incomprehensible idealism. By presenting Kant as only elucidating a ground that had always been present and fundamental to a common{-}sense foundation of religion, Reinhold seeks to show that, although the \textit{Critique }is hard to read, the results of Kant's work are indeed far from incomprehensible or detrimental to religion. 

Taken together, the third and fourth `Briefe' present a structure similar to Reinhold's earlier `Skizze', discussed in Chapter 2. In that article, he presented a sketch of a history of religion according to which the initial harmony between God and mankind was lost due to the incomplete development of reason. With the full development of reason, the harmony would be found again, but at another level, where it would not only be felt but also understood (cf. section 4 of Chapter 2). True to his promise not to lead his correspondent ``into the depths of speculation from which Kant had unearthed so many previously undiscovered treasures of the human spirit'' (16; Erster Brief, 125), Reinhold does not expound the reasoning behind the Kantian solution, that is, the moral argument, in any detail {--} not even in the sketchy version that Kant himself had given up to that point. Instead, he is content with claiming that practical reason provides a ground for this conviction ``by making up, in a satisfying way, for that which must forever remain indemonstrable'' (64; Vierter Brief, 141). Rather than presenting Kant's arguments, he provides an elaborate structure of his own, centering on the development of human reason throughout history. It is in this respect that his account is closer to his own previous works on Enlightenment than to Kant's writings. 


\subsection{The final four `Briefe'}


Like the first two `Briefe' and the third and fourth taken together, the final four \textit{Merkur}{-}`Briefe' form more or less a unity. Again, there was a discontinuity in the publication, as the series was resumed in May 1787 while the fourth installment had been published in February 1787. Roughly, these final four `Briefe' show a structure similar to the third and fourth, but with the focus upon the conviction that the soul continues to exist after the death of the body. This is to say that Reinhold seeks to establish how different philosophical opinions regarding the nature of the soul arose from the structure of the human mind and how the development of these opinions throughout history led to a seemingly insolvable conflict, to be solved only by Kant's \textit{Critique}. The fact that this took him four, instead of two, installments is explained by his desire to elaborate, in the final two installments, on the historical backgrounds of the conception of the soul as a simple, thinking substance. Keeping the structure of the last four `Briefe' in mind, this section is divided into two subsections, the first of which will focus on analyzing Reinhold's discussion of the grounds for the belief in a future life, while the second will discuss his considerations regarding the history of the concept `soul'.


\subsubsection{Grounding the conviction of the continuing existence of the human soul}


The title of the fifth article in the series of `Briefe' (May 1787), `The result of the Critique of Reason concerning the future life', parallels the title of the second. The installment opens with the claim that there are great similarities between the nature and fate of both fundamental truths of religions, that is, the convictions that God and the future life exist (F\"{u}nfter Brief, 168). Reinhold refers to the conviction that there will be life after death as a `postulate of practical reason' (\textit{Postulat der praktischen Vernunft}). Although this is terminology that a modern philosophical reader would immediately associate with Kant, Reinhold's use of it is very remarkable, since Kant had not yet used it at the time and was only to introduce the phrase in the second \textit{Critique}. In the first \textit{Critique} God and the afterlife are termed `presuppositions' (\textit{Voraussetzungen}), while the term `postulate' is predominantly used in the context of the `postulates of empirical thinking'.\footnote{ Cf. \textit{AA} 5:122, 124, 132; cf. \textit{KrV}, A 811, A 218; \textit{KrV}, A 633{-}634 refers to ``postulating'' the existence of something that is a condition for necessary practical laws. } The closest Kant had come at the time to introducing the term `postulate' in a practical context is in the `Orientation'{-}essay, in which he calls the ``rational faith'' a ``postulate of reason'' and stresses that although holding something to be true in this manner is different from knowledge, it is certain in its own way.\footnote{ Cf. \textit{AA} 8:141. } Since Reinhold almost certainly derived the term `postulate' from this context, the importance of this essay for the development of his understanding of Kant as expressed in the `Briefe' is obvious. His fifth article in the series claims that as we have a moral interest in being convinced of God's existence, so we have a moral interest in believing there is an afterlife. 

Das \textbf{Moralische} hingegen (\ldots ) gr\"{u}ndet sich auf das entweder \textbf{gef\"{u}hlte}, oder \textbf{deutlich erkannte }Bed\"{u}rfni\ss{}, welches die Vernunft n\"{o}thiget, zum Behuf ihrer moralischen Gesetze eine Welt, in welcher Sittlichkeit un Gl\"{u}ckelichkeit in vollkommenster Harmonie stehen, anzunehmen, und die anschauende Erkenntni\ss{} dieser Harmonie f\"{u}r die Zukunft zu erwarten. (F\"{u}nfter Brief, 168{-}169)

The moral intrest, in contrast (\ldots ) is grounded on a need that is either felt or distinctly cognized. This need requires reason, on behalf of its moral law, to assume a world in which morality and happiness stand in most perfect harmony and to expect an intuitive cognition of this harmony in the future. (66) 

Historically, this conviction appeared somewhat later than the conviction that God exists, because the ``expectation of reward and punishment presuppose faith in a judge'' (67; F\"{u}nfter Brief, 171). Systematically, however, the conviction that a moral world is possible and the expectation to experience this world in the future would appear to be prior to the conviction that God exists. The versions of the moral argument for the existence of God provided by Kant in the first \textit{Critique }and in the `Orientation'{-}essay both proceed from the need to presuppose the possibility of a moral world.\footnote{ Cf. \textit{KrV}, A 808{-}810; \textit{AA} 8:139. In the second \textit{Critique} the postulate of immortality is treated prior to the postulate of the existence of God as well. Cf. \textit{AA} 5:122{-}132. } Since Reinhold does not discuss the moral argument in any detail, the incongruity between the historical priority of the conviction that God exists and the systematic priority of the conviction of the immortality of the soul is not obvious from the `Briefe'. Reinhold certainly does not seem to be bothered by the difference and stresses the similarity of the fates met by these two core convictions of religion, which provides him with a reason to abstain from discussing the conviction of the immortality of the soul in detail. Apart from the similar historical fates of the two fundamental religious convictions, the main similarity is their foundation upon a felt moral need that was only to be articulated in a rational manner by Kant. The common{-}sense basis of the conviction of the immortality of the soul is ``the concept of a good or bad fate after death that is determined by the moral course of one's life prior to death'' (68; F\"{u}nfter Brief, 173). Since, as Reinhold claims, it took some time before Kant came along and firmly grounded this conviction in reason by giving this need a rational articulation, the previous, fallacious, grounds proposed for this conviction were historically necessary in order to pave the way for this final articulation. 

Diese h\"{o}chstwichtige Entdeckung war so lange unm\"{o}glich (\ldots ) als nicht einerseits die Evidenz des Sittengesetzes durch die Fortschritte der moralischen Cultur jenen Grad von St\"{a}rke erreicht hatte, die der \textbf{Einzige} Ueberzeugungsgrund von den Grundwahrheiten der Religion haben mu\ss{}, andererseits aber die spekulative Vernunft in ihrem Selbsterkenntnisse so weit gekommen, das sie die Unm\"{o}glichkeit sowohl historischer als spekulativer Beweise f\"{u}r das Daseyn und die Beschaffenheiten von Gegenst\"{a}nden, die ausserhalb der Sinnenwelt liegen, einsehen mu\ss{}te. (F\"{u}nfter Brief, 175)

[This most important discovery was not possible (\ldots ) as long as the evidence of the moral law had not reached, through the advances of moral cultivation, the degree of strength that the sole ground of conviction in the basic truths of religion must have. Moreover, it was not possible as long as speculative reason had not come far enough in its self{-}cognition for it to gain insight into the impossibility of historical as well as speculative proofs regarding the existence and nature of objects lying outside the world of sense.] (69{-}70)

Because of the similarity of the two convictions, their moral origins and the fate they met through history, Reinhold simply forgoes any explanation of what a Kantian articulation might look like. Instead, he proceeds by showing why the historical and metaphysical grounds for this conviction that have been proposed throughout history are spurious, that is, not at all related to morality. 

 Thus, the remainder of the fifth installment is dedicated to showing that historical religion, with its hyperphysical ground of cognition has only served to separate religion and morality. As that ground is supposed to be taken from supernatural revelation, it is beyond the reach of reason and ``there is no necessary connection, illuminative to reason between the moral law and future rewards and punishments,'' because the divine will prescribing that law must be considered to be incomprehensible (72; F\"{u}nfter Brief, 180). The hyperphysical ground of cognition turns out to hinder morality rather than contribute to it. Again, the Reformation is hailed as the first step towards true Enlightenment. 

Seit dem es den Protestanten gelungen hat, sich von der Bothm\"{a}\ss{}igkeit der unfehlbaren Ausleger des unbegreiflichen Willens loszumachen, hat sich ihre religi\"{o}se Moral mit starken Schritten der Moral der Vernunft gen\"{a}hert. (F\"{u}nfter Brief, 184)

[Ever since Protestants managed to free themselves from subordination to the infallible explicators of the incomprehensible will, their religious morality has drawn closer to the morality of reason with giant strides.] (74) 

With this by now familiar note on the importance of the Reformation, Reinhold need not waste any more ink on the hyperphysical ground of the conviction of the existence of a future life. His discussion of the metaphysical ground is postponed to the sixth installment of the `Briefe', aptly titled `Continuation of the preceding letter: the united interests of religion and morality in the clearing away of the metaphysical ground for cognition of a future life' (July 1787). As Reinhold's fictitious correspondent is portrayed as an adherent of this ground of cognition, the dismissal of the metaphysical grounds for the conviction of a continuation of the existence of the soul deserves extra attention. The aim of this article is therefore to show that 

durch denselben Erkenntni\ss{}grund, der Ihnen die Ueberzeugung vom zuk\"{u}nftigen Leben, und das Ansehen der Vernunft zugleich festzusetzen, und folglich das Interesse der Religion und der Moral zu vereinigen schien, in der Sache selbst dieses Interesse nicht weniger getrennt und entzweyt werden m\"{u}sse, als durch den entgegengesetzte \textbf{Historischen}, der doch durch jenen \"{u}berfl\"{u}ssig gemacht werden sollte. (Sechster Brief, 70)

the interests of religion and morality would, in truth, be no less divided and opposed to each other by the very ground of cognition that seems to you to establish at once both the conviction in a future life and the reputation of reason {--} and consequently to unite the interests of religion and morality {--} than they would be by the opposing historical ground, which was supposed to have been made superfluous by this metaphysical ground. (77{-}78)

In order to convince his readers that the metaphysical ground of cognition does not at all deliver what it promises, Reinhold first stresses his use of a narrow definition of this metaphysical ground of cognition. First of all, he does not understand by it any ground based on reason, since ``the moral faith that the Critique of Reason establishes is built entirely upon grounds of reason'' as well (78; Sechster Brief, 70). The truly metaphysical ground of cognition being based on the concept of the soul as a simple substance, Reinhold secondly narrows down his argument to direct it at a certain use of this concept only. He states that it is certainly legitimate to use the concept in ``defending against its opponents the basic religious truth of a future life'' (78; Sechster Brief, 71).\footnote{ Kant allows for a similar use of the concept of the soul as a defense against materialism. Cf. \textit{KrV}, A 383{-}384.} If the concept is not used merely in defense against those who deny the immortality of the soul, for instance because they believe the soul is material, but in order to claim something about the nature of the soul as it is in itself, it will become contradictory, so Reinhold warns. 

sobald er mehr als die blo\ss{}e Verschiedenheit der Pr\"{a}dikate des innern und \"{a}ussern Sinnes, so bald er das absolute Subjekt unsres denkenden Ichs, das eigentliche und unbegreifliche Wesen unsrer Seele ausdr\"{u}cken soll, so widerspricht er sich selbst: indem er ein blo\ss{}es Pr\"{a}dikat der Anschauung zum Subjekte an sich selbst macht, einem ganz unbekannten Dinge, in sofern dasselbe unbekannt ist, eine bekannte Beschaffenheit beylegt, und einen leeren Begriff zum wirklichen Gegenstand umschaft. (Sechster Brief, 72)

[It contradicts itself as soon as it is supposed to express more than the mere difference between predicates of inner and of outer sense, or as soon as it is supposed to express the absolute subject of the thinking I, the genuine and incomprehensible essence of our soul. And it does this by making a mere predicate of intuition into a subject in itself, by attaching a known nature [\textit{Beschaffenheit}] to a wholly unknown thing (\ldots ) and by transforming an empty concept into an actual object.] (79)

It is precisely this misuse of the `psychological concept of reason' that Reinhold calls the metaphysical ground of cognition for immortality. However, his criticism on it will not focus on the legitimate and illegitimate uses of this concept, but rather on its ``influence on the common interest of religion and morality'' (80; Sechster Brief, 73). Reinhold's way of dealing with the metaphysical grounds of cognition for this conviction runs more or less parallel to his discussion of the metaphysical grounds for the conviction that God exists. There he answered his correspondent's worries concerning the desirability of toppling the metaphysical grounds by pointing out the relation of these grounds to the unification of morality and religion, instead of presenting a Kantian argument against the validity of these arguments. Parallel to his presentation of the impossibility to gain speculative knowledge of God, Reinhold presents the impossibility to know the nature of the soul because it is not an object of a possible perception. 

(\ldots ) das \textbf{Subjekt} derselben [i.e. of inner sense, das der Verstand zwar als \textbf{Etwas} \"{u}berhaupt denken, aber nie als \textbf{etwas Bestimmtes erkennen} kann, weil es ausser dem Gesichtskreise aller m\"{o}glichen Wahrnehmung liegt. (Sechster Brief, 74)

[The understanding can certainly think of this subject [of inner sense] as something in general, but can never cognize it as something determinate because it lies outside the horizon of all possible experience.] (80)

To those who seek to determine the concept by predicating substantiality and simplicity, he replies that these determinations do not at all furnish the concept of the soul with an actual object, as they contradict the criteria by which actuality is established. 

Alle wirkliche Gegenst\"{a}nde, denen wir eine Subsistenz ausser unsren Begriffen einr\"{a}umen, m\"{u}ssen \textbf{irgendwo}, das hei\ss{}t, \textbf{im Raume} daseyn; und alles, was sich von diesen Gegenst\"{a}nden erkennen l\"{a}\ss{}t, ist blosses Pr\"{a}dikat. Dem einem dieser Kriterien der Wirklichkeit wiederspricht der Begrif des Einfachen, und dem andern der Begrif des Subjektes, das kein Pr\"{a}dikat seyn kann. (Sechster Brief, 75).

[All actual objects to which we grant subsistence outside our concept must exist somewhere {--} that is, in space {--} and all that can be cognized with regard to these objects are mere predicates. The concept of the simple contradicts one of these criteria of actuality, and the concept of a subject that cannot be a predicate contradicts the other.] (80{-}81)

The citation purports to show that thinking the soul as a simple substance, which is the basis of the metaphysical ground of cognition of immortality does not constitute knowledge of the soul at all, but rather attributes empty determinations to it. It is forever to remain empty, for it cannot be filled with perception. Moreover, Reinhold believes that putting such a concept of the soul at the foundation of the metaphysical ground of cognition is detrimental to the unity of morality and religion. As the demonstrations of the continuity of the soul after the death of the body concern precisely the soul as far as it cannot be known, that is, the soul in its capacity of simple substance, the survival of such a soul is of no interest at all to an actual person and thus will not provide a morally relevant ground for the religious tenet of the continuing existence of the soul. 

Da also die metaphysisch demonstrirte Fortdauer nach dem Tode nur dasjenige trift, was er von seinem Selbste \textbf{nicht} kennt, alles dasjenige aber, was er w\"{a}hrend seines Lebens kennen gelernt hat, entweder gerade zu von der k\"{u}nftigen Existenz ausschlie\ss{}t, oder wenigstens dar\"{u}ber in Ungewisheit l\"{a}\ss{}t; so mu\ss{} dem konsequenten Denker sein \textbf{k\"{u}nftiges }Daseyn in der \textbf{unsichtbaren Welt} ungef\"{a}hr eben so gleichg\"{u}ltig seyn als sein \textbf{voriges} Daseyn im \textbf{Reiche der M\"{o}glichkeiten}. (Sechster Brief, 80)

[Since, therefore, the metaphysical demonstration of survival after death concerns only what he does not know about himself, while everything that he has become acquainted with over the course of his life either excludes his future existence straight away or at least leaves him in uncertainty with regard to it, a consistent thinker must be just about as indifferent to a future existence in an invisible world as to a former existence in a realm of possibilities.] (83)

Having established that the metaphysical ground of cognition cannot succeed in demonstrating the immortality of the soul in a way that preserves the unity of morality and religion because of the emptiness of the determination of the metaphysical concept of the soul, the remainder of the article indicates how filling up this emptiness with fantasy is detrimental to this unity as well. Reinhold does not fail to refer to the pantheism controversy again as an ``expression of the perplexity in which reason finds itself when it becomes aware of the incongruity between its essential needs and its previous means for satisfying these needs'' (87; Sechster Brief, 86). The sixth `Brief' closes with the promise to elucidate this incongruity in the following installment, on the history of the metaphysical concept of the soul. 


\subsubsection{Brief history of the concept of the soul }


And indeed, Reinhold's seventh article in the series of `Briefe' bears the title `A sketch of a history of reason's psychological concept of a simple thinking substance' (August 1787). It opens with a move similar to the ones we have already encountered regarding the historical accounts of the development of the grounds of cognition for the basic truths of religion: the features of the concept of the soul as a simple substance originate from the structure of our cognitive faculty and have been present in an elementary form from the first development of reason. 

Gleich mit der ersten Morgend\"{a}merung der Vernunft mu\ss{}te sich das denkende Ich, den Gesetzen des \textbf{Bewu\ss{}tseyns} gem\"{a}\ss{}, von jeder seiner \textbf{gedachten }Vorstellungen, und folglich auch schon darum vom K\"{o}rper, in so ferne dieser unter jenen Vorstellungen vorkam, unterscheiden. Eben so machten die Gesetze der \textbf{Sinnlichkeit} die wesentliche Unterscheidung zwischen den Gegenst\"{a}nden des inneren und des \"{a}usseren Sinnes, das hei\ss{}t zwischen den \textbf{Vorstellungen} in uns, und den \textbf{Dingen} ausser uns nothwendig. In so ferne nun alle Vorstellungen in uns dem \textbf{Ich} als ihren \textbf{Subjekte }anh\"{a}ngen, der \textbf{K\"{o}rper} aber in die Reihe der Dingen ausser uns geh\"{o}rt, mu\ss{}te der beym Bewu\ss{}tseyn \textbf{gedachte} Unterschied zwischen dem \textbf{Ich} und dem \textbf{K\"{o}rper}, einerseits an den \textbf{Vorstellungen }die durch den inneren Sinn, andererseits aber an dem \textbf{K\"{o}rper}, der durch den \"{a}usseren Sinn dargestellt wurde, auch sogar in der Anschauung gegeben seyn. (Siebenter Brief, 142{-}143)

[Right at the first dawning of reason, the thinking I, in conformity with the laws of consciousness, had to distinguish itself from every one of the representations it was thinking and consequently also from the body, particularly in so far as this body appeared among those representations. Similarly, the laws of sensibility made necessary the essential distinction between representations in us and things outside us. Now, in so far as all representations in us, attach to the I as their subject, while the body belongs to the order of things outside us, the distinction that is thought in consciousness between the I and the body {--} between representations presented through inner sense, on the one hand, and the body that is presented through outer sense, on the other {--} had to be given already in intuition as well.] (89)

Even though the `I' was distinguished from the body from the beginning of the development of human reason, Reinhold continues, the nature of this distinction was not thought clearly at all. It will come as no surprise that Reinhold credits Kant with finally discovering the ground of this distinction by investigating the structure of human reason, showing this ground to be the following rule of reason: ``it is not possible to think the subject of the predications of inner sense by means of the predicates of outer sense'' (90; Siebenter Brief, 144). It is upon this rule of reason that the concept of the soul as a simple substance is based, being slowly developed from the rule and only determined in full by Kant's \textit{Critique} (Siebenter Brief, 145). This means that the distinction between body and soul, made on the basis of the nature of our cognitive faculty, is prior to the concept of the soul as a simple thinking substance. The latter concept is empty and only legitimately functions as a rule of reason. As such it protects the expectation of a future life ``from any possible refutation on account of the death and dissolution of the body'' (93; Siebenter Brief, 148). Since this concept is empty, fantasy is likely to try and fill it, as Reinhold already indicated in the previous installment. In antiquity the soul was thought of as an invisible body, made of very fine matter, like air or ether (cf. 94{-}95; Siebenter Brief, 150{-}152). The immortality of the soul was not safeguarded by its simplicity, but by its mere invisibility, so that this fantasy had no detrimental influence whatsoever on the belief in a future life. 

 The element of the conception of the soul that was of crucial importance in this regard was the characterization of the soul as a power of thinking. To be more precise, the understanding of the connection and division of thinking and sensing proved to be crucial with respect to the belief in the continuing existence of the soul after the death of the body. Again, so Reinhold, it is only because of the \textit{Critique} that we now do know the proper relation between sensibility and understanding. 

Bis auf die Erscheinung der \textbf{Critik der Vernunft}, durch welche zuerst die \textbf{Sinnlichkeit }als \textbf{Receptivit\"{a}t unsres Erk\"{a}nntni\ss{}verm\"{o}gens} von der \textbf{Receptivit\"{a}t }der \textbf{sinnliche Werkzeuge} mit v\"{o}lliger Bestimmtheit unterschieden, die erstere f\"{u}r einen wesentlichen Theil unsres Erkenntni\ss{}verm\"{o}gens, der vor aller Empfindung, und vor aller Receptivit\"{a}t der Organe (\ldots ) im Gem\"{u}the vorhanden ist, erkl\"{a}rt, und ihre wesentliche Zusammenwirkung mit dem \textbf{Verstande} bey aller wirklichen \textbf{Erkenntni\ss{}} gezeigt worden ist {--} war das eigentliche \textbf{Verh\"{a}ltni\ss{}} der \textbf{Sinnlichkeit }zum \textbf{Verstande} ein tiefes Geheimnis geblieben. (Siebenter Brief, 154{-}155)

[Prior to the appearance of the Critique of Reason, by which sensibility, as the receptivity of our faculty of cognition, was distinguished with complete determination of the receptivity of our sense organs, the true relation of sensibility to the understanding remained a deep mystery. For the Critique of Reason explained for the first time sensibility as an essential part of our faculty of cognition that is present in the mind before all sensation and before all receptivity of the organs (\ldots ), and it showed the essential cooperation of sensibility with the understanding in all actual cognition.] (97)

Although considerations concerning the various philosophical opinions of the Greek schools on rationality and sensibility fill most of the remainder of the seventh and eighth `Briefe', Reinhold's account of these opinions need not concern us in detail here. Instead, it will be worthwhile to focus upon his presentation of the Kantian solution to the problem he has identified. According to Reinhold, ``the true relation of sensibility and understanding remained a deep mystery,'' prior to Kant's \textit{Critique} (97; Siebenter Brief, 155). The Greeks dealt with this mystery by either ``assuming two different souls {--} one thinking and one sensing {--} in order to explain the distinction between understanding and sensibility'' or by ``abolishing that distinction, on account of the very same hypothesis that had been forged for its explanation, in order to save the unity of the soul'' (98; Siebenter Brief, 156). The first strategy is attributed to both Plato and Aristotle, while the second is attributed to both the Stoics and Epicure. Whether they made too large a distinction between sensibility and understanding, or none at all, all the ancient schools failed to grasp ``how understanding and sensibility were supposed to be essentially different and yet essential parts of one and the same faculty of cognition'' (101; Siebenter Brief, 160). At the end of the seventh `Brief' it turns out that those who conflate sensibility and understanding do not ascribe immortality to the human soul, because they take the soul to be ``immediately connected to the flesh through sensation'' (103; Siebenter Brief, 164). Those who strictly separate the thinking soul from a sentient soul, however, ascribe immortality only to the former. With this, Reinhold presumes to have established that the ancient philosophical opinions on the mortality or immortality of the human soul were closely related to the respective concepts of the faculty of cognition, rather than connected to a concept of ``the substratum of the soul'' (103; Siebenter Brief, 165).

 The final, eighth installment of the `Briefe' continues the discussion on ancient philosophy and is aptly titled `Continuation of the preceding letter: The master key to the rational psychology of the Greeks' (September 1787). Using, once again, the didactic tool of objections from his correspondent, Reinhold opens the article by contrasting his view on the ancient philosophy of the soul with that of other authors of his day, like Platner and Meiners.\footnote{ \textit{Letters}, 106; Achter Brief 251. The actual references to Meiners, \textit{Geschichte des Ursprungs, Fortgangs, und Verfalls der Wissenschaften in Griechenland und Rom} (Lemgo: Meyer, 1781{-}1782) and Platner, \textit{Philosophische Aphorismen nebst einigen Anleitungen zur Philosophiegeschichte} (Leipzig: Schwickert, 1776) were inserted only in the first volume of \textit{Briefe} (1790). Cf. \textit{Briefe I}, 292. } The most important point to deal with, however, is the objection that the differences of opinion Reinhold had attributed to the ancient philosophers and hence their misunderstanding of the faculty of cognition amounted to nothing but a dispute about words. Reinhold responds to this objection in the following manner. 

Es giebt eine \textbf{Verschiedenheit der Meynungen }\"{u}ber die Natur des Denkens und Empfindens, die eine Folge der \textbf{verschiedenen richtigen Gesichtspuncte} ist, aus welchen verschiedene Denker das Erkenntni\ss{}verm\"{o}gen betrachtet haben. Es giebt aber auch eine \textbf{Verschiedenheit unter jenen Meynungen}, die eine Folge des noch \textbf{nicht entdeckten einzigen Gesichtspunctes} ist, aus welchem sich \textbf{alle \"{u}brigen Gesichtspunkte vereinigen lassen}. (Achter Brief, 255{-}256)

[There is a difference of opinion regarding the nature of thinking and sensing that is a consequence of the different, correct viewpoints from which different thinkers considered the faculty of cognition. But there is also a difference among those opinions that is a consequence of a single point of view not having been discovered yet, one by which all other viewpoints can be united.] (109)

By stating that the differences of opinion among philosophers are real yet admit of a unification from a higher viewpoint, Reinhold has paved the way for bringing in the Kantian philosophy again. In the seventh `Brief', he had already pointed out the importance of the Kantian conception of sensibility as an a priori part of the human faculty of cognition. Here, in the eighth `Brief', he introduces this element of Kant's philosophy again, this time under the name of `pure sensibility'.

\textbf{Kant }hat in dem erstgenannten Werke eine neue, oder wenigstens \textbf{bisher ganz verkannte Quelle der menschlichen Erkenntni\ss{} }entdeckt {--} die \textbf{reine Sinnlichkeit}. Sie ist weder Th\"{a}tigkeit der Organisation, noch Reitzbarkeit der Organe, sondern \textbf{das Verm\"{o}gen der Seele \"{u}berhaupt afficiert zu werden}, und besteht aus den in unsrem Erkenntni\ss{}verm\"{o}gen vorhandenen Bedingungen, welche jeder Anschauung (unmittelbare Vorstellung) eines Gegenstandes zum Grunde liegen. Sie ist die \textbf{subjektive} Beschaffenheit des Anschauungsverm\"{o}gens, und hei\ss{}t, weil alle Anschauungen durch sie bestimmt werden, die allgemeine \textbf{Form} derselben. Sie ist die \textbf{Receptivit\"{a}t der Seele}, welche \textbf{vor} allen Eindr\"{u}cken durch die Organe vorhergehen mu\ss{}, weil sie bey jedem derselben vorausgesetzt wird. (Achter Brief, 264) \footnote{ From the fact that Reinhold gives several descriptions here and from the circumstance that he would substantially rewrite the paragraph three years later, it is clear that he is struggling to find a way of making Kant's achievement intelligible to a wider audience. Cf. \textit{Briefe I}, `Eilfter Brief,' 308{-}316; \textit{Letters}, Appendix H, 201{-}205.}

[In this work Kant discovered a new, or at least heretofore wholly unrecognized, source of human cognition {--} pure sensibility It is neither the activity of the sense organs nor their excitability, but rather the faculty of the soul for being affected in general, and it consists in the conditions present in our faculty of cognition that lie at the basis of every intuition (immediate representation) of an object. It is the subjective constitution of the faculty of intuition, and it is called its universal form, because all intuitions are determined through it. It is the receptivity of the soul, which must precede all impressions from the sense organs because it must be presupposed by each of them.] (114{-}115) 

Reinhold's use of the term `pure sensibility' is novel and not taken directly from Kant.\footnote{ Kant does use the term once in the B{-}edition of the first \textit{Critique} (cf. \textit{KrV}, B 107/108). This use appears to be unrelated to the way in which Reinhold employs the term. It certainly does not play a similar, crucial role. Reinhold's use does appear to be related to Schmid's in the second edition of his \textit{W\"{o}rterbuch}. Cf. C. Ch. E. Schmid, \textit{W\"{o}rterbuch zum leichtern Gebrauch der Kantischen Schriften} (Jena: Cr\"{o}ker, 1788; 2nd edition), s.v. `Sinnlichkeit', 318. However, since that edition dates from 1788 and I have been unable to locate a copy of the first edition of 1786, it is impossible to establish whether Reinhold's use is influenced by Schmid's or vice versa, although the former appears to be more likely. } It does, however, figure as an element in his earliest reception of Kant, as it found an expression in the letter he wrote to Voigt, containing his plans regarding the Kantian philosophy. There, listed as point VII under `external grounds', we find the following. 

Bed\"{u}rfni\ss{} einer entscheidenden Antwort auf die Frage \"{u}ber den \textit{Ursprung der Begriffe}. Nachtheile sowohl der bisherigen Lehre der \textit{angebohrnen Begriffe} (auch selbst im Leibnitzischen Sinne) als auch vom \textit{empyrischen}\footnote{ Reinhold's spelling of the word `empirisch' is idiosyncratic. Apparently he used this spelling frequently, as Wieland felt compelled to correct him. In a postscript he writes: ``Il faut toujours ecrire \textit{empirisch} / empirique, empiricus / non pas empyrisch; weil dies wort nicht von \pi υ\ensuremath{\rho}, \textit{Feuer}, sondern von \pi \ensuremath{\varepsilon}ι\ensuremath{\rho}\ensuremath{\omega}, und zun\"{a}chst von \ensuremath{\varepsilon}μ\pi \ensuremath{\varepsilon}ι\ensuremath{\rho}ι\ensuremath{\alpha} herkommt.'' \textit{RK} 1:258, Letter 59, August 29, 1787, from Wieland. As, in the same letter, Wieland refers to reading Reinhold's eighth `Brief', it is not far{-}fetched to assume that he had had to correct Reinhold's idiosyncrasy there in a context similar to this one. The printed version of the `Briefe' consistently has `empirisch'. }\textit{ Ursprunge} der Begriffe (auch selbst im Lockischen Sinne). Kant berichtiget die Leibnizische und Lockische Lehre und vereiniget das Wahre von beyden; in dem er eine bisher verkannte \"{a}usserst wichtige Quelle menschlicher Erkenntni\ss{}: Reine Sinnlichkeit entdeckt, die bisher bekannten {--} Verstand, und Vernunft genau bestimmt, und ihr Verh\"{a}ltni\ss{} (als der Form) zur Empfindung (als der Materie aller Erkenntni\ss{}) deutlich angiebt.\footnote{ \textit{RK}, 1:156, letter 35, beginning of November 1786, to Voigt. } 

[Need for a decisive answer to the question concerning the \textit{origin of concepts}. Disadvantages of both the current doctrine of \textit{innate concepts} (even in the sense of Leibniz) and the doctrine of the \textit{empirical origin }of concepts (even in the sense of Locke). Kant corrects the doctrines of Leibniz and Locke and combines the truth of both, because he discovers a very important source of human knowledge that was hitherto unknown, namely pure sensibility, determines those that were already known, namely understanding and reason, and clearly states their relation (as the form) to sensation (as the matter of all knowledge).]

At first sight the context in which the term `pure sensibility' is introduced in the eighth `Brief' does not concern the origin of cognition. Yet there are some significant similarities. In both the letter to Voigt and the eighth `Brief' Reinhold refers to `pure sensibility as a `source' (\textit{Quelle}) of cognition. Moreover, following the passage cited, the paragraph in the `Brief' continues in the following manner. 

Eben darum aber weil sie blo\ss{}es Verm\"{o}gen, blo\ss{}e subjective Form, blo\ss{}e Receptivit\"{a}t ist, m\"{u}ssen ihr die Gegenst\"{a}nde \textbf{gegeben}, oder vielmehr mu\ss{} sie durch Gegenst\"{a}nde \textbf{afficirt werden}; und dieses Afficirtwerden der reinen Sinnlichkeit durch Gegenst\"{a}nde ist es was \textbf{Kant Empfindung}, empirische Anschauung nennt. [In]dem nun die \textbf{reine Sinnlichkeit}, die \textbf{Form}, die \textbf{Empfindung }aber die \textbf{Materie} der empirische Anschauung liefert: so giebt es keine sinnliche Erkenntni\ss{}, keine \textbf{unmittelbare Vorstellung} eines Gegenstandes ohne reine Sinnlichkeit und Empfindung. (Achter Brief, 264{-}265)

[But precisely because it is a mere faculty, a mere subjective form, a mere receptivity, objects must be given to it {--} or rather, it must be affected by objects. And the affecting of pure sensibility is what Kant calls sensation [\textit{Empfindung}], empirical intuition. Now because pure sensibility provides the form while sensation supplies the matter for empirical intuition, there can be no sensory cognition, no immediate representation of an object, without pure sensibility and sensation.] (115)

As in the letter to Voigt, Reinhold, in this eighth `Brief' stresses the role of `pure sensibility' as the form that is applied to the matter of sensation in order to have empirical intuition. From this it is clear that, although he uses the odd terminology `pure sensibility', Reinhold has the Kantian a priori forms of intuition in mind. Later on in the same paragraph he considers the matter{-}form relation more extensively. 

Der \textbf{Verstand} bezieht sich also in seinen \textbf{wesentlichsten} Wirkungen auf reine Sinnlichkeit und Empfindung, so wie sich Sinnlichkeit und Empfindung, in so fern Gegenst\"{a}nde nicht blo\ss{} durch sie \textbf{gegeben}, sondern auch \textbf{erkannt} werden sollen, auf den Verstand beziehen. \textbf{Reine Sinnlichkeit} liefert also die \textbf{Form} {--} \textbf{Empfindung }den \textbf{Inhalt} der \textbf{Anschauung}; \textbf{Anschauung} liefert den \textbf{Inhalt}, \textbf{Verstand} die \textbf{Form} des \textbf{Begriffes}; so da\ss{} es ohne \textbf{Zusammenwirkung} von reine Sinnlichkeit, Empfindung und Verstand keine Erkenntni\ss{} eines wirklichen Gegenstandes geben kann. (Achter Brief, 265)

[In its most essential operations, the understanding thus relates to pure sensibility and sensation, just as sensibility and sensation relate to the understanding, in so far as objects are not merely to be given through them but also cognized. Hence pure sensibility supplies the form of intuition, and sensation its content; intuition supplies the content of the concept, and the understanding its form {--} so that without the cooperation of pure sensibility, sensation, and the understanding, there can be no cognition of an actual object.] (115;) 

Since, in the end, Reinhold does add a note on the origins of knowledge, the only difference between the introduction of `pure sensibility' here and in the letter to Voigt is the focus. As the term is introduced in the eighth `Brief' in the context of ancient philosophy, Leibniz and Locke and their ideas on the origins of cognition would have been somewhat out of place.\footnote{ Reinhold may have had `On the amphiboly of concepts of reflection' in mind, from Kant's first \textit{Critique}, and especially the passage on A 271, where Kant accuses Leibniz of `intellectualizing' appearances and Locke of `sensitivizing' concepts, and stresses that understanding and sensibility only yield objective knowledge in conjunction. This appears to be exactly what Reinhold stresses at this point in the eighth `Brief', even if Leibniz and Locke are not mentioned here. } Here, it is much more important for Reinhold to stress that Kant's philosophy has produced a result that can help understand Greek philosophy. 

 Reinhold's way of putting the Kantian philosophy forward as the single viewpoint from which the differences between ancient philosophical opinions can finally be properly understood is an application of the strategy he had already used with regard to the question of the grounds for the conviction of God's existence. As mentioned earlier, that question led directly to a sharp conflict of grounds, expressed in the pantheism controversy. Reinhold's way of relating the question regarding the conviction of the immortality of the soul to a similar conflict is indirect. Having shown that the two traditional grounds for this conviction fail to deliver what they promise because they do not have the proper relation to morality (cf. 2.2.1 of the present chapter), he focuses on the (mis)understandings of the concept of the soul generated by fantasy, that have a bearing on the convictions regarding immortality. From that point, he can use a similar strategy as in the case of the grounds for the conviction of God's existence. In the earlier case, the nature of human reason was being misunderstood, with the Kantian conception of `practical reason' providing the new perspective from which the previous misunderstandings could be understood. In the case of the conception of the soul the human faculty of cognition as a whole is misunderstood and `pure sensibility', although not a term used by Kant, is introduced as the solution to these misunderstandings. 

 However continuous the methodology of the final two `Briefe' may be with that of the previous installments, there are also elements that point ahead, towards the methodology of the \textit{Versuch}, which will be discussed in more detail in the following chapter. At the beginning of the seventh article the representing `I' is distinguished sharply from both its representations and the objects of these representations, which distinctions were to become very prominent in the so{-}called \textit{Satz des Bewu\ss{}tseins}.\footnote{ Lazzari has already pointed out a connection between the opening of the seventh `Brief' and Reinhold's objections against Platner's arguments regarding immortality in the context of the historical genesis of Reinhold's \textit{Satz des Bewu\ss{}tseins}. Cf. Lazzari, `Zur Genese von Reinholds ``Satz des Bewu\ss{}tseins''.' } Reinhold's preoccupation in general with properly distinguishing and connecting the elements of the human faculty of cognition is striking in light of his later interests and is illustrated by the following passage. 

Bis auf diese Theorie [Kant's] {--} welche unser Erkennen blo\ss{} auf Gegenst\"{a}nde einschr\"{a}nkt die der Sinnlichkeit gegeben werden k\"{o}nnen, und folglich alle Erkenntni\ss{} \textbf{von Dinge an sich selbst} und \textbf{ausser der sinnlichen Vorstellung}, f\"{u}r unm\"{o}glich erkl\"{a}rt {--} mu\ss{}te der eigentliche Unterschied sowohl als der Zusammenhang zwischen Denken und Empfinden \textbf{nothwendig misverstanden }werden. (Achter Brief, 266)

[Prior to this theory [Kant's] {--} which restricts our cognizing to only those objects that can be given to sensibility and, consequently, declares as impossible all cognition that is of things in themselves and that goes beyond sensory representation {--} the genuine distinction as well as connection between thinking and sensing was necessarily misunderstood. (115;)

From this passage it is very clear that Reinhold believed that Kant was the first and only philosopher who had got the difficult relation between thinking and sensing right, that is, who knew how to distinguish them but also showed their intimate connection. Moreover, this passage is illustrative of the direction into which Reinhold was developing, as it contains a brief characterization of Kant's theory that shows Reinhold's tendency to develop what Karl Ameriks has termed a `short argument' for idealism, that is, an argument to the effect that we cannot know things in themselves, based solely on the thought that things in themselves cannot be experienced.\footnote{ Cf. Ameriks, \textit{Kant and the Fate of Autonomy}, 125{-}129. } Although it takes the form of an argument only in the \textit{Versuch}, the thought appears to be clearly prefigured in Reinhold's description of the Kantian philosophy. 

The final four \textit{Merkur}{-}`Briefe', then, show both a consolidation and an elaboration of the position introduced in the first four. Reinhold's claims regarding the achievements of Kant's philosophy are consolidated by the consistent application of his historical way of arguing in favor of the special position of this philosophy. The general line of the historical arguments is closely related to the way Reinhold had presented the history of religion in his `Skizze'. According to these arguments, human history has been through several phases, characterized by different stages of the development of human reason in general. The most important truths for humanity have been available from the beginning, in a pre{-}reflected form, as needs felt. Through the subsequent uneven development of human reason these truths have been elucidated, but only partly, in different one{-}sided manners. It is only with the investigation into the nature of human reason undertaken by Kant that humanity has gained insight into the way reason functions. The building has been finished and the one{-}sided earlier efforts can be understood from the newly found perspective. Reinhold also remains constant in his approach of the Kantian results from external, not from internal grounds. He does not refer to the \textit{Critique }with regard to specific claims, nor does he provide any arguments for the claim that Kant has solved the problems he allegedly solved. 

 Nevertheless, the final four `Briefe' do show a more thorough awareness of the Kantian philosophy. Reinhold's stress on the importance of `pure sensibility', for instance, means that the area of interest of the `Briefe' is widened from the field of philosophy of religion and morality to epistemology, which is assigned a crucial role in solving the problems in the field of the philosophy of morality and religion. Notwithstanding the first steps taken into the direction of the \textit{Versuch}, it is clear that the historical methodology would remain firmly in place within the context of the `Briefe', focusing on external arguments: the above{-}mentioned `Skizze' was to figure as a part of the final, twelfth `Brief' in the first volume of \-\textit{Briefe \"{u}ber die Kantische Philosophie}.\footnote{ Reinhold, \textit{Briefe I}, `Zw\"{o}lfter Brief,' 358{-}371. Cf. \textit{Briefe I}, Bondeli ed., 339, n. 401.} It fills the place of a promised ninth installment in the \textit{Merkur} concerning the influence of the concept of the soul on religion and morality. 


\section{Evaluation: Practical reason and pure sensibility}


The previous sections of the present chapter have introduced both the context and the content of Reinhold's `Briefe \"{u}ber die Kantische Philosophie' as first published in \textit{Der Teutsche Merkur} in 1786{-}1787. In order to have a clear starting point for our account of the developments in Reinhold's reception of the Kantian philosophy, it will be useful to pay specific attention to the role of the concept `practical reason' in these articles. Since the concept only entered his vocabulary with the `Briefe', his usage is obviously associated with Kant. The analysis of the `Briefe' provided above has shown that, true to his intentions, Reinhold does not provide his readers with Kantian intricacies regarding the exact nature and function of practical reason. His use of the concept nevertheless yields valuable insights into his understanding of Kant. This final section will evaluate the importance of the concept `practical reason' as employed by Reinhold in especially the first four `Briefe'. Understanding the way in which the term is used in that context will allow us to identify a more or less parallel concept, `pure sensibility' in the later installments of the `Briefe.


\subsection{Practical reason}


From the first two installments it was already clear that the concept `practical reason' is of crucial importance to Reinhold's argument. We have seen that the general line of argument, as introduced in the first two `Briefe', rests upon the claim that there is a crisis regarding the status of reason in respect of the foundation of the basic truths of religion. Given the pantheism controversy, this claim was likely to be granted readily by Reinhold's audience. However, as an advocate of the Kantian philosophy Reinhold had advanced the further claim that the Kantian philosophy could give an answer to the problems; an answer, moreover, that would supersede the misguided claims made by previous philosophy and theology. Practical reason is presented as the crucial ingredient of Kant's philosophy with respect to achieving this aim. 

Indem sie [the new answer] den von der praktischen Vernunft gebothenen Glauben festsetzt, st\"{u}rzet sie die Lehrgeb\"{a}ude der \textbf{apodiktischen Beweise} und des \textbf{blinden Glaubens} um, und stiftet durch die gl\"{u}ckliche Vereinigung der gel\"{a}uterten Hauptgr\"{u}nde ven beyden Lehrgeb\"{a}uden ein neues System, in welchem die Vernunft anmassend, und der Glaube blind zu seyn aufh\"{o}ren, und anstatt sich, wie bisher, zu widersprechen, in ewiger Eintrach sich wechselseitig unterst\"{u}tzen. (Zweyter Brief, 134{-}135)

[In so far as it is founded on a faith commanded by practical reason, this answer topples the doctrinal structures of both apodictic proofs and blind faith and establishes a new system through a most successful union of the clarified principal arguments of both doctrinal structures. In the new system, reason ceases to be presumptuous and faith ceases to be blind, and instead of opposing one another as before they mutually support one another in perpetual harmony.] (22)

As shown, Reinhold provides elaborate historical accounts to argue that it was due to the previous uneven development of human reason that its nature and capacities had been hitherto misunderstood. This had resulted in a crisis regarding the status of reason, which in fact was a crisis regarding the status of rationalist metaphysics, especially with regard to the existence of supersensible objects. However, rationalist metaphysics is the result of speculative reason only. Therefore, the shortcomings of this type of metaphysics are not to be regarded as objections against the capacities of reason as a whole, including practical reason. Since the proper concept of reason differentiates between speculative and practical reason, it softens the opposition between reason and faith, which had dominated the pantheism controversy. Reason as such is still opposed to \textit{blind} faith, but no longer to faith as such, as practical reason establishes a moral faith, based on rational grounds (Sechster Brief, 70). In the context of the pantheism controversy the only two alternatives, as presented by Jacobi, were speculative reason and blind faith, which were diametrically opposed. The new, properly investigated conception of reason is opposed to blind faith only, while including rational grounds for a moral faith. Reinhold claims that, by means of the introduction of `practical reason', Kant has found a way to rise above the previous opposition, by identifying a middle ground between reason and faith, rational faith, founded upon practical reason. It alone provides a rational ground for the conviction that God exists and that the soul is immortal. 

Die \textbf{Kritik der Vernunft }hat (\ldots ) dargethan: \quotedblbase Da\ss{} es der \textbf{spekulativen }Vernunft eben so unm\"{o}glich sey, die \textbf{Unsterblichkeit der Seele}, als das \textbf{Daseyn der Gottheit} zu \textbf{demonstriren}; da\ss{} hingegen die \textbf{praktische} Vernunft durch eben dasselbe \textbf{Postulat}, wodurch sie ein h\"{o}chstes Princip der sittlichen und nat\"{u}rlichen Gesetze voraussetzt, auch die Erwartung einer k\"{u}nftigen Welt nothwendig mache, in welcher Sittlichkeit und Gl\"{u}ckseligkeit nach der Bestimmung jenes h\"{o}chsten Princips in vollkommenster Harmonie stehen m\"{u}ssen.`` (Vierter Brief, 117{-}118) \footnote{ For similar passages, cf. 20; Zweyter Brief, 131 and 38; Dritter Brief, 20.}

[(\ldots ) the Critique of Reason has shown `that it is just as impossible for speculative reason to demonstrate the immortality of the soul as it is for it to demonstrate the existence of the deity', and `that practical reason, on the contrary, through the same postulates by which it presupposes a highest principle of moral and natural laws, also makes necessary the expectation of a future world in which morality and happiness must stand in most perfect harmony according to the determination of the highest principle'.] (50)

There are two claims at stake here. First, speculative reason cannot \textit{demonstrate} the existence of God. Secondly, the conviction of the existence of God is rationally justified by practical reason, which \textit{postulates} this existence. As the analysis in the previous sections has shown, Reinhold does not provide a clear or complete account of the reasons why practical reason can postulate this existence. He appears to construe a picture in which practical reason, in contrast to speculative reason, does not exclude a connection to sensibility. This may then, in a way, make up for the lack of intuition due to which speculative reason cannot assert the existence of supersensible objects. 

We may also investigate Reinhold's views on practical reason in a more general way, abstracting from its role in providing a rational ground for the fundamental tenets of religion. With regard to the question what practical or moral reason is and does in general, it is clear that Reinhold understands it as a mediating capacity in several different but related contexts. First, related to its function in solving the disputes on faith and reason, practical reason mediates between apodictic proofs and blind faith by overcoming both and using the truth of both while dispensing with their falsity. Thus, practical reason is not only presented as capable of providing a rational ground for religion, which speculative reason cannot do. It is also presented as commanding a faith that mediates between apodictic proofs and blind faith by superseding both (Zweyter Brief, 134{-}135, cited at the beginning of this section). Secondly, this characteristic of practical reason leads to mediation in a different field. Since, in Reinhold's schematic accounts, the philosophers are commonly associated with the search for apodictic proofs, whereas the common man is portrayed as subjected to blind faith and superstition, the new solution is accessible to both and thus also mediates on a social level, as the truth that arises from the unification is accessible to everyone, the common man and the philosopher alike.

(\ldots ) und wenn sich der Weise gen\"{o}thiget sieht, ein \textbf{h\"{o}chstes Wesen} als Princip der sittlichen und physischen Naturgesetze vorauszusetzen, welches m\"{a}chtig und weise genug ist, die Gl\"{u}ckseligkeit der vern\"{u}nftigen Wesen, als den nothwendigen Erfolg der sittlichen Gesetze, zu bestimmen und w\"{u}rklich zu machen: so f\"{u}hlt sich au der gemeinste Mann gedrungen, einen k\"{u}nftigen \textbf{Belohner }und \textbf{Bestrafer }jener Handlungen anzunehmen, die sein Gewissen (\ldots ) billiget und verwirft. In der \textbf{Kantischen }Antwort ist es also ein und ebenderselbe Vernunftgrund, welcher dem aufgekl\"{a}rtersten sowohl als dem gemeinsten Verstande Glauben gebiethet; und zwar einen Glauben der die strengste Pr\"{u}fung des einen aush\"{a}lt, und den gew\"{o}hnlichsten F\"{a}higkeiten des andern einleuchtet. (Zweyter Brief, 136{-}137)

[And just as the sage feels it necessary to presuppose a highest being as the principle of the moral and physical laws of nature, a principle that is wise and powerful enough to determine and bring about the happiness of rational beings as a necessary consequence of the moral laws, so too the most common man feels compelled to accept a future rewarder and punisher of the actions that his conscience approves and condemns (\ldots ). In the Kantian answer it is thus one and the same ground of reason that offers faith to the most enlightened as well as to the most elementary understanding {--} that is, a faith that stands up to the most rigorous examination by the former and is illuminating for the most ordinary capacities of the latter.] (23) 

Thirdly, practical reason does not only mediate between the different grounds proposed for the belief in God's existence, but also between speculative reason and sensibility in general. These capacities of the human mind can be regarded as the bases of the different perspectives discussed above. Apodictic proofs have their origin in speculative reason, whereas blind faith is supported by revelation and miracles, that is, by information gathered by sensibility. With regard to the issues of religion both these faculties, when operating on their own lead to excesses. 

Isolierte Sinnlichkeit, vernunftloses Gef\"{u}hl, blindes Glauben reissen unaufhaltsam zum Fanatismus dahin; Isolierte Vernunft, kalte Spekulation, ungeregelte Wi\ss{}begierde f\"{u}hren, wenns hoch k\"{o}mmt, zum frostigen, gr\"{u}belnden, unth\"{a}tigen Deismus. \textbf{Vernunft und Gef\"{u}hl} hingegen in \textbf{ihrer Vereinigung}, {--}die Elemente der Sittlichkeit, {--}bringen den \textbf{moralischen Glauben }hervor, und machen (\ldots ), den einzigen, reinen und lebendigen Sinn aus, den wir f\"{u}r die Gottheit haben. (Dritter Brief, 33)

[Isolated sensibility, feeling without reason, and blind faith pull inexorably toward fanaticism; isolated reason, speculation, and the unrestricted desire to know lead at best to icy, carping, inactive deism. Yet when they are unified, reason and feeling {--} the elements of morality {--} give rise to moral faith and constitute (\ldots ) the only pure and living meaning we have for the deity.] (46)

Left to themselves, the two poles of human cognition, sensibility and reason, lead to opposite and extreme results. While sensibility on its own can lead to nothing more than blind faith, reason on its own leads to speculation, but not to action. Both are needed to produce moral faith. Morality is thus presented as the combination of two elements of our faculty of cognition which appear to be opposites but need to be combined. 

 It is interesting to note that ``isolated reason'' appears to be speculative reason here, which would render the distinction between speculative and practical reason not one within reason or between different forms of reason, but rather a distinction between reason on its own and reason connected to the senses. Thus instead of an opposition of reason and intuition with a distinction within reason (as appears to be the Kantian picture), we find that for Reinhold isolated reason and isolated sensibility are opposed, but practical reason mediates between them and combines the two roots of human knowledge. 

 Reinhold's deviation from Kant in this respect and his insistence on mediation between reason and sensibility reflects his pre{-}Kantian concerns regarding Enlightenment, discussed in detail in the second chapter, especially in section 3. Some of his key writings of that period will be briefly recapitulated below as an illustration of the consistency of his approach to reason. Already in Vienna, Reinhold expressed the thought that Enlightenment in practice needed both reason and sensibility. We have seen this especially clearly in the speech on the `Wehrt einer Gesellschaft', in which the importance of a good heart, or virtue, was stressed. Without it, the noblest (rational) ideas would not be put into action, as they could not be asserted with sufficient force in the actual life of the agent, where sense impressions distract the attention. In order to compete with these, the good ideas must, by means of a good heart, or virtue, become similar to these sense impressions, for the way to action goes, as Reinhold puts it, ``through sensation [\textit{Empfindung}].''\footnote{ Reinhold, `Der Wehrt einer Gesellschaft,' 68r.} Enlightenment in practice thus presupposes a good heart as well as a good mind, sensibility as well as reason. 

Moreover, from other writings, such as `M\"{o}nchthum und Maurerey' (1784) and `Gedanken \"{u}ber Aufkl\"{a}rung' (1784) it is clear that Reinhold believed that true Enlightenment is Enlightenment in practice. With regard to `M\"{o}nchthum and Maurerey' we have seen that Reinhold thinks of human reason as closely related to human nature. Man's natural and rational behavior in turn is understood in terms of proper action in society, as a husband, father, and citizen.\footnote{ Reinhold, `M\"{o}nchthum und Maurerey,' 183. } The article makes very clear that according to Reinhold the type of reason that is to be furthered by Enlightenment is a practical type of reason, focusing on being rational in our relation to other people and to society in general. Although Reinhold himself does not use the term `practical reason' at all in this context (and neither does he contrast it with speculative reason), we may say that Reinhold's conception of reason, as it surfaces from this article, is thoroughly practical (in a common{-}sense understanding of the term, that is, not in the Kantian sense). It is hard to tell whether Reinhold had a distinct conception of reason at this time and whether he would distinguish between a speculative and a practical use. We cannot deny, however, that he primarily associates reason with activities in the practical sphere, with being a concrete human being with concrete relations to other human beings rather than with speculation.

 In the tripartite article `Gedanken \"{u}ber Aufkl\"{a}rung' Reinhold looks at Enlightenment from a scientific rather than a personal perspective. He provides a brief history of the sciences culminating in the elaborate systems of rational metaphysics. However, although the system of concepts thus built has been finished, it is not a proper building at all, but only the scaffolding, so Reinhold argues.\footnote{ Reinhold, `Gedanken \"{u}ber Aufkl\"{a}rung,' 4{-}5. Reinhold uses a similar metaphor in the `Briefe.' Cf. 43; Dritter Brief, 29. } As the philosophers are not aware of this, they do not proceed to the construction of the building itself. The unceasing labor on the scaffolding has resulted in ever more refined analyses of concepts, which thereby become more and more abstract. This means that the common notions become disconnected from and useless for human life. The real importance of Enlightenment is situated in the efforts to overcome this gap between scientific concepts and real life. This is to be done by means of concepts that can build a ``bridge between speculation and action.''\footnote{ Reinhold, `Gedanken \"{u}ber Aufkl\"{a}rung,' 6. } 

 As shown in our second chapter, Reinhold's explication of `bridging concepts' uses the concept `father' as an example, which resonates in the third `Brief'. By means of these intermediate concepts, which are common to humanity as such, the distinct idea of the philosopher and the confused idea of the common man can be connected. Again, human relations are the basis for the connection of rationality to real life. It is this process, then, that Reinhold calls Enlightenment, the process of dissemination of the concepts developed by reason throughout society. 

 Although Reinhold's pre{-}Kantian writings recapitulated here discuss the relation between reason and feeling, mind and heart in different manners, their common denominator is that Enlightenment combines rationality and action. This requires that the rational and sensible capacities of man are connected properly. With respect to society, Enlightenment must aim to bring about a unification between those who predominantly use their minds and those who predominantly rely on their senses in life. The proper use of reason is connected to real life, which is related to the complex of human relations that people have when they are a part of society. Reason, as an essentially human property, belonging to human nature, expresses itself properly in connection with the daily reality of human life, that is, in connection with social relations. The concepts that describe these relations are therefore properly called common notions and are the basis of the solution to the problem of the excessive speculative use of reason. The situation in which the proper use of reason is as yet to be found is the result of the uneven development of man's cognitive capacities throughout history.

 Even if the actual term `practical reason' only entered Reinhold's vocabulary with the `Briefe' and is thus closely associated with his getting acquainted with Kant's philosophy, it will be clear from the above brief recapitulation of his earlier writings that the seeds for the way in which the `Briefe' present practical reason had been sown already in his pre{-}Kantian years. In the `Briefe' the concept of `practical reason' serves to present his argument in support of the Kantian philosophy. As speculative reason is not able at all to answer the questions it has raised, practical reason is introduced as an authority that can provide answers, and thus overstep the boundaries set on speculative reason, without sacrificing the rational character of the answers. This is the essential feature Reinhold needs to be able to present Kant as the philosophical hero of the age. Reinhold's hints at the solution that practical reason offers point towards reconciliation of speculative reason and sensibility and suggest that his view differs from Kant's. Practical reason appears either to mediate between pure speculative reason and sensibility, or to combine them. This is not in line with the Kantian conception viewing the distinction between speculative and practical reason as one within reason. In Reinhold's picture speculative reason is devaluated to a one{-}sided form of reason, deaf to the senses. Nevertheless, it is understandable that Reinhold, when focusing on the first \textit{Critique} in 1786 and 1787, could have thought that this was indeed Kant's intention. In the first \textit{Critique} there are remarks that imply that the sphere of the practical presupposes a connection to the sensible world of desires and inclinations.\footnote{ Cf. \textit{KrV}, A 15; A 569. } The interpretation Reinhold gives of Kant's philosophy, according to which practical reason mediates between speculative reason and sensibility, is, however, more strongly connected to Reinhold's own, pre{-}Kantian conceptions of reason than to the scattered and imprecise remarks of the master himself. 

In his early works the term `practical reason' itself is never employed, but Reinhold stresses the importance of reason being practical time and again, in the sense of being connected to human action in the world. At first, he does not need to distinguish between speculative and practical reason, since reason is so closely connected with human nature that its use in social practice is presented as essential in `M\"{o}nchthum and Maurerey'. Later, in `Gedanken \"{u}ber Aufkl\"{a}rung', pure speculative reason is presented simply as reason gone adrift, disconnected from the reality of human life. In both cases, the proper, or complete, use of reason is practical, which in Reinhold's vocabulary means `connected to the real world and sensibility'. This understanding of `practical' also underlies his reception of Kant's first \textit{Critique}, as the `Briefe' clearly testify. Reason and sensibility are frequently presented as opposites, which opposition is solved by practical reason, combining rationality with being informed by the real, concrete world. 


\subsection{Pure sensibility}


In a way the emphasis laid so far in this section on `practical reason' as a central concept in Reinhold's `Briefe' is somewhat artificial, or may appear unbalanced. For actually, this concept is mainly prominent in the third and fourth `Briefe', with the first two installments building up to its introduction. In the later `Briefe' the term hardly occurs, as Reinhold discusses the solution of the problems regarding the grounds for the conviction of the immortality of the soul in even less detail than he had discussed the conviction that God exists. While it is true that the analysis given of Reinhold's use of the Kantian term `practical reason' is not directly applicable to the final four `Briefe', it still offers very valuable insights into Reinhold's reception of Kant especially in relation to his pre{-}Kantian thoughts on reason. Moreover, the analysis is of indirect importance, for we have seen that the argumentative structure of the final four `Briefe' runs more or less parallel to that of the third and fourth. In them, Reinhold introduces the seemingly Kantian term `pure sensibility', which then performs a function similar to that of `practical reason' in the earlier installments. As shown above that Reinhold introduces the term independently of Kant's usage to refer to the `Kantian discovery' of the a priori forms of intuition. Since apparently there is something a priori to sensibility, this discovery is very important in overcoming the opposition between Leibnizian and Lockean theories of cognition. Leibniz had identified man's rational capacities as the sole source of cognition, whereas Locke had maintained that all cognition was rooted in sense experience. The discovery of a `source of cognition' as Reinhold called it, that a priori shapes sense experience and is thus related to both our rational and our sensible capacities means that both Leibniz and Locke were partly right and partly wrong and that their differences can be reconciled. In the `Briefe' the term is introduced in order to give a novel interpretation of the opinions of the ancient philosophers on the immortality of the soul. The point of introducing it is to stress the importance of a proper understanding of the relation between our sensible and our rational capacities, that is, recognizing that these capacities must be distinguished, yet are at the same time intimately related.

In this case, as in the case of `practical reason' we are dealing with a feature of the structure of the human faculty of cognition, ever present and active, but only acknowledged recently, since the thorough investigation undertaken by Kant in his first \textit{Critique}. Moreover, Kant's discovery, as Reinhold puts it, of both `practical reason' and `pure sensibility' provides a new standpoint in philosophy, one that allows us, so Reinhold, to put the previous controversies and misunderstandings in perspective. Thus, in both cases, he introduces a feature of our faculty of cognition, allegedly discovered by Kant, as an argument for the value of the Kantian philosophy, claiming that these discoveries yield crucial insights not only into the structure of the human mind, but into the structure of the history of philosophy as well. After all, for Reinhold the history of philosophy is closely related to the development of man's mental capacities. 

 It is further important to note that the two newly introduced concepts, `practical reason' and `pure sensibility', are not only similar in their role of supporting the Kantian philosophy. They appear to have a similar content as well. That is to say, a content that appears to function in a similar way, since, as always, Reinhold is not very interested in the technical details in these `Briefe'. Both the discoveries Kant is credited with, namely those of `practical reason' and of `pure sensibility', are presented as extremely important by Reinhold because they refer to the element or elements of the human faculty of cognition that can mediate between abstract rationality on the one hand and concrete sensation on the other. We have seen that `practical reason' is presented as a mediation between isolated (that is, speculative) reason on the one hand and isolated feeling or sensibility on the other. Similarly, `pure sensibility' mediates between sensation, the input to our sense organs on the one hand, and the understanding, the faculty of concepts on the other. Judging by diversity of terms, it is clear that Reinhold still did not have, at that point, a fully developed system of the human faculties in mind, but rather still worked within the familiar framework, in which `heart', `feeling' and `sensibility' are associated with the concrete, real world, whereas `reason', `understanding' and `thought' are associated with the abstract world of concepts. We have already encountered this schematic dichotomy in Reinhold's works on Enlightenment.\footnote{ Cf. Chapter 2, section 3. } The fact that it continues to play such an important role in his first reception of Kant indicates that his becoming acquainted with the Kantian philosophy did not significantly alter his views on the nature of man's mental capacities. Given the circumstance that Kant's theory of cognition was to change the philosophical landscape forever and, as Reinhold admits, contained the seeds of a philosophical revolution, the fact that Reinhold's thoughts on the nature of the human mind underwent little change is remarkable indeed. He apparently conceived of the Kantian philosophy as very suitable for the ideas he already had about the proper relation between our rational and sensible capacities. 

 A further point of similarity between the introduction of `practical reason' in the earlier `Briefe' and `pure sensibility' in the later ones is that these discoveries of Kant do not only both have a mediating function between `reason' and `sensibility', they are also both strongly connected to morality. That is, both are presented in a way showing that they are necessary conditions for the successful unification of religion and morality. The case is most clear, of course, with the introduction of practical or moral reason. This type of reason is presented as directly providing the moral foundation for the conviction that God exists, and also for the conviction that the human soul is immortal. `Pure sensibility' contributes albeit indirectly, to the unification of religion and morality as well. Its discovery showed that the previous ways of understanding the relation between thinking and sensing had been misguided. These mistaken understandings, which either conflated the two or introduced too sharp a distinction between them, did have a detrimental influence upon either the conviction of the immortality of the soul or at least upon the unity of religion and morality in this regard. We have seen that Reinhold went to great lengths to argue that the opinions of ancient philosophers on the (im)mortality of the soul were not related to their thoughts on its (im)materiality, but rather to their conceptions of the nature of our cognitive capacities. In this manner, the discovery of `pure sensibility' is very significant with regard to religion and morality, just as the discovery of `practical reason'.

 Finally, it must be pointed out that, for Reinhold, both these discoveries are actually discoveries and not philosophical inventions of some kind. That is to say, Reinhold presents Kant as having uncovered a number of essential features of the structure of the human mind. The undiscovered existence of these features had caused feelings which in turn had inspired the mistaken views on the human mental capacities that were being superseded by the Kantian philosophy. On the one hand, this way of presenting Kant's work allows Reinhold to elaborate on the historical development of man's mental capacities, claiming that the grounds discovered by Kant were always there and have always worked towards the same goal, but that they can only be fully understood after various necessary misconceptualizations. On the other hand, presenting the terms introduced as discoveries of the structure of the human mind means that Kant has successfully identified grounds that are common to all of humankind. Since the Kantian discoveries concern fundamental traits of the human mind, the problems created by the previous misguided standpoints can be solved in a manner that is accessible to philosophers and common men alike.\footnote{ For more on the way in which Reinhold combines systematic and historical approaches to philosophy with the effort to make philosophy popular, that is, accessible to philosopher and layman alike, cf. Ameriks, \textit{Kant and the Historical Turn}, especially Chapter 8. } The connection to Reinhold's Enlightenment ambitions can hardly be overlooked. Those things that he hoped Enlightenment would achieve are now presented as achievements of the Kantian philosophy. 

 The single most important feature of the Kantian philosophy, then, is in Reinhold's eyes that it alone is able to overcome the distinctions between rationality and sensibility that resulted from the uneven development of human reason throughout human history. Reinhold's enthusiasm for the new philosophy follows from his pre{-}Kantian conviction that philosophy actually needed to overcome these distinctions as is testified by his earlier works. In the `Briefe' he pulls this feature of the Kantian philosophy into focus by introducing two Kantianizing terms, `practical reason' and `pure sensibility'. These terms denote the newly discovered facts concerning the structure of our faculty of cognition that enabled Kant to indeed overcome the deadlock of previous philosophy. In light of the early reception of Kant's work throughout Germany, it is important to note that Reinhold's use of these terms is more related to the perspective from which he wants to present the Kantian philosophy than to anything Kant had claimed with regard to them up to that point. Reinhold's use of the term `practical reason' is not so much inspired by Kant's moral philosophy, but rather by his own desire to find a way in which the abstract world of reason and philosophy can be connected to the real, concrete world of feeling. His use of the term `pure sensibility' to denote the a priori forms of sensibility is unrelated to Kant's preoccupation with synthetic a priori judgments and the status of mathematics, but rather stresses overcoming the distinction between our rational and sensible capacities and their role in bringing about cognition, a distinction which has been made too sharply in the history of philosophy. Both terms are used by Reinhold in a context in which the foundation of religion upon morality is the main issue. His claim that this foundation has been achieved by Kant in a way that is in principle understandable to everyone and that is the inevitable final stage of the development of reason, strongly links the `Briefe' to his pre{-}Kantian interests. His employment of Kantianizing terms serves his overall Enlightenment purpose, rather than explaining Kant. 

