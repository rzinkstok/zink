
\chapter{Reinhold's life and works}


The present study aims to shed light on Reinhold's appropriation of the Kantian philosophy from his personal and philosophical background. Therefore, this chapter is devoted to introducing the man and his work in such a way as to contribute to the understanding of the relevant writings. It does not aim to be a full biographical account,\footnote{ For a more detailed account of Reinhold's life up to his move to Kiel, cf. Onnasch, introduction to \textit{Versuch}. } since it only considers Reinhold's life up to his departure from Jena to Kiel. This does not imply that Reinhold's life or philosophical development after that move does not deserve our attention; it simply falls outside the scope of this investigation. The biographical details included and the works briefly introduced in this chapter are important within this framework because they are relevant to the argument that is presented in this study. The information serves as a backdrop against which the story of Reinhold's philosophical development will unfold in the following chapters. 


\section{Vienna (1757{-}1783)}


There has been a long tradition of mistaken accounts of Reinhold's early life, starting with the biography\footnote{ Ernst Reinhold, \textit{Karl Leonhard Reinhold's Leben und litterarisches Wirken nebst einer von Briefen Kant's, Fichte's, Jacobi's und andrer philosophirender Zeitgenossen an ihn} (Jena: Friedrich Frommann 1825). } written by his son, Ernst Reinhold, and, in a sense, with Reinhold himself. Karl Leonhard himself apparently believed he was born on October 26, 1758 in Vienna, but nowadays, as a result of the work done by the editors of Reinhold's correspondence, we know that the year was in fact 1757.\footnote{ The correct year of Reinhold's birth was first published in Lauth, `Nouvelles R\'{e}cherches,' 593.} Reinhold grew up in Vienna as the eldest son in a middleclass family, his father being a military official who had been given a desk job after having sustained an injury.\footnote{ \textit{RK} 1:1, n. 2. Cf. \textit{RLW}, 3. } According to Ernst Reinhold, young Karl's character was formed by his pious mother, who instilled a sense of religiosity into him, of the kind that lives ``not in a fanciful mind, but in a loving heart.''\footnote{ \textit{RLW}, 3{-}4. } Ernst Reinhold here apparently refers to his father's writings on religion and Enlightenment, in which a warm and loving religion is valued, and for which the religious education Karl received from his mother was to be a recurring source of inspiration. It was an important factor in his decision to join the Jesuit order as a novice at the age of fifteen. Ernst Reinhold states that his father's Jesuit grammar{-}school teachers would have had no trouble winning over the smart pupil with his susceptible mind.\footnote{ \textit{RLW}, 4.} 

Reinhold's Jesuit career was broken off within a year, with the dissolution of the Jesuit order in 1773, which occasioned him to write to his father, thereby leaving us the only primary source regarding his life with the Jesuits. This letter shows two important things. First of all, the young novice is apparently being educated in an atmosphere of penance. Severely shocked, Reinhold views the abolition as a punishment for the immorality of the world in general and the lack of enthusiasm (\textit{Lauigkeit}, lit. `tepidity') of the novices.\footnote{ \textit{RK} 1:1, Letter 1, September 13, 1773, to Karl \"{A}gidius Reinhold. } Placing the blame on the novices may seem exaggerated, but it makes sense to the extent that special prayers and penances had been ordered to avert the dissolution by the spring of 1773, but to no avail.\footnote{ \textit{RK} 1:2{-}3.} Secondly, the letter shows that although he is willing to do what it takes in order to be a proper monk, Reinhold is clearly struggling to fulfill the requirements imposed on him by his own piety. He feels guilty about thinking of his relatives too much and interprets his attachment to them as tricks of the devil trying to bind him to the world. In passing he mentions having had similar problems in the beginning of his monastic life.\footnote{ \textit{RK} 1:5{-}6. } Since he is told to wait half a year before joining another order, Reinhold asks permission to return to the parental home, where he requires a separate room so that he will not have to meet any women, not even his sisters.\footnote{ \textit{RK} 1:7.} The monastic life has clearly led to tensions for Reinhold. Especially the requirement to distance himself not only physically from his loved ones but also in thought causes anxiety. The reunion with them under the same roof seems to give him more pain than joy, which indicates that he took his commitment to monastic life very seriously at this time. 

 It is no great surprise, therefore, that he was not to stay with his natural family for a long time. Contrary to the Jesuit instruction, he joined the Barnabite order within two months. Like the Jesuits, the Barnabites, or Clerics Regular of St. Paul, were a reformation order, established as a reaction to the Reformation and focused on well{-}educated clerics and apostolic work. Reinhold's choice for an order concentrating on learning and a practical form of Christianity, instead of a contemplative, closed order tells us that he was eager to learn and intent on putting the knowledge acquired to good use. After his probation, Reinhold studied philosophy and theology at the Barnabite college of St. Michael in Vienna.\footnote{ For a detailed account, see `Reinhold's Werdegang im Orden der Barnabiten', \textit{RK}, 1:386{-}393. } 

Probably the most important figure in his life at this time was his philosophy teacher and friend Paul Pepermann (1745{-}1784), with whom he kept in touch after he left Vienna.\footnote{ \textit{RK}, Letters 33, 36, 73, 111.} Pepermann not only taught the standard curriculum, but also encouraged Reinhold to go beyond the textbooks. In his letter to Reinhold dated November 5, 1786, Pepermann expresses some dissatisfaction with regard to the material he had to teach. He disapprovingly mentions the textbook writers Storchenau and Bertieri\footnote{ Sigmund von Storchenau (1731{-}1797) was a Jesuit professor of logic and metaphysics. Apart from textbooks he wrote books against the modern world view. Guiseppe Bertieri (1734{-}1804) was an Augustinian theology professor at the Vienna University.} and regrets that his own knowledge of metaphysics does not go much beyond Malebranche and Wolff.\footnote{ \textit{RK} 1:159{-}160, Letter 36, November 5, 1786, from Don Paulus Pepermann. } Wolff can be considered standard in this context, since Wolffianism was indeed en vogue in Vienna after the dissolution of the Jesuit Order.\footnote{ For details on the development of philosophy in Vienna in this period, see Sauer, \textit{\"{O}sterreichische Philosophie}, second chapter. } The choice of Malebranche seems less obvious, as both his \textit{De la recherche de la verit\'{e}} (1674{-}1675) and his \textit{Trait\'{e} de la nature et de la grace} (1680) were on the Catholic Index of prohibited books and he was accused of Spinozism and atheism. Apart from this, Pepermann, who had spent his youth in Britain, also introduced Reinhold to the English language, and ``the classical philosophers, poets and historians of that nation.''\footnote{ \textit{RLW}, 16{-}1. However, cf. \textit{RK}, 1:138, n. 2, which reports that Pepermann came from Vienna. } 

Although in his later writings he would not be very positive about monasticism in general, Reinhold used the term ``unm\"{o}nchisch'' to describe the Barnabite order. He says that not only was there no obstruction on his way to intellectual education; he ``even found encouragement and reward.''\footnote{ Reinhold, \textit{Ehrenrettung der Lutherischen Reformation}, `Vorbericht' (Jena 1789), cited from: \textit{RK}, 1:11, n. 3. } He would eventually become a teacher himself, first in Mistelbach and from 1782 at St. Michael's in Vienna,\footnote{ In Mistelbach Reinhold taught Church history, Rhetoric (\textit{eloquentia sacra}), Mathematics and Philosophy; in Vienna he lectured on philosophy and was \textit{Novizenmeister}. \textit{RK}, 1:390{-}392.} which activity he would later describe as follows. 

Drey Iahre hindurch hatte er philosophische Vorlesungen nach dem leibnitzischen Systeme gehalten, und die Schriften des grossen Stifters desselben, so wie seines w\"{u}rdigen Gegners \textit{Locke}, waren ihm keineswegs nur aus den neuern philosophischen Produkten unsrer Landesleute bekannt. (\textit{Versuch}, 51{-}52)

[For three years, he had lectured according to the Leibnizian system, and the works of its great founder, as well as those of his worthy opponent Locke, were by no means only known to him from the recent philosophical products of our compatriots.]

Reinhold's knowledge of Leibniz and Locke in the original is a further indication that his education and possibly his teaching as well went beyond the traditional textbooks.\footnote{ At this time both Leibniz's \textit{Theodic\'{e}e} (1710) and his \textit{Nouveaux essais sur l'entendement humain} (published posthumously in 1765) would have been available. Locke may have been available to Reinhold in English, through Pepermann.} However, he was to enjoy the life of a Vienna philosophy teacher only for about a year, as on November 10, 1783, he was sent to the Barnabite college of St. Margaret in Moos, to replace the priest there. This employment was truly short{-}lived. Without the consent of his superiors he left for Leipzig on the 19th with the help Christian Friedrich Petzold (1743{-}1788), professor of philosophy at the University of Leipzig. There are at least two (second{-}hand) sources stating a love affair as one of the main reasons for Reinhold's departure.\footnote{ Cf. \textit{RK} 1:5, n. 4. Friedrich Schiller states in a letter to Christian K\"{o}rner that a girl Reinhold was planning to marry robbed him of his ecclesiastical status. Karl B\"{o}ttiger speaks of a barber's daughter, who decided not to elope with Reinhold for fear of her parents. Gerhard Fuchs further mentions a French letter to a certain Therese. Contrary to his suggestion, this letter appears to have been addressed to one of Reinhold's sisters rather than to the mysterious lover. Unfortunately it is missing from \textit{RK}. Notwithstanding bringing up this letter, Fuchs does not view the alleged affair as the primary motivation for Reinhold's departure from Vienna. Cf. Fuchs, \textit{Karl Leonhard Reinhold {--} Illuminat und Philosoph}, 33. Ernst{-}Otto Onnasch, on the other hand, does present the affair as a relevant factor contributing to Reinhold's decision to leave Vienna. Cf. Onnasch, introduction to \textit{Versuch}, [XXVIII].} If we are to take Ernst Reinhold's word for it, his father had been planning to leave Austria for some time, and took action when Petzold visited Vienna on family business in the summer of 1783.\footnote{ Cf. \textit{RLW}, 20{-}21.} 

 In order to explain his stealthy departure, we need to focus on the extracurricular activities Reinhold had been undertaking from 1782, the time of his return to Vienna as \textit{Novizenmeister}. Born and bred in Vienna, Reinhold had many friends and acquaintances in the capital. Some of his former fellow Jesuit novices had not entered another order but were pursuing secular careers, serving the reformist government.\footnote{ This circle of politically engaged young men who were active in literature included Aloys Blumauer, Joseph Franz Ratschky, Martin Prandstetter, Johann von Alxinger, Gottlieb Leon, Johann Pezzl, Joseph Richter und Franz Xaver Huber. For more elaborate information on Freemasonry in Vienna and the circle Reinhold belonged to, see Rosenstrauch{-}K\"{o}nigsberg, \textit{Freimaurerei im josephinischen Wien. }} After leaving Vienna, Reinhold kept in touch with several of these friends, notably Aloys Blumauer (1755{-}1798), Johann von Alxinger (1755{-}1799) and Gottlieb Leon (1757{-}1832). Serving the Austrian government in those days was tantamount to being involved in Enlightenment activities. From 1780 Emperor Joseph had continued, in a more radical way, the reforms begun by his mother, Maria Theresa. In 1781 he granted freedom of religion to Christian denominations other than Catholicism and extended the freedom of the press, which led to an enormous increase in pamphleteering activities. The reforms were aimed at a rationalization and centralization of government and included the formation of an intellectual elite in state service, on the basis of merit instead of birth. Apart from university professors this new elite included many of Reinhold's friends, so that he made a number of new acquaintances, like Joseph von Sonnenfels (1732{-}1817), who, as an important scholar of Law, was one of the principal advisers at the Habsburg court.\footnote{ Rosenstrauch{-}K\"{o}nigsberg, \textit{Freimaurerei im josephinischen Wien}, 13.} Another important figure was the mineralogist Ignaz von Born (1742{-}1791), a former Jesuit and adviser to the court on mining and the mint. Through his contacts with such persons, Reinhold became part of Austria's intellectual elite. 

Members of this elite frequently belonged to Masonic Lodges, another place where merit was more important than rank or birth. In 1783 Reinhold wrote to Blumauer requesting to be admitted to the Freemasons, and from that letter we can gather that he felt oppressed by his ecclesiastical status. 

Schliessen Sie daraus, wie heftig meine Begierde seyn m\"{u}sse, mich als ein durch einen un\"{u}berdachten Schwur von der Kette der Menschheit abgerissenes Glied, an diese Gesellschaft, und durch sie an den edelsten Theil der Menschheit wieder anzuschliessen, (\ldots ).\footnote{ \textit{RK} 1:11{-}12, Letter 1.1, shortly before April 16, 1783, to Aloys Blumauer. This effectively refutes Gliwitzky's view that Reinhold cannot reasonably have felt oppressed by his monastic status. Gliwitzky did not know this letter. Cf. Gliwitzky, `Carl Leonhard Reinholds erster Standpunktwechsel,' 19, 67. }

[Please infer from this the ardor of my desire to connect myself, as a member torn off from the chain of humanity by an ill{-}considered oath, to this society [Freemasonry], and through it to the noblest part of humanity again.]

Strikingly, in this passage Reinhold returns to a metaphor he had used earlier in the letter to his father. There he wrote that ``the attachment to flesh and blood is (\ldots ) one of the strongest chains (\textit{Kette}) with which Satan binds us to earth.''\footnote{ \textit{RK} 1:5, Letter 1, September 13, 1773. } It is clear that at fifteen and still intent on being a monk, Reinhold made every effort to break the devilish chains, whereas ten years later he regrets having been in a position that required him to be cut off from the rest of humanity. As he could no longer blindly believe everything he had been told to believe when he aspired to be a monk, it was only natural to give in to the desire not to shut out the world. What he sought in Freemasonry was what he had missed in his early days as a novice, being away from home in the Jesuit College, a sense of being connected to the rest of the world, to those who mattered to him. As is clear from the letter to his father, he found a way to deal with this distress, interpreting it as a devilish trick to bind him to the world. A couple of years later, through his friends in Freemasonry, Reinhold sought to reconnect to the world. Within a Masonic Lodge all the worldly differences in status and wealth would fall away, which explains Reinhold's conviction that by becoming a Freemason he would be able to reconnect to mankind, even if he was a monk, that is, formally disconnected.

  After his admittance as a member of the Vienna Lodge \textit{Zur wahren Eintracht} on April 30, 1783, Reinhold not only frequented its meetings, he also participated in a more active manner by writing and reading speeches.\footnote{\label{footnote:_Ref223691918} \textit{RK} 1:394{-}395, `Reinholds freimaurerischer Werdegang in der Loge `Zur wahren Eintracht'.' Two of the speeches that were not subsequently published are extant in manuscript. `\"{U}ber die Kunst des Lebens zu gen\"{u}ssen', ms. Haus{-}, Hof{-}, und Staatsarchiv Wien, Vertrauliche Akten, Kart. 73, fol. 74{-}75, read by Blumauer on June 18, 1783; `Der Wehrt einer Gesellschaft h\"{a}ngt von der Beschaffenheit ihrer Glieder ab', ms. in Haus{-}, Hof{-}, und Staatsarchiv Wien, Vertrauliche Akten, Kart. 73, fol. 64{-}68, read by Reinhold himself on September 5, 1783.} From the letter to Blumauer just cited it is already apparent that by joining Freemasonry Reinhold in a way distanced himself from the monastic ideology.\footnote{ Cf. Batscha, \textit{Karl Leonhard Reinhold}, 22. } One of his early speeches for the Lodge refers to the trickery of monks to recruit youngsters, confirming that monastic life had proven different from what he had expected it to be.\footnote{ `Der Wehrt einer Gesellschaft h\"{a}ngt von der Beschaffenheit ihrer Glieder ab,' see footnote \ref{footnote:_Ref223691918}. } At the same time he adopted the Enlightenment ideology of his friends. This is most apparent from his contributions to a periodical that Blumauer edited at the time, the Vienna \textit{Realzeitung}, ``the organ of the Viennese Enlightenment elite around Born.''\footnote{ Fuchs, \textit{Karl Leonhard Reinhold {--} Illuminat und Philosoph}, 18.} In October 1782, when Reinhold started as a reviewer, Blumauer had just taken over the editorship of the \textit{Realzeitung} and had introduced a new section, in which most of Reinhold's reviews were published, on `Theologie und Kirchenwesen'.\footnote{ Rosenstrauch{-}K\"{o}nigsberg, \textit{Freimaurerei im josephinischen Wien}, 70. The reviews have been photomechanically reprinted in Batscha, \textit{Karl Leonhard Reinhold}.} Most of the reviews are short and of little philosophical interest. They show that religion was much debated in Austria at this time in a wide range of publications. The reviews bear the distinct stamp of Enlightenment. Authors who claim Enlightenment endangers religion are severely criticized by Reinhold, whereas books and pamphlets that are critical of the Baroque Catholicism of late eighteenth{-}century Austria are welcomed, usually with some critical notes on readability.\footnote{ Cf. Batscha, \textit{Karl Leonhard Reinhold}, 41.} In the following chapter the contents of some of these reviews will be dealt with.\footnote{ For a brief overview of the themes discussed in the reviews and the development of Reinhold's Enlightenment in this period, cf. Fuchs, \textit{Karl Leonhard Reinhold {--} Illuminat und Philosoph}, 26{-}31; Batscha, introduction to \textit{Karl Leonhard Reinhold}. } 

Reinhold's move from monasticism to Freemasonry was of vital importance to his development as a philosopher, although it must not be forgotten that his education primed him for Enlightenment with Leibniz, Wolff and Locke.\footnote{ Cf. Fuchs, \textit{Karl Leonhard Reinhold {--} Illuminat und Philosoph}, 31.} His new Enlightenment environment\footnote{ Cf. Jacob, \textit{Living the Enlightenment}, 35, styling freemasonry a ``cult of Enlightenment.''} was especially encouraging him to read and think and write. The Lodge \textit{Zur wahren Eintracht} held special meetings, the so{-}called \textit{\"{U}bungslogen}, where members discussed scientific topics on which one of them had lectured. Reinhold's speeches were also intended to be read on these occasions.\footnote{ \textit{RK} 1:17, n. 17. } One of them, `\"{U}ber die Pflicht des Maurers sich zu freuen', is lost; of two others `\"{U}ber die Kunst des Lebens zu gen\"{u}ssen' and `Der Wehrt einer Gesellschaft h\"{a}ngt von der Beschaffenheit ihrer Glieder ab', an edition is forthcoming.\footnote{ Karianne Marx and Ernst{-}Otto Onnasch, `Zwei Wiener Reden Reinholds. Ein Beitrag zu Reinholds Fr\"{u}hphilosophie' in George di Giovanni ed., \textit{Reinhold and the Enlightenment}. } With its `exercise'{-}meetings, the Lodge \textit{Zur wahren Eintracht} was especially suited for the education of its members. Moreover, it was connected to the secret society of Illuminates, which had been established in 1776 in Bavaria by Adam Weishaupt (1748{-}1830). Several members of \textit{Zur wahren Eintracht}, including Reinhold, were involved in the Illuminate order.\footnote{ For a list see \textit{RK} 1:75{-}76, n. 13.} While Freemasonry meant a retreat from society and its prejudices, Illuminatism took a more active perspective and aimed at changing society in the direction of a universal brotherhood of man. More important, however, than political influence was the education of young, bright students.\footnote{ Being a member of a secret society for education, hidden from the absolutist government, and thus able to read books that were officially not allowed, however, also entailed a political statement. Cf. Agethen, \textit{Geheimbund und Utopie}, 30{-}36. On the political implications of Masonic sociability, see also Jacob, \textit{Living the Enlightenment}.} 

 A speech written in 1782 by the founder of the Illuminate order himself, Adam Weishaupt, provides a glimpse of what the Illuminates taught these students (including our Reinhold). This `Anrede an die neu aufzunehmenden Illuminatos dirigentes'\footnote{ Reprinted in van D\"{u}lmen, \textit{Der Geheimbund der Illuminaten}, 166{-}194. Further references are in the text. For a more elaborate summary than the one given here, see Agethen, \textit{Geheimbund und Utopie}, 106{-}111.} was meant to be read to those who were admitted to the degree of \textit{Illuminatus dirigens} (leading illuminate).\footnote{ For a schematic account of the degrees, see van D\"{u}lmen, \textit{Der Geheimbund der Illuminaten}, 292. } As will be seen in the following Chapter, themes that occur in Reinhold's Enlightenment writings render it highly probable that Reinhold was aware of this text by Weishaupt, or at least of Illuminatist texts of a similar substance. It depicts `real Masonry' as the salvation of mankind, retrieving man's original innocence and happiness. Weishaupt describes the natural state of man as innocent and happy, as man's needs were small and easily met. However, this paradise was not to last; it rather serves as an indication of man's destiny (van D\"{u}lmen, 170). According to Weishaupt, humanity's needs are the driving force for the development of humanity and society (169). With the evolution of society new needs come into being and man drifts away from the state of nature, at the expense of original freedom and equality. However, the darkest hour is just before dawn and as the situation worsens, the need for freedom becomes stronger (177). This need can be met through morality, which will teach people to live like a family again. This morality can only be achieved by means of ``secret schools of wisdom'' (179). Man has to learn to desire in a rational way, has to be enlightened about his own situation. According to Weishaupt, Enlightenment should be more than words, more than abstract, speculative knowledge that has no effect on the heart. Rather, it is knowledge of oneself and of the dependence on the assistance and goodwill of others (183). Christian doctrine is introduced as exemplifying the morality needed to change society. Christianity is presented as a significant progression from Judaism, since the concept of God as a Father and humanity as His children makes morality easy to practice (187). However, humanity was not saved by Christianity, because religion was soon used for despotism. Pure Christianity was kept secret within the organization of Freemasonry (190). Now that Freemasonry itself has become corrupted, it is up to the chosen few to try and establish true morality in their circles (191{-}192). From this brief summary it will be clear that the Illuminatist vision of human history is designed to show the importance of and the need for its own organization and activities, the main goal of which is to make people free and equal again. Reinhold was to be profoundly influenced by this view on human history, particularly by the imagery used regarding Christianity. Especially the role played by the metaphor of family, conveyed by Christianity, would come to be of importance in his philosophical work. Further, the thought that human needs are the driving force behind the development of societies would return as part of Reinhold's views on the history of philosophy. 

As shown above, the adoption of Enlightenment ideology entailed a rejection of the monastic way of life for Reinhold. Yet, since leaving your order was a crime, he had no real choice but to remain in the Barnabite order, while writing against monks and being a member of that other society, the Freemasons. This combination does not necessarily require a split personality. Reinhold's correspondence with Pepermann shows that the former's later statement that his order was \textit{unm\"{o}nchisch} is probably close to the truth. At any rate, the order was open enough for its teachers, like Pepermann, to provide their pupils with more than just the standard textbook philosophy. It was apparently also open enough for Reinhold to visit the Masonic meetings almost weekly and to denounce monastic life before a general audience. However, though the practice of the Barnabite order was friendly, Reinhold may still have felt oppressed by the monastic system which, at least in theory, prevented him from leading a normal life, that is, a life in accordance with human nature. In another of his Masonic speeches he portrays monasticism as aimed at disciplining human nature, by denying its members the rights of freedom, property and procreation by means of the vows of obedience, poverty and celibacy.\footnote{ Reinhold, `M\"{o}nchthum und Maurerey. Eine Rede von Br. R\textasteriskcentered \textasteriskcentered ,' \textit{JF}, 1784 IV, 167{-}188.} The letter to Blumauer shows that Reinhold regretted taking these vows. He clearly distances himself from monastic theory, while enjoying the freedom the order allowed him in practice. Thus, it was the openness of the Barnabite order that enabled Reinhold's escape. By teaching him philosophy in an open atmosphere, it loosened the grip of blind faith and thus in a way already enlightened him. By allowing him to meet his friends it opened the door for new contacts and another social network, in which freedom of thought was common. In the end it was this Masonic network that physically enabled Reinhold to leave, because it provided him with the contacts necessary for his survival as a refugee.


\section{Leipzig (1783{-}1784)}


Since leaving a monastic order without dispensation from one's vows was a criminal offence, Reinhold was in fact a fugitive, not just a refugee, in need of assistance. Here the Masonic organization proved to be very useful indeed. It has already been shown that his flight was facilitated by Petzold. While in Leipzig, where he attended the philosophical lectures of Ernst Platner (1744{-}1818) and others, Reinhold was financially supported by the Vienna Lodge \textit{Zur wahren Eintracht}, both directly and indirectly, through payments for the material he supplied for the \textit{Journal f\"{u}r Freymaurer}.\footnote{ Batscha, \textit{Karl Leonhard Reinhold}, 28. } This magazine was distributed only among other Lodges, so censorship did not apply to it in the same way as it did to normal publications.\footnote{ Rosenstrauch{-}K\"{o}nigsberg, \textit{Freimaurerei im josephinischen Wien}, 66. Cf. \textit{RK}, 1:18{-}19, Letter 2, April 19, 1784, from Ignaz von Born.} Of the speeches Reinhold delivered himself, only one, `Ueber den Hang zum Wunderbaren', was published.\footnote{ Reinhold, `Ueber den Hang zum Wunderbaren,' \textit{JF}, 1784 III, 123{-}138. } His other publications in the \textit{Journal f\"{u}r Freymaurer}\footnote{ Reinhold, `M\"{o}nchthum und Maurerey'; Reinhold, `Ueber die kabirischen Mysterien. von Br. R\textasteriskcentered \textasteriskcentered ,' \textit{JF}, 1785 III, 5{-}48; Reinhold, `Ueber die wissenschaftliche Maurerey', \textit{JF}, 1785 III, 49{-}78; Reinhold, `Ueber die Mysterien der alten Hebr\"{a}er' \textit{JF}, 1786 I, 5{-}79; Reinhold, Ueber die gr\"{o}\ss{}ern Mysterien der Hebr\"{a}er, \textit{JF} 1786 III, 5{-}98. The latter two articles were later published together as \textit{Die Hebr\"{a}ischen Mysterien oder die \"{a}lteste religi\"{o}se Freymaurerey} (Leipzig: G\"{o}schen 1788) [under Reinhold's Illuminatist pseudonym `Br. Decius']. } were either read out by somebody else or were adaptations of texts written by others, as is clear from his correspondence with Born in 1784.\footnote{ \textit{RK}, Letters 2 and 3. } It is therefore likely that they were written when Reinhold had already left Vienna. It is unclear to what extent Reinhold is to be regarded as the author of the article on scientific Freemasonry. It does not appear to be typical of his work and it lacks the indication `von Br. R\textasteriskcentered \textasteriskcentered ' in the title. It appears in the same issue as the article on the Kabirian mysteries {--} which we know is Reinhold's {--} yet a letter from Blumauer to Bertuch suggests that there is only one piece by Reinhold in that issue.\footnote{ ``Sie erhalten hier den 3ten Band des 2ten Jahrg. unseres Maurerjournals, dessen Herausgabe meiner Krankheit wegen etwas verz\"{o}gert wurde. Sie werden darin eine treffliche Arbeit unseres Reinholds finden.'' Letter from Blumauer to Bertuch, September 30, 1785, cited from: Rosenstrauch{-}K\"{o}nigsberg, \textit{Freimaurerei im josephinischen Wien}, 236{-}237. } Thematically, the speeches and articles are diverse, but the subject is usually in some way connected to Freemasonry. In `M\"{o}nchthum und Maurerey', Reinhold compares the monastic and Masonic orders `with regard to Enlightenment'. In `Ueber den Hang zum Wunderbaren' he is concerned for the moral and emotional education of Masons, from an Illuminatist point of view.\footnote{ Cf. Fuchs, \textit{Karl Leonhard Reinhold {--} Illuminat und Philosoph}, 38.} In several other articles he discusses the mysteries of the ancient world, which were believed to be the origin of Freemasonry. 

 Apart from his Masonic activities, Reinhold attended Platner's lectures Although the extent to which this influenced his later philosophy is not clear, there are certain features of Platner's philosophical outlook that deserve to be highlighted here, because they suggest that some aspects of his philosophy may indeed have been important for Reinhold's philosophical development. Platner can be described as an adherent of the Leibnizian{-}Wolffian school, who placed great emphasis on the notion of representation and the power of representation (\textit{Vorstellungskraft}). This is striking in the light of Reinhold's later theory of the faculty of representation, which also has the notion of representation at its centre. Platner was also partly an \textit{Aufkl\"{a}rer}, who saw more in morality than in religion. His lectures in Leipzig were very popular among students, most probably with Reinhold as well. In the winter semester of 1783/1784, the time of Reinhold's stay in Leipzig, Platner lectured on the basis of his own \textit{Philosophische Aphorismen}.\footnote{ \textit{RK }1:15, n. 2. Ernst Platner, \textit{Philosophische Aphorismen nebst einigen Anleitungen zur philosophischen Geschichte }(Leipzig: Schwickert 1776); Ernst Platner, \textit{Philosophische Aphorismen nebst einigen Anleitungen zur philosophischen Geschichte. Anderer Theil} (Leipzig: Schwickert 1782). } This work is not only concerned with the systematic development of philosophical concepts, but also with their history. Platner took a special interest in the history of philosophy and aimed at systematically relating historical doctrines.\footnote{ Fuchs, \textit{Karl Leonhard Reinhold {--} Illuminat und Philosoph}, 34.} In his presentation Platner shows signs of being influenced by \textit{Popularphilosophie}, a development within the Leibnizian{-}Wolffian tradition, seeking to apply philosophy to real life, outside academia.\footnote{ For more on \textit{Popularphilosophie}, see B\"{o}hr, \textit{Philosophie f\"{u}r die Welt}.} However, Reinhold was not to adopt Platner's philosophy for good, as in his \textit{Versuch }he counted him among the many philosophers who had misunderstood Kant (cf. \textit{Versuch}, 155).\footnote{ For the later relation of Reinhold and Platner, see Onnasch, introduction to \textit{Versuch}, [XXX{-}XXXI, LXXIII{-}LXXIV].}

Reinhold remained in Leipzig for half a year (November 1783{-}May 1784). His stay at a Protestant university had become known in Vienna, obstructing the negotiations of his friends to bring about his return without fear of prosecution. Therefore he was instructed to go to Weimar as the prot\'{e}g\'{e} of the famous poet Christoph Martin Wieland, who was held in great esteem, if not adored, in Vienna.\footnote{ \textit{RK} 1:15{-}16, Letter 2, April 19, 1784, from Ignaz von Born. See \textit{RK}, 1:23{-}24, Letter 3, June 9, 1784, from Ignaz von Born, for the adoration of Born for Wieland. } 


\section{Weimar (1784{-}1787)}


Maybe the friendly reception by Wieland and his family\footnote{ Cf. \textit{RLW}, 24. } and the possibility to make a living in Weimar inspired Reinhold to convert to Protestantism, and he was formally accepted in the Protestant church by Herder, who was the General Superintendent of the Lutheran clergy at the Weimar court. If his stay in Leipzig had frustrated his friends' attempts to secure his safe return to Vienna, converting to Protestantism made his possible return no longer an issue and entailed a final break with the Catholic Austrian environment in which he grew up, although, as mentioned earlier, he did keep in touch with Pepermann and his Masonic friends. The topics of his correspondence vary, from issues concerning the Lodge and his contributions for the \textit{Journal f\"{u}r Freymaurer }to the latest news and gossip. He still received more than just moral support from Vienna. As he stated later, at the time of his arrival in Weimar he depended on the kindness of Wieland and Blumauer.\footnote{ \textit{RK} 1:180{-}181, Letter 41, January 26, 1787, to Nicolai.}

The reception of Reinhold as a friend and later a member of the Wieland family was warm and kind. His new situation also provided him with a new livelihood, as he started writing for \textit{Der Teutsche Merkur} almost immediately upon his arrival in Weimar, his first review appearing in June 1784, in the \textit{Anzeiger}, the review section of the \textit{Merkur}.\footnote{ Reinhold, review of \textit{Ideen zur Philosophie der Geschichte der Menschheit}, by Herder, \textit{Anzeiger des Teutschen Merkur}, June 1784, LXXXI{-}LXXXIX.} This monthly periodical had been established in 1773 by Wieland after the example of the \textit{Mercure de France}, as a literary magazine. Among the early subscribers was Sonnenfels in Vienna.\footnote{ Wahl, \textit{Geschichte des Teutschen Merkur}, 12.} When Reinhold arrived in Weimar the \textit{Merkur} was gradually recovering from a downfall in the early 1780s. Wieland had entered into a partnership with Friedrich Justin Bertuch (1747{-}1822), who, being more of an entrepreneur than Wieland, had the money and a plan to get the \textit{Merkur} up on its feet again. Bertuch attracted new authors, who gradually disappeared after Reinhold became an important author for the magazine.\footnote{ Cf. Wahl, \textit{Geschichte des Teutschen Merkur}, 164.} When Reinhold married Wieland's eldest daughter Sophie on May 16, 1785, Wieland helped the young couple financially by raising the remuneration for reviewers by two{-}thirds, from which Reinhold benefited most.\footnote{ Cf. Wahl, \textit{Geschichte des Teutschen Merkur}, 169.} As the relation between Wieland and Bertuch worsened, the former wanted to see the \textit{Merkur} in family hands, again to help Reinhold out financially. In July 1786 Reinhold took Bertuch's place, as fellow editor, not as investor, of course.\footnote{ Cf. Wahl, \textit{Geschichte des Teutschen Merkur}, 170; \textit{RK} 1:48, Letter 10, after May 16, 1785, from Anna Dorothea Wieland .} He would remain active for \textit{Der Teutsche Merkur} until 1788, when other projects demanded his attention. During this period he wrote many reviews and articles, the earliest of which clearly reflect that he started feeling at home in the Protestant North, which he often compares favorably to the situation in Austria, where Enlightenment clashed with the Catholic Church. According to Ernst Reinhold, his father almost filled the review section (\textit{Anzeiger}) of the \textit{Merkur} by himself.\footnote{ \textit{RLW}, 25. Cf. Wahl, \textit{Geschichte des Teutschen Merkur}, 168.} Most of these reviews have not been officially attributed to Reinhold, and apart from the review of Herder's \textit{Ideen zur Philosophie der Geschichte der Menschheit}, of which more will be said in Chapter 3, they are of little philosophical interest and will not be discussed here. Apart from the reviews already attributed to him,\footnote{ Reinhold, review of \textit{Ueber die Einsamkeit}, by Zimmermann, \textit{Anzeiger TM}, August 1784, CXIII; Reinhold, review of \textit{Briefe \"{u}ber die Schweiz}, 1. und 2. Theil, by Meiners, \textit{Anzeiger TM}, December 1784, CLXXVII; Reinhold, review of \textit{Oeuvres de Valentin Jamerai Duval, prec\'{e}d\'{e}s des Memoires sur sa Vie}, \textit{Anzeiger TM}, December 1784, CLXXVIII.} the one of Born's \textit{Iohannis Physiophili Specimen Monachologiae }is most probably also by his hand.\footnote{ Review of \textit{Iohannis Physiophili Specimen Monachologiae}, \textit{Anzeiger TM}, June 1784, XCII{-}XCIV.} It need not be discussed in great detail here, since it mainly contains praise of the author for having discovered a new, unnatural animal species, the monk. Further, it provides an opportunity for ridiculing Patritius Fast, which strengthens the case for Reinhold's authorship.\footnote{ Pater Patritius Fast was the Rector of the Metropolitan Church in Vienna and is mentioned frequently in the \textit{Realzeitung\- }reviews as an adversary of Enlightenment and a superstitious man. His opinion on Jesus's circumcision, earned him a third `P' in Reinhold's circle, for `\textit{praeputium'}, `foreskin'.} The citations of Fast run parallel to text used by Reinhold earlier in the \textit{Realzeitung},\footnote{ Reinhold, review of \textit{Vollkommene Abfertigung des Freund Werklins mit seinem vollkommenen Widerlegungsschreiben}, by Geisttrich,' \textit{RZ}, August 12, 1783 [316{-}319].} which renders Reinhold's authorship even more likely. 

From July 1784 onward, articles by Reinhold's hand appeared in the \textit{Merkur} as well. The first piece is the rather short `Die Wissenschaften vor und nach ihrer Sekularisation. Ein historisches Gem\"{a}hlde'.\footnote{ Reinhold, `Die Wissenschaften vor und nach ihrer Sekularisation. Ein historisches Gem\"{a}hlde,' \textit{TM}, July 1784, 35{-}43 (Photomechanical reprint in Batscha, \textit{Karl Leonhard Reinhold}, 398{-}406).} It is clear that Reinhold's interest here lies with the main lines of history, not with the correctness of the details. The next article, `Gedanken \"{u}ber Aufkl\"{a}rung' is much more elaborate; starting in the issue of July, it is continued in the August issue and concluded in the September issue of 1784.\footnote{ Reinhold, `Gedanken \"{u}ber Aufkl\"{a}rung', \textit{TM}, July, 3{-}22; August, 122{-}133; September, 232{-}245 (Photomechanical reprint in Batscha, \textit{Karl Leonhard Reinhold}, 352{-}396).} It is one of the first attempts to give some theory of Enlightenment. It was published before Kant's `Beantwortung der Frage: Was ist Aufkl\"{a}rung?' \footnote{ Immanuel Kant, `Beantwortung der Frage: Was ist Aufkl\"{a}rung?', \textit{Berlinische Monatsschrift}, December 1784, 481{-}494. According to Wahl, Kant reacts to Reinhold, rather than to Z\"{o}llner. Wahl, \textit{Geschichte des Teutschen Merkur}, 189. } which explicitly reacted to the question posed by Johann Friedrich Z\"{o}llner (1753{-}1804) in his `Ist es rathsam, das Eheb\"{u}ndnis nicht ferner durch die Religion zu sanciren?'\footnote{ Johann Friedrich Z\"{o}llner, `Ist es rathsam, das Eheb\"{u}ndnis nicht ferner durch die Religion zu sanciren?' \textit{BM}, December 1783, 516.} Kant's answer is dated September 30 and mentions only Mendelssohn's answer\footnote{ Moses Mendelssohn, `Ueber die Frage: was hei\ss{}t aufkl\"{a}ren?' \textit{BM}, September 1784, 193{-}200. } to the same question, not Reinhold's pieces for the\textit{ Merkur}. Although Reinhold does not refer to Z\"{o}llner, his efforts to understand and determine the concept of Enlightenment suggest that he aimed to answer Z\"{o}llner's question. In the first installment Reinhold determines the concept of Enlightenment negatively, paying special attention to those who claim to promote Enlightenment, without knowing what it is. In the second part of his article, Reinhold gives a more positive determination, explaining `Enlightenment' with regard to individuals and nations. Finally, he shows that it is not an empty concept by explaining the rational capacity of human beings in general and their need for Enlightenment, given the history of society. Reinhold's `Ueber die neuesten patriotischen Lieblingstr\"{a}ume in Teutschland', published in August and September 1784, is explicitly based on his knowledge of the situation in Vienna.\footnote{ Reinhold, `Ueber die neuesten patriotischen Lieblingstr\"{a}ume in Teutschland. Aus Veranlassung des 3. und 4. Bandes von Hrn. Nicolai's Reisebeschreibung', \textit{TM}, August, 1784, 171{-}186; September, 246{-}264.} The subtitle reveals that Christoph Friedrich Nicolai's (1733{-}1811) \textit{Reisebeschreibung}\footnote{ Christoph Friedrich Nicolai, \textit{Beschreibung einer Reise durch Deutschland und die Schweiz im Jahre 1781}, Berlin 1783{-}1796, 12 volumes. This work was also known in Vienna, given that Reinhold refers to it in one of his last reviews for the \textit{Realzeitung} [337], although at that time probably not the 3rd and 4th volume, to which the article in the \textit{Merkur} refers\textit{. }} occasioned this article, which aims at providing a more realistic perspective on the Enlightenment in Germany. Reinhold's next article, `Schreiben des Pfarrers zu \textasteriskcentered \textasteriskcentered \textasteriskcentered  an den H[erausgeber]. des T[eutschen]. M[erkurs].' provided the first occasion for Reinhold to react to Kant.\footnote{ Reinhold, Schreiben des Pfarrers zu \textasteriskcentered \textasteriskcentered \textasteriskcentered  an den H[erausgeber]. des T[eutschen]. M[erkurs]. Ueber ein Recension von Herders Ideen zur Philosophie der Geschichte der Menschheit,' \textit{TM}, February, 1785, 148{-}174. } Although, as we shall see below (Chapter 3, section 1) Reinhold was not familiar with the Kantian philosophy at that moment in time, he reacted to Kant's negative review of Herder's \textit{Ideen}, which he himself had reviewed positively. Another major article from his pre{-}Kantian period was the tripartite `Ehrenrettung der Reformation', which was a reaction to a book on German history.\footnote{ Reinhold, `Ehrenrettung der Reformation gegen zwey Kapitel in des k.k. Hofraths und Archivars Hrn. M.I. Schmidts Geschichte des Teutschen,' \textit{TM}, February, 1786, 116{-}142; March, 193{-}228; April, 42{-}80.} Next to some smaller articles reacting to other publications,\footnote{ Reinhold, `Berichtigungen und Anmerkungen \"{u}ber eine Stelle aus der Brosch\"{u}re Faustin, oder das philosophischen Jahrhundert. Zweytes B\"{a}ndchen. S. 83,' \textit{TM}, March, 1785, 267{-}277; `Revision des Buches: Enth\"{u}llung des Systemes der Weltb\"{u}rger{-}Republik,' \textit{TM}, May, 1786, 176{-}190; `Auszug einiger neueren Thatsachen aus H. Nikolais Untersuchung der Beschuldigungen der Herrn Prof. Garve u.s.w.,' \textit{TM}, June, 1786, 270{-}280. } Reinhold also published an article on the history of religion, entitled `Skizze einer Theogonie des blinden Glaubens'.\footnote{ Reinhold, `Skizze einer Theogonie des blinden Glaubens,' \textit{TM}, June, 1786, 229{-}242. } Since this article was the last to be published before the `Briefe \"{u}ber die Kantische Philosophie', a project started in August 1786, yet also figures as a part of the `Zw\"{o}lfter Brief' in the 1790 book edition (\textit{Briefe I}), it is indicative of a continuity between Reinhold's pre{-}Kantian and Kantian phases.\footnote{ \textit{Briefe I}, `Zw\"{o}lfter Brief', 358{-}371. Cf. \textit{Briefe I}, Bondeli ed., 339, n. 401.} 

Apart from his writing for the \textit{Merkur} and the \textit{Journal f\"{u}r Freymaurer} Reinhold also produced his first book(let) in his pre{-}Kantian period, albeit anonymously.\footnote{ Reinhold [anonymously], \textit{Herzenserleichterung zweyer Menschenfreunde, in vertraulichen Briefen \"{u}ber Johann Caspar Lavaters Glaubensbekenntnis} (Frankfurt and Leipzig 1785). Due to the Leipzig censorship, the work was printed in Halle. Cf. \textit{RK} 1:199, Letter 43, March 23, 1787, to Nicolai.} The \textit{Herzenserleichterung zweyer Menschenfreunde} was written before he started studying Kant. It has been argued by Reinhard Lauth that this reaction to Johann Caspar Lavater (1741{-}1801), a philosopher from Switzerland, is connected to Reinhold's contacts with fellow Illuminate Johann Joachim Christoph Bode (1730{-}1793).\footnote{ Lauth, `Nouvelles R\'{e}cherches,' 614.} The work is written as a dialogue between Lichtfreund and Wahrmund, who discuss matters of religion. Lavater is criticized severely by both participants in the dialogue, who appear to differ mainly on the question whether they should be seriously worried by Lavater's publications. While discussing Lavater in particular, and matters of religion more generally, the booklet also provides an evaluation of Enlightenment in much the same way that Reinhold would later do again in his `Briefe \"{u}ber die Kantische Philosophie'. 

 Appearing in installments in \textit{Der Teutsche Merkur}, Reinhold's `Briefe \"{u}ber die Kantische Philosophie' (1786{-}1787) discuss the Kantian philosophy in relation to Enlightenment. They open with a discussion between Reinhold and a fictional correspondent, who is depressed by the lack of progress made by Enlightenment in northern Germany. Reinhold, however, interprets the widespread confusion in metaphysics as a sign that a reformation of philosophy is near. Diagnosing the confusion as a misunderstanding among philosophers about the conception of reason, he introduces Kant's \textit{Critique of Pure Reason }as the solution. By specifying what speculative reason can and cannot achieve and by assigning a special role to practical reason, Kant has found the proper conception of reason, which relegates the previous misconceptions to the past, or so Reinhold claims (without really bothering to prove it). His argument for this interpretation of Kant mainly consists in historically arguing for the correctness of his own diagnosis of the problem of metaphysics. The `Briefe' not only made Reinhold famous overnight, but also put the \textit{Merkur }in the frontline of journals that propagated Kantian philosophy, like the \textit{Allgemeine Literatur{-}Zeitung} and the \textit{Berlinische Monatsschrift}, although Wieland never became a Kantian.\footnote{ Wahl, \textit{Geschichte des Teutschen Merkur}, 170{-}171, 188.} 


\section{Jena (1787{-}1794)}


After the success of the `Briefe' Reinhold became extraordinary professor of philosophy at the University of Jena in 1787, to which town he moved in the summer of that year. Both Voigt and Herder supported his appointment, although Reinhold did not possess any formal qualification. In September he asked the philosophy department if they would be so kind to grant him the title of \textit{Magister}, now that he would be lecturing at a university.\footnote{ \textit{RK} 1:267{-}268, Letter 64, September 20, 1787, to the University of Jena (Philosophical department). } Thus, Reinhold's appointment appears to be somewhat strange, as his main merit in philosophy at the time consisted in the first few `Briefe' in \textit{Der Teutsche Merkur} and a handful of other articles. It has been suggested by Kurt R\"{o}ttgers that Reinhold's appointment had more to do with politics than with philosophical merit.\footnote{ Kurt R\"{o}ttgers, `Die Kritik der reinen Vernunft und K.L. Reinhold', 795.} Reinhold was indeed acquainted with both Voigt and Herder, possibly even with Karl August, Duke of Saxony{-}Weimar{-}Eisenach himself, who was an Illuminate, like Reinhold.\footnote{ \textit{RK} 1:226, n. 8.} From Reinhold's correspondence with Voigt it is clear that the former has hopes of professional advantage by discussing the situation of Kantian philosophy with Voigt. The letter in question will be discussed in Chapter 3. Reinhold started his academic activities in Jena in the winter term of 1787/1788, although his first lectures had not been included in the official catalogue of the university. From that time on he would lecture in Jena up to the winter term of 1793/1794, after which period he moved to Kiel as an ordinary professor. During this six{-}year span Reinhold would lecture on Kant's first \textit{Critique} almost every term, albeit under varying titles.\footnote{ See Neuper, \textit{Das Vorlesungsangebot an der Universit\"{a}t Jena von 1749 bis 1854}. For a detailed account of Reinhold's lecturing activities, see Onnasch, introduction to \textit{Versuch}, `Reinholds Vorlesungen in Jena.'} Apart from this he also lectured almost every term on Logic and Metaphysics, first according to Platner's handbook,\footnote{ Probably Ernst Platner, \textit{Philosophische Aphorismen}.} later on the basis of his own notes. A further recurring subject is Aesthetics, first according to Eberhard's handbook,\footnote{ Probably Johann August Eberhard, \textit{Theorie der sch\"{o}nen K\"{u}nste und Wissenschaften} (Berlin, 1783). } later again based on his own notes. In the beginning he also lectured on \textit{Oberon}, a poetic work by his father{-}in{-}law, Wieland. By the winter term 1789/1790 this subject is being replaced by the `History of Philosophy according to his own propositions'. The fact that Reinhold gradually abstains from lecturing on the basis of textbooks of others shows that as a lecturer he gradually acquired the confidence to teach his own views. 

 Reinhold's lectures on Kant's first \textit{Critique }formed a substantial part of his academic activity. According to his letter to Kant, dated January 19, 1788, he approached the work from a historical perspective, elucidating the situation of philosophy immediately before the \textit{Critique} was published and the ``necessity of overcoming the misunderstanding that has divided the philosophical world in four parties.''\footnote{ \textit{RK} 1:315, Letter 84, January 19, 1788, to Immanuel Kant. } Moreover, he announces that he is planning to publish his efforts in this field in order to ``provide the \textit{Critique of Pure Reason} with better prepared readers.''\footnote{ \textit{RK} 1:315. } Within two years, however, Reinhold was not to publish an introduction to Kant's philosophy, but his \textit{Versuch einer neuen Theorie des menschlichen Vorstellungsverm\"{o}gens}, which one may translate as \textit{Essay on a New Theory of the Human Faculty of Representation}. By choosing this title, Reinhold places his book in the tradition of Locke and Leibniz, who both wrote \textit{Essays} on human understanding.\footnote{ John Locke, \textit{An Essay concerning Human Understanding} (1690); Gottfried Wilhelm Leibniz, \textit{Nouveau essais sur l'entendement humain} (written 1704, published 1764). } These \textit{Essays} in a way exemplify the conflict in early modern philosophy, since Locke had presented an empiricist vision of human understanding, while Leibniz' \textit{Nouveau essais} were composed as a refutation of Locke. We have seen earlier that in the introduction to his \textit{Versuch} Reinhold presents himself as a reader of Locke as well as Leibniz. He stresses his knowledge of Locke in the original by citing from Locke's \textit{Essay} in English. Apart from suggesting erudition, the reference to Locke and Leibniz also serves to call attention to Reinhold's perspective on the recent history of philosophy. The opposing Essays by Locke and Leibniz can be seen as symptoms of the general lack of determination of what is to be understood by `human understanding'. With his \textit{Versuch} Reinhold proposes nothing less than to end this deadlock once and for all by providing the philosophical public with a theory of the faculty of representation. In his lectures, Reinhold presented logic and metaphysics following a similar scheme, as he announced `Logic and Metaphysics according to Leibniz, Locke and others'.\footnote{ Cf. Fuchs, \textit{Karl Leonhard Reinhold {--} Illuminat und Philosoph}, 171{-}172, n. 131. Apart from the catalogue of lectures from the Unversity of Jena, Fuchs has also consulted the `Lektionszettel', which describe the content of the lectures in more detail. }

 In the First Book of the \textit{Versuch}, concerning the need for a theory of the faculty of representation, Reinhold claims that philosophers cannot reach an agreement on the precise determination of possible human knowledge, because they do not have the same concept of knowledge. In order to remedy this, the concept of knowledge needs to be explicitly derived from the concept of representation, on the basic features of which everyone agrees, or should agree, as far as Reinhold is concerned. It is clear that for Reinhold this is not only a matter of philosophical importance. As long as the question what we can know cannot be answered in a manner satisfactory for all, the most important questions concerning religion and morality cannot be answered. In the Second Book, Reinhold develops a theory of the faculty of representation on the basis of universally accepted (\textit{allgemeingeltend}) premises. `Representation' and its conditions are analyzed according to the two essential components `material' (\textit{Stoff}) and `form' (\textit{Form}). The Third Book then proceeds to derive the most relevant `results' of Kant's philosophy from the theory of the faculty of representation just established. It analyzes the faculty of knowledge by focusing on a theory of sensibility', theory of understanding and theory of reason.

 As Reinhold's efforts to provide the Kantian project with a secure foundation inspired mixed reactions in the philosophical community, Reinhold's literary activities in his Jena years were mainly related to the \textit{Versuch }and its subsequent defense, a phase in his work which is now known as \textit{Elementarphilosophie}. In response to criticism, two volumes of \textit{Beytr\"{a}ge zur Berichtigung bisheriger Mi\ss{}verst\"{a}ndnisse der Philosophen }were published in 1790 and 1794, relating to theoretical and practical philosophy respectively. His shorter \textit{\"{U}ber das Fundament des philosophischen Wissens} was published in 1791 and is considered the most accessible presentation of the \textit{Elementarphilosophie}.\footnote{ Cf. Frank, \textit{Unendliche Ann\"{a}herung}, 217. } Reinhold also continued the project of the `Briefe \"{u}ber die Kantische Philosophie', by publishing two volumes, the first of which (1790) contains the `Briefe' that had appeared in the \textit{Merkur} in 1786{-}1787, albeit in a revised and expanded version. The second volume of \textit{Briefe \"{u}ber die Kantische Philosophie} (1792) is more related to practical philosophy proper and contains the development of Reinhold's own position on the freedom of the will, which deviates notably from Kant's.

 Reinhold's activities in Jena, however, were by no means limited to writing books and articles, although it is obvious from the above summary that writing and editing must have taken up much of his time. The remainder of his time was invested in his academic life, and with considerable success. His lectures were well attended and were supplemented by \textit{privatissime} lectures at his house. Reinhold also participated in a lecturers' society, building relations with his colleagues as well as with his students. When he traded Jena for Kiel in 1794, to take up an ordinary professorate, his students were so sorry to see him go that they tried to convince him to stay.\footnote{ \textit{RLW}, 63{-}66. } Reinhold was, of course, deeply touched by the gesture, but moved to Kiel anyway, where he would remain until his death in 1823. 

