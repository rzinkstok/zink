
\chapter{Introduction}


Nachdem also von den Reichen und M\"{a}chtigen in der gegenw\"{a}rtigen philosophischen Welt f\"{u}r das neue Evangelium der reinen Vernunft so wenig zu hoffen ist. (\ldots ): so glaubte ich meine eigene Mittelm\"{a}ssigkeit viel eher f\"{u}r ein Merkmal als f\"{u}r ein Hinderni\ss{} des Berufes ansehen zu m\"{u}ssen, den ich (\ldots ) zu f\"{u}hlen anfieng: die Gr\"{u}nde meiner innigsten \"{U}berzeugung von der \textit{Realit\"{a}t} und \textit{dem ungemeinen Nutzen} dieser Wissenschaft, meines Gleichen (\ldots ) vorzulegen.\footnote{ \textit{RK} 1:153, Letter 35, beginning of November 1786, to Christian Gottlob von Voigt.} 

[Since nothing much was to be expected from the high and mighty in the philosophocal world with respect of the new gospel of pure reason, I considered my own mediocrity to be a characteristic, rather than an obstacle with regard to the calling I began to feel: to present to my peers the grounds for my innermost conviction of the \textit{reality} and \textit{the incredible usefulness} of this science.]

Although he was an influential figure in his own day, Karl Leonhard Reinhold's (1757{-}1823) contribution to the German reception of Kant's Critical project was obscured for a long time. It is only in the last couple of decades that his works have become the subject of scholarly attention on a significant scale. The present study is a contribution to this rising interest in Reinhold's philosophy, aiming to add a new perspective by focusing on the development of Reinhold's practical philosophy before and during the time when he saw and presented himself as a supporter of the Kantian philosophy. This focus is reflected in the title \textit{The Usefulness of the Kantian Philosophy} which is based on Reinhold's description of his plans regarding this new philosophy laid down in a letter to Christian Gottlob von Voigt (1743{-}1819) dating from the beginning of November 1786, a passage from which is cited above. Reinhold projects two volumes, the first of which is to discuss the \textit{Nutzen}, that is, the usefulness of the Kantian philosophy. With regard to volume on the \textit{Realit\"{a}t}, or the inner grounds of this philosophy, nothing is specified. The external grounds, however, relating to its usefulness, are presented in an overview and can be traced in his published work. These external grounds are not to be taken from the Kantian work, ``but rather from the current state of philosophy and the most urgent scientific and moral needs of our time.''\footnote{ \textit{RK} 1:153.} Thus Reinhold sets out to promote the Kantian philosophy on the basis of grounds that are to be found outside it, in the wider philosophical and cultural context.  The main thesis of the present study is that Reinhold's efforts to propagate the Kantian project in this manner, that is, on the basis of these external grounds, are not only expressed in his `Briefe \"{u}ber die Kantische Philosophie' (1786{-}1787), but also in his \textit{Versuch einer neuen Theorie des menschlichen Vorstellungsverm\"{o}gens} (1789) and in the second volume of the book edition of the \textit{Briefe \"{u}ber die Kantische Philosophie }(1792). Although these works differ in subject, size and projected audience, I will argue that they are the result of Reinhold's initial reception of Kant. This is not the case for some of his other works from the same period, such as the two volumes of \textit{Beytr\"{a}ge} (1790 and 1794), which are the result of Reinhold's reactions to criticism his \textit{Versuch} received. As will be clear from the following, this is a novel approach.\footnote{ Karl Ameriks has pointed out the relevance. Cf. Ameriks, `Reinhold's \textit{first }Letters on Kant,' 13; Ameriks, introduction to Letters. } It seeks to present Reinhold's philosophy from the point of view of his own philosophical development. In order to understand why this approach is new, it will be useful to look at how his philosophy came to be obscured for a long time, and how it was rediscovered in the last couple of decades. 

The relatively obscure status of Reinhold in the history of philosophy is at least partly due to his own attitude towards philosophy, which was characterized by a tendency to keep an open mind towards new theories. As we shall see throughout the present study, he had definite ideals of what philosophy should accomplish and spent his life searching for the system that would achieve these goals. The net result of his open attitude was that he did not develop one system to which he held on for the rest of his life, but rather tried to achieve the aims of philosophy in reaction to the philosophical systems available at a given time. Due to his own philosophical development, but to a large extent also due to the rapid changes occurring in German philosophy at the end of the eighteenth century, Reinhold shifted allegiance several times. This, in the end, earned him the reputation of a philosophical lightweight, whose preferences changed with the weather.\footnote{ Schelling called Reinhold a ``vom Wind umhergetriebene Rohr'' and also a ``Schwachkopf.'' His changes of system were interpreted as a sign of ``philosophische Imbecilit\"{a}t'' (Hegel). Later commentators have interpreted Reinhold's tendency to work in reaction to other philosophers as unmanly (Kuno Fischer) and as clinging to authority, supposedly a remnant form his former Catholicism (Karl Rosenkranz). Cited in Sch\"{o}nborn, \textit{Reinhold. Eine annotierte Bibliographie}, 21{-}22.} 

In actual fact, Reinhold had the fortune of living and working in a period when philosophy was constantly developing, in which developments he played an active and in many respects crucial role. During the last two decades of the eighteenth century Immanuel Kant (1724{-}1804) was publishing his Critical project, which was soon to be overturned by the German Idealists. Since Reinhold's philosophy was itself part of the philosophical turmoil resulting from the reception and critical development of the Kantian philosophy, it is not hard to understand how his contributions got obscured. In order to make sense of this short period in which so many developments took place, and in which the sheer amount of philosophical works grew tremendously because of the rise of scholarly journals and reviews,\footnote{ Cf. Lazzari, \textit{Das Eine, was der Menschheit Noth ist}, 20{-}21.} historians of philosophy have had to make their choices. Nothing was more natural than to try and understand the period from the point of view of one of the philosophical perspectives which had maintained their influence well into the nineteenth century. This means that the end of the eighteenth century was either interpreted from a (neo{-}) Kantian perspective, or from the point of view of the later Idealists. With regard to this, Alexander von Sch\"{o}nborn has stated that ``Reinhold has been the victim of philosophical myth{-}building.''\footnote{ Sch\"{o}nborn, \textit{Reinhold. Eine annotierte Bibliographie}, 25.} From the Kantian perspective, Reinhold represents the first move away from Kant, a move away that is more radically represented by the philosophies of Johann Gottlieb Fichte (1762{-}1814), Friedrich Wilhelm Joseph Schelling (1775{-}1854) and Georg Wilhelm Friedrich Hegel (1770{-}1831). From this perspective, Reinhold's contribution is not viewed positively, since it did not live up to Kantian expectations. From a Hegelian point of view, it has been all too easy to rebuke Reinhold for not going far enough and ignore his importance altogether.\footnote{ Hegel, in fact, owed much to Reinhold, even if he refused to acknowledge it. Cf. Onnasch, `Hegel zwischen Fichte und der T\"{u}binger Fichte{-}Kritik,' 173. } From both perspectives, the story of the philosophical developments in late eighteenth{-}century Germany could be presented without needing to pay special attention to the role of Reinhold. His influence at the time, though, is not only relevant for a more complete understanding of his own philosophy, but is also important with regard to our picture of later German philosophy. It is not uncommon to present the post{-}Kantian philosophy as a single line from Kant, via Fichte, to Schelling and Hegel, with a small role for Reinhold between Kant and Fichte. It is important to realize, however, that there are, in fact, two lines of reception of the Kantian philosophy here. Fichte found his own way to Kant and then built on Reinhold's thoughts to criticize it. Schelling and Hegel, on the other hand, first became acquainted with the Kantian philosophy through Reinhold's `Briefe \"{u}ber die Kantische Philosophie' (1786{-}1787) and \textit{Versuch einer neuen Theorie des menschlichen Vorstellungsverm\"{o}gens} (1789), which gives them a different starting{-}point for the reception of the Critical Philosophy.\footnote{ For a more detailed account of these two lines of reception, cf. Onnasch, introduction to \textit{Versuch}, [XV{-}XXII].}

 Thus, Reinhold came to be regarded as merely a minor figure, whose philosophy was rightly forgotten. Fortunately, this fate has changed over time, first slowly and almost unintentionally, with an edition of his \textit{Briefe \"{u}ber die Kantische Philosophie}, appearing in 1923 to mark the first centenary of his death. This edition, however, was expressly undertaken to contribute to Kant's glory, rather than to initiate a `Reinhold Renaissance'.\footnote{ Raymund Schmidt, introduction to \textit{Briefe \"{u}ber die Kantische Philosophie von Carl Leonhard Reinhold}, 5. } Apart from a few early monographs,\footnote{ Herbert Adam, \textit{Carl Leonard Reinholds philosophischer Systemwechsel }(Heidelberg: Carl Winters Universit\"{a}tsbuchhandlung, 1930); Magnus Selling, \textit{Studien zur Geschichte der Transzendentalphilosophie. I: Karl Leonhard Reinholds Elementarphilosophie in ihrem philosophiegeschichtlichen Zusammenhang. Mit Beilagen Fichte's Entwicklung betreffend} (Uppsala: Lundequistska Bokhandeln, 1938). Alfred Klemmt, \textit{Karl Leonhard Reinholds Elementarphilosophie. Eine Studie \"{u}ber den Ursprung des spekulativen deutschen Idealismus} (Hamburg: Meiner, 1958).} interest in his work has only developed gradually from the 1960s onwards. After the edition of the volumes of the \textit{Briefe}, other works have been made available to a more general public by means of annotated edition or reprint. Reinhold's main work, \textit{Versuch einer neuen Theorie des menschlichen Vorstellungsverm\"{o}gens, }was reprinted in 1963, followed by the early writings on Enlightenment in the monograph of Zwi Batscha in 1977.\footnote{ Karl Leonhard Reinhold, \textit{Versuch einer neuen Theorie des menschlichen Vorstellungsverm\"{o}gens} (Darmstadt: Wissenschaftliche Buchgesellschaft, 1963); Photomechanical reprint of the first edition (Prague and Jena: Widtmann and Mauke, 1789); Zwi Batscha, \textit{Karl Leonhard Reinhold. Schriften zur Religionskritik und Aufkl\"{a}rung 1782{-}1784 }(Bremen and Wolfenb\"{u}ttel: Jacobi Verlag, 1977).} These were followed by a partial reprint of the so{-}called \textit{Fundamentschrift }in 1978\footnote{ Karl Leonhard Reinhold, \textit{\"{U}ber das Fundament des philosophischen Wissens/\"{U}ber die M\"{o}glichkeit der Philosophie als strenge Wissenschaft}, ed. Wolfgang Schrader (Hamburg: Meiner, 1978). This book contains a reprint of the main part of \textit{Ueber das Fundament des philosophischen Wissens, nebst einigen Erl\"{a}uterungen \"{u}ber die Theorie des Vorstellungsverm\"{o}gens} (Jena: Mauke, 1791), in which Reinhold also included two reactions to reviews of the \textit{Versuch}. It further contains a reprint of one of the essays in \textit{Beytr\"{a}ge I}: `Ueber die M\"{o}glichkeit der Philosophie als strenge Wissenschaft.' } and the first volume of the scholarly edition of Reinhold's correspondence in 1983, of which the second volume was published only recently.\footnote{ Faustino Fabianelli, Eberhard Heller, Kurt Hiller, Reinhard Lauth, Ives Radrizzani, Wolfgang Schrader, \textit{Karl Leonard Reinhold Korrespondenzausgabe}. First volume (1773{-}1788) 1983; second volume (1788{-}1790) 2008. } The fundamental bibliography of Reinhold's published works by Alexander von Sch\"{o}nborn has made a crucial contribution to unlocking the corpus of Reinhold texts. A further important impetus behind Reinhold scholarship has been Martin Bondeli's monograph on the problems of deduction in Reinhold's works from the \textit{Versuch }up to 1803.\footnote{ Martin Bondeli, \textit{Das Anfangsproblem bei Karl Leonhard Reinhold. Eine systematische und entwicklungsgeschichtliche Untersuchung zur Philosophie Reinholds in der Zeit von 1789 bis 1803} (Frankfurt am Main: Klostermann, 1995).} More recently, Faustino Fabbianelli has edited the two volumes of \textit{Beytr\"{a}ge zur Berichtigung bisheriger Mi\ss{}verst\"{a}ndnisse der Philosophen}, written by Reinhold in reaction to criticism regarding the \textit{Versuch}.\footnote{ Karl Leonhard Reinhold, \textit{Beitr\"{a}ge zur Berichtigung bisheriger Missverst\"{a}ndnisse der Philosophen. Erster Band, das Fundament der Elementarphilosophie betreffend}, ed. Faustino Fabbianelli (Hamburg: Meiner, 2003); Karl Leonhard Reinhold, \textit{Beitr\"{a}ge zur Berichtigung bisheriger Missverst\"{a}ndnisse der Philosophen. Zweiter Band, die Fundamente des philosophischen Wissens, der Metaphysik, Moral, moralischen Religion und Geschmackslehre betreffend}, ed. Faustino Fabbianelli (Hamburg: Meiner, 2004).} Another significant contribution to the availability of Reinhold's works has been made by the English translation of the `Briefe \"{u}ber die Kantische Philosophie'. Apart from making these crucial articles available in English, this translation notes the most significant additions and changes in the first volume of the \textit{Briefe} with respect to the original articles.\footnote{ Karl Leonhard Reinhold, \textit{Letters on the Kantian Philosophy}, ed. Karl Ameriks, transl. James Hebbeler (Cambridge: CUP, 2005). } Currently, Reinhold's \textit{Gesammelte Schriften} are being published; the new editions of the volumes of the \textit{Briefe \"{u}ber die Kantische Philosophie} by Martin Bondeli recently appeared as the first results of this project.\footnote{ Karl Leonhard Reinhold, \textit{Briefe \"{u}ber die Kantische Philosophie.} \textit{Erster} \textit{Band}, ed. Martin Bondeli (Basel: Schwabe, 2007), Volume 2/1 of Karl Leonhard Reinhold, \textit{Gesammelte Schriften}; Karl Leonhard Reinhold, \textit{Briefe \"{u}ber die Kantische Philosophie.} \textit{Zweiter Band}, ed. Martin Bondeli (Basel: Schwabe, 2008), Volume 2/2 of Karl Leonhard Reinhold, \textit{Gesammelte Schriften}.} A modern edition of the \textit{Versuch, }by Ernst{-}Otto Onnasch, is forthcoming,\footnote{ Karl Leonhard Reinhold, \textit{Versuch einer neuen Theorie des menschlichen Vorstellungsverm\"{o}gens}, ed. Ernst{-}Otto Onnasch (Hamburg: Meiner, forthcoming).} as well as an English translation.\footnote{ This translation is being made by Professor Tim Mehigan, Otago University, New Zealand. } Of some of Reinhold's works, his \textit{Fundamentschrift},\footnote{ Reinhold, `The Foundation of Philosophical Knowledge' in \textit{Between Kant and Hegel: Texts in the Development of Post{-}Kantian Idealism}, ed. George di Giovanni and H. S. Harris (New York: State University of New York, 1985; revised edition: Indianpolis: Hackett, 2000), 51{-}103. } his\textit{ Verhandlungen \"{u}ber die Grundbegriffe und Grunds\"{a}tze der Moralit\"{a}t }(1798)\footnote{ Reinhold, `The Fundamental Concepts and Principles of Ethics,' in Sabine Roehr, \textit{A Primer on German Enlightenment} (Columbia: University of Missouri Press, 1995), 157{-}251. } and his `Gedanken \"{u}ber Aufkl\"{a}rung',\footnote{ Reinhold, `Thoughts on Enlightenment,' in \textit{What is Enlightenment? Eighteenth{-}Century Answers and Twentieth{-}Century Questions}, ed. James Schmidt (Berkeley and Los Angeles: University of California Press, 1996), 65{-}77.} parts have been translated into English. Reinhold's philsosophy has also been studied in France and Italy, with the result that (partial) translations into the French and Italian have also been published in the last couple of decades.\footnote{ An integral translation in Italian of the \textit{Versuch }has been published by Faustino Fabbianelli. Karl Leonhard Reinhold, \textit{Saggio di una nova teoria della facolt\`{a} umana della rappresentazione}, ed. Faustino Fabbianelli (Florence: Le Lettere, 2006). For an overview of earlier modern translations of Reinhold's works, see the list of `\"{U}bersetzungen' in \textit{Beytr\"{a}ge I}, Fabbianelli ed., LII. }

 The editions mentioned have not only contributed to the wider availability of Reinhold's works, they have also made significant contributions to Reinhold scholarship by means of introductions and annotations. Parallel to these scholarly works directly related to Reinhold texts, there has also been an important increase in monographs and articles dedicated to Reinhold's philosophy and its place within the philosophical field of late eighteenth{-}century Germany. The initial impetus behind the surge of scholarly activity concerning Reinhold has come from the circle of Fichte scholars, most notably Reinhard Lauth, who initiated wider interest in Reinhold by editing a collection of articles and by starting the annotated edition of Reinhold's correspondence.\footnote{\label{footnote:_Ref233188211} Reinhard Lauth ed., \textit{Philosophie aus einem Prinzip. Karl Leonhard Reinhold. Sieben Beitr\"{a}ge nebst einem Briefekatalog aus Anla\ss{} seines 150. Todestages} (Bonn: Bouvier Verlag Herbert Grundmann, 1974).} The Fichtean background of the scholars first interested in Reinhold has had important consequences for the focus of the research being undertaken initially. The phase of Reinhold's philosophical development called `\textit{Elementarphilosophie}' (between 1789 and 1794) received primary attention. Usually this phase is considered to comprise Reinhold's \textit{Versuch}, the two volumes of \textit{Beytr\"{a}ge} and the \textit{Fundamentschrift}, with a focus on the latter work, which is considered ``der b\"{u}ndigste Ausdruck und die sicherste Form der Elementarphilosophie.''\footnote{ Wolfgang Schrader in his introduction to the partial reprint of \textit{\"{U}ber das Fundament des philosophischen Wissens}, VII, citing Kuno Fischer. } In these works Reinhold aims to establish philosophy upon a (single) fundamental first principle. It is in this effort and its failure that his main relevance for philosophy is seen.\footnote{ With respect to this, Michael Gerten has suggested that one of the reasons of Reinhold's impopularity is the identification of foundational philosophy with fundamentalism and totalitarianism. Gerten, `Begehren, Vernunft und freier Wille: Systematische Stellung und Ansatz der praktischen Philosophie bei K. L. Reinhold,' 154. } In the introduction to the aforementioned collection of essays, Manfred Zahn identifies two reasons for which Reinhold's philosophy is of interest. The first reason is that Reinhold played a decisive role in the dissemination of Kant's philosophy. Secondly, his most important accomplishment is ``da\ss{} er als erster auch den entscheidenden Mangel in der systematischen Durchf\"{u}hrung des transzendentalphilosophischen Programms durch den Kritizismus gesehen hat und durch seinen eigenen \textit{Versuch einer neuen Theorie des menschlichen Vorstellungsverm\"{o}gens }zu beseitigen suchte.''\footnote{ Manfred Zahn, `Einleitung' in Lauth (ed.), \textit{Philosophie aus einem Prinzip}, 1.} This view on the place of Reinhold in the philosophical field of his day implies that the way in which Reinhold prefigured Fichte is the most promising line of inquiry. Indeed, most of the early research on Reinhold relates to his position between Kant and Fichte, identifying the points of dissatisfaction with Kant's philosophy and the points that became more prominent in Fichte's.\footnote{ Apart from the collection mentioned in footnote \ref{footnote:_Ref233188211}, see, for instance, Klemmt, \textit{Karl Leonhard Reinholds Elementarphilosophie}; Selling, \textit{Studien zur Geschichte der Transzendentalphilosophie. I: Karl Leonhard Reinholds Elementarphilosophie}. } In another, more recent, line of inquiry regarding Reinhold's philosophy followed by Manfred Frank, Reinhold's position between Kant and the German \textit{Fr\"{u}hromantik }plays a central role. This approach also focuses on Reinhold's \textit{Elementarphilosophie}, and especially on the consequences drawn by his students from its collapse in 1792.\footnote{ Manfred Frank, \textit{Unendliche Ann\"{a}herung: Die Anf\"{a}nge der philosophischen Fr\"{u}hromantik} (Frankfurt am Main: Suhrkamp, 1997). } Frank's research is based on the results of the massive research program of Dieter Henrich, focusing on the historical circumstances in Jena and the \textit{T\"{u}binger Stift} that brought about this collapse.\footnote{ Dieter Henrich, \textit{Grundlegung aus dem Ich: Untersuchunge zur Vorgeschichte des Idealismus. T\"{u}bingen{-}Jena (1790{-}1794) }(Frankfurt am Main: Suhrkamp, 2004). This work can be regarded as the general synthesis of prolonged investigations, spanning more than two decades. } 

 Without questioning the legitimacy and value of these approaches, it must be noted that they are significantly limited in two ways. First, they focus on a relatively short period of Reinhold's philosophical activity, roughly the years around 1790. Reinhold, however, was a prolific writer both before and after this time. Since his philosophical development was not limited to a single system of thought, focusing on a specific period always carries the risk of ignoring important factors that lie outside this period. Both the earlier and later works contain important clues as to Reinhold's ideas of what philosophy is and the direction he believes it should take.\footnote{ Alexander von Sch\"{o}nborn stresses the relevance of Reinhold's later work in his `Reinholds letztes Werk: Anfang im Ende.'} Although the phase of \textit{Elementarphilosophie} may be the period of Reinhold's greatest influence on his contemporaries,\footnote{ For the influence of Reinhold's \textit{Elementarphilosophie} in Scandinavia, see Vesa Oittinen, `Ein nordischer Bewu\ss{}tseinsphilosoph: ``Reinholdianische'' Themen by G. I. Hartman.' For the influence on his moral philosophy on Schiller, see Sabine Roehr, `Zum Einflu\ss{} K. L. Reinholds auf Schillers Kant{-}Rezeption.'} we need to go beyond this period to understand how Reinhold came to be such a leading figure. Secondly, the interest in the phase of \textit{Elementarphilosophie} has implied a thematic emphasis on system, foundation and theoretical philosophy in general,\footnote{ See, for instance, Bondeli, \textit{Das Anfangsproblem bei Karl Leonhard Reinhold}.} at the expense of religion, morality and practical philosophy in general. A premise of scholarship centering on the \textit{Elementarphilosophie }is that the main and most interesting point of Reinhold's philosophy is ``da\ss{} Kants Philosophie (\ldots ) so lange unbegr\"{u}ndet bleiben mu\ss{}, wie ihre S\"{a}tze sich nicht als Konsequenz eines in sich evidenten obersten Grundsatzes rechtfertigen lassen.''\footnote{ Frank, \textit{Unendliche Ann\"{a}herung}, 152. } The focus is thus on Reinhold's attempt to deduce the totality of philosophy from one first principle. Again, this attempt and its failure may be one of the more influential aspects of Reinhold's philosophy, yet this perspective overlooks the circumstance that his authority did not derive from his thoughts on the foundations of the Kantian philosophy, but rather on his presentation of the Kantian results in a practical context.

With the general increase in scholarly work on Reinhold, however, the interest in especially Reinhold's early philosophical development and in his practical philosophy has also increased. It is no longer the case that Reinhold's philosophy is mainly studied with Kant or Fichte in mind. With regard to the importance of Reinhold's early philosophical development, and especially his Masonic engagement, the work of Gerhard Fuchs deserves to be mentioned for showing that this engagement has significantly impacted Reinhold's philosophical writings, throughout his life.\footnote{ Gerhard Fuchs, \textit{Karl Leonhard Reinhold {--} Illuminat und Philosoph. Eine Studie \"{u}ber den Zusammenhang seines Engagements als Freimaurer und Illuminat mit seinem Leben und philosophischen Wirken} (Frankfurt am Main: Lang, 1994).} Alessandro Lazzari has shown that the focus on theoretical philosophy in the literature does not do justice to Reinhold's idea of the task of philosophy, which is of a practical nature.\footnote{ Alessandro Lazzari, \textit{Das Eine, was der Menschheit Noth ist. Einheit und Freiheit in der Philosophie Karl Leonhard Reinholds (1789{-}1792)} (Stuttgart{-}Bad Cannstatt: Frommann{-}Holzboog, 2004).For the relevance of Reinholds practical philosophy, see further Lazzari, `K. L. Reinholds Behandlung der Freiheitsthematik zwischen 1789 und 1792' and Gerten, `Begehren, Vernunft und freier Wille.'} With these developments, important gaps in our understanding of Reinhold are being filled in. The broadening of the perspective on Reinhold is also apparent from the collections of papers of several \textit{Reinhold{-}Tagungen}.\footnote{ \textit{Die Philosophie Karl Leonhard Reinholds}, ed. Martin Bondeli and Wolfgang Schrader (Amsterdam: Rodopi, 2003); \textit{Philosophie ohne Beynamen: System, Freiheit und Geschichte im Denken Karl Leonhard Reinholds}, ed. Martin Bondeli and Alessandro Lazzari (Basel: Schwabe, 2004); \textit{K.L. Reinhold: Alle soglie dell'Idealismo}. Archivio di Filosofia, volume 73 (2005), issue 1{-}3; \textit{Karl Leonhard Reinhold and the Enlightenment}, ed. George di Giovanni (Dordrecht: Springer, forthcoming).}

The present study relates to this filling of the gaps in our understanding of Reinhold's philosophy. Rather than looking at the details of the \textit{Elementarphilosophie}, this study concentrates on Reinhold's way toward it. By focusing on the development of Reinhold's understanding of the Kantianizing term `practical reason' his initial and later understanding of the Kantian philosophy will be investigated. This leads to a picture that is very different from the one mentioned above, according to which Reinhold's main merit lies in pointing out some of the weaknesses of the Kantian system and trying to repair them. In the present study, Reinhold's theory of the faculty of representation appears as a way to present the unique and important discoveries made by the critical philosophy, rather than as an attempt to remedy its faulty foundation. Although my findings will have a bearing on how we are to view Reinhold's \textit{Elementarphilosophie}, the foundational aspect of his work in the early 1790s falls beyond the scope of this study. Rather, this study provides a background against which the need to do foundational work arose. This perspective is a result of bringing the strong continuity between Reinhold's pre{-}Kantian writings and his efforts on behalf of the Kantian philosophy to the fore. Reinhold's frame of reference is not so much Kant, but rather his Enlightenment ideal of what philosophy should do for mankind. In order to substantiate this claim, the present study starts with investigating Reinhold's thoughts on philosophy before he knew about Kant, after which it shows how this determined his reception of the Kantian philosophy and how it gave rise to the need to come up with a solid foundation of it. This means that although the focus of the study is on aspects of Reinhold's philosophy in its `Kantian phase', the relationship of Reinhold's presentation of the Kantian philosophy to the actual letter and spirit of the master himself is a sideline in the investigation. Rather than showing Reinhold as criticizing Kant, he is presented as creatively employing the Kantian philosophy within his own Enlightenment framework. 

This intention, to show Reinhold's authentic development as a spokesman on behalf of the Kantian philosophy, has implications for the methodology employed in this study. Of course, if we really want to know what Reinhold's intentions and motivations were in dealing with the Kantian project in the way he did, we would have to ask him and hope for a thruthful answer. As in all history of philosophy, or history in general, this is unfortunately not possible. However, studying Reinhold's texts from different periods, before and after he became acquainted with the Kantian philosophy, reveals that there are some persistent continuities in Reinhold's thought. These enable us to understand why, initially, Reinhold read Kant with a strong interest in morality and religion and why, later, he chose to present the new philosophy from the point of view of a theory of the faculty of representation. This approach requires a methodological focus on the sources, Reinhold's writings, rather than on previous interpretations of these writings. Rather than providing an explicit discussion with the available literature, the present study aims to provide an interpretation that is guided by what Reinhold himself thought interesting (as evidenced by his writings). I believe this interpretation will be a useful addition to existing interpretations. 

One of the premises of the present study is that in order to understand how Reinhold came to interpret the Kantian project in the way he did, it is imperative to understand his background, the tools with which he worked his way into an understanding of the Kantian philosophy. \textbf{Chapter 1} presents the first, factual, building blocks that are needed for a proper understanding of Reinhold's background. It sketches his life from his education in Vienna up to his move to Kiel in 1794. During this twenty{-}year journey Reinhold stopped over in Leipzig, Weimar and Jena. Yet his journey to Kiel did not only take him to different geographical places. He also travelled from Catholicism to Protestantism, from being a monk and priest to being an \textit{Aufkl\"{a}rer}, a `Kantian' and an \textit{Elementarphilosoph}, from writing poetry, reviews and Masonic speeches to publishing a 600{-}page monograph and many articles dealing with the results of the Kantian philosophy. This remarkable journey will serve as the backdrop against which the argument of the present study will unfold. Apart from introducing Reinhold as an historical figure, the first chapter also briefly introduces his works in so far as they originate in the period discussed. The aim of introducing Reinhold and his works in this manner is to present a preliminary account of his interests and activities during the period investigated. 

The remainder of the study is organized chronologically. Thus, \textbf{Chapter 2} investigates Reinhold's earliest works, that is, the works predating his acquaintance with the Kantian philosophy. Reinhold produced a great variety of writings during this period, ranging from reviews and Masonic speeches to articles and small books. A common theme of these writings is their interest in \textit{Aufkl\"{a}rung}, Enlightenment.\footnote{ According to Sch\"{o}nborn Reinhold can be called the ``Idealtypus des Aufkl\"{a}rers.'' Sch\"{o}nborn,\textit{ Reinhold. Eine annotierte Bibliographie}, 10. } Since Reinhold was among the very first authors in the German speaking world who addressed the question `What is Enlightenment?', it is clear that he had definite ideas about the nature and tasks of Enlightenment; ideas that were at least partly shaped by the context of his education in the Vienna of Empress Maria Theresa and her son Joseph.\footnote{ For Reinhold's debt to the `reformist Catholicism' of Maria Theresa and Joseph II, see Batscha, \textit{Karl Leonhard Reinhold}.} The first step towards understanding Reinhold's engagement for Enlightenment consists of an investigation of his efforts on behalf of clarifying the concept of Enlightenment, that is, answering the question `What is Enlightenment?' The chapter then proceeds thematically, presenting Reinhold's thoughts on, first, the role of history, and, secondly, the importance of involving both mind and heart, both reason and the senses in order to achieve Enlightenment. It is clear that Reinhold's thoughts on both these subjects are related to his involvement in Freemasonry and the Order of Illuminati. Further, this chapter addresses the consequences of Reinhold's views on Enlightenment. It will be shown how the two themes mentioned above are related to his criticism of blind, superstitious forms of religion and how he thinks these problems may be remedied. In the end the chapter will provide an evaluation of the compatibility of his sometimes radical statements on the nature of religion with his statements elsewhere that true Enlightenment will not harm true religion. In this manner the chapter presents a multi{-}faceted account of Reinhold's views on Enlightenment, which considers different kinds of writings and approaches the subject from different angles. 

For a proper understanding of Reinhold's interpretation of the Kantian project, however, it is not enough to concentrate on his philosophical background; there is also an important historical question that needs to be investigated, as to how and why Reinhold started studying the Kantian philosophy. This will be dealt with in \textbf{Chapter 3}. Unfortunately, this question does not admit of a straightforward answer, since we lack conclusive sources regarding the historical facts concerning Reinhold's `conversion to Kantianism'. By presenting several plausible stories of how the Viennese refugee became interested in Kant, the chapter seeks to show that, although we do not have all the facts we would like to have, we can still put together a reasonably plausible picture if we compare the different perspectives. One of these perspectives focuses on the indirect influence of Johann Gottfried Herder (1744{-}1803), with whom Reinhold got acquainted through his benefactor in Weimar, Christoph Martin Wieland (1733{-}1813). Reinhold reviewed Herder's \textit{Ideen}; so did Kant. Reinhold then sought to defend Herder against Kant's objections, yet the way in which Kant responded may well have interested Reinhold for his criticism of metaphysics. Another perspective focuses on the letter Reinhold produced about a year and a half after the skirmish with Kant, in November 1786, to Christian Gottlob von Voigt, in which he presents his plans with regard to the Kantian philosophy. The story emerging from that letter strongly suggests that Reinhold had political reasons to be involved in the Kantian philosophy. These reasons are wholly absent from the third perspective, the account that Reinhold gives of his conversion to Kant in the Preface to his \textit{Versuch}. This stylized account presents the story of an intellectual and religious crisis, which was solved by the Kantian philosophy. Evaluating these three stories, the chapter aims to bring them together by taking a closer look at the first letter Reinhold wrote to Kant, and noting the influence of the reviewing activities of Christian Gottfried Sch\"{u}tz (1747{-}1832) in the \textit{Allgemeine Literatur{-}Zeitung.}

The consequences of the way Reinhold most likely made himself familiar with the Kantian philosophy will become clear in \textbf{Chapter 4}, which discusses the first products of his pen dedicated to the Kantian philosophy, the `Briefe \"{u}ber die Kantische Philosophie', appearing in \textit{Der Teutsche Merkur} in 1786 and 1787. The chapter presents these articles in their historical context, which is dominated by the pantheism controversy about correspondence between Moses Mendelssohn (1729{-}1786) and Friedrich Heinrich Jacobi (1743{-}1819) regarding the alleged Spinozism of the late Gotthold Ephraim Lessing (1729{-}1781). The analysis of the contents of Reinhold's articles in this chapter shows that they are strongly related to his pre{-}Kantian writings. Having thus presented the \textit{Merkur}{-}`Briefe' in their historical context and as a continuation of Reinhold's previous interests, the chapter turns to his employment of Kantianizing terminology in them. Although his use of the terms `practical reason' (\textit{praktische Vernunft} ) and `pure sensibility' (\textit{reine Sinnlichkeit}) suggests a strong influence of the Kantian philosophy, Reinhold in fact employs these terms in a way that profoundly differs from anything found in the writings of Kant up to that point. Strikingly, he employs both terms to call attention to the feature of the Kantian philosophy that is most relevant to him: the necessity of combining or unifying reason and sensibility, or the spontaneous and receptive capacities in the human cognitive faculty. `Practical reason' is presented as the way in which reason and sensibility come together to provide a rational ground for the crucial religious conviction that there is a God. `Pure sensibility' plays a more indirect role in a similar argument regarding the rational ground for the conviction that the human soul will have a continued existence after the body has died. The term represents the unity of the receptive and spontaneous capacities needed for human cognition. Reinhold's use of this Kantianizing terminology goes hand in hand with the historical way of arguing that was the hallmark of his writings on Enlightenment. 

Reinhold's reception of Kant took on a new shape in his substantial monograph \textit{Versuch einer neuen Theorie des menschlichen Vorstellungsverm\"{o}gens}, published in 1789, when he was already teaching at the University of Jena. \textbf{Chapter 5} will present the work in its historical context and analyze its structure. It will be shown that Reinhold's efforts to provide a theory of the faculty of representation as a premise for the Kantian theory of cognition are strongly related to the project of the \textit{Merkur}{-}`Briefe', in which the most relevant feature of the Kantian philosophy was deemed to be its potential for understanding the receptive and spontaneous cognitive capacities as producing cognition as a result of their unified activities. The theory of the faculty of representation as presented in the Second Book of the \textit{Versuch} states that in any representation both the receptivity and the spontaneity of the capacity for representation must be involved. Since any cognition is a form of representation, this also holds good for all forms of cognition. According to Reinhold, presenting a theory of the faculty of representation as a premise for the Kantian theory of cognition should help people to understand the Kantian philosophy better. His efforts to increase the acceptance of the Kantian philosophy are no longer presented with the Kantianizing terms `practical reason' and `pure sensibility' but in Reinhold's own terms of a theory of a faculty of representation. By 1789, Reinhold's previous use of especially `practical reason' could no longer seriously serve the function it had in the \textit{Merkur}{-}`Briefe', namely that of signaling the potential of the Kantian philosophy for forging a connection between rationality and sensibility. In the meantime Kant had published his second \textit{Critique}, which stressed the purity of practical reason and warned against the influence of sensibility. The fact that, at the very end of his \textit{Versuch,} Reinhold does discuss `practical reason' is, as will be argued in this chapter, strongly related to the appearance of the second \textit{Critique}, and especially to its review by August Wilhelm Rehberg (1757{-}1836). Reinhold's answer to this review in his \textit{Versuch} shows that his theory of the faculty of representation causes tension when it is also to be the basis of practical philosophy. The premise that the basic actions of the human mind always involve some level of both receptivity and spontaneity does not combine well with Reinhold's need, fuelled by Rehberg's review of the second \textit{Critique}, to establish absolute freedom, or pure spontaneity. 

The subsequent development of Reinhold's practical philosophy, culminating in the publication of the second volume of \textit{Briefe \"{u}ber die Kantische Philosophie, }shows that the ad hoc solution presented regarding practical reason and the freedom of the will in the \textit{Versuch} was not satisfactory. In the years following the publication of the \textit{Versuch }Reinhold not only defended and revised his \textit{Elementarphilosophie }but also developed a theory of the freedom of the will in several articles. This development of Reinhold's position on the freedom of the will is presented and analyzed in \textbf{Chapter 6\textit{.}} It starts with the position in the \textit{Versuch}, through the articles that appeared in \textit{Der Neue Teutsche Merkur} up to the second volume of the \textit{Briefe \"{u}ber die Kantische Philosophie}, which was published in 1792 and contains, apart from new material, adapted versions of the earlier articles. In his efforts to establish a free will, Reinhold comes to distinguish sharply between practical reason and the will. By 1792 it is clear that the mediation between reason and sensibility is no longer situated in `practical reason' (as it was in the \textit{Merkur}{-}`Briefe') but rather in the freedom of the will, which is situated in the capacity to choose between following the moral law or the natural law of desire, whenever it is confronted with the question whether to satisfy a particular desire. This understanding of the kind of freedom that is needed for morality, the freedom to choose the morally right or the morally wrong way, was Reinhold's pre{-}Kantian starting{-}point on freedom.

