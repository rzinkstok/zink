
\chapter{Reinhold's Way to Kant}


In the first and second chapters we have established that Reinhold's early philosophical interests were closely connected to the life he led as a monk, and then as a Freemason and an \textit{Aufkl\"{a}rer}. Although in this early period he did not put forward a systematic philosophical theory of his own, his reviews, articles and speeches are by no means a random collection of opinions. They show a coherence that indicates that he had definite ideas about man and society and especially the role of history and religion. This means that when he became acquainted with the Kantian philosophy, his understanding of it was inevitably influenced by his previous philosophical concerns. Before presenting the evidence of the importance of that influence in the following chapters, the present chapter aims at establishing the circumstances under which Reinhold came into touch with Kant's philosophy and his initial reaction to it. In order to evaluate properly Reinhold's reception of Kant it is important to understand what his motivation for studying the Kantian philosophy was in the first place. 

After Reinhold's first introduction to the Kantian philosophy in the summer of 1785,\footnote{\label{footnote:_Ref230831075} By means of Sch\"{u}tz's review of Schultz's \textit{Erl\"{a}uterungen}, which explicitly considered Kant's first \textit{Critique} and \textit{Prolegomena} as well. The review provides an overview of the structure of the first \textit{Critique} and pays special attention to Kant's \textit{Stufenleiter der Vorstellungen} and the moral argument for the existence of God\textit{.} It appeared in several parts in the \textit{ALZ} of July 12 (nr. 162), July 14 (nr. 164), July 30 (nrs. 178 and 179 and supplement) 1785. It is included in Landau ed., \textit{Rezensionen zur Kantischen Philosophie. 1781{-}1787}, 147{-}82. Reinhold indeed acknowledges the importance of this review for his first understanding of the Kantian project. Cf. \textit{RK}, 1:271{-}273, Letter 66, October 12, 1787, to Kant. For the possibly crucial role of Schultz's \textit{Erl\"{a}uterungen}, cf. Onnasch, introduction to \textit{Versuch} [LII{-}LIII].} it took roughly a year before he publicly endorsed this novel philosophy in his `Briefe \"{u}ber die Kantische Philosophie', the publication of which started in August 1786. During this year, he studied Kant, although this left no obvious traces in his works or his correspondence of that time. Considering his activities in this period, one may even wonder where he found the time and energy required to study such a demanding work as Kant's first \textit{Critique}.\footnote{ Cf. Onnasch, introduction to \textit{Versuch} [LVIII{-}LX]. } Reinhold did not only marry Sophie Wieland, he also worked hard to make a living, translating the French \textit{Biblioth\`{e}que universelle de dames} (\textit{Allgemeine Damenbibliothek}), and contributing to both \textit{Der Teutsche Merkur} and the \textit{Journal f\"{u}r Freymaurer}.\footnote{ `Ehrenrettung der Reformation gegen zwey Kapitel in des k.k. Hofraths und Archivars Hrn. M.I. Schmidts Geschichte des Teutschen,' \textit{TM}, February, 1786, March, and April; `Ueber die Mysterien der alten Hebr\"{a}er', \textit{JF} 1786 I; `Revision des Buches: Enth\"{u}llung des Systemes der Weltb\"{u}rger{-}Republik,' \textit{TM}, May 1786; `Skizze einer Theogonie des blinden Glaubens,' \textit{TM}, June, 1786; `Auszug einiger neuern Thatsachen aus H. Nikolais Untersuchung der Beschuldigungen des Hrrn. Prof. Garve u.s.w.,' \textit{TM}, June, 1786. } Although we lack direct sources that can tell us in detail how Reinhold managed to juggle those balls and at the same time find the time and peace of mind required for studying Kant, there are other ways to shed light on the process by which he got acquainted with Kant's philosophy. First, there is the salient fact that Reinhold, before studying Kant, found himself attacking him in defense of Herder's \textit{Ideen zur Philosophie der Geschichte der Menschheit }(1784), which Kant had reviewed in the \textit{ALZ}.\footnote{ Kant, review of \textit{Ideen zur Philosophie der Geschichte der Menschheit}, by Herder, \textit{ALZ}, 1785, January 6, (nr. 4 and supplement), 17{-}20 and 21{-}22; \textit{AA} 8:45{-}55. Asked by Sch\"{u}tz to review Herder's book in July 1784, Kant's review was only published in January 1785 in the \textit{ALZ. }Cf. \textit{AA} 10:394 (Sch\"{u}tz to Kant, July 10, 1784); \textit{AA} 10:396 (Sch\"{u}tz to Kant, August 23, 1784).} As the minor controversy between Reinhold and Kant involves wider philosophical issues, such as the relation between metaphysics and empirical inquiry, it can help us gain some insight into Reinhold's views on metaphysics on the eve of his introduction to Kant's philosophy proper. Moreover, it provides a clue as to how the Kantian philosophy may have come to interest Reinhold. Secondly, there is Reinhold's plan for two volumes on the Kantian philosophy as expressed in a draft of a letter to Christian Gottlob von Voigt, written when only the first two of his `Briefe \"{u}ber die Kantische Philosophie' had been published. This letter can help us assess Reinhold's original view on Kant's philosophy and the circumstances under which it was developed. Finally, Reinhold himself provided an elaborate account of his turn of Kant in the Preface of his \textit{Versuch}. Although this account, serving a specific purpose, may lack objectivity, it does tell us how Reinhold liked to see and present himself with regard to Kant's philosophy. 

 The current chapter is devoted to carefully assessing the sources presented above regarding Reinhold's earliest reception of the philosophy of Kant. The first section will focus upon the Herder reviews. It will be established that Reinhold's praise of Herder is closely related to his earlier thoughts on Enlightenment and that this precisely opens up the possibility for a later sympathetic reading of Kant, as Reinhold's criticism turns out to be misplaced. The second section of this chapter is devoted to Reinhold's letter to Voigt. Based on his own information we will be able to determine how much Reinhold knew and understood of Kant at the time, when he was working on the `Briefe' as well. Moreover, since Reinhold presents the outlines of a plan for a work on Kant's philosophy, it will be possible to compare this to his actual works regarding the Kantian project, and thus put these in a framework of his original intentions regarding to Kant's philosophy. The third section critically presents Reinhold's own \textit{post factum }statements regarding his Kantian turn, as presented in the \textit{Versuch}. The fourth and final section will evaluate these different accounts and bring them together. 


\section{The controversy surrounding Herder's Ideen: a story of a polemical exchange}


Reinhold's work for \textit{Der Teutsche Merkur} started almost immediately upon his arrival in Weimar in May 1784 with a review\footnote{ Reinhold, review of \textit{Ideen zur Philosophie der Geschichte der Menschheit}, by Herder, \textit{Anzeiger TM}, June 1784, LXXXI{-}LXXXIX. } of Herder's \textit{Ideen zur Philosophie der Geschichte der Menschheit}, which had just come from the press.\footnote{ On May 4, 1784, Herder sent Johann Wilhelm Ludwig Gleim the first copy of the first part of his \textit{Ideen}. Dobbek and Arnold ed., \textit{Johan Gottfried Herder. Briefe. F\"{u}nfter Band}, Letter 25, May 4, 1784, to Sophie Dorothea and Johann Wilhelm Ludwig Gleim.} In this book Herder aims at establishing the place of man in the world by adducing empirical material enabling a classification of creatures, and showing how geological, climatological and biological facts are related to the special plan God/Nature has had for mankind. Reinhold's positive opinion is not only obvious from his actual judgment, but also from his introduction and his treatment of quotations from the work. Two striking remarks deserve our special attention. First, Reinhold describes Herder as someone who sees ``all in one and one in all,''\footnote{ Reinhold, review of \textit{Ideen}, \textit{Anzeiger TM}, June 1784, LXXXVI.} which may indicate that Reinhold felt attracted to the Spinozist thoughts of Herder. As pointed out in the previous chapter Reinhold entertained Spinozist ideas at some point. Herder's philosophy of nature, according to which God and Naure were almost identified, would have attracted him.\footnote{ Cf. Onnasch, introduction to \textit{Versuch }[XLII, XLIV]; Bondeli, `Von Herder zu Kant, zwischen Kant und Herder, mit Herder gegen Kant,' 206.} In his `Ueber den Hang' he called upon his fellow Masons to study nature in order to discover the destination of man implied in it.\footnote{ Cf. section 4 of the previous Chapter; Reinhold, `Ueber den Hang,' 137.} He may also have been attracted to Herder's humanism and his way of linking human behavior in society to human nature and human reason. As we have seen in section 3.2 of the previous chapter, Reinhold had attempted something similar in his `M\"{o}nchthum und Maurerey', written before Herder's \textit{Ideen}.\footnote{ On April 19, 1784, Born tells Reinhold that this article cannot be published in a public journal. \textit{RK} 1:19, Letter 2, from Born. The article was published in the \textit{Journal f\"{u}r Freymaurer} later that year. For an evaluation of this article, cf. sections 1 and 3.2 of the second Chapter. } We have also seen that his description of Enlightenment may be related to Herder's thoughts on turning \textit{Vernunftf\"{a}higkeit }into actual \textit{Vernunft}.\footnote{ Cf. section 1 of the previous Chapter; Sauer, \textit{\"{O}sterreichische Philosophie}, 100, n. 54.} Reinhold's positive opinion is thus associated with his own philosophical pre{-}occupations at the time. The analysis of Reinhold's counter review below (1.2) will show that Reinhold was also interested in Herder's emphasis on empirical evidence, which he contrasts to the Kantian a priori approach to history. A second remark worth mentioning in Reinhold's original review is a critical note regarding the status of the `organic forces' (\textit{organische Kr\"{a}fte}) introduced by Herder. Reinhold hopes that Herder will explain this in more detail in the parts to follow.\footnote{ `Ideen, von Herder,' \textit{Anzeiger TM}, June 1784, LXXXIX.} This point is especially interesting since Kant was overtly critical of these forces and their status within Herder's theory as well, presenting this as one of his main objections to Herder's project. Unlike Kant, Reinhold was clearly attracted to the content of that project, although he did express some reservation with regard to the clarity of its method. 

Owing to Kant's negative review of the \textit{Ideen}, the work became controversial. According to Herder, Kant's campaign against him also included Kant's own history of philosophy, as published in his `Idee zu einer allgemeinen Geschichte in weltb\"{u}rgerlicher Absicht' in the \textit{Berlinische Monatsschrift }of November 1784.\footnote{ Kant, `Idee zu einer allgemeinen Geschichte in weltb\"{u}rgerlicher Absicht,' \textit{BM}, November, 1784, 385{-}411, \textit{AA} 8:17{-}31.} Herder reacted by supporting the counter{-}review by Reinhold, to which Kant reacted in turn. The current section discusses the dispute between Reinhold and Kant regarding Herder's \textit{Ideen zur Philosophie der Geschichte der Menschheit} as a means of determining Reinhold's views on metaphysics and philosophical method on the eve of his introduction to the Kantian philosophy. First, Kant's criticism of Herder's philosophy of history will be assessed, followed by Reinhold's defense of Herder in his counter review against Kant. Since Kant's reply to the counter review ended the dispute in a way that may explain why Reinhold would all of a sudden be interested in the Kantian project, it will be briefly discussed as well. 


\subsection{Kant versus Herder}


Although Kant's criticism of Herder is expressed most directly and clearly in his review, it will be worthwhile to take a brief look at his `Idee zu einer allgemeinen Geschichte' as well, if only because Herder felt the two pieces were part of one Kantian campaign against him.\footnote{ \textit{Herder. Briefe}, Letter 91, Herder to Christian Gottlob Voigt, January 11 1785..} Kant himself, however, does not mention Herder's book at all in the article and explicitly claims that a notice in the \textit{Gothaische gelehrte Zeitungen} of February 1784 occasioned it.\footnote{ `Verm\"{o}ge einer Nachricht\ldots ', \textit{Gothaische gelehrte Zeitungen}, zw\"{o}lftes St\"{u}ck den elften Februar, 1784, 95. Cf. Landau ed., \textit{Rezensionen}, 64.} Nevertheless, Herder's book may have inspired Kant to go public with his vision on human history, without being prepared to acknowledge this openly.\footnote{ Cf. Bondeli, `Von Herder zu Kant,' 210.} The similarity of the titles and Kant's ignoring Herder's efforts are conspicuous in this respect. 

The article opens with a definition of history as the account of human action, which, on a large scale, shows regularities.\footnote{ Cf. \textit{AA} 8:17. } Kant thinks that human behavior is to be understood as following natural laws and serving a hidden goal, as if there were a plan of nature. However, it is impossible to write a systematic (\textit{planm\"{a}\ss{}ig}) history, because human actions are neither actually planned on a large scale nor purely instinctive, like animal behavior (Kant, `Idee', \textit{AA} 8:17). Therefore, the philosopher should try to discover a guideline (\textit{Leitfaden}) to understanding history, so that others will be able to use it to actually write a history of mankind (\textit{AA} 8:18). Immediately, the crucial difference between Kant's and Herder's ideas on history stands out. Whereas Herder attempts to derive the special status of man from nature in general, Kant instead assumes this special status and aims at figuring out the fitting plan behind it. Kant and Herder thus move in opposite directions. At the end of the article Kant returns to his proposed method for the study of history guided by an idea of the plan of nature and warns against claiming knowledge of the actual elements of this plan, which again can be taken as a criticism of Herder's method, which is directed at obtaining knowledge of the plan of nature (cf. \textit{AA} 8:29). According to Kant the idea of a plan should only be taken as a \textit{Leitfaden} to seek the plan in history. \footnote{ In more general Kantian terms, the idea of a plan of human history has to be understood as an idea in the regulative sense, that is, as an ideal that serves to guide research by providing an ideal endpoint, as we know from the Appendix to the Transcendental Dialectics, titled `On the Regulative Use of the Ideas of Pure Reason' in the first \textit{Critique} (A 644/B 672). } Instead of replacing the empirical study of history, philosophical history, with an idea as a \textit{Leitfaden}, will assist it.

 Whereas Kant's own perspective on history seems to make a statement by completely ignoring Herder's attempts in the same field, his review of the first part of the \textit{Ideen} makes abundantly clear that he and Herder differ when it comes to the question of the relation between the a priori and the empirical in the study of human history. Kant opens his critical discussion with what he understands to be Herder's aim of this first part of the \textit{Ideen}. 

While avoiding all metaphysical investigations, the spiritual nature of the human soul, its persistence and progressions in perfection are to be proven from the analogy to natural formation of matter, mainly in its organization.\footnote{ \textit{AA} 8:52; translation by Allen W. Wood, cited from \textit{The Cambridge Edition of the Works of Immanuel Kant: Anthropology, History and Education}, 130{-}131. In the following the pagination of this translation will follow the reference to the Academy Edition in brackets. } 

Kant is disturbed by the way Herder proceeds to achieve this aim. Herder needs to ``assume'' spiritual forces and an ``invisible realm of creation'' (\textit{AA} 8:52), which, naturally, does not agree with Kant's demands for proper science, since these forces and this realm escape experience. Apart from his reservations with respect to `assuming' non{-}material forces and an invisible realm of creation, Kant finds Herder's idea of a gradual progress among the different species shocking. Although Kant is aware of the striking similarities between species, he is much more comfortable with the thought that these are simply due to the great abundance of species, rather than to an intimate relation between them, since the assumption of a ``single original species'' leads to ``monstrous'' ideas, from which reason will recoil (\textit{AA} 8:54 [132]). The nature of these tremendous ideas is not exactly clear, nor is the way in which they will ``wreak great devastation among the accepted concepts'' (\textit{AA} 8:54; [133]). The objection focuses on the idea of the unity of organic forces and especially the use Herder makes of this idea, which is, according to Kant, beyond the capacity of human reason. Given Kant's own philosophical point of view, the objection against Herder's speculative use of ideas implies that Herder cannot achieve knowledge of the things he speaks of. Herder is guilty of the mistake introduced in `On the Regulative Use of the Ideas of Pure Reason', which is characterized by wanting to go ``beyond every given experience'' (\textit{KrV}, A 644{-}645/B 672{-}673; \textit{CPR}, 591). Although Herder claims that his account of the history of humanity is based on empirical evidence, his conception of organic forces leads him well outside the field of empirical knowledge. The idea of these organic forces is not merely a regulative ideal, guiding inquiry, but rather constitutive for the knowledge supposedly gained. From Kant's point of view, this use of ideas is illegitimate. Moreover, Kant has little patience with things that are principally beyond experience; they cannot be used to explain anything (cf. \textit{AA} 8:53). The method employed by Herder is not that of serious, responsible philosophy and can, according to Kant, at best be called dogmatic metaphysics (cf. \textit{AA} 8:53{-}54). This criticism must have been especially hard to swallow for Herder, who, in his earnest attempt to be scientific, claims that his philosophical history of humanity starts from empirical facts and nothing else. This claim, however, was not only contested by Kant, but also caused confusion for Reinhold, who, in all other respects, appeared to be perfectly pleased with the work. It is clear, then, that the method of Herder's attempts in the field of the philosophy of human history was novel and, to put it mildly, not entirely transparent to his contemporaries. 

Kant's strong wording of ``monstrous ideas'' that would ``wreak devastation'' among our concepts may also carry a more general reproach. He may be trying to associate Herder's ideas with the undesirable consequences of materialism or Spinozism by speaking strongly against those ideas of Herder that suggest a primacy of the material organization of nature and the unity of a universal power or species grounding the manifold of natural species. The most obvious suggestion of materialism can be found in Kant's discussion of the relation between Herder's organic forces and the human soul. According to Kant, Herder connects the ``thinking principle in the human being'' too strongly to the material organism, which clears the way for understanding the human soul not as a substance, but rather as the ``effect on matter of an invisible and universal nature that works within it and animates it'' (\textit{AA} 8:53; [132]). Kant's hesitation to attribute this position to Herder indicates that he believes it to be untenable. Indeed, it could be perceived as a version of Spinozism, according to which the only substance is nature, determining the lives and actions of everything including man. 

Kant's strategies in discrediting Herder's project of a philosophy of the history of man are intimately connected to the strategies of his own critical philosophy. Claiming to use only facts, or empirical data, Herder makes a grave mistake in assuming unperceivable organic forces. This point of Kant's criticism is linked to his own theory of knowledge, which does not allow knowledge of things that are outside the realm of possible experience. Further, also from his own critical perspective, Kant objects to Herder's illegitimate use of ideas, going beyond the proper limits of reason that Kant has set in his \textit{Critique of Pure Reason}. As we have gathered from his `Idee zu einer allgemeinen Geschichte', Kant is not averse to the project of a philosophical history; it is just that he believes that Herder has chosen the wrong method for the subject. We have seen that Kant thinks that the empirical method Herder claims to use is not at all empirical, as the latter introduces the idea of invisible forces. Herder's use of ideas, however, is not up to the Kantian standard, which triggers Kant's objection that Herder commits dogmatic metaphysics. Kant's alternative, as published in the \textit{Berlinische Monatsschrift}, is supposed to show the right way of dealing with ideas when studying history. Finally, Kant objects to Herder's insistence on the empirical, which carries the strong suggestion of materialism, regarded as unwanted by Kant. 


\subsection{Reinhold versus Kant}


Herder and his friends regarded Kant's article on history and the review of Herder's \textit{Ideen }as an unfair attack on his work, especially because the criticism came from his former teacher.\footnote{ Cf. \textit{Herder. Briefe}, Letter 96, February 14, 1785, to Hamann, and Letter 98, February 25, 1785, to Jacobi. Cf. also Sch\"{u}tz to Kant, February 18, 1785, \textit{AA} 10: 398{-}400. Herder had followed lectures by Kant in 1762{-}1764. See Irmscher ed., \textit{Immanuel Kant. Aus den Vorlesungen der Jahre 1762 bis 1764}.} In response to Kant's review and probably encouraged by Wieland and/or Herder, Reinhold produced his `Schreiben des Pfarrers zu \textasteriskcentered \textasteriskcentered \textasteriskcentered  an den H[erausgeber]. des T[eutschen]. M[erkurs]. Ueber eine Recension von Herders Ideen zur Philosophie der Menschheit', in which he seeks to discredit the anonymous reviewer (Kant).\footnote{ Reinhold, `Schreiben des Pfarrers zu \textasteriskcentered \textasteriskcentered \textasteriskcentered  an den H. des T. M. Ueber eine Recension von Herders Ideen zur Philosophie der Menschheit,' \textit{TM}, February, 1785, 148{-}174.} Upon seeing a draft Herder expressed his gratitude and suggested some changes.\footnote{ \textit{Herder. Briefe}, Letter 93, Herder to Wieland, End of January 1785. } As his instructions appear to have been followed (some almost literally), the article can be regarded as Herder's opinion on Kant as he was prepared to express in public.\footnote{ Cf. R\"{o}ttgers, `Die Kritik der reinen Vernunft und K.L. Reinhold,' 793.} The picture of the anonymous reviewer (Kant) emerging from Reinhold's article is that of an old{-}fashioned metaphysician who is neither willing nor able to think beyond his own system, and thus cannot properly judge Herder's effort in this novel field.\footnote{ Cf. Malter ed., \textit{Immanuel Kant in Rede und Gespr\"{a}ch}, nr. 301: Hamann to Herder, February 4, 1785. Making a rebuke of this kind corresponds to Herder's views on the method of literary criticism, which emphasised the importance of judging a work internally, that is, from the perspective of the author's intentions. Cf. Beiser, \textit{The Fate of Reason}, 141. } As the philosophical background of the reviewer is put forward only as a hypothesis, Reinhold's article does not reflect Herder's anger to its full extent and may be seen as a contribution towards reconciliation, a ``\textit{Vers\"{o}hnungsversuch}'' as Kurt R\"{o}ttgers has called it.\footnote{ R\"{o}ttgers, `Die Kritik der reinen Vernunft und K.L. Reinhold', 793. One problem of this interpretation may be that it is not clear why Herder would initiate any reconciliation with Kant. Attempts at reconciliation would be more likely to have come from the Kantian side, since Herder's anger about the review may have threatened the attempts at gaining influence at the University of Jena. Notwithstanding the unlikelihood of Herder initiating a kind of reconciliation, the changes he suggested show a reluctance to wage war with Kant over metaphysics.} 

 Reinhold starts his counter{-}review by describing the contemporary situation in metaphysics as preoccupied with analyzing ``traditional abstractions'' and keeping experience at a distance (`Schreiben des Pfarrers', 152). Of course, Reinhold had already made this point in his article on Enlightenment, as shown in the first section of the previous chapter. Metaphysics should not be rejected entirely, of course, but we need to be aware that it may not be the only form of true knowledge (cf. 154). After this introduction Reinhold uses his hypothesis that the reviewer is a metaphysician of the kind described above to argue that the reviewer cannot have understood Herder properly (cf. 157{-}158). He believes it is no wonder that the metaphysician and Herder differ when it comes to the meaning of `philosophy of history', since Herder's attempt arises from dissatisfaction with the traditional a priori view on history and seeks to fill the ``tremendous gap'' between experience and speculation (159). Reinhold points out that metaphysics is not the proper science for the subject matter of history. The actual course that history has taken cannot be established a priori; rather, the actual course of history is prior to historical speculations showing that humanity has necessarily become what it is. History needs facts (cf. 159{-}160). The primary value of the \textit{Ideen }should be seen precisely in its collection of facts for the sake of a future philosophy of history. Judging this work fairly, according to Reinhold, entails judging the facts, instead of ignoring them (cf. 161). The first conclusion he draws is that the reviewer cannot possibly appreciate Herder's work because of his preconception of philosophy, in which facts are irrelevant. He then focuses on the freedom of thinking that Kant, with mixed feelings, attributed to Herder, reading this hesitation as a sign of metaphysical orthodoxy (cf. 164). For Reinhold, there is no reason to be afraid of freethinking or its results. He clearly interprets Kant's suggestion of reason recoiling from certain ideas as a sign of clinging to old ideas. 

All in all, Reinhold's hypothesis that the reviewer is an old{-}fashioned metaphysician results in the judgment that the latter cannot fully appreciate the value of Herder's work, as he is too attached to his own system to understand Herder's plan, which threatens his own position and doctrine when it comes to the history of man. The hypothesis explains the negative tone of the review and at the same time shows Reinhold's ignorance of Kant and his inability to value Kant's criticism properly. As Kant criticized Herder from his own perspective, so Reinhold attacks Kant from his. This becomes clear when looking at the final passages of the article, in which Reinhold presents his own views on the proper relation between metaphysics and history. He is convinced 

da\ss{} wir ohne Metaphysik so wenig eine Philosophie, als ohne Erfahrung eine Geschichte haben w\"{u}rden. Philosophie im engsten und Geschichte im weitesten Verstande sind die Beiden Pole des gesammten menschlichen Wissens. (173{-}174) 

[that without metaphysics we would not have philosophy just as we would not have history without experience. Philosophy in the narrowest and history in the widest sense are the two poles of the totality of human knowledge.]

Reinhold clearly distinguishes the two sciences of philosophy and history with regard to the sources of their knowledge, metaphysics and experience, respectively. Although earlier he presented these sciences as opposites, because of these different sources, his final statements unambiguously state that we should not focus exclusively on either one. 

Ich verlange sie [philosophy and history] auch nur in so ferne als sie das [poles of the totality of human knowledge] sind, einander entgegen zu setzen. Nur mu\ss{} man nicht vergessen, da\ss{} die Gegenden die zwischen beyden liegen die ergiebigsten sind, und da\ss{} ein ewiges Reisen um den einen herum nie eine Reise um die Welt werden kann. (174)

[It is only in so far as philosophy and history are the poles of the totality of human knowledge that I wish to oppose them to one another. It must not be forgotten that the regions lying in between the poles are the most fertile and that continuously circling one of them will not constitute a trip around the world.]

This passage can be easily connected to Reinhold's ideas on philosophy and human nature as discussed in the previous chapter (section 3.2). In his early works Reinhold time and again stressed the importance of the connection between man's rational capacities and his sensibility. In his Masonic writings the necessary connection was strongly linked to the idea that thoughts can only be transformed into actions through sensibility, or the heart, as he calls it in the context of acting. Likewise, in his essay on Enlightenment, the connection of abstract concepts to sensibility was advocated because of its practical value. In his `Schreiben des Pfarrers', however, the practical angle is absent, and the focus is on philosophy (metaphysics/a priori knowledge of the Leibnizian variety) and history (empirical knowledge) as the poles of human knowledge. This development in Reinhold's thought confirms that his pre{-}Kantian views express a single coherent view of man and world, so that he can easily employ the rationality{-}sensibility distinction in the theoretical field of epistemology next to his earlier use in a more practical context. The new focus brings with it some new terminology, as Reinhold refers to rational, a priori cognition as philosophical and to empirical cognition as historical. Nevertheless, we find the same stress on the thought that there is no real opposition between man's empirical and rational capacities and on the thought that ideally the two capacities should work together in unison. In the context of a theory of knowledge this means that all human knowledge should ideally be a combination of the a priori and the empirical somewhere in between those poles. Formulated in this manner, it is not at all hard too see why Reinhold, once he became more closely acquainted with Kant's philosophy, came to be interested. After all, Kant also stresses that we can only have cognition when the a priori structure of the mind and the given manifold are combined, when sensibility and the understanding work together. 

 At the time, however, Reinhold found the balance between metaphysics and experience in Herder's work rather than in his image of Kant, whose criticism he tries to eliminate by simply portraying Kant as a relic of old{-}fashioned metaphysics. For Kant the issue was more complex, for on the one hand, he wanted history based on an idea, while on the other hand he set strict limits upon the legitimate use of ideas. Reinhold ignores such complications and presents a simple picture. Metaphysics, or a priori philosophy, is indeed limited, but experience can supplement philosophy here. This is especially true for the field of history. In the ideal situation, metaphysics and experience work together. Although his judgment of Kant is mistaken in the sense that Kant is not quite the old{-}fashioned metaphysician that Reinhold takes him to be, he is right in judging that Kant is criticizing Herder from his own philosophical point of view. In his reply, Kant clarifies this position. 


\subsection{Kant versus Reinhold}


Although a counter{-}review could have been disadvantageous to Kant, Reinhold's was not. Kant could not help but replying, since the reputation of being an old{-}fashioned metaphysician was not at all what he was looking for. In his reaction to the `Schreiben des Pfarrers', published in March 1785 in the \textit{ALZ},\footnote{ Kant, `Erinnerungen des Recensenten des Herderschen Ideen zu einer Philosophie der Geschichte der Menschheit (Nro. 4) und Beil. der Allg. Lit.{-}Zeit.) \"{u}ber ein im Februar des Teutschen Merkur gegen diese Recension gerichtetes Schreiben.' \textit{ALZ}, `Anhang zum M\"{a}rzmonat der Allgemeine Literatur{-} Zeitung,' March 1785. \textit{AA} 8:56{-}58.} on the invitation of Sch\"{u}tz, who had revealed the identity of the `Pfarrer' as well,\footnote{\textit{AA} 10:398{-}400 (Sch\"{u}tz to Kant, February 18, 1785).} Kant was in a position to restate his views on Herder's \textit{Ideen }in a less aggressive way, while dismissing the misdirected criticism. Since Reinhold's ignorance of Kant's actual philosophy was obvious, it was not difficult for him to do so. Almost immediately, Kant declared that he wholeheartedly agrees with Reinhold's negative picture of the metaphysician.

In his article, the pastor quarrels much with a metaphysician whom he has in mind, and who, as he imagines him, is wholly spoiled for all instruction through the paths of experience, or where they do not complete the matter, for inferences in accordance with the analogy of nature, and who wants everything to fit his last of fruitless scholastic abstractions. The reviewer can well tolerate this quarrel, since in this he is fully of one opinion with the pastor, and the review itself is the best proof of that. (\textit{AA} 8:56 [134])

Since the attack is directed at the wrong target, which Reinhold could have known, his article is refuted with a simple: read before you write. Kant is now in a position to restate his original arguments showing that Reinhold's criticism does not affect them. 

First, Kant agrees with Reinhold on the unfitness of metaphysics as a historical method. Instead of praising Herder for collecting biological and geological facts, he introduces an alternative basis for human history, namely human \textit{actions}.

But since he [i.e., the reviewer/ Kant] believes himself rather well acquainted with the materials for an anthropology, and likewise somewhat with the method of their use in attempting a history of humanity in the whole of its vocation, he is convinced that these materials may be sought neither in metaphysics nor in the cabinet of natural history specimens by comparing the skeleton of the human being with that of other species of animals; least of all does the latter lead to his vocation for another world; but that vocation can be found solely in his \textit{actions}, which reveal his character. (\textit{AA} 8:56 [134])

Thus, Kant appears to differ from Herder mainly with regard to the domain of human history, that is, the \textit{kind} of empirical facts deemed relevant. Understandably, as his reviewing activities were anonymous, Kant does not refer to his own essay on history, which, as we have seen, also takes human actions to be history's subject matter. There, it also became clear that Kant relates history exclusively to the development of humanity, already assuming its special position in nature, whereas Herder seeks to deduce this special position from facts of nature, which are not primarily related to human activity. Herder's preference for biological facts was understood by Kant as an attempt to reduce human properties like reason to material characteristics. Apart from presenting his ideas on the subject matter of human history, Kant warns against making the empirical the sole basis of our historical knowledge, as experience has its limits, being unable to warrant any necessities by itself (cf. \textit{AA} 8:57). Like the original review, Kant's reply to the \textit{Pfarrer }is clearly connected to his own philosophy. This also holds good for his reply to Reinhold's charge that recoiling from certain ideas is a sign of metaphysical orthodoxy. 

It is merely the \textit{horror vacui} of universal human reason, namely to \textit{recoil} where one runs up against an idea in which \textit{nothing at all} \textit{can be thought}, and in this regard the ontological codex might well serve as a canon for the theological, and indeed precisely for the sake of tolerance. (\textit{AA} 8:57 [135])

Again, Kant criticizes Herder from the perspective of his own philosophy, which does not allow crossing the borders of possible experience and employing ideas of objects that cannot be known by human reason. 

Kant defends himself by restating those points of his review that were both critical of Herder and of traditional metaphysics, thus proving Reinhold's hypothesis wrong. In this way, possibilities for a connection between Kant and Reinhold open up,\footnote{ R\"{o}ttgers considers Reinhold's ``Bildung einer Kantschule'' an effect of Kant's Herder review. R\"{o}ttgers, `Die Kritik der reinen Vernunft und K.L. Reinhold,' 791. } as Kant, like Reinhold, assumes a position in the middle between purely rational metaphysics and purely empirical history. Thus, the controversy concerning Herder's \textit{Ideen} may have played an interesting role in Reinhold's philosophical development. He was not yet familiar with Kantian philosophy, but the Kantian criticism of Herder may have aroused Reinhold's interest in Kant. Kant's position in this controversy was likely to attract Reinhold's attention, since the final outcome of the discussion and the position that Kant assumed in his reply to him may actually have been quite close to his own views. As we have seen above, he had already expressed an interest in finding a method that would involve both `poles of human knowledge'. Moreover, Kant's insistence on human actions as the basic facts of a history of humanity would have opened up possibilities for a connection to the practical interests predominant in his earlier writings. Thus in the story emphasizing the role of the Herder controversy in Reinhold's conversion to Kant coincidence and friendship play an important role. More importantly, it emerges that the first point of contact between Kant and Reinhold is their insistence that sensibility and reason must work together in the production of knowledge. 


\section{Voigt's request and Reinhold's plan: a story of political and economic interests}


Through Wieland's contacts and his own productivity for \textit{Der Teutsche Merkur}, of which his review and counter{-}review of Herder's \textit{Ideen} were only examples, Reinhold soon became a well{-}known author in Weimar. His articles for the \textit{Merkur} had shown his enthusiasm for Enlightenment and his role in the Herder controversy indicates that he was appreciated in the humanist circle to which Wieland and Herder belonged, along with other members of the Weimar court, such as Goethe. Moreover, if Ernst Reinhold is right about his father writing most of the reviews for the \textit{Anzeiger }(\textit{RLW}, 25), Reinhold would have proven himself capable of processing heaps of written material with remarkable speed. These circumstances may have been on Christoph Gottlob von Voigt's mind when he asked for Reinhold's opinion regarding Kant's philosophy. Apart from being an advisor to the Weimar court, Voigt was also curator of the University of Jena, where Kantianism was advancing.\footnote{ For this process and especially the role played by Christian Gottfried Sch\"{u}tz, cf. Schr\"{o}pfer, \textit{Kant's Weg in die \"{O}ffentlichkeit}. } This university was a joint venture of several principalities, one of which was Saxony{-}Weimar{-}Eisenach. The humanist court circles in Weimar, which included Herder, were none too pleased with Kantianism. They were, however, pleased with Reinhold, who might therefore play a mediating role between the Kantian university and the humanist Weimar court.\footnote{ Cf. R\"{o}ttgers, `Die Kritik der reinen Vernunft und K.L. Reinhold,' 792.} As Kurt R\"{o}ttgers indicates, the university may have had a definite interest in employing the young talented author as well, because the authorities depended on attracting students from abroad. Reinhold would indeed prove to be a major asset in this respect.\footnote{ Cf. R\"{o}ttgers, `Die Kritik der reinen Vernunft und K.L. Reinhold,' 799.}

 The situation sketched above shows that political factors may well have been of importance with regard to Reinhold's introduction to the Kantian philosophy. As shown in the previous section, he may have taken a personal interest in the novel philosophical system as well, after the public exchange with Kant, whose mild reaction to his counter{-}review of Herder's \textit{Ideen} may first have drawn his attention to Kant, which prompted Voigt to request Reinhold's views on the new philosophy. We do not know the precise timing of this request; hence it is not possible to establish with certainty whether Reinhold was already interested in Kant and Voigt reacted to that, or that the latter acted in a more pro{-}active way, by putting Reinhold on the Kantian track.\footnote{\label{footnote:_Ref232306650} R\"{o}ttgers suggests that the `Briefe' project was the direct result of Voigt's request which he therefore dates before the summer of 1786. This implies that Voigt's efforts were crucial in Reinhold's turn to Kant. Cf. R\"{o}ttgers, `Die Kritik der reinen Vernunft und K.L. Reinhold,' 794{-}795. Ernst{-}Otto Onnasch, on the other hand employs a later letter from Reinhold concerning the break in the `Briefe \"{u}ber die Kantische Philosophie' to claim that Voigt only contacted Reinhold after the publication of the first two `Briefe' published in August 1786, and that the request must therefore be dated after August. Cf. Onnasch, introduction to \textit{Versuch }[LXVIII]. Assuming that Voigt's specific request, to which the `Briefe' of 1787 were the response, was indeed dated between August and October 1786, this does not disqualify R\"{o}ttgers's suggestion that Reinhold and Voigt may have had contacts concerning the Kantian philosophy earlier, say, in the spring of that year. This means that both suggestions to put a date on Voigt's approaching Reinhold are not conclusive. } All we know is that he did, at some point, request Reinhold to report on the ``\textit{Einflu\ss{} der Kantischen Philosophie},'' for in the beginning of November 1786, Reinhold wrote to him in response to this request. Today, we have only a draft of this letter, which is not even complete.\footnote{ \textit{RK} 1:145{-}157, beginning of November 1786, Letter 35, to Christian Gottlob von Voigt. In the following, references to this letter will be in the text. } 

The letter has the form of a small treatise, with different sections covering different aspects of the influence of the Kantian philosophy and Reinhold's own plans regarding it. It is introduced by a brief note showing that Reinhold's answer to Voigt's request means more to him than just providing someone with information on the Kantian philosophy. Reinhold, whose first child has just been born, appears very eager to provide Voigt with the information he needs, for he is aware of the latter's intentions. 

Der Gedanke an die \textit{Absicht }Ihrer Frage st\"{a}rkt mich in meinem Kampfe gegen eine, bisher mir unbekannte, \"{A}ngstlichkeit, die mich bey jeder Wahrnehmung des Misverh\"{a}ltnisses zwischen meiner gegenw\"{a}rtigen h\"{a}uslichen und politischen Lage bef\"{a}llt. Aber auch jeder Anblick meines Kindes legt mir meine alte Neigung zur Philosophie n\"{a}her ans Herz, und n\"{o}thiget mich in ihr eines der n\"{a}chsten Mittel zu w\"{u}nschen und zu hoffen, wodurch ich den Staat, der mich bisher aufgenommen und gesch\"{u}tzt hat, auch zu jener th\"{a}tigen Unterst\"{u}tzung, die mir nun immer unentbehrlicher wird, bewegen k\"{o}nnte. (\textit{RK} 1:145)

[The thought of the \textit{purpose} of your question encourages me in the struggle against an anxiety that I did not yet know, coming over me whenever I perceive the discrepancy between my current private and political situation. Yet every look at my child also brings me closer to my old inclination towards philosophy and forces me to wish and hope to find in it the means by which I could move the state that has currently accepted and protected me to that active support that I can less and less do without.]

From this touching picture, it is clear that Reinhold has reason to hope that his report on the influence of the Kantian philosophy will help him provide for his family. Voigt may have suggested that he might be able to find Reinhold a place at the University of Jena, that is, of course, upon the condition that Reinhold, who possessed no official academic qualifications, proved himself worthy of such a job.\footnote{ Cf. R\"{o}ttgers, `Die Kritik der reinen Vernunft und K.L. Reinhold,' 794{-}795.} Both the letter and the `Briefe \"{u}ber die Kantische Philosophie', resumed in January 1787, could be such a qualification. 

 In a move comparable to his `more precise determination' of the concept `Enlightenment' in his `Gedanken \"{u}ber Aufkl\"{a}rung',\footnote{ Cf. Chapter 2, section 1. } Reinhold stresses that we must first determine the proper nature of the Kantian philosophy, or its ``characteristic concept'' (146). However, being a completely new science, the Critical project is hard to define. Simply giving a ``description'' of its most important results will not do either, for these cannot be proven without citing the whole \textit{Critique} itself (146{-}147). The \textit{Critique}, however, has been misunderstood by a number of important reviewers (cf. 148). In listing several reviews and remarks regarding Kant by well{-}known authors of the day, Reinhold shows off his factual knowledge of the most relevant judgments on the Kantian project.\footnote{ Reinhold refers to the following reviewers: Johan Georg Heinrich Feder (1740{-}1821), Christoph Meiners (1747{-}1810), Johann August Eberhard (1739{-}1809), Christian Garve (1742{-}1798), Christian Gottlieb Selle (1748{-}1800), Dietrich Tiedemann (1748{-}1803), Platner and Johann August Heinrich Ulrich (1746{-}1813). For details of their reviews, cf. \textit{RK} 1:149{-}151, nn. 7{-}22. } He claims that all these professional philosophers have misunderstood Kant, referring to Kant himself in the \textit{Prolegomena},\footnote{ Cf. \textit{AA} 4:377.} and to Eberhard's remark that the \textit{Critique }is ``obscure and incomprehensible'' \footnote{ Reinhold refers to \textit{Neue Litteratur{-}Briefe}, first volume (Berlin 1786), 21, where this claim is attributed to Eberhard. He does not refer directly to any publication of Eberhard himself. Cf. \textit{RK} 1:150, n. 12. } (148{-}150). Nevertheless, he, the inexperienced refugee, claims to have understood Kant. He argues that it is precisely because he was not already committed to an academic position and a philosophical system, that he was able to appreciate the new philosophy (cf. 153). After having read Kant's first \textit{Critique} ``three times'' he claims to have felt a calling to share the grounds for his ``most intimate conviction of the \textit{reality }and the \textit{incredible utility} of this science'' with philosophically interested readers (153). In order to carry out his plan, Reinhold suggests a two{-}phase strategy. First, the ``external grounds'' (\textit{\"{a}u\ss{}ere Gr\"{u}nde}) for accepting the Kantian philosophy are to be discussed, which, as the name suggests, cannot be found in the \textit{Critique }itself, but are related to the moral needs of the time (153). Reinhold also calls these grounds the ``usefulness'' (\textit{Nutzen}) of the Kantian philosophy. Discussing the external grounds of the Kantian philosophy will lead to a better appreciation of this philosophy since it will be presented as the answer to some of the most pressing contemporary questions regarding religion, morality and philosophy in general. The letter breaks off at the discussion of the ``internal grounds'' (\textit{innere Gr\"{u}nde}) or ``actuality'' (\textit{Realit\"{a}t}) of the Kantian philosophy. Although the precise meaning of the term `internal grounds' is unclear, Reinhold paraphrases it as ``organization of the Kantian system itself'' (153). In contrast to the external grounds, the internal grounds must be related to the Kantian philosophy itself. They provide reasons for accepting the Kantian system by appealing to the force of Kant's arguments, rather than by pointing out its use with regard to external circumstances. 

Such, in a nutshell, were Reinhold's plans regarding the Kantian philosophy in the late autumn of 1786. In order to gain insight into Reinhold's way to Kant, it is important to assess the extent to which this plan was carried out. At the time of writing Reinhold had already started his `Briefe \"{u}ber die Kantische Philosophie', as he mentions in the letter to Voigt. These articles appear to be intended to treat the external grounds or results of Kant's philosophy, as Reinhold would later, in the first volume of his \textit{Beytr\"{a}ge} (1790), also state himself. 

Der Plan meiner k\"{u}nftigen Arbeiten hat nun zwei Hauptteile, wovon mich der eine in den\textit{ Briefen \"{u}ber die Kantische Philosophie}, der andere in den \textit{Beitr\"{a}gen} besch\"{a}ftigen wird. In jenen werde ich die \textit{Folgen}, die Anwendbarkeit, und den Einflu\ss{}; in diesen die \textit{Gr\"{u}nde}, die Elemente, und eigentliche Prinzipien der \textit{Kritischen Philosophie} zu entwickeln suchen.\footnote{ Cf. Reinhold, \textit{Beytr\"{a}ge I} , IV [Fabbianelli, 4]. The way that things are stated in the letter to Voigt suggests that both projected volumes could be part of the `Briefe' project, although Reinhold does not commit to any title yet for either the whole project or the two projected volumes. } 

[The plan of my future work now has two main parts, one of which will concern me in the \textit{Briefe \"{u}ber die Kantische Philosophie}, the other in the \textit{Beytr\"{a}ge}. In the former I will seek to develop the \textit{effects}, applicability and the influence; in the latter the \textit{grounds}, elements and the proper principles of the \textit{Critical philosophy}.] 

Although the `Briefe' had already appeared in \textit{Der Teutsche Merkur} when Reinhold made the above announcement in 1790, the future tense refers to the fact that he was publishing an expanded volume of them in the same year, to be followed by a second volume of \textit{Briefe }in 1792. However, not only Reinhold's later comment on the `Briefe' in relation to his \textit{Beytr\"{a}ge} indicates that they were intended to treat the external grounds for the Kantian philosophy. The structure of the `Briefe' that appeared in \textit{Der Teutsche Merkur} is very similar to the structure of the external grounds as presented in the letter to Voigt. The themes introduced in the first seven points of \S  6 of the letter to Voigt appear in the same order in the `Briefe'. After an introduction, two problems regarding the philosophy of religion are discussed: What is the role of reason with regard to the conviction that God exists? and What is the role of reason with regard to the conviction that there is an afterlife? Regarding both problems it is argued that philosophy had not been able to solve them satisfactorily, and that Kant's investigation into the nature of reason has come to the rescue. Since the `Briefe' only cover the first seven of sixteen points listed in the letter to Voigt, it is clear that the original plan was not brought to completion. In the Preface of the\textit{ Versuch }Reinhold mentions that he had to stop working on the `Briefe' because of his academic pursuits.\footnote{ Reinhold, \textit{Versuch}, 58. } This makes sense, for the `Briefe' published in \textit{Der Teutsche Merkur} appear to be an unfinished whole. At the end of the seventh `Brief' Reinhold announces the continuation of the history of the idea of the soul. He does not keep this promise in the eighth `Brief', in which he elaborates instead on Greek philosophy. In the edition of 1790, however, the last (twelfth) `Brief' is an addition, which achieves a kind of conclusion by summarizing the influence of the misunderstood tenets of religion throughout history.\footnote{ `Zw\"{o}lfter Brief: Winke \"{u}ber den Einflu\ss{} der unentwickelten und mi\ss{}verstandenen Grundwahrheiten der Religion auf b\"{u}rgerliche und moralische Kultur,' \textit{Briefe I}, 332{-}371. } As part of this `Brief' had already been published separately as `Skizze einer Theogonie des blinden Glaubens' before Reinhold started to work on Kant (cf. section 4 of Chapter 2), its appearance here in the context of the Kantian philosophy shows the continuity in Reinhold's thought between his pre{-}Kantian and his Kantian work. In this respect I share Gerhard Fuchs's criticism of R\"{o}ttgers's contention that Reinhold's way of responding to Voigt's request is only explicable as a purely political move, just going for the perspective that would ensure the biggest impact. Although this may have played a role, there is no denying that the moral{-}religious perspective chosen is very close to Reinhold's pre{-}Kantian work. Thus, the thematic choices Reinhold made were by no means purely political.\footnote{ Cf. R\"{o}ttgers, `Die Kritik der reinen Vernunft und K.L. Reinhold,' 794, 796; Fuchs, \textit{Karl Leonhard Reinhold}, 63, n. 44. } 

In order to understand Reinhold's plan as expressed in the letter to Voigt better we must understand what he means with the phrase `result of the Kantian philosophy'. From the letter it is clear that he considers the influence of Kant and the results of the Kantian philosophy predominantly in relation to Enlightenment (cf. 146). He refers to the main results as the ``merits regarding our intensive Enlightenment,'' indicating that he believes that with Kant Enlightenment has gained depth or intensity (151). In contrast to his first opinion on Kant in the context of the Herder dispute, Reinhold no longer thinks of Kant as an old{-}fashioned metaphysician, but rather values his influence on Enlightenment especially for his ``scattering of the metaphysical delusions'' (146). The second section of the letter presents a long list of the ``main results,'' which can be summarized by the claim that Kant has provided a map of reason's capacities on the basis of which he has been able to decide important questions regarding religion and morality about which philosophers have been arguing for a long time, by pin{-}pointing the misunderstanding between them (cf. 146{-}147). Among other things, these questions concern the existence of God, the soul and human freedom. The main result of the Kantian philosophy is a complete inventory of what can and cannot be decided by reason with respect to these issues (cf. 147). No matter how impressive these results are, they cannot convince anyone of the Kantian philosophy on their own. Anyone wishing to present the results of the Kantian philosophy must ``present the \textit{How} next to the \textit{What} simultaneously'' (147). This implies that showing the results of the Kantian philosophy in this broad sense involves, at least, indicating the basis of those results, which suggests that they need to be connected to the internal grounds of the Kantian philosophy. 

In 1786, however, Reinhold intends to connect the results of the Kantian philosophy to the external grounds, presumably because he first needs to convince people of the relevance of Kant's project (cf. 153). This is the main objective in the `Briefe'. On the other hand, the First Book of the \textit{Versuch} presents the problems of philosophy that may count as external grounds for the Kantian philosophy, but the brute fact of their solution is not put forward explicitly as a `result' of this philosophy. Instead, Reinhold speaks of more specific results, such as the impossibility for speculative reason to prove the existence of God. He apparently uses the phrase `result of the Kantian philosophy' in two different senses. In a very broad sense the phrase is connected to the external grounds for the Kantian philosophy. In this sense, the result of the Kantian philosophy is `that it has solved the problems of the philosophy of religion'. In a narrower sense the phrase refers to the more specific results connected to the internal grounds of the Kantian philosophy, such as the exact limits of reason, which are the actual outcome of the investigation of reason. Thus, we might say that while the `Briefe' aimed to present the `result' of the Kantian philosophy in the first sense, the aim of the \textit{Versuch} as a whole is to present the results of the Kantian philosophy in the second sense.\footnote{ Alessandro Lazzari may have underestimated the connection to the internal grounds, expressed by Reinhold's claim that the results of the Kantian philosophy can only be presented properly with the `How' of these results. Thus, he exclusively interprets the `results of the Kantian philosophy' that the \textit{Versuch} aims to establish in terms of positive statements regarding the reality of the ideas of God, immortality and freedom. Lazzari, \textit{Das Eine, was der Menchheit Noth ist}, 45{-}46. This would of course have been Reinhold's ultimate aim. Yet the one thing conspicuously lacking from the \textit{Versuch} is a concluding chapter that shows how the results established in the Third Book contribute to the solution of the problems introduced in the First Book. Reinhold does not spell out how the results in the narrower sense will suffice to solve the problems posed in the First Book. For a further discussion of this issue see Chapter 5. } 

From Reinhold's letter to Voigt, written to communicate his plans with regard to the Kantian philosophy, we can now draw several conclusions regarding Reinhold's perspective on Kant at the time of his first public support for the latter's philosophy. 1) Reinhold had personal interests of a political and economic nature to work on Kant, as he hoped this work would secure him a position at the university, which in turn would enable him to provide for his family to some extent, and be less dependent on his father{-}in{-}law. 2) He understood the Kantian philosophy in a context of Enlightenment, which means that for him there was no discontinuity with his earlier philosophical pursuits. 3) He was aware at least of some of the available comments on Kant and of Kant's \textit{Prolegomena }and \textit{Grundlegung zur Metaphysik der Sitten}. 4) The plan that Reinhold presented to Voigt in 1786 continued to be the framework within which his subsequent works regarding Kant were placed. 5) These works deal with the results of the Kantian philosophy in various ways. On the one hand, Reinhold discussed the results in a general sense, as solutions to contemporary philosophical problems, mainly in the `Briefe'. On the other hand, he discussed them more specifically and investigated their internal grounds, bringing forward his own theory of the faculty of representation to provide these grounds, in the \textit{Versuch }and the \textit{Beytr\"{a}ge}. An account story of Reinhold's way to Kant on the basis of this letter is likely to stress the non{-}philosophical considerations Reinhold had for turning to Kant. It promised him some financial security and the possibility to be less dependent on the whims of his father{-}in{-}law, in short, the opportunity of a lifetime. Nevertheless, it becomes clear from the way Reinhold introduces Kant that his turn to Kant is not a life{-}changing event, but that his interest in the Kantian philosophy can be understood from the point of view of his previous philosophical interests in Enlightenment. 


\section{Reinhold's own statements regarding his interest in Kant: a story of religious and intellectual crisis}


In the first section of this chapter we have come upon a plausible story telling us that Reinhold's interest in Kant got aroused by the polemical exchange between them regarding Herder's \textit{Ideen}. According to that story Reinhold's first interest in Kant would have been the latter's way of dealing with the `poles of human knowledge', that is, philosophical/metaphysical and historical/empirical cognition. In the second section another story has been explored according to which Reinhold's first interest in Kant was triggered by political and economic motives. These stories need not be mutually exclusive, however. It may well have been the case that Reinhold's philosophical interest at the time coincided with an opportunity to benefit from them financially as well as socially.\footnote{ The accounts of Reinhold's turn to Kant presented by R\"{o}ttgers (`Die Kritik der reinen Vernunft und K.L. Reinhold') and also by Onnasch (introduction to \textit{Versuch}) assume an even stronger connection between these two stories; they link Voigt's interest in Reinhold's opinion on the Kantian philosophy to his contribution to the controversy over Herder's \textit{Ideen}. Given the delicate relations between the Weimar court and the University of Jena this may be plausible, but as long as there is no concrete historical evidence of such a link, it should be treated for what it is, a plausible speculation. } 

 One source, which is generally accepted more or less at face value, has not been considered yet: Reinhold's own statements regarding his allegiance to Kant. Reinhold elaborately justified his endorsement of the Kantian philosophy in a much cited passage from the Preface to the \textit{Versuch}, which will be the topic of this third section. As it is of vital importance to understand the background against which Reinhold presents his study of Kant's \textit{Critique}, the passage in which he introduces himself and his reasons for endorsing the Kantian philosophy will be cited here extensively. 

Er [Reinhold] glaubt, die Vorkenntnisse, die bey einer metaphysischen Lekt\"{u}re vorausgesetzt werden, besessen zu haben, als er 1785 dieses [Kant's] System zu studieren anfieng. Zehn Iahre hindurch war speculative Philosophie sein Hauptstudium gewesen, (\ldots ) Drey Iahre hindurch hatte er philosophische Vorlesungen nach dem leibnitzischen Systeme gehalten, und die Schriften des gro\ss{}en Stifters desselben, so wie seines w\"{u}rdigen Gegners \textit{Locke}, waren ihm keineswegs nur aus den neuern philosophischen Produkten unsrer Landesleute bekannt. (\textit{Versuch}, 51{-}52)

[He believes to have been in the possession of the knowledge required for reading metaphysics, when, in 1785, he started to study this [Kant's] system. For ten years, speculative philosophy had been his main subject of study (\ldots ). For three years, he had lectured according to the Leibnizian system and the works of its great founder, as well as those of his worthy opponent Locke were by no means only known to him from the recent philosophical products of our compatriots.] 

Reinhold opens his autobiographical paragraphs by introducing himself as someone who knows his philosophy. Obviously, this is important to him, as he is a relatively new appearance on the stage of professional philosophy, who has only received his academic title of \textit{Magister} by courtesy of the University of Jena.\footnote{ Cf. \textit{RK} 1:267{-}268, September 20, 1787, Letter 64, to the philosophy department of the University of Jena.} The statement quoted above is generally true, although it may be slightly exaggerated as regards speculative philosophy having been his main occupation for the past ten years. Reinhold did teach philosophy at the Barnabite college, presumably according to Wolff, and it is highly probable that he indeed knew original work of both Leibniz and Locke. Even if he had no formal academic education, this familiarity with the major philosophical schools dominating the scene in the 1780s would suffice to establish his credentials. However, he believes that it took more than a well{-}prepared mind to digest the Kantian philosophy. 

Zu dieser Vorbereitung des Kopfes (\ldots ), kam bey ihm noch ein dringendes Bed\"{u}rfni\ss{} hinzu, auf einem neuen Wege seinem Herzen die Ruhe wiederzufinden, die er auf dem Felde der Speculation verloren und auf allen ihm bekannt gewordenen Wegen vergebens gesucht hatte. (52)

[To this preparation of his mind (\ldots ) was added in his case a pressing need to find, on a new path, the peace for his heart that he had lost in the field of speculation and that he had sought in vain on all the paths known to him.] 

These remarks are structured in a way that is strongly reminiscent of Reinhold's pre{-}Kantian outlook on Enlightenment and the task of philosophy, as we have seen in the previous chapter. Apart from the philosophical preparation of his \textit{mind}, or rational capacities, Reinhold says his \textit{heart} provided the necessary motivation to study Kant. This heart of his was not satisfied at all by the other current philosophical systems, and he had high hopes that the Kantian philosophy would be able to help him out in this respect, for it was new and different from everything he had encountered before. He continues by presenting the story of how his heart came to lose its peace. 

Durch seine Erziehung war ihm \textit{Religion }nicht nur zur ersten, sondern gewissermassen zur einzigen Angelegenheit seiner fr\"{u}heren Lebensjahre gemacht. (52)

[Because of his education, \textit{religion} had become not only the first, but in a sense the only interest of his early life.]

Reinhold's earliest extant letter, to his father, cited in the first chapter, shows that he is not exaggerating in this case. There had indeed been a time, when he could have been described as a religious ascetic, chastising himself over his own negligence in the face of the imminent suppression of the Jesuit Order. His description of the following steps in his philosophical development is harder to judge, as there are fewer sources to corroborate his own account. 

Die philosophische Kritik des Geschmackes (\ldots ) verleitete ihn unvermerkt auf das Gebieth der speculativen Philosophie, und er hatte kaum einige Schritte auf derselben zur\"{u}ckgelegt, als er den Grund seiner bisherigen Gl\"{u}ckseligkeit mit Schrecken ersch\"{u}ttert f\"{u}hlte. Vergebens versuchte er sich hinter die Bollwerke der Ascetik zur\"{u}ckzuziehen und dem Kampfe mit den Zweifeln auszuweichen, die ihn drohend und einladend von allen Seiten best\"{u}rmten. Es war ihm unm\"{o}glich geworden, blind, wie vorher, zu glauben, und er sah sich bald genug gezwungen, sich auf Diskretion den Feinden seiner Ruhe zu \"{u}berlassen, die ihm mit Wucher wiederzugeben verhie\ss{}en, was sie ihm genommen hatten. (52{-}53)

[Philosophical critique of taste (\ldots ) seduced him unwittingly into the domain of speculative philosophy, and he had barely taken his first steps in this field as with fear he felt the ground of his past happiness shaking. In vain he tried to retreat behind the bulwarks of ascetics and to avoid the battle with his doubts, which from all sides came to him both inviting and threatening. It had become impossible for him to believe blindly, as before, and soon enough he found himself forced to deliver himself to the discretion of the enemies of his peace, who promise to return with high rates of interest what they had taken from him.] 

Although we lack concrete evidence that it was indeed through aesthetics that Reinhold's interests in philosophy were aroused, the fact that many of his Vienna friends had literary aspirations, as well as Reinhold himself, makes it rather plausible.\footnote{ Reinhold's first known publication was a poem, published in 1777. Cf. Sch\"{o}nborn, \textit{Karl Leonhard Reinhold}, 65. } Of course, he also studied philosophy in the Barnabite college. As we have seen in the first chapter, Paul Pepermann played an important role in encouraging Reinhold with respect to his philosophical studies. The crucial claim that Reinhold makes regarding his first steps upon the path of philosophy is that once he began his studies, his (blind) faith began to crumble and he experienced a serious philosophical crisis, which continued for several years. With regard to the loss of his blind faith, however, his writings show no evidence of such a crisis.\footnote{ Cf. Gliwitzky, `Carl Leonhard Reinholds erster Standpunktwechsel,' 19, 67.} 

Nun war Metaphysik die Hauptangelegenheit seines einsamen, sorge{-} und gesch\"{a}ftefreyen Lebens geworden. Allein am Ende einer vielj\"{a}hrigen Periode, w\"{a}hrend welcher er alle vier Hauptsysteme der Reihe nach angenommen und aufgegeben hatte, war er nur dar\"{u}ber mit sich selbst einig geworden, da\ss{} ihm die Metaphysik zwar mehr als einen Plan, sich bald mit seinem Kopfe, bald mit seinen Herzen abzufinden, aber keinen einzigen vorzulegen hatte, der die ernsthaften Forderungen von beyden zugleich zu befriedigen vermochte. (53)

[Now, metaphysics had become the main occupation of his solitary life, free of worries and business. Yet at the end of a period of many years, during which he accepted and rejected all of the four main systems one after the other, the only thing he was sure of was that metaphysics had more than one plan to offer that would allow him to now agree with his mind, then with his heart, but none that was able to satisfy the serious demands of both at the same time.]

In all of his early writings Reinhold is already firmly on the side of Enlightenment and against blind faith and superstition. None of them shows any clear signs of his subsequently accepting and abandoning the four main systems of metaphysics. Earlier in the Preface Reinhold had identified these four systems as ``spiritualism, materialism, dogmatic skepticism and supernaturalism'' (21). It is clear that in his youth Reinhold had favored supernaturalism, or religious orthodoxy. Given that the term `spiritualism' refers to the Leibnizian philosophy, according to which the basic substances in the world are of a spiritual nature, it is not far{-}fetched to assume that Reinhold had adopted spiritualism as well at some point. After all, the Leibnizian{-}Wolffian philosophy formed the basis of philosophical education in Vienna in those days.\footnote{ For details on the development of philosophy in Vienna in this period, see Sauer, \textit{\"{O}sterreichische Philosophie}, second chapter.} That Reinhold's ideals regarding Enlightenment and religion may have been more radical than is apparent from the writings that were intended for a wider audience has been shown in the previous chapter. His Masonic and Illuminatist ideology may have incorporated materialist elements. Thus Reinhold's own description of his philosophical development appears to be correct. We only lack evidence for the phase termed `dogmatic skepticism', but again, it would not be far{-}fetched at all to assume that Reinhold was familiar with the works of Hume, probably made available to him by Pepermann.\footnote{ Reinhold certainly took an interest in different forms of scepticism, as will be apparent from the discussions of the \textit{Versuch }and \textit{Briefe} \textit{II} in the fifth and sixth chapters of this study. He also wrote the introduction to the new German translation of Hume's \textit{Essay concerning Human Understanding}. Reinhold, `Ueber den philosophischen Skepticismus' in \textit{David Humes Untersuchung \"{u}ber den menschlichen Verstand} neu \"{u}bersetzt von M. W. G. Tennemann (Jena: Verlag der akademischen Buchhandlung, 1793), i{-}lii.} 

 Although there is evidence supporting the claim that Reinhold was indeed familiar with the four main systems of philosophy and knew them from personal study, there is no compelling evidence indicating that Reinhold indeed had adopted and rejected these systems in succession in a strong sense, except in the case of supernaturalism. His `accepting and rejecting' should be understood in terms of his having evaluated these systems, being aware of their strong points, but also of their limitations. As he did not venture on the field of metaphysics himself, there are no clear signs of his working within a systematic framework or abandoning one system for another. The only clear framework from which he worked, is, as shown in the previous chapter that of his Enlightenment engagement. His writings suggest that he is quite comfortable as an eclectic \textit{Aufkl\"{a}rer}, not committed to any particular system, but rather commenting on current affairs and developments in philosophy from his own point of view. Reinhold, as he appears from his pre{-}Kantian writings, has more in common with the \textit{Popularphilosophen} of his days than with the main systems he has identified.\footnote{ For Reinhold as \textit{Popularphilosoph}, see di Giovanni, `Die \textit{Verhandlungen \"{u}ber die Grundbegriffe und Grunds\"{a}tze der Moralit\"{a}t} von 1798.' The circumstance that Reinhold's thought shared features with the \textit{Popularphilosophie} does not mean, however, that he can be said to belong to this school or be directly influenced by \textit{Popularphilosophen}. Cf. Onnasch, introduction to \textit{Versuch}, [XXXVI]. } In the introduction to the \textit{Versuch}, however, he is overly critical of this type of philosophy, making abundantly clear that this is not what he would want to be associated with (cf. 23{-}24; 122{-}123; 133{-}140). Since Reinhold's writings show no signs of his adopting and rejecting these systems, his account of his philosophical development here may well have been designed to contribute to the goal of dissociating his pre{-}Kantian philosophical work from the works of the \textit{Popularphilosophen}. Instead, he presents his pre{-}Kantian development as a kind of quest for a way to satisfy the demands of his mind and his heart. 

Der peinliche Gem\"{u}thszustand, der bey ihm eine sehr nat\"{u}rliche Folge dieser Ueberzeugung war, und die Begierde, desselben es koste auch was es wolle los zu werden, waren die ersten und st\"{a}rksten Triebfedern des Eifers und der Anstrengung, womit er sich dem Studium der \textit{Kritik der reinen Vernunft }hingab, nachdem er an derselben unter andern auch den Versuch wahrzunehmen glaubte, die Erkenntni\ss{}gr\"{u}nde der Grundwahrheiten der Religion und der Moral von aller Metaphysik unabh\"{a}ngig zu machen. (53{-}54)

[The painful state of mind that for him naturally resulted from this conviction, and the desire to get rid of it, whatever it would take, were the first and strongest incentives of the industry and effort with which he started to study the \textit{Kritik der reinen Vernunft}, since he believed to perceive in it, among other things, the attempt to make the grounds of cognition of the fundamental truths of religion and morality independent of all metaphysics.]

Since Reinhold has argued earlier in the Preface that it would be nearly impossible for professional philosophers to understand the new philosophy presented by Kant (cf. 40{-}41), he can now use his own philosophical predicament and the lack of academic employment to make plausible that he did have the energy and the spare time to study and understand Kant, and the right frame of mind, of course. According to his claim here, the reason for starting his study of Kant was that he saw in it an attempt ``to make the grounds of cognition of the fundamental truths of religion and morality independent of all metaphysics.'' The reader is left to fill in how this would satisfy both his mind and heart, but it must be something along the following lines. Religion and morality satisfied his heart, but they did not satisfy his head, for their fundamental truths could not be philosophically justified. Several metaphysical systems had promised Reinhold such justification, but had failed to deliver the kind of foundation that would satisfy his heart. Such is the philosophical predicament that Reinhold claimed led him to study Kant's first \textit{Critique}, in which he saw a possible solution, that is, a non{-}metaphysical justification of the basics of religion and morality. He relates his efforts and their results. 

Bey der ersten \"{a}usserst aufmerksamen Durchlesung sah er nichts als einzelne schwache Lichtfunken aus einem Dunkel hervorschimmern, das sich kaum bei der \textit{f\"{u}nften }ganz verloren hatte. Ueber ein Iahr lang enthielt er sich fast aller andern Lekt\"{u}re, zeichnete sich die Haupts\"{a}tze des Werkes, die er verstanden zu haben glaubte so wohl, als die er wirklich nicht verstanden hatte, besonders auf, und verfertigte mehr als einen mi\ss{}lungenen Auszug des Ganzen. Alles, was er auf diese Weise Anfangs herausbrachte, waren Bruchst\"{u}cke, die ihm theils aus andern Systemen entlehnt, theils schlechterdings unvereinbar schienen. Allein so wie er rastlos fortfuhr, einerseits durch wiederholtes Lesen aus dem Werke selbst neuen Stoff auszuheben, andererseits aber das Ausgehobene aneinander zu r\"{u}cken: erg\"{a}nzten sich die Bruchst\"{u}cke allm\"{a}lich zu aneinander passenden Theilen, verschwanden Dunkelheiten, die ihm vorher un\"{u}berwindlich, und Ungereimtheiten, die ihm ganz entschieden deuchten, und am Ende stand das Ganze im vollen Lichte einer Evidenz vor ihm da, die ihn um so mehr \"{u}berraschte, je weniger er sie seinen vorigen Erfahrungen und Grunds\"{a}tzen zufolge in der speculativen Philosophie f\"{u}r m\"{o}glich gehalten hatte. (54{-}56)

[When he first read it highly attentively, he did not see anything but small singular sparks shimmering from a darkness that had barely resolved upon his \textit{fifth} reading. For over a year he abstained from almost all other reading, made notes of the main claims that he believed he had understood as well as of those that he had not understood at all and produced more than one failed outline of the whole. In the beginning he produced nothing but bits and pieces, which seemed to him partly taken from other systems, partly incompatible with one another. Yet in this way, carrying on without rest, on the one hand collecting new material from repeated reading, on the other hand combining the bits and pieces, the fragments came together as parts belonging together, the problems that had seemed unconquerable and the incompatibilities that had seemed decisive started to disappear, and in the end, the whole was there, in the full light of day, with an evidence that surprised him all the more, since, as a result of his previous experience and principles, he had deemed this impossible.]

The above passage creates a strong image of our lonesome philosophical hero, driven by the need to overcome his philosophical problems, studying feverishly, working very hard to make sense of it all. Even if reading Kant's first \textit{Critique} five times in little over a year must be considered humanly possible,\footnote{ Cf. Schultz, \textit{Erl\"{a}uterungen}, 8. Schultz was able to read the first \textit{Critique }and produce a decent book on it in about nine months. } in Reinhold's case it would have been a truly remarkable achievement. We already indicated above that he did not really have the leisure he claimed to have had. Marrying and moving out of his in{-}laws' house would have been a strain already, but his activities did include other reading and writing.\footnote{ Cf. Onnasch, introduction to \textit{Versuch }[LIX{-}LX] .} Reinhold wrote to Nicolai that he did not publish much in the \textit{Merkur} of 1785, as he worked on his \textit{Herzenserleichterung} and the \textit{Damenbibliothek}, the translation of which he took on ``des lieben Brodes willen.''\footnote{ \textit{RK} 1:183, January 26, 1787, Letter 41, to Nicolai.} In the beginning of the following year, he published an article against Schmidt's \textit{Geschichte der Teutschen}, so he must have read that as well.\footnote{ Reinhold, `Ehrenrettung der Reformation,' \textit{TM}, February, 1786, 116{-}142; March, 193{-}228; April, 42{-}80.} Although Reinhold does not actually claim that he read the first \textit{Critique} five times in just the first year, we saw earlier that he did state in his letter to Voigt that he read it three times.\footnote{ \textit{RK} 1:153, beginning of November 1786, Letter 35, to Voigt. The combination of these statements also implies that Reinhold was probably still trying to make sense of the \textit{Critique} at the time when he wrote the `Briefe' and replied to Voigt's request, since the obscurity of that work had ``barely disappeared'' upon his fifth reading of it. } Reinhold may be given the benefit of the doubt her, but his claim that he has abstained from almost all other reading is, to say the least, an exaggeration. Based on what we know about his personal life and other activities in this period, the best we can say is that it may be true that the \textit{Critique} was the main object of purposeful study; it is certainly not true that his attention was as undividedly devoted to it as he makes out.

From Reinhold's description of the way he proceeded with his studies, it is clear that making sense of Kant's \textit{Critique} was by no means easy. As he describes how he kept trying to make summaries and was confused by the combination of things he recognized and things he could not understand, his joy at finally getting some understanding of the new system is not hard to imagine. It is also clear that Reinhold's access to Kant was mediated by his philosophical experience up to that point.\footnote{ Cf. Onnasch, introduction to \textit{Versuch} [LXIII{-}LXIV].} Although he started out identifying the bits in Kant that he knew from elsewhere, the final result of his study was still, in his own words, ``unexpected.'' 

Er begn\"{u}gt sich also hier zu bekennen: da\ss{} ihm durch die neuerhaltenen Principien alle seine philosophischen Zweifel auf eine Kopf und Herz vollkommen befriedigende, f\"{u}r immer entscheidende, obwohl ganz unerwartete Weise beantwortet sind. (\ldots )

Seine eigenen Angelegenheiten waren ins reine gebracht, und es erwachte in ihm der Wunsch etwas beyzutragen, da\ss{} ein Gut, in dessen Besitze er sich so gl\"{u}cklich f\"{u}hlte, auch von andern erkannt und benutzt w\"{u}rde. Er suchte in seinen \textit{Briefen \"{u}ber die Kantische Philosophie }auf die Kritik der Vernunft vorz\"{u}glich durch diejenigen \textit{Resultate }aufmerksam zu machen, die sich aus derselben f\"{u}r die Grundwahrheiten der Religion und der Moral ergeben. (56{-}57)

[It is therefore with pleasure that here he affirms that for him all his philosophical doubts are answered by the newly received principles in a way that fully satisfies his mind and his heart; that is forever decisive, although it was totally unexpected. (\ldots ) His own affairs were now sorted and the desire awoke in him to make an effort so that the good he so happily possessed would also be known and used by others. In his `Briefe \"{u}ber die Kantische Philosophie' he tried to get the Critique of Reason noted mainly by means of those \textit{results} that follow from it for the fundamental truths of religion and morality.]

Reinhold's personal joy at finding a philosophical system that could satisfy both his heart and his mind, combined with the realization that this result had been produced in a way that was different from anything else he had seen up to that point form the ingredients of his desire to share the outcome of his study with a wider audience. Again, his self{-}presentation clearly serves the purpose of making plausible that he, the relative newcomer on the philosophical stage, has a right to speak on Kant. After all, the philosophical heavyweights of the day had already denounced the Critical philosophy as incomprehensible. The circumstantial evidence regarding his strong motivation to get to the core of it combined with the philosophical credentials presented earlier, and the fact that he was not yet part of the academic establishment with its vested interests, supports the image that Reinhold was the right man to understand the revolutionary potential of Kant's philosophy. The realization that due to his particular circumstances he was one of the few people who had access to this new way of justifying the truths of religion and morality is presented as his motivation to spread the news. The extraordinary circumstances surrounding the new philosophy called for an extraordinary way of communicating its merits to a wider audience.

Er hatte bald genug eingesehen, da\ss{} diese Resultate aus den neuen Principien nur f\"{u}r diejenigen streng bewiesen werden konnten, welche das kantische Werk selbst studiert und durchg\"{a}ngig verstanden h\"{a}tten. Da er nun dieses Studieren und Verstehen vielmehr erst zu bef\"{o}rdern w\"{u}nschte, als schon voraussetzen durfte, so blieb ihm nichts als der Versuch \"{u}brig, diese Resultate unabh\"{a}ngig von den kantischen Pr\"{a}missen aufzustellen, sie an bereits vorhandene Ueberzeugungen anzukn\"{u}pfen, ihren Zusammenhang mit den wesentlichsten wissenschaftlichen und moralischen Bed\"{u}rfnissen unsrer Zeit, ihren Einflu\ss{} auf die Beylegung alter und bisher unentschiedener Zwiste in der philosophischen Welt und ihre Uebereinstimmung mit dem was die gr\"{o}\ss{}ten philosophischen K\"{o}pfe \"{u}ber die grossen Probleme der speculativen Philosophie gedacht haben, sichtbar zu machen. (57{-}58)

[He had understood soon enough that these results from the new principles could only be strictly demonstrated to those who had studied and thoroughly understood the Kantian work themselves. Since he, however, wished to further this study, rather than assuming it, the only thing for him to do was to try and establish these results independently of the Kantian premises, to connect them to convictions already present, to show their relation to the most essential scientific and moral needs of our time, to make visible their influence on the settlement of old and as yet undecided disputes in the philosophical world and their agreement with what the greatest philosophical minds have though regarding the important problems of speculative philosophy.] 

As Reinhold has presented himself as one of the very few who have been able to understand Kant, selling the all{-}important results of his philosophy cannot proceed solely from the basis of that philosophy. The Kantian philosophy was not understood widely and deeply enough to sell itself. Therefore, publicly endorsing the new system meant introducing it independently of the contested \textit{Critique} itself. This, as we shall see in the following chapter, is indeed what Reinhold had done in the `Briefe'. He presented reasons for taking the Kantian philosophy seriously that were independent of and external to the \textit{Critique }itself; they were to be found in the historical situation of the philosophical debates of the day. 

 When valuating Reinhold's description of his conversion to Kantianism in the Preface to the \textit{Versuch}, it becomes clear that he does not present the whole story. The more mundane interests that must have played a role in his decision to write his `Briefe', as discussed previously are absent. In other respects, the account is not exactly accurate, as the historical circumstances render it highly implausible that he studied Kant with the intensity he suggests. In other words, in this Preface Reinhold does not aim to present the complete factual story of his introduction to and study of Kant. He is deliberately sketching a picture of himself that will both justify his endorsement of Kant and establish his credentials as a philosopher fit to discuss Kant's philosophy. By recounting his previous activity in the field of metaphysics, Reinhold seeks, first, to make plausible that he has been able to understand Kant and, secondly, to justify the (practical) point of view from which he chose to present the main results of his philosophy in the `Briefe'.


\section{Evaluation: weaving the stories together}


While the previous sections have each presented different reasons for Reinhold to go and study Kant, the time has now come to weave those different perspectives together to reach a coherent understanding of the process through which Reinhold came to Kant. First, let us briefly recapitulate the main perspectives that were discussed above. The account of the controversy surrounding Herder's \textit{Ideen} suggests that Reinhold came upon Kant almost by coincidence. Writing against Kant's review, Reinhold showed an interest in questions regarding scientific methodology and especially the relation between the empirical and the metaphysical `poles of human knowledge'. This theoretical interest can be traced back to the more practical Enlightenment context of his pre{-}Kantian works. As Kant responded to the criticism mildly and invitingly, Reinhold may have taken up studying his work because he was interested in his solution to this problem of scientific methodology. His allegiance to Herder would have been more of a personal than of a philosophical nature, even if Herder's Spinozism may have been attractive to him. Although Herder may not have liked his young friend's resolve to study Kant, their friendship did not hinder Reinhold's philosophical ambitions, nor, apparently, vice versa. 

 In contrast to this picture, Reinhold's letter to Voigt conveys the image of cold{-}blooded politics. It shows that the context of Reinhold's public endorsement of the Kantian philosophy was one in which he was invited to do so, with prospects for his financial situation and career. To be sure, Reinhold does express his enthusiasm for the practical results of the Kantian philosophy in the letter, but his plan also bears witness to a marketing strategy, not only for the Kantian philosophy, but also for his own person. He is clearly intent on appearing as someone who is up to date on Kant. Moreover, he seeks to present the one shortcoming for acquiring an academic position, his lack of official academic education, as an actual advantage for understanding the Kantian philosophy. From the letter accompanying the plan it is obvious that Reinhold expected to benefit (socially, financially) from sharing his opinion on the influence of the Kantian philosophy. 

 This picture again differs from the version that Reinhold presents in the Preface to his \textit{Versuch}, which portrays him as a young hero on a quest to find the holy grail of philosophy, a system that can satisfy both the demands of his mind and of his heart. It has already been shown that in some places at least his claims on the intensity with which he studied Kant in the year preceding his `Briefe \"{u}ber die Kantische Philosophie' are definitely exaggerated. On the one hand the presentation in the Preface of the \textit{Versuch} shares its aim with the letter to Voigt, in that it seeks to present Reinhold as the one thinker in the philosophical field who is in a position to explain Kant's achievements. Both sources turn his status as an academic outsider into an asset in this respect. On the other hand, however, these two sources do not agree at all on Reinhold's motivations for studying Kant. The story of the philosophical hero on an all{-}important quest appears to be at odds with the more mundane considerations presented in the letter to Voigt. Reinhold's quest was by no means undertaken for merely intellectual goods. 

 Since the sources are at odds, or at the very least only present partial pictures of how Reinhold came to be a propagator of the Kantian philosophy,\footnote{ A link has been claimed between the first and second perspectives discussed above, that is, between the controversy regarding Herder's \textit{Ideen} and Voigt's request that Reinhold write something on the Kantian philosophy. Cf. R\"{o}ttgers, `Die Kritik der reinen Vernunft und K.L. Reinhold,' 792{-}795; Onnasch, introduction to \textit{Versuch}, [LXVIII{-}LXIX]. However, since there is no concrete evidence that Voigt's motivation behind the request to Reinhold was related to the tension between the Jena University and the Weimar court, this must remain historical speculation. } our knowledge of his first interest and understanding of Kant is limited. Yet this knowledge is of the highest importance if we want to understand Reinhold's reception of the Critical philosophy. The relation between the different sources discussed thus far and the relative weight that needs to be attached to the separate pictures they convey can be elucidated by looking at another source of Reinhold's own hand, his first letter to the master himself, dated October 12, 1787. As we have seen that he is prone to practice `self{-}stylization',\footnote{ Cf. Onnasch, introduction to \textit{Versuch }[LXI]. R\"{o}ttgers speaks of \textit{Selbststilisierungen }with regard to Reinhold's first letter to Kant. Cf. R\"{o}ttgers, `Die Kritik der reinen Vernunft und K.L. Reinhold,' 794.} this source needs to be assessed with as much caution as the others. 

After revealing himself to Kant as the author of the `Schreiben des Pfarrers' and apologizing for the ``unphilosophische Philosophie'' of the Pfarrer, Reinhold thanks Kant for bringing about a wholesome revolution in his ``Gedankensystem.''\footnote{ \textit{RK}, 1:271, October 12, 1787, Letter 66, to Kant. } He describes his introduction to the Kantian system as follows. 

Der von \textit{Ihnen }entwickelte \textit{moralische Erkenntni\ss{}grund} der Grundwahrheiten der Religion, das einzige Morceau das mir aus dem ganzen in der Litteraturzeitung gelieferten Auszuge \textit{Ihres} Werkes verst\"{a}ndlich war, hat mich zuerst zum Studium der Kritik d. r. V. eingeladen. Ich ahndete, suchte und fand in derselben das kaum mehr f\"{u}r m\"{o}glich gehaltene Mittel, der unseeligen Alternative zwischen Aberglauben und Unglauben \"{u}berhoben zu seyn. (RK 1: 271{-}273)

[The \textit{moral ground of cognition} of the fundamental truths of religion, developed by \textit{you}, the only bit in the whole overview of \textit{your} work provided in the \textit{Litteraturzeitung} that I could understand, was the thing that invited me to the study of the \textit{Critique of Pure Reason}. I suspected, sought and found in it the means that I barely believed possible to rise above the unholy alternatives of superstition and non{-}belief.]

Like the Preface to the \textit{Versuch}, Reinhold's first letter to Kant presents the moral and religious perspective on the \textit{Critique} as his first and main point of interest. It was even, he admits, the only bit he understood at first, although it is not made clear whether it was the review alone that sparked his interest in the Kantian philosophy. Given the reference to the `moral ground of cognition of the fundamental truths of religion', Reinhold must refer to the review by Sch\"{u}tz of Schultz's \textit{Erl\"{a}uterungen}, which also takes Kant's \textit{Critique of Pure Reason} and \textit{Prolegomena} into account. \footnote{ See footnote \ref{footnote:_Ref230831075}. } The review, also by Sch\"{u}tz, of Kant's \textit{Groundwork of the Metaphysics of Morals}, does not refer to the moral argument at all, but it does present the Kantian philosophy as a beginning of a new era in philosophy.\footnote{ Sch\"{u}tz, review of \textit{Grundlegung zur Metaphysik der Sitten}, by Kant, \textit{ALZ}, April 7 (nr. 80) 1785. Included in Landau, \textit{Rezensionen}, 135{-}139. } In his book on the importance of Sch\"{u}tz for the dissemination of the Kantian philosophy, Horst Schr\"{o}pfer discusses these two reviews and some others that are certainly or almost certainly from his hand.\footnote{ Schr\"{o}pfer, \textit{Kant's Weg in die \"{O}ffentlichkeit}, 207{-}278.} Sch\"{u}tz not only sought to present Kant's works to his readers, but also referred to the Kantian philosophy in reviewing works by others. For our purposes it is relevant to point out that Sch\"{u}tz was very much attracted by the moral theology of Kant, and used the so{-}called pantheism controversy between Mendelssohn and Jacobi as an excellent opportunity to point out the Kantian solution to the problem.\footnote{ Thus Sch\"{u}tz reviewed Mendelssohns \textit{Morgenstunden} (\textit{ALZ}, 1786, January 2 (nr.1) and January 9 (nr. 7)) (Landau, \textit{Rezensionen}, 249{-}261) and quite possibly also Jacobi's \textit{Ueber die Lehre des Spinoza }(\textit{ALZ}, 1786, February 11 (nr. 36)) (Landau, \textit{Rezensionen}, 271{-}276) and Wizenmann's \textit{Die Resultaten der Jacobischen und Mendelssohnsen Philosophie} (\textit{ALZ}, May 26 (nr. 125) and May 27 (nr. 126) 1786) (Landau, \textit{Rezensionen}, 383{-}398). For argumentation regarding Sch\"{u}tz's authorship of the latter two reviews, see Schr\"{o}pfer, \textit{Kant's Weg in die \"{O}ffentlichkeit}, 228{-}229. } We shall return to this controversy and its role in the formation of Reinhold's `Briefe' in the following chapter. For now it will suffice to point out that the \textit{Allgemeine Literatur{-}Zeitung} in the period in which Reinhold started to study the Kantian philosophy offered several reviews (by Sch\"{u}tz) that stressed the relevance of the new philosophy for questions of morality and religion. These reviews are very likely to have had their influence upon Reinhold's interpretation and subsequent presentation of Kant.\footnote{ The circumstance that the \textit{ALZ} published these reviews in which Kant was presented in a light that appears to have influenced Reinhold's perspective on the Kantian philosophy, as we shall see in more detail in the following chapter, renders it plausible that Reinhold started out to study the Kantian philosophy out of personal curiosity, rather than from political motives. Onnasch's contention that Voigt only contacted Reinhold after the first two `Briefe \"{u}ber die Kantische Philosophie' had been published in August 1786, gains plausibility in comparison to R\"{o}ttgers's thought that these first `Briefe' are already a reaction to Voigts request. Cf. footnote \ref{footnote:_Ref232306650}. The following chapter deals with the publication history of the `Briefe' in more detail. } 

 In his letter to Kant Reinhold presents himself in a way that is similar to the presentation in the Preface of the \textit{Versuch} discussed earlier. The stress in his letter on practical issues was not only confirmed by his `Briefe' but is also very understandable from the point of view of his previous interests, which, as demonstrated in the previous chapter, focused on religion and Enlightenment. It is no wonder, then, that the only \textit{morceau} he could understand at first was related to those interests. In the \textit{Versuch} Reinhold also admits that first he was able to understand only those items in Kant's philosophy that he was already familiar with (cf.\textit{ Versuch}, 55). His stress on the `moral ground of cognition' of religion also goes very well with the direction his own interests in religion were taking. Already in his defense of Herder, Reinhold had confidently asserted that ``our morality no longer depends on a metaphysical system of the nature of spirits.''\footnote{ Reinhold, `Schreiben des Pfarrers', 164. Cf. also his \textit{Herzenserleichterung}, discussed in Chapter 2, section 4. } This shows that he was keen on making morality independent of traditional metaphysics, presumably because it could not satisfy the demands of his heart, i.e. his religious and moral feelings. It is no coincidence, then, that the leading question Reinhold poses in his letter to Voigt concerns the influence of the Kantian philosophy in particular on the ``scattering of the metaphysical delusions'' (\textit{RK }1:146). Thus, one of the common points behind the different perspectives appears to be Reinhold's dislike of traditional metaphysics, for which he found support in Kant. Against this background, the perspective centering on the Herder controversy gains plausibility, as the outcome of the debate was that both parties, Kant and Reinhold, were critical of traditional a priori metaphysics. The question still open to Reinhold at the end of the controversy was how to produce the ideal mix of a priori and empirical. Having read Kant, Reinhold probably thought he could now criticize metaphysics more effectively. 

 If criticizing traditional metaphysics was the prime aim of Reinhold's first plans regarding the Kantian philosophy, it is easily understandable why Herder, after being offended by Kant, would lend his approval to the proposal to provide Reinhold with an extraordinary professorship, which would entail lecturing on Kant. Apart from wishing his young friend well, Herder must have thought that appointing an avowed anti{-}metaphysician at the University of Jena was a good idea. However, in his official report on Reinhold, he could not help expressing his doubts about the value of the Kantian philosophy.\footnote{ Cf. \textit{RK} 1:200, n. 4. } If we understand that Reinhold's main interest in Kant was based on his critical attitude towards the metaphysical foundation of religion and morality, it becomes clear that Reinhold indeed could play a reconciling role between Herder and Kant. 

 Also regarding the aims and methods of Reinhold's anti{-}metaphysics his letter to Kant provides a valuable clue to providing a more integrated account of Reinhold's interest in Kantianism. He stresses that it was the `moral ground of cognition' (\textit{moralische Erkenntni\ss{}grund}) that tempted him to study the first \textit{Critique}, as it was the only thing he could understand in the review of it. We have already seen that, in the letter to Voigt, he already indicated that the solution of the problems with the fundamental truths of religion had to be sought not in the direction of a new metaphysical foundation, but rather in the structures of the human mind. With his adoption of Kantianism, Reinhold thus continues his criticism of traditional metaphysics. Moreover, founding the truths of religion on the structure of the human mind was a move he understood, given his earlier work on the history of religion. The previous chapter has shown how Reinhold repeatedly connects the developments in religion with the developments of the cognitive capacities of human reason. The structure and level of development of the human mind determine the form religion takes at a given time in history. This form can vary in keeping with the capacities of different groups of people at different times. Underlying the idea that the structure of reason plays a large part in determining the shape of religion is Reinhold's radical naturalist conception of religion, according to which religion can be explained by natural and psychological processes. Of course, this is not what Kant had in mind, as his insights into the structure of reason are meant to be neither psychological nor historical.  

With the central passage of Reinhold's first letter to Kant, acknowledging his authorship of both the `Briefe \"{u}ber die Kantische Philosophie' and the `Schreiben des Pfarrers' the efforts to provide an account for his `conversion' to Kantianism have come full circle. We are now in a position to evaluate the different sources and present a coherent account of the process of Reinhold's becoming interested in Kant. His own statements regarding this process in both the Preface to his \textit{Versuch} and the letter to Kant are incomplete and at points definitely exaggerated. They serve the specific purpose of conveying that he, Reinhold, is a fit expositor of the Kantian philosophy. He expressly presents his motivations for studying Kant as intrinsically philosophical or even existential; the more mundane incentives that must also have played a role, witnessing his letter to Voigt, are conveniently left out. He further suggests that his efforts to penetrate the Kantian philosophy turned out to be a full{-}time job for a year, which is incompatible with his other activities at the time. Nevertheless, without taking the intensity of both his crisis and his study too literally, the core of Reinhold's own claims is that the Kantian philosophy provided a solution to a serious problem he experienced and which clearly emerges from the variety of sources discussed throughout this chapter. Reinhold wants to establish the fundamental truths of religion on a basis that is neither supernatural, nor metaphysical. A long time ago he had departed from the supernaturalism of his youth and now, as pointed out in the previous chapter, he had come to find the metaphysics of his age lacking. Already he must have been already working on alternative foundations, presumably within an Illuminatist framework, which led to a naturalistic conception of religion. However, when it is founded on a metaphysical basis such as Spinozism, this religion cannot sufficiently engage Reinhold's heart. Again, in the previous chapter, it has become apparent that Reinhold was not aiming at such a metaphysical basis, but tried to ground the historical development of religion on the development of the structure of the human mind. The fact that the `\textit{moralische Erkenntni\ss{}grund}' in the \textit{ALZ}{-}review of the first \textit{Critique} was presented in a psychological form, from which Kant would soon distance himself,\footnote{ That is, the claim that religion is required as a motivation or an incentive for morality. Both Sch\"{u}tz's review and Schultz's \textit{Erl\"{a}uterung }cite the \textit{Critique }stating that, without the ideas of God and an afterlife moral action would be admirable, but we would lack incentives to act morally. Cf.\textit{KrV}, A 813; Schultz, \textit{Erl\"{a}uterungen}, 176; Sch\"{u}tz's review, 128 (Landau, \textit{Rezensionen}, 180). Kant would clearly distance himself from that position in `Was hei\ss{}t: sich im Denken orientiren?' (\textit{BM}, October 1786, 306; \textit{AA} 8: 134). In the following chapter we will return to the significance of this text for Reinhold. } no doubt contributed to Reinhold's easy understanding of this `Kantian' point. Enthusiasm for facts in order to criticize metaphysics sounds very familiar against the background of Reinhold's first reaction to Kant in defense of Herder's \textit{Ideen}. 

 We can conclude that Reinhold started studying the Kantian philosophy because it interested him and because he (rightly) believed that it could pay off as well. In spite of the high level of self{-}stylization in his own accounts of the process, they are honest in the sense that Reinhold did indeed see new possibilities in the Critical philosophy. Possibilities, that is, to realize his own goals, shaped by his Illuminatist background. These goals entailed establishing religion on a non{-}metaphysical basis, taking the structure of the human mind and its development over time as its starting point. Apart from offering a model for doing this, the Kantian philosophy must also have attracted Reinhold because it stressed the importance of the cooperation of our sensible and rational capacities. 

