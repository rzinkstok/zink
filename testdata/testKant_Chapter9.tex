
\chapter{ `Practical reason' in the Versuch einer neuen Theorie des menschlichen Vorstellungsverm\"{o}gens}




The previous chapter showed that Reinhold initially used the Kantian term `practical reason' in a way that nicely fitted his own pre{-}Kantian thoughts on human reason and Enlightenment. The `Briefe' were never intended as an exposition on Kant's thoughts regarding practical reason. Rather, Reinhold employed the term in order to apply a form of Kantianism to his interpretation of Enlightenment. Addressing key issues with which German philosophy was dealing at the time, the `Briefe' were a profound success and Reinhold became an extraordinary professor in Jena.\footnote{ Not everyone, of course, was convinced that Reinhold had found the right way of arguing for the Kantian philosophy. The Jena professor of metaphysics Ulrich, for instance, had been lecturing on Kant earlier, but turned against Kant around the time when Reinhold became his colleague. Cf. \textit{RK} 1:274{-}275, Letter 66, October 12, 1787, to Kant; \textit{RK} 1:316{-}17, Letter 84, January 19, 1788, to Kant; \textit{RK} 1:339, Letter 94, March 1, 1788, to Kant; for Kant's reaction cf. \textit{RK} 1:345, Letter 96, March 7, 1788, from Kant. } His intention to lecture on the \textit{Critique of Pure Reason} entailed that he had to look into it more deeply than he had done before. He also became more aware of the fact that the first \textit{Critique }was the center of controversy and that the work he had put forward as the panacea for the philosophical problems of his age was not universally appreciated. At the same time, Kant's \textit{Kritik der praktischen Vernunft} (1788) saw the light and was noted by critics. The result of Reinhold's more intensive occupation with Kant is the \textit{Versuch einer neuen Theorie des menschlichen Vorstellungsverm\"{o}gens }(1789). This 600{-}page main work is the impressive expression of a second, more ambitious phase of Reinhold's occupation with Kant's philosophy. Instead of merely attempting to make the usefulness of the Kantian philosophy known to a wide audience, Reinhold now tries to secure unanimous support for the philosophy he had fused with his own. 

 The present chapter continues the investigation of Reinhold's reception of Kant on the basis of his use of the Kantian concept `practical reason'. There are several reasons why the \textit{Versuch }is highly significant in this respect. Reinhold's new academic status and his growing awareness of the many controversies surrounding Kant's philosophy required a more intensive study of Kant's works and those of his critics. The publication of Kant's second \textit{Critique}, dealing with practical reason is likely to have generated some pressure on Reinhold's views concerning practical reason, especially as he had employed the term, as we have seen in the previous chapter, not so much out of interest in Kant's moral philosophy, but rather because it provided him with the desideratum for his views on human nature, a faculty that mediates between (speculative) reason and sensibility. Conspicuously, explicit references to the second \textit{Critique} are missing from the \textit{Versuch}. Moreover, the new \textit{Critique }was scrutinized immediately by August Wilhelm Rehberg, with whom Reinhold had had a polemical encounter earlier.\footnote{ Rehberg was a Hannover politician and author. Philosophically, he flirted both with Kantianism and Spinozism. For more on Rehberg and Kant, cf. Schulz, \textit{Rehbergs Opposition gegen Kants Ethik}; di Giovanni, \textit{Freedom and Religion}, section 4.3 `Rehberg and Kant', 125{-}136. } These and other historical circumstances surrounding the production of the \textit{Versuch }will be discussed in the first section of the current chapter. 

Apart from these external factors, there are also features of the \textit{Versuch} itself that give rise to the expectation that Reinhold's reception of Kant entered a new phase with that book, and in particular his conception of practical reason. The \textit{Versuch}, first of all, does not deal exclusively with the external grounds for the Kantian philosophy (cf. the letter to Voigt, discussed in Chapter 3). Rather, it appears to advance to the internal grounds as well. This distinction between the internal and external grounds, however, cannot be applied to the \textit{Versuch }in a straightforward way, since the work is not constructed as an exposition of Kant, but rather as a new theory, on the human faculty of representation, intended as a foundation of the Kantian theory of cognition. The very thought that the Kantian theory of cognition, heralded as the `Evangelium der reinen Vernunft' in the `Briefe', should need of a new and proper foundation implies a critical attitude towards Kant, which is likely to have had implications for Reinhold's conception of `practical reason' as well. The second section of the present chapter will discuss the general structure of the \textit{Versuch}, focusing on the First Book, from which we can gain useful insights into Reinhold's aims with this work, and on the final part of the Third Book, the `Theory of Reason', including the curious section entitled `Grundlinien der Theorie des Begehrungsverm\"{o}gens'. It is only in this final part that the terminology of `practical reason' returns. 

Parallel to the previous chapter, the final section of this chapter will analyze Reinhold's views on `practical reason' as they emerge from the \textit{Versuch}. They will not only be connected to his previous views, as presented in the `Briefe', but also to Rehberg's review of Kant's second \textit{Critique}. In several places this section will supplement Alessandro Lazzari's important study.\footnote{ Lazzari, \textit{Das Eine, was der Menschheit Noth ist}.} Firstly, it will stress the continuities between the `Briefe' and the central theme of the \textit{Versuch}, the theory of the faculty of representation. This confirms that Reinhold's practical interests were a driving force behind the development of his theoretical philosophy. Secondly, the final section will point out the relevance of Rehberg's review of Kant's second \textit{Critique}, which explains how and why Reinhold came to consider `practical reason' at the end of the \textit{Versuch}, in the `Grundlinien'. From a different perspective, namely that of investigating Reinhold's understanding of `practical reason', Lazzari's conclusion that Reinhold's theory of the faculty of representation gets him into trouble when it comes to justifying an absolute conception of freedom will be confirmed. 


\section{Historical context of the Versuch}


As Reinhold started lecturing at the University of Jena, his chosen subjects were aesthetics and `Introduction to the Critique of Reason for beginners'.\footnote{ Cf. \textit{RK} 1:274, Letter 66, October 12, 1787, to Kant.} In January 1788 he informed Kant of his method.  

Ich diktire die Theorien der Sinnlichkeit, des Verstandes, und der Vernunft in Aphorismen; in welchen ich von einer getreuen Schilderung des Zustandes in welchem die Kr. d. V. unsre spekulativen Ph[ilosoph]ie nat\"{u}rliche Theologie und Moral gefunden hat, ausgieng, die Nothwendigkeit einer Beilegung des Misverst\"{a}ndni\ss{} das die Philosophische Welt in vier Partheyen 1 \textit{Supernaturalisten} 2 (Naturalisten) \textit{Skeptiker}, 3 (Dogmatiker) \textit{Pantheisten }oder \textit{Atheisten}. 4 \textit{Theisten} trennt; so wie den Grund und Ursprung des Misverst\"{a}ndnisses, die unbestimmten und falschen Vorstellungsarten vom Erkenntni\ss{}verm\"{o}gen (\ldots ).\footnote{ \textit{RK} 1:315, Letter 84, January 19, 1788, to Kant. } 

[I dictate the theories of sensibility, understanding and reason in aphorisms; in which I started from a true picture of the circumstances in which the Critique of Reason had found our speculative philosophy, natural theology and morality and in which I showed the necessity of overcoming the misunderstanding that divides the philosophical world in four parties\footnote{ The presentation of the four parties in philosophy here is not very clear. The numbered order of italicised parties goes together with the indications `naturalists' and `dogmatists' between parentheses. I think we must understand Reinhold's fourfold division in accordance with the following scheme. Philosophers can be divided into supernaturalists (the first party) and naturalists. The naturalists can be divided into skeptics (the second party) and dogmatists, which are divided into a pantheist/atheist party (third) and a theist party (fourth). In this way all parties are in opposition to some other party over some philosophical issue and both the numbered order of parties and the parenthetical remarks are taken into account. } (1 s\textit{upernaturalists}; 2 (naturalists) \textit{skeptics}; 3 (dogmatists) \textit{pantheists }or \textit{atheists}; 4 \textit{theists}) and the ground and origin of this misunderstanding, that is, the indeterminate and false ways of representing the faculty of cognition (\ldots ).]                         

This method of introducing Kant's \textit{Critique} appears to be clearly related to the project of the \textit{Versuch}. The elements it mentions {--} the theories of the different capacities of the faculty of cognition and the misunderstandings of the philosophers prior to the Kantian \textit{Critique} {--} have both found their way into the \textit{Versuch}, in the Third and First Books respectively.\footnote{ Upon the assumption that Reinhold, in his lectures, limits himself to the `external grounds' for the Kantian philosophy, Onnasch claims that the `theories of sensibility, understanding and reason' as mentioned in the letter to Kant cited above, have not found their way into the Third Book of the \textit{Versuch}, dealing with `internal grounds' but into the First Book. Cf. Onnasch, introduction to \textit{Versuch}, [LXXXVII]. Although this Book discusses `sensibility' and `reason,' these discussions, in my opinion, cannot be called `theories' dictated in aphorisms. Their main aim is to show that there has hitherto been a lack of agreement among philosophers regarding the faculty of cognition. According to Lazzari, the Theories of Sensibility, Understanding and Reason as they are found in the Third Book of the \textit{Versuch}, may be based on the theories mentioned in this letter, adapted, of course, to the situation that in the \textit{Versuch} they are related to the theory of the faculty of representation, which did not yet exist when Reinhold wrote this letter. Cf. Lazzari, \textit{Das Eine, was der Menschheit Noth ist}, 86, n. 24. Onnasch's interpretation requires that the `theories' here ar understood as referring not to the Kantian theories of sensibility, understanding, reason, but rather to such theories in previous philosophy. Lazzari's interpretation, on the other hand, rests on the assumption that the `Theories' refer to the Kantian philosophy. The matter cannot be conclusively resolved on the basis of the text of this letter. } The structure of dictating aphorisms matches the structure of the \textit{Versuch }as well since it consists of numbered sections, followed by an explanation or argument. What is clearly absent from Reinhold's account of his lectures is the theory of the faculty of representation, which either did not form part of his lectures, or, more likely, did not yet exist at this time. 

 Although Reinhold already hints at the possible publication in the letter to Kant cited above,\footnote{ Cf. \textit{RK} 1:315, Letter 84.} the first mention of a definite plan to publish a book on the basis of these lectures is only found in a later letter to Kant. Reinhold announces that an ``Introduction to the Critique of Reason'' will be published by Blumauer and appear in the autumn of 1788.\footnote{ \textit{RK} 1:337, Letter 94, March 1, 1788, to Kant. } He even published a notice in the `Anzeiger' of \textit{Der Teutsche Merkur}, stating that he was working on a ``theory of the faculty of cognition'' in which the ``results of the Critique of Reason'' would be presented in a systematic way together with the vitiation of the criticisms made against it as well as their origin, which can be found in a misunderstanding of the principles.\footnote{ \textit{Anzeiger TM}, 1788, June, LXXII. } Wisely, he did not present a date by which the readers of the \textit{Merkur} could expect this book, for it was nowhere near completion by the autumn of 1788. In October, Reinhold complains that his other activities have barely left him time to compile the materials for his theory of the faculty of cognition. He now plans its publication for Easter 1789, and expects that printing will start in December 1788.\footnote{ Cf. \textit{RK} 2:28, Letter 132, October 10 1788, to Schack Hermann Ewald. } It is in this letter to Schack Hermann Ewald that we find the first mention of the theory of the faculty of representation, when Reinhold introduces the aim of the ``booklet.'' In keeping with his promise in the \textit{Merkur} he plans to do the following: 

die bisherigen Misverst\"{a}ndnisse der Crit. der V. ohne Polemik und auf eine leicht fa\ss{}liche Weise f\"{u}r die jenigen, die anders \"{u}berzeugt werden wollen hinwegr\"{a}umen (\ldots ). Ich bin n\"{a}mlich auf die \textit{Pr\"{a}missen }gerathen, welche der Kantische \textit{Theorie des Erkennens} vorangeschickt werden m\"{u}ssen, und die in einer genauen Theorie des \textit{Vorstellens} liegen.\footnote{\label{footnote:_Ref211313789} \textit{RK} 2:28{-}29, Letter 132.}

[clear away the misunderstandings of the Critique of Reason, without polemics and in a way that is easily understandable for those who will let themselves be convinced. (\ldots ) I have actually found the \textit{premises} which have to precede the Kantian \textit{theory of cognition} and which consist in a precise theory of \textit{representation}.]

During 1788 the focus of the projected work appears to have shifted significantly. At first, Reinhold's main aim was to ``provide the \textit{Critique} with prepared readers.''\footnote{ \textit{RK} 1:315, Letter 84.} By the end of the year a new element had been introduced, a theory of representation which has to precede the Kantian theory of cognition not as an \textit{introduction} but as its \textit{premise}. Reinhold's efforts to counter the misunderstandings of the Kantian philosophy take on a new form. Although it appears as if the decision to provide the Kantian philosophy with new premises requires a shift in Reinhold's perspective,\footnote{ For the view that the newer project essentially differs from the intended publication of his introductory lectures, cf. Onnasch, introduction to \textit{Versuch} [LXXXV{-}LXXXIX]. For the thought that the theory of the faculty of represesentation, or at least the \textit{Satz des Bewu\ss{}tseins} resulted from Reinhold's work on his `Kantian theory of sensibility', cf. Lazzari, \textit{Das Eine, was der Menschheit Noth ist}, 86, n. 24.} he himself did not see a real break. His claim that he had worked on the ``Theorie des Vorstellungsverm\"{o}gens'' for four years\footnote{\label{footnote:_Ref213408642} Cf. \textit{RK} 2:169, Letter 175, October 12, 1789, to Nicolai. ``Die Theorie des Vorstellungsverm\"{o}gens habe ich \textit{vier Jahre} unter der Feder gehabt.'' Obviously, Reinhold refers to the whole work here, as his discovery of the theory of the faculty of representation as presented in the Second Book of the \textit{Versuch} dates from 1788 as we have seen above, cf. note \ref{footnote:_Ref211313789}. } suggests that he thought of his efforts on behalf of the Kantian philosophy as a continuous project, starting with his first studies of Kant.

 The actual writing of the \textit{Versuch} as has come down to us, however, began in December 1788, when Reinhold signed a contract with the Jena publisher Mauke and his business partner Widtmann.\footnote{ Cf. \textit{RK} 2:44, Letter 139, December 21, 1788, from Wieland. } Soon after, we find him complaining about his workload and the resulting lack of time to keep up with all the attacks on the Kantian philosophy that had appeared.\footnote{ Cf. \textit{RK} 2:46, Letter 140, End of December 1788 and January 8, 1789, to Karl Wilhelm Justi. } The work on the \textit{Versuch }did not progress as smoothly as Reinhold would have liked, and it was not even near completion by Easter 1789. This must at least partly have been due to the fact that Reinhold fell seriously ill. He appears to have suffered some kind of stroke, becoming temporarily paralyzed and was recovering by the end of February.\footnote{ Cf. \textit{RK} 2:53, Letter 143, February 18, 1789, from Wieland; \textit{RK} 2:58, Letter 144, February 23, 1789, to Nicolai. } Although it is not clear to what extent this thwarted the plan to publish the work by Easter, it was, by that time, clear that publication would not take place before the autumn of 1789.\footnote{ Cf. \textit{RK} 2:63, Letter 145, February 26, 1789, to G\"{o}schen. } 

 The Preface, however, was ready by the beginning of April,\footnote{ Reinhold sent a copy to Kant as a birthday present. Cf. \textit{RK} 2:69, Letter 148, April 9, 1789, to Kant. } and was published both as a separate booklet, and in the \textit{Merkur} of April and May 1789.\footnote{\label{footnote:_Ref211316039} Reinhold, \textit{Ueber die bisherigen Schicksale der Kantischen Philosophie} (Jena: Mauke 1789); Reinhold, `Ueber das bisherige Schicksal der Kantischen Philosophie,' \textit{TM}, April, 1789, 3{-} 37; May, 113{-} 135. The separate publication is identical with the Preface of the \textit{Versuch}, which explains the fact that the Preface is dated April 1789, although the \textit{Versuch }was published only in the autumn of that year. It is unclear whether the separate publication or the \textit{Merkur\-} version was written earlier. The fact that the separate publication was published only by Mauke shows that it was not part of the original deal with Mauke and Widtmann. Reinhold did not only seize the opportunity to alert the public to his forthcoming book, but also to gain supplementary income. Cf. Onnasch, Introduction to \textit{Versuch} [XCVIII{-}XCIX].} By that time Reinhold claimed to have the ``small work'' lying in his desk ``for the most part finished,'' while the first sheet was being prepared for print.\footnote{ \textit{RK} 2:81, Letter 153, April 30, 1789, to Ewald. } By the end of May, the First Book had been printed, as Reinhold sent its final sheets to Hufeland.\footnote{ \textit{RK} 2:122, Letter 161, end of May/ begin June 1789, to Gottlieb Hufeland. } The printing of the Second Book appears to have begun immediately when the First was finished since Reinhold informed Kant on June 14 that it would be ready within six weeks.\footnote{ \textit{RK} 2:131, Letter 164, June 14, 1789, to Kant. } From the same letter it appears that the final Third Book did not at that time lie ready in Reinhold's desk yet, since he speaks of it in the future tense. ``The third book will contain the application to the theory of the faculty of cognition.''\footnote{ \textit{RK} 2:131, Letter 164.} It is most likely that this book was only written during the summer of 1789. Indeed, the state of the Third Book suggests that Reinhold was under some pressure to finish the work in time, as it is not as neatly polished as the earlier parts of the work.\footnote{ Cf. Onnasch, `Introduction to \textit{Versuch }[CIX{-}CXI]; Lazzari, \textit{Das Eine, was der Menschheit Noth ist}, 154{-}155. } 

 Our account of the circumstances surrounding the publication of Reinhold's first main work {--} he himself consistently and modestly refers to it as \textit{B\"{u}chlein} or \textit{Werkchen {--} }is not complete with only a look at the publication process itself. While producing the work, Reinhold was also very concerned about its reception and he took his measures to ensure the attention of potential reviewers, trying to create some sort of media hype as we might call it. For instance, when he sent Hufeland the final sheets of the First Book, he used the occasion to complain {--} Hufeland co{-}edited the \textit{ALZ} {--} about the lack of attention his works had received up until that moment in that magazine.\footnote{ Cf. \textit{RK} 2:123{-}124, Letter 161.} Particularly the tardiness of the \textit{ALZ} in reviewing \textit{Ueber die bisherigen Schicksale der Kantische Philosophie} appears to be a source of concern.\footnote{ Cf. footnote \ref{footnote:_Ref211316039}. } Of course, this booklet had been published as an appetizer to create an eager anticipation of the main work with the readers. For that reason, it must have been of the highest importance to Reinhold that it was noted in the magazines and did not get too much criticism. By May 1789, Reinhold had every reason to be unhappy with the reception so far. As he tells Hufeland, he is not pleased that the \textit{ALZ} has not been the first to review his booklet {--} the first review, in the \textit{G\"{o}ttingische Anzeigen von gelehrten Sachen},\footnote{ Buhle, review of\textit{ Ueber die bisherigen Schicksale der Kantischen Philosophie}, by Reinhold, \textit{G\"{o}ttingische Anzeigen von gelehrten Sachen}, nr. 84, May 25, 1789.} was not to his liking.\footnote{ Cf. \textit{RK} 2:123, Letter 161.} His suspicion that Rehberg would be the reviewer for the \textit{ALZ }did not ease his worries and he anticipated having to write a counter{-}review to redeem himself.\footnote{ He writes to Hufeland: ``Ist H. Rehberg \textit{ungerecht}: so werde ich mir zum erstenmal in meinem Schriftstellerleben selbst Gerechtigkeit verschaffen.'' Cf. \textit{RK} 2:124, Letter 161. }

 Reinhold had his reasons for expecting trouble if Rehberg were to review his work. Preceding the work on the \textit{Versuch}, Reinhold and Rehberg had been engaged polemically as a result of the latter's book \textit{\"{U}ber das Verh\"{a}ltni\ss{} der Metaphysik zur Religion} (1787). Reinhold had reviewed the work for the \textit{ALZ}, questioning the author's understanding of Kant.\footnote{ Reinhold, review of \textit{Ueber das Verh\"{a}ltni\ss{} der Metaphysik zur Religion}, by Rehberg, \textit{ALZ}, June 26 (Nr 153b),1788. For the claim that Reinhold is the author of this (second) \textit{ALZ} review of Rehberg's book and for a overview and evaluation of the issues contested cf. Schulz, \textit{Rehbergs Opposition}, third chapter, 77{-}175, esp. 81{-}83. Although the review is not listed in Sch\"{o}nborn's bibliography or the \textit{Korrespondenzausgabe}, several authors have followed Schulz in attributing it to Reinhold. Cf. Onnasch, annotations to \textit{Versuch}; di Giovanni, \textit{Freedom and Religion}, 126.} Rehberg had reacted to the review in the \textit{Merkur}, questioning Reinhold's understanding of Kant in turn.\footnote{ Rehberg, `Erl\"{a}uterung einiger Schwierigkeiten der nat\"{u}rlichen Theologie,' \textit{TM}, September, 1788, 215{-}233. Cf. \textit{RK}, 2:17, Letter 126, September 6 1788, from Wieland, who informs Reinhold of including Rehberg's article in the \textit{Merkur}. } Reinhold was thus not only concerned about the time it took for a review of the \textit{Bisherige Schicksale} to appear in the \textit{ALZ}, he also doubted that this review would be favorable. He continued his complaints regarding the lack of attention for his works on the part of the \textit{ALZ} in his letter to Kant of June 14. Again, both the tardiness and the expected contents of the review worried Reinhold. He expressly states that he believes that Rehberg has ``only half understood the \textit{Critique}.''\footnote{ \textit{RK}, 2:133, Letter 164, June 14, 1789. Reinhold's statement in the Intelligenzblatt of the \textit{ALZ} that the \textit{ALZ}{-}reviewer of the \textit{Versuch} (Rehberg) ``has completely understood'' Kant's \textit{Critique} (Reinhold, `Erkl\"{a}rung' in `Intelligenzblatt', \textit{ALZ}, December 12 (Nr. 137), 1789, 1138{-}1140; cf. \textit{RK}, 2:133, note 35) becomes more understandable when we realise that Reinhold was not aware that Rehberg was the author of that review. Cf. note \ref{footnote:_Ref211316289} below.} Reinhold's prayers were answered, since the review of the \textit{Bisherige Schicksale} was published June 23 and was not negative.\footnote{ [anonymous] review of \textit{Ueber die bisherigen Schicksale der kantische Philosophie}, by Reinhold, \textit{ALZ,} June 23 (Nr. 186), 1789, 273{-}276. } It appears, however, that the review (of the \textit{Bisherige Schicksale}) was not Rehberg's after all. For in his review of the \textit{Versuch\-} {--} which was certainly not to Reinhold's liking {--} Rehberg states that the Preface had been reviewed earlier by somebody else.\footnote{\label{footnote:_Ref211316289} Cf. Rehberg, review of \textit{Versuch einer neuen Theorie des Vorstellungsverm\"{o}gens}, by Reinhold, \textit{ALZ}, November 19 (Nr. 357), 1789, 414{-}424; continued November 20 (Nr. 358) 425{-}429. This would also explain why Reinhold would think that Schultz, the author of the \textit{Erl\"{a}uterungen} was the reviewer, given that he assumed that Rehberg had reviewed the \textit{Schicksale}. Even the editors of Reinhold's correspondence appear to be confused by the matter, as they indicate that Reinhold first ascribed the review to Rehberg and later to Schultz. Cf. \textit{RK} 2:186, note 3. However, Reinhold only ascribed the review of the \textit{Schicksale} to Rehberg. Cf. \textit{RK} 2:124, Letter 161, end of May/ begin June 1789, to Hufeland; \textit{RK} 2:133 Letter 164, June 14, 1789, to Kant. Given the claim in the review of the \textit{Versuch} that the \textit{Schicksale} had been reviewed by someone else, it would make sense for Reinhold to think that Rehberg was not the author of the review of the \textit{Versuch}. Despite his friendly relations to the publishers and editors of the \textit{ALZ}, Reinhold was not always aware of the ins and outs of the reviews. } The contents of the review of the \textit{Versuch }occasioned new complaints from Reinhold to Hufeland and a reaction in the `Intelligenzblatt' of the \textit{ALZ}.\footnote{ Reinhold, `Erkl\"{a}rung' in `Intelligenzblatt,' \textit{ALZ}, December 12 (Nr. 137), 1789, 1138{-}1140; Rehberg, `Antwort' in `Intelligenzblatt', \textit{ALZ}, January 30 (Nr. 15), 1790, 118{-}120. Cf. \textit{RK} 2:185{-}188, Letter 181, between November 20 and December 2, 1789, to Hufeland. } 

  His complaints regarding the \textit{ALZ} were by no means the only measures Reinhold took to try and ensure a positive reception of his first main work. Similar requests not to let certain individuals review his \textit{Versuch} went to Nicolai, the publisher of the \textit{Allgemeine deutsche Bibliothek} {--} with a similar lack of success.\footnote{ Cf. \textit{RK} 2:170, Letter 175, October 12, 1789, to Nicolai; \textit{RK} 2:192, Letter 183, November 29, 1789, to Nicolai. Against Reinhold's wishes, the \textit{Versuch} was critically reviewed by Hermann Andreas Pistorius in \textit{Allgemeine deutsche Bibliothek}, 1791, vol. 101, nr. 2, 295{-}318. } Reinhold's haste to finish the work before the Michaelmas book fair must be seen in the same light. After all, he had gone to great lengths to ensure that everybody knew that the new Jena professor was about to publish his textbook on Kant in the autumn of 1789. Not only had he published the Preface as an appetizer, the various parts of the First Book had also appeared in several magazines.\footnote{\label{footnote:_Ref211329357} Section 1 had been published in \textit{Der Teutsche Merkur}, June, July, 1789. Section 2 had been published in \textit{Berlinische Monatsschrift}, July 1789. Sections 3, 4 and 5 had been published in \textit{Neues deutsches Museum}, July, August, September, 1789. Although the versions are not identical, the changes made merely concern style and typography. Cf. Sch\"{o}nborn, \textit{Karl Leonhard Reinhold }, 73{-}74. Moreover, a summary of the first eleven sections of the Second Book was published in \textit{Der Teutsche Merkur}. `Fragmente \"{u}ber das bisher allgemein verkannte Vorstellungs{-}verm\"{o}gen,' \textit{TM}, October, 1789, 3{-}22. The relevant deviations from the \textit{Versuch }in all of these works are noted in Reinhold, \textit{Versuch}, ed. Onnasch. } From a marketing perspective, postponing publication any further would not have been a good idea, even if its philosophical and structural qualities might have benefited from a delay in publication. 


\section{Structure and aims of the Versuch}


The eventual result of Reinhold's hard work consists of four main parts. First, there is the lengthy Preface on the ``fate of the Kantian philosophy up till now,'' which had been published separately before. The work itself, then, is divided into three books, each consisting of sections numbered continuously, and containing a short statement or aphorism, followed by an explanation. The first of these books is entitled `Abhandlung \"{u}ber das Bed\"{u}rfniss einer neuen Untersuchung des menschlichen Vorstellungsverm\"{o}gens', in which Reinhold argues, as he had done in the `Briefe', that the given problems regarding the philosophy of religion and morality necessitate a critique of reason. He also argues, however, that this critique must be preceded by and founded upon an investigation of the faculty of representation, since the central concept of Kant's first \textit{Critique}, cognition (\textit{Erkenntnis}) is but a species of the more general concept of representation (\textit{Vorstellung}) (cf.\textit{ Versuch}, 189). This investigation, intended as foundation for the Kantian philosophy, is presented in the Second Book of the \textit{Versuch}, entitled `Theorie des Vorstellungsverm\"{o}gens \"{u}berhaupt'. In the Third Book, the `Theorie des Erkentnissverm\"{o}gens \"{u}berhaupt', Reinhold claims to show how the Kantian results follow from that foundation. In doing so, Reinhold roughly follows the structure of Kant's first \textit{Critique}.\footnote{ That is, Reinhold presents the theories of sensibility, understanding and reason respectively, in a way that is reminiscent of Kant's division of the \textit{Critique }in Transcendental Aesthetics, Analytics and Dialectics. Reinhold's theory of sensibility leads to the presentation of the forms of intuition, space and time; the theory of the understanding results in the presentation of the categories; and the theory of reason leads to the transcendental ideas (all in Reinhold's understanding, of course). For a further overview of the structure of the Third Book of Reinhold's \textit{Versuch }in relation to Kant's first \textit{Critique}, cf. Lazzari, \textit{Das Eine, was der Menschheit Noth ist}, 75{-}81.} He tacitly assumes that these results solve the problems discussed in the First Book. The general structure of the \textit{Versuch }can thus be described as follows. First, as in the `Briefe', serious problems in the field of the philosophy of morality and religion are introduced and attributed to misunderstandings among the philosophers. The Kantian philosophy is credited with providing a solution for these problems, as in the `Briefe'. In contrast to the `Briefe', however, this solution has to be grounded in a new theory of the faculty of representation. From this theory, the results of the Kant's first \textit{Critique }are thought to follow, which means that the \textit{Versuch }as a whole can be characterized as a work of theoretical philosophy. The focus on the first \textit{Critique }is also apparent from the absence of any explicit reference to either Kant's \textit{Groundwork of the Metaphysics of Morals }or the second \textit{Critique}.\footnote{ Reinhold cites from the \textit{Groundwork} and is certainly inspired by Kant's table of material determining grounds of the will from the second \textit{Critique}, yet does not explicitly acknowledge these works as a source. Cf. \textit{Versuch}, 102{-}117. } Nevertheless, the Kantian results that it seeks to establish do have a practical character, since they are to provide a rational ground for the basic concepts of religion and morality (God, the soul, freedom). We have seen that in the `Briefe' Reinhold attached a crucial importance to practical reason with regard to such an effort to provide these rational grounds. 

Although one would therefore expect Reinhold to work out these rational grounds from practical reason in greater detail in his first main work, this is not what happens in the course of the Third Book. Reinhold's strategy in following Kant's first \textit{Critique} rather appears to be showing how the ideas of God, the soul and freedom are necessarily related to our cognitive faculties, which in turn would tell us what can and cannot be done with these ideas. That is, they cannot be used to ground knowledge of supersensible objects, yet have an important function in organizing knowledge and are rationally sound. The exact manner in which these ideas can be used to ground religion and morality is not part of the scope of the \textit{Versuch}.\footnote{ According to Lazzari, the structure of the Third Book shows that it aims to establish the idea of absolute freedom in a much stronger sense. Cf. Lazzari, \textit{Das Eine, was der Menschheit Noth ist}, 63, 81{-}82; cf. footnote \ref{footnote:_Ref229216414} of the present chapter. In the final section of the current chapter I argue for a different interpretation of that structure, according to which Reinhold, in line with the first \textit{Critique}, only seeks to establish the logical possibility of freedom. Of course, Kant's second \textit{Critique }aims to establish something beyond the logical possibility of freedom, namely the practicality of pure reason. Cf. \textit{AA} 5:3.} Given these circumstances, it is surprising to find that `practical reason' figures prominently at the very end of the work, in a curious chapter entitled `Grundlinien der Theorie des Begehrungsverm\"{o}gens'. The peculiarities at the very end of the work provide us with at least one good reason to examine the structure of the \textit{Versuch} in more detail.

 There is another good reason, however, which is connected to Reinhold's initial plans regarding the Kantian philosophy as expressed in his letter to Voigt from 1786. As we have seen in the second section of Chapter 3, Reinhold intended to discuss both the `external grounds' or benefits and the `internal grounds' of the Kantian philosophy. It also appeared that Reinhold's `Briefe' correspond to the first seven points he planned to discuss as external grounds.\footnote{ Cf. Chapter 3, section 2. } Thematically, these external grounds return in the First Book of the \textit{Versuch}. There is, however, a difference in perspective. While the `Briefe' focused on the \textit{content }of the solution, that is, on the Kantian discovery that practical reason can provide rational grounds for certain religious beliefs, while speculative reason fails to do so, the First Book of the \textit{Versuch} stresses the \textit{need} for an investigation of the faculty of representation, leaving the actual results of this investigation for the remainder of the work to answer. This is where one might expect a discussion of the internal grounds, which remained unspecified in Reinhold's letter to Voigt. However, since Reinhold at this point no longer just intends to present Kant's philosophy proper {--} if he ever did {--} but has become quite convinced that the premises for the Kantian theory of cognition have not been provided yet, it becomes hard to understand what, at this point, would be the internal grounds for the Kantian philosophy. This is, of course, partly due to the initial lack of determination given of that term in Reinhold's letter to Voigt. As we have seen, the internal grounds concern a presentation of the Kantian arguments that would display the Kantian system as an actual and sound system of philosophy. Given Reinhold's enterprise of providing, in his theory of the faculty of representation, the missing premises for the Kantian theory of cognition, it is doubtful that he can still maintain that this theory is in fact completely sound.\footnote{ Cf. Lazzari, \textit{Das Eine, was der Menschheit Noth ist}, section 1.3, esp. 51{-}52. } The nature of the project of providing the premises for the Kantian philosophy indicates that the earlier distinction between the external and internal grounds for the Kantian philosophy is not very helpful. Since Reinhold initially already failed to specify the internal grounds, it is impossible to pinpoint them in the \textit{Versuch}, where the emphasis is no longer solely on the Kantian theory of cognition but also on Reinhold's own theory of representation. 

Similarly, there are some difficulties in determining the nature of the Kantian `results' at this point. It is clear that Reinhold seeks to establish these results on a more firm basis by explicitly providing the hidden premises of the Kantian theory of cognition. As the Third Book presents the outlines of that theory as founded upon Reinhold's theory of the faculty of representation, they appear to be the Kantian `results'. If this is the case, however, there appears to be a mismatch between these `results' and the `results of the Kantian philosophy' as they figured in the context of the external grounds.\footnote{ Cf. Chapter 3, section 2. } In that context, it is a result of the Kantian philosophy that it has provided the means to solve urgent philosophical matters such as the status of the conviction that God exists. In the \textit{Versuch} it is tacitly assumed that the `results of the Kantian philosophy' as presented in the Third Book are `results' in the second sense as well, that is, that they meet the needs sketched in the First Book. However, there is no hint in the \textit{Versuch} that the book is actually trying to establish the results of the Kantian philosophy in the second sense.\footnote{\label{footnote:_Ref229216414} Lazzari, on the other hand, claims that the aim of the \textit{Versuch }is precisely to establish these results in the second sense. Cf. Lazzari, \textit{Das Eine, was der Menschheit Noth ist}, 42, 46, 63, 81{-}82. Although it is certainly true that the \textit{Versuch} is the first step towards universally acceptable claims concerning the fundamental truths of religion and morality, this does not entail that the theory of the faculty of representation itself should be seen as directly providing the grounds of cognition for these truths. Cf. Lazzari, \textit{Das Eine, was der Menschheit Noth ist}, 48{-}49. }

For now, let it suffice to say that the relation of this work to Reinhold's earlier work on Kant and to his plans regarding the Kantian philosophy cannot be described very neatly. Although the main idea to consider both external and internal grounds appears to remain upright, the fusion of Reinhold's own theory with that of Kant makes it hard to affix these labels in the case of the \textit{Versuch}. The following overview of the structure of the work aims to provide a more detailed insight into Reinhold's aims when writing it. 


\subsection{Preface}


The separate publication of \textit{Ueber die bisherigen Schicksale der Kantischen Philosophie }was intended to prepare the general public for the \textit{Versuch}, and Reinhold indeed received positive comments.\footnote{ From Johann Heinrich Abicht and Karl Heinrich Heydenreich, whom Reinhold had sent copies. Even Christian Garve, although he did not share Reinhold's opinion on Kant, found some friendly words for its author. Cf. \textit{RK} 2:104, Letter 158, May 14 1789, from Abicht; \textit{RK} 2:152, Letter 169, July 20, 1789, from Heydenreich; \textit{RK} 2:159, Letter 171, August 14, 1789, from Garve. } As a Preface it serves a similar purpose, that is, it aims at convincing the reader that the book that lies before him is worthwhile and that its author knows what he is talking about.\footnote{ For Reinhold's presentation of his philosophical history provided in the Preface, cf. Chapter 3, section 3. } In order to make these points Reinhold opens with a move similar to that of the first two `Briefe': he establishes that the German philosophical world is in chaos, which is a result of the decline of the Leibnizian{-}Wolffian philosophy which is no longer as solid as it was, because of the combination of this system with experience (cf. \textit{Versuch}, 1{-}8). The chaos shows that there is a lack of universally valid (\textit{allgemeing\"{u}ltig}) principles among philosophers. Kant's \textit{Critique} is intended to remedy this but his project has been misunderstood (cf. 12). In the `Briefe' the nature of reason is presented as being misunderstood, and in the Preface the Kantian project is said to be misunderstood. Here Reinhold's first aim, then, is to show that there is ``no absolute impossibility that the Critique of Reason has been misunderstood by its opponents and defenders alike'' (18). Without actually claiming that Kant's first \textit{Critique} does in fact contain universally valid principles, Reinhold points out that the thought that it might contain such principles is indeed compatible with the reception it has had (cf. 41).

 In order to strengthen his case, however, Reinhold adds some more specific notes on the situation of German philosophy and the nature of Kant's \textit{Critique}. The first step here is to ascribe the lack of universally valid principles to ``a general misunderstanding, common to all [philosophical] sects'' (41). Since, according to Reinhold's construction, the controversies between the different sects concern the cognizability (\textit{Erkennbarkeit}) of supersensibles, Kant's investigation of what can be known in general may provide the solution to the general misunderstanding, by showing the one{-}sidedness of the different perspectives of all sects. This hypothesis of a general misunderstanding thus not only explains the lack of unison among philosophers, it also explains the fate Kant's philosophy met (cf. 49). 

 The remainder of the Preface aims at making a case for the author as a legitimate spokesman on behalf of the Kantian philosophy. As we have seen in the third section of Chapter 3, Reinhold sketches quite a dramatic image of his philosophical career up to this point, aiming not only at convincing the reader that he knows his philosophy (cf. 51{-}52), but also emphasizing his unique position in the philosophical world, not being attached to a single system. In this `professional autobiography' his advocacy for Kant is also presented as a natural result of its salutary effects on Reinhold's personal faith. We have already seen in Chapter 3 that Reinhold probably had less noble reasons as well for making a serious effort to understand Kant. As the sincerity of Reinhold's statements regarding the time he initially invested in his study of Kant is doubtful,\footnote{ Cf. Chapter 3, section 3. } one may easily become cynical with regard to the statements about himself in this Preface. Although most commentators acknowledge that Reinhold produces a stylization of himself here, his remarks on being saved from the twin evils of \textit{Aberglauben }and \textit{Unglauben} are generally taken seriously.\footnote{ For a very strong instance of taking Reinhold's statements about his philosophical and religious development at face value cf. Lauth, `Nouvelles R\'{e}cherches'. Cf. also Batscha, \textit{Karl Leonhard Reinhold}, 11, 17; Sauer, \textit{\"{O}sterreichische Philosophie}, 59. In the most recent literature, however, Reinhold's enthusiasm for Kant is viewed from a broader perspective. Cf. Bondeli, introduction to \textit{Briefe I}, XXIV{-}XXVII; Onnasch, introduction to \textit{Versuch} [XXXII, LVII{-}LVIII]. } Given the discussion in Chapter 2 on Reinhold's Enlightenment ideals, however, there is not much reason to exclude these statements from the stylized picture Reinhold paints of himself here. He obviously had a clear insight into the spirit of the times and in what would be appreciated by the general public. He presents himself as the personification of the recent philosophical history in Germany. After going through different phases that did not quite work, he has been saved from intellectual and religious crisis by the Kantian philosophy. 


\subsection{Book I: The need for a theory of the faculty of representation}


With a citation from Locke's \textit{Essay concerning Human Understanding }on the title page (69) of the First Book Reinhold presents himself as an ``under{-}labourer'' and his aims as ``removing some of the rubbish, that lies in the way to knowledge.''\footnote{ Cf. John Locke, \textit{An Essay concerning Human Understanding}, ed. Peter. H. Nidditch, 10. } Like the Preface, the First Book had been published in several parts earlier that year.\footnote{ Cf. footnote \ref{footnote:_Ref211329357}.} It discusses a problem very similar to that of the `Briefe': given the animosity among philosophers regarding the foundations of morality and religion, we need to investigate how our convictions in these fields are even possible. Such an investigation should provide the common ground upon which a universally valid and universally accepted structure of philosophy of morality and religion can be erected. The terminology of `universally accepted' principles (\textit{allgemeingeltend}) as different from principles that are merely universally valid (\textit{allgemeing\"{u}ltig}) is novel in comparison to the `Briefe' (cf. 71{-}76) and reflects the fact that Reinhold had become more aware of Kant's critics. Although he himself was firmly convinced that the Kantian theory of knowledge was universally valid, and had presented it as such in his `Briefe', the writings of Kant's critics must have driven the point home that not everybody was receptive to this universal validity, in other words that the universally valid theory was by no means universally accepted (\textit{allgemeingeltend}). In the Preface, as we have seen above, Reinhold argued that it was very well possible that the Kantian philosophy was universally valid, but almost universally misunderstood. 

 In comparison to the `Briefe' the First Book of the \textit{Versuch }approaches problems of the philosophy of religion and morality in a more systematic way. Following his discussion on the lack of universal validity in philosophy, he specifies four areas in which this is conspicuous: the grounds of cognition of the fundamental truths of 1) religion and 2) morality and the principles of 3) morality and 4) natural rights (cf. 75). After modestly announcing that his only goal here is to point out the lack of universally valid principles in these areas, Reinhold continues with a specification of his understanding of `grounds of cognition of the fundamental truths of religion and morality'. 

Es bedarf wohl kaum erinnert zu werden, da\ss{} hier unter \textit{Religion }und \textit{Moralit\"{a}t}, kein wissenschaftliches System der Theologie und der Moral, sonder die Inbegriffe gewisser Neigungen und Th\"{a}tigkeiten des Willens verstanden werden, die man mit diesen Namen bezeichnet. Die Ueberzeugungen durch welche diese Neigungen und Th\"{a}tigkeiten zun\"{a}chst m\"{o}glich werden, nenne ich \textit{Grundwahrheiten}; und die zureichenden Gr\"{u}nde dieser Ueberzeugungen \textit{Erkenntni\ss{}gr\"{u}nde} (nicht der Gegenst\"{a}nde, sondern) der \textit{Grundwahrheiten der Religion und der Moralit\"{a}t}. (75{-}76)

[It is hardly needed to point out that here by \textit{religion }and \textit{morality }is not meant a scientific system of theology and morality but the quintessence of certain dispositions and activities of the will that are designated by this name. The convictions through which these dispositions and activities are first possible I call the \textit{fundamental truths}; and the sufficient grounds for these convictions \textit{grounds of cognition }(not of objects but) of the \textit{fundamental truths of religion and morality}.]

Applied to religion, discussed immediately following this specification (76{-}89), Reinhold stresses that he is not considering religion as an abstract system, but rather focuses on the actual religious inclination of people and the convictions that make these inclinations possible {--} in the case of religion the convictions that God exists and that there is an afterlife. It is the sufficient ground for these \textit{convictions} that Reinhold calls `grounds of cognition'. If universally accepted grounds of cognition should be established at some point, they are sufficient grounds of conviction of the existence of these objects, not proofs.\footnote{ It is no coincidence that Reinhold came to replace the unfortunate terminology of `grounds of cognition' (Erkenntni\ss{}gr\"{u}nde) with `grounds of conviction' (Ueberzeugungsgr\"{u}nde) in \textit{Briefe I}. } This clarity about the status of the grounds of cognition is an improvement in comparison to the `Briefe'. Another innovation is the systematic presentation of the parties in the conflict regarding the grounds for the fundamental truths of religion. Reinhold himself was very aware of this innovation, having published it beforehand as `Neue Entdeck'.\footnote{ `Neue Entdeck.' in \textit{ALZ}, September 25 (nr. 231a), 1788, 831{-}832. } (New discovery). Instead of the two parties introduced in the `Briefe' from a diachronic perspective (hyperphysicists and metaphysicians) Reinhold now presents a synchronic perspective of four parties (dogmatic theists, atheists, dogmatic skeptics and supernaturalists). Further, the account deviates from the `Briefe' in that it lacks any discussion of the conviction that the soul is immortal whatsoever. 

 With the discussion of the ground of cognition of the fundamental truth of morality (89{-}98) we enter territory that had not been covered by the `Briefe'. \footnote{ The addition of a discussion on the lack of universally accepted principles regarding the principles of morality (freedom) and natural right is foreshadowed in Reinhold's plan as communicated to Voigt. Point XV mentions the ``need for \textit{universally valid first principles} for other sciences, for instance, \textit{natural right}.'' \textit{RK} 1:157, Letter 35. This may include morality as well. The issue of freedom is also mentioned under point X, in the context of the antinomies of reason. Cf. \textit{RK} 1:156. } Reinhold opens with the definition of morality as the ``\textit{intended} agreement of voluntary (\textit{willk\"{u}rlich}) actions with the laws of reason'' (89). Since this agreement, so Reinhold continues, should depend on the choice (\textit{Willk\"{u}r}) of the actor, it is presupposed that this actor has the ability to enforce these laws of reason in the face of opposing demands of sensibility, which ability is called freedom (cf. 89{-}90). Thus, the conviction of freedom is the fundamental truth of morality. Because Reinhold sees it as the task of philosophy to provide grounds for this conviction and given that these grounds should go beyond the immediate common{-}sense conviction of our freedom, philosophers are required to establish at least the `logical possibility' of freedom, which is contrasted with its metaphysical possibility (cf. 93). Unfortunately, however, the philosophical world is divided on this issue along the same lines as it is in the case of the fundamental truths of religion.

 In relation to the third area specified above, Reinhold sets out to show a lack of universal acceptance regarding the principle of morality (99{-}117). This \textit{principle} is not to be confused with the \textit{fundamental truth} of morality. While the latter concerns the non{-}impossibility of human freedom, the former concerns the grounds of obligation (\textit{Verbindlichkeit}) of the moral law (cf. 99). Without any explicit reference, this section leans heavily on Kant's second \textit{Critique} and can be regarded as an explication of the Table of Practical Material Determining Grounds (\textit{AA} 5:40). The result of this discussion is that there are many ways to define the nature of the obligation implied by the moral law, but those who have done so in a way depending on pleasure and pain (material determining grounds) have not succeeded in producing any universally accepted principle of morality. The fourth point, regarding the lack of a universally accepted principle in the field of natural right, is not discussed in any detail. Reinhold merely indicates (117{-}120) that in this field the disagreements are even bigger, as philosophers do not even agree on the concept of `right'. 

 Now that Reinhold has established the omnipresent lack of universally accepted (\textit{allgemeingeltend}) principles, the second section (120{-}141) of his First Book considers whether this lack could be due to a lack of universally valid (\textit{allgemeing\"{u}ltig}) principles. This leads to the critical doubt whether philosophy is able to provide universal grounds of cognition and principles at all (cf. 120). It is important to Reinhold to distinguish this kind of doubt from dogmatic skepticism as well as from the ``unphilosophical skepticism'' of the \textit{Popularphilosophen }(130{-}141). Critical doubt differs from dogmatic skepticism because it doubts whether philosophy (in which dogmatic skepticism is but one party) can provide universally valid grounds, whereas the dogmatic skeptic, far from doubting the universal validity of his own grounds, doubts the possibility of objective truth (cf. 131). Since the \textit{Popularphilosophen} are, like Reinhold, not themselves a party in the philosophical conflicts as Reinhold has sketched them, and in a way also doubt the capacity of philosophy in general to provide sufficient grounds, Reinhold takes great pains to convince his readers that the critical skepticism he advocates is to be sharply distinguished from the `popular' skepticism. He refers to it as an `unphilosophical' kind of skepticism, since it stems from an antipathy against philosophical reason (cf. 132{-}133). This antipathy in turn is grounded upon inscrutable common sense, and therefore unphilosophical, whereas Reinhold's critical doubt is based upon a comparison of the various philosophical systems from the point of view of the Kantian philosophy and the observation that none of them has been able to provide us with universally valid principles and grounds of cognition. Unlike the two other kinds of skepticism, critical doubt necessitates a new kind of inquiry, which is to establish precisely the differences and agreements between the various parties in order to see whether anything universally valid can be found (cf. 140).

 The third section (141{-}146) of the First Book takes a further step, as it establishes that this new inquiry is of a transcendental nature, focusing on the question: ``\textit{How} are those universally valid grounds of cognition and principles possible?'' (141). This question can only be answered, so Reinhold continues in the fourth section (146{-}188), after an answer to the following question has been found: ``What are the limits of the human faculty of cognition?'' (146). This means that the investigation that is to be undertaken is ``neither \textit{hyperphysics} nor \textit{metaphysics} but rather \textit{Critique}'' (148 n.). The section is devoted to showing that there is currently no agreement among philosophers as to their understanding of the human faculty of cognition. They do not agree on the meaning of `reason', not on the meaning of `sensibility', and not on the meaning of `faculty of cognition' in general. This disagreement, which hinders the all{-}important investigation into the possibility of universally valid grounds of cognition and universally valid principles, can only be solved, so Reinhold claims in the fifth and final section (188{-}192) of the First Book, by starting not from cognition, but from the faculty of representation. The reason for this is that `representation' is prior to `cognition' in the sense that any concept of `cognition' presupposes a concept of `representation', but not vice versa (cf. 189). Moreover, ``representation is the \textit{only thing} about the reality of which \textit{all} philosophers agree'' (190). Thus, it is the only universally valid presupposition, even if not all philosophers share the same concept of `representation'. Reinhold hopes, however, that his \textit{Versuch} will succeed in providing such a degree of clarity that everybody can grasp the proper concept of `representation'. 

 It is only with this very last section that Reinhold distances himself from the position held in the `Briefe' that the Kantian theory of cognition is the ultimate solution to the philosophical disputes of his age. The fact that the Kantian theory of knowledge was nearly universally misunderstood necessitated a new theory, one of the faculty of representation, which is to be at once both universally valid and universally accepted and will thus pave the way for the Kantian theory of cognition. 


\subsection{Book II: The theory of the faculty of representation}


Like the First Book, the Second opens with a citation from Locke's \textit{Essay} which claims that the only immediate objects of the mind are its own ideas.\footnote{ Cf. Locke, \textit{Essay}, 525. Reinhold has added emphasis on `immediate' by means of italics. } It must be noted that Reinhold does not accept Locke's use of the term `idea' as referring to the basic unit of mental activity. He would prefer ``representation in general'' (cf. 317). The first seven sections of the Second Book (\S \S  6{-}12) contain an introductory determination of the `faculty of representation'. Reinhold distinguishes three determinations of this concept, varying in width. In the first, very broad, sense the faculty or capacity of representation (\textit{Vorstellungsverm\"{o}gen}) is the totality of conditions for representation (cf. 195; 217). Although `\textit{Verm\"{o}gen}' is usually translated as `faculty', Reinhold does not aim to provide some sort of faculty psychology. Rather, his aim is transcendental, concerning the internal and external conditions required for representation.\footnote{\label{footnote:_Ref229468623} Although the aphorism of the second section of the Second Book (\S  7) contains a version of the `Satz des Bewu\ss{}tseins' ``in statu nascendi'' (Bondeli, \textit{Das Anfangsproblem bei Karl Leonhard Reinhold}, 56), the \textit{Versuch }does not appear to propose a strict `deduction' of the rest of the work from that section as its principle, but is only taken as the basis for the further theory of the faculty of representation insofar as it is the first and most general description of some of the conditions of representation, the external conditions. Although the internal conditions (material and form) of representation are related to object and subject, there is no relation of strict deduction of the latter conditions from the claim in \S  7. Rather they are presented as resulting from the conceptual analysis of `representation'. } The representing subject and represented object are among the external conditions. They are implied by the consciousness of representation, but are distinct from the representation itself, even if they are necessary conditions for representation (cf. 202). In its narrower meaning, the faculty of representation only includes the internal conditions of representation (cf. 202). This narrower meaning is to be distinguished from the narrowest meaning, which only considers the conditions for `representation' as a genus, excluding the specific kinds of representations (intuition, concept, idea) in their variation (cf. 218{-}219).

 This gives us a general idea of Reinhold's theory of the faculty of representation. If `mere representation' is the genus of different kinds of representation, it will contain only what is common to these kinds. The faculty of representation in the narrowest sense is the set of conditions for this genus and the theory of this faculty is an investigation of these conditions. Before actually embarking upon this investigation, Reinhold uses \S \S  13{-}14 to clarify its nature. First of all, he explains that it is not an investigation into the nature of the soul or the nature of objects.\footnote{ For this reason I cannot agree with Bondeli's claim that Reinhold's way of dealing with the faculty of representation as the foundation of all philosophy entails that this faculty resembles a Herderian `erzeugende Originalgattung', as ``ein h\"{o}chstes Verm\"{o}gen das sich zu realisieren und das dadurch Folgeprinzipien zu begr\"{u}nden vermag.'' Cf. Bondeli, `Von Herder zu Kant,' 227. As indicated above, Reinhold's understanding of the term `Vorstellungsverm\"{o}gen' as the condition for representation excludes understanding it in the sense of faculty psychology, as a part of the mind to be distinguished from other parts. Reinhold's explicit description of his theory as a conceptual analysis of `representation' shows that he is not aiming to establish foundational relations between faculties, but rather between our concepts of them. } It can only be undertaken with the concept of mere representation (\textit{blosse} \textit{Vorstellung}) as a starting point. Any other representation, be it of the soul or of something else, is necessarily only a particular representation, the conditions of which are not necessarily the conditions of representation in general (cf. 221). He adds that although both the representing subject and the represented object are indeed conditions of representation, they are only external conditions. As the investigation concerns the faculty of representation in its narrowest sense, it does not deal with the origin of representation, but rather with its nature (cf. 222). The question to be answered is: ``what can and must be thought in the \textit{concept of representation}?'' (223). The impossibility to give a definition does not mean that it cannot be expounded and discussed (cf. 227). This discussion will entail an exposition of the essential characteristics (\textit{wesentliche Merkmale}) which in turn will provide us with a criterion of representability (\textit{Vorstellbarkeit}), for anything that is in contradiction with these characteristics cannot be represented. 

 These characteristics of representation, then, are `material' (\textit{Stoff}) and `form' (\textit{Form}), reflecting the double relation of representation to both the subject and the object. The material of a representation relates to the object (\S  15), whereas the form relates to the subject (\S  16) of a representation. Representation itself is the unity of this material and form. Since every representation has a form, Reinhold continues in section 17, a thing in itself, that is, a thing as it is independent of the form of representation, cannot be represented as such, for representing it would entail giving it the form of representation. Reinhold presents this as the shorter route to Kant's thesis that we only know appearances, not things in themselves (cf. 254).\footnote{ Cf. Ameriks, \textit{Kant and the Fate of Autonomy}, 125{-}135. } It now becomes clear why Reinhold chose that particular citation from Locke to put on the title page of this Second Book. The thought that the immediate objects of the mind are representations supports Reinhold's thought that things cannot be known as they are in themselves. 

The next step is to determine the nature of the material and form of representation. In section 18 Reinhold establishes that the material must be given, whereas the form must be produced (\textit{hervorbringen}) by the subject. Referring to the systems of Locke and Leibniz, he argues that the concept of representation had been confused insofar as Locke overemphasized the element of givenness, whereas Leibniz exclusively focused on the production by the subject. It turns out that their systems have to be combined as both the given material and the produced form are essential characteristics of representation (cf. 260{-}261). The discussion of Leibniz and Locke again, as in the `Briefe', stresses the need to combine sensibility and understanding. There, the discussion focused on sensibility and understanding as elements of the faculty of cognition. Here, in the \textit{Versuch}, the essential dichotomy and the need to combine receptive and spontaneous elements is situated at a more fundamental level, that of the faculty of representation. 

 From these characteristics of representation the essential characteristics of the faculty of representation can be inferred, as conditions for the possibility of representation. On the one hand, the faculty of representation must have receptivity, in order to receive the given material (\S  19). On the other, it must have spontaneity, in order to produce the form of representation (\S  20). Reinhold insists that these characterizations are neutral concerning the nature of the representing subject, which must be regarded as the ``merely \textit{logical substrate}'' of the predicates `receptivity' and `spontaneity', and which cannot be represented in itself (273). Likewise, mere material and mere form of representation cannot be represented in themselves (\S  22). Nevertheless, Reinhold claims that the material of a representation is a manifold, whereas the form is a unity (\S  24). Hence, the nature or form of receptivity is the ``manifold in general'' (\textit{Mannigfaltigkeit \"{u}berhaupt}), while the form of spontaneity consists in the synthesis of this manifold. (\S \S  25{-}26) These forms are prior to all representation and are given to the subject as subjective material, that is, material that does not come from outside the faculty of representation (\S  27). Here we find the \textit{Versuch}'s version of Reinhold's ideas on `pure sensibility' as found in the `Briefe'.\footnote{ Cf. Chapter 4, section 2.2.2 and section 3.2. } That is, he stresses that both spontaneity and receptivity must be part of the faculty of representation prior to actual representations. This means that they can be represented `purely', that is, without reference to other objects (cf. 294). We get at these forms by means of a conceptual analysis of the concept of `mere representation', which is an abstraction from particular representations (cf. 295).\footnote{ In this analysis of the abstracted forms of representation without reference to the actual content of particular representations, Reinhold's method resembles the later phenomenological method of Edmund Husserl. See Klemmt, \textit{Karl Leonhard Reinhold's Elementarphilosophie}, 58{-}68. Frederick Beiser likewise explicitly discusses `Reinhold's Phenomenological Project.' See Beiser, \textit{The Fate of Reason}, 247{-}252. Martin Bondeli is critical of the attempts to place Reinhold in a `phenomenological undercurrent of German Idealism,' more specifically of Klemmt's interpretation.  See Bondeli, \textit{Das Anfangsproblem bei Karl Leonhard Reinhold}, 15, 17. For the most recent assessment of Reinhold's phenomenonology, see Oittinen, `\"{U}ber einige ph\"{a}nomenologische Motive in Reinhold's Philosophie.'}

Since the faculty of representation, however, only provides ``the determined possibility to receive a manifold and (\ldots ) give it unity through synthesis'' the actuality of representation requires something more, something outside the faculty of representation (296). This external element necessary for the actuality (\textit{Wirklichkeit}) of representation is objective material, or material that is given from outside the faculty of representation (\S  28). This forms the basis of Reinhold's `refutation of idealism' as he continues in the next section: ``The existence of objects outside us is therefore just as certain as the existence of a representation in general'' (299).\footnote{ Cf. \textit{KrV}, B 275. It is interesting to note that Reinhold places the proof for the reality of things outside us at the end of the theory of the faculty of representation, whereas, in Kant's first \textit{Critique} it is found in the Transcental Analytics, which would be the theory of the understanding in the structure of the \textit{Versuch}. This position of Reinhold's refutation of idealism may relate to the lack of a transcendental deduction of the categories. He was critical on Kant's use of the `synthetic unity of apperception' in relation to the deduction of the categories. Cf. Onnasch, `Vor\"{u}berlegungen zur Herleitung der Urteilsformen und Kategorien,' especially section 2.} The Second Book concludes with the description of the different kinds of representation and their relation to mere representation (\S \S  33{-}37).


\subsection{Book III: The theory of the faculty of cognition}


The Locke{-}citation on the title page (319) of the Third Book draws attention to the limited nature of our knowledge: its scope is narrower than that of our ideas.\footnote{ Cf. Locke, \textit{Essay}, 539. Reinhold's citation breaks off in the middle of a long and complex sentence. The gist of that sentence is that although it is obvious that our knowledge cannot extend beyond our ideas, it would be nice if the boundaries of knowledge would coincide with those of our ideas, but this is not case, hence we are often in doubt concerning our ideas. } Although Reinhold would not use `idea' as the most general term for `representation' (cf. 317), the citation indicates that his theory of the faculty of cognition (\textit{Erkenntnisverm\"{o}gen}), to be expounded in this Book, will be narrower than the theory of the faculty of representation. It must be noted, however, that Reinhold was not entirely satisfied with the title he had chosen for the Third Book. In his \textit{Beytr\"{a}ge zur Berichtigung bisheriger Mi\ss{}verst\"{a}ndisse der Philosophen} he presents the theories of sensibility, understanding and reason as belonging to the theory of the faculty of representation, rather than the theory of the faculty of cognition. In \textit{Beytr\"{a}ge I }Reinhold describes the relation between representing and cognizing in the following manner: 

Die Formen der Vorstellungen sind nur \textit{in wieferne} sie mit den Vorstellungen auf die \textit{Objekte} derselben bezogen werden, \textit{Formen des Erkennens}; in \textit{wieferne} sie hingegen mit den Vorstellungen nicht auf Objekte, sondern auf das \textit{Subjekt} bezogen werden, sind sie die bestimmten \textit{Formen des Begehrens}.\footnote{ \textit{Beytr\"{a}ge I}, 277 [Fabbianelli 192].}

[The forms of representations are only \textit{forms of cognition insofar} as they are, with the representations, related to the \textit{objects }of those representations; on the other hand, \textit{insofar} as they are not with the representations related to objects, but rather to the \textit{subject}, they are determinate \textit{forms of desiring}.] 

The first eight sections (\S \S  38{-}45) form a transition from the theory of the faculty of representation to the theory of the faculty of cognition by means of a discussion of consciousness. Consciousness, according to Reinhold, in general consists of ``the being related [\textit{Bezogenwerden}] of mere representation to the object and the subject'' (321). Reinhold's discussion of consciousness as an introduction to his theory of the faculty of cognition is a continuation of his theory of the faculty of representation, extending it from the internal conditions of representation to the external conditions, that is, the representing subject and the represented object. This fits in with the understanding of the relation between the theory of the faculty of representation and the theory of the faculty of cognition as presented in the passages from the \textit{Beytr\"{a}ge }cited above. Reinhold further stresses that consciousness is not itself a representation, but rather a ``double act of the subject, through which the representation is assigned [\textit{zueignen}] to the object with respect to its material, and to the subject with respect to its form'' (324). This double action is in both cases an act of both connecting (\textit{verbinden}) and distinguishing (\textit{trennen}), as connecting a representation to the subject entails distinguishing it from the object and vice versa (cf. 324{-}325). As representation and consciousness are intimately connected, consciousness containing the external conditions for representation, Reinhold maintains that there are no representations without consciousness (cf. 327). He further distinguishes between consciousness of the representation, of the representing subject and of the represented object. The precise details of these distinctions need not concern us here. 

Finally, the transition to the theory of cognition is made when Reinhold announces that ``consciousness of the object is called \textit{cognition in general}, in so far as in this consciousness the representation is related to the determined object'' (340). In cognition the object is not only connected to and distinguished from the representation (as in consciousness in general) but it is again ``\textit{represented as }distinct'' from the representation (341). Cognition entails representing the object as an object that differs from the representation itself. This requires that the object was earlier represented immediately, that is, was an object of intuition. This intuition can then be the object of another representation, a concept, in which the object of the intuition is thought, that is represented as something which is represented (cf. 344). The conclusion of the introduction is therefore that the ``\textit{faculty of cognition in general} consists of a faculty of \textit{intuitions} and a faculty of \textit{concepts}'' (349). 


\subsubsection{Theory of sensibility}


The theory of sensibility comprises twenty{-}one sections (\S \S  46{-}66) and roughly corresponds to Kant's Transcendental Aesthetics in that it establishes the forms of intuition, space and time. In the opening section Reinhold stresses time and again that he is not concerned with the question as to the nature of the representing subject but only with the nature of the faculty of representation. That is, he aims to answer the question: ``What must be the nature of the faculty of representation, if it is to be capable of sensible representation?'' (351). First the special nature of sensible representation must be established, which is done in the following manner: ``Mere representation is called \textit{sensible} insofar as it is formed immediately by the way in which receptivity is affected'' (356). The material of the sensible representation is immediately given as the faculty of representation is being affected and its form can only consist in the synthesis ``of the given insofar as it is given'' (357). In accordance with this description sensibility is defined as ``the ability [\textit{Verm\"{o}gen}] to come to have representations by the way in which receptivity is affected'' (362). This way, then, in which receptivity can be affected is either from the outside or from the inside, resulting in the distinction between outer and inner sense (\S \S  50{-}51). Like receptivity and spontaneity, outer and inner sense have forms that are a priori determined in the faculty of representation. As the form of outer sense comprises the possibility to receive a manifold that consists of parts that are outside of one another (\textit{aussereinander befindliche Theile}) (cf. 378), it is no wonder that this form is itself the material for the a priori representation of mere space (cf. 389). Similarly, the form of inner sense comprises the possibility to receive a manifold of sequential parts (\textit{nacheinanderfolgende Theile}) (cf. 381) and is itself the material of the a priori representation of mere time (cf. 402). Reinhold establishes the forms of sensibility in a different way than Kant had done; for instance, he does not appear to care for mathematics at all. Nevertheless, the conclusion of his theory of sensibility neatly matches the result of Kant's Transcendental Aesthetics: ``\textit{Space }and \textit{time} are essential conditions of all appearances, but not of things in themselves'' (419).\footnote{ Cf. \textit{KrV}, A 43/B 60. }


\subsubsection{Theory of the understanding}


Reinhold's theory of the understanding comprises ten sections (\S \S  67{-}76) and roughly matches Kant's Transcendental Analytics as Reinhold deals with acts of judgment and the categories. In contrast to sensible representation, or intuition, which is immediately formed by the way in which receptivity is affected, a concept is immediately formed by the way in which spontaneity acts, that is, through synthesis (cf. 423). The representation that comes to be in this way has as its material some material that had already been represented in an intuition. The understanding or the ``ability to come to have representations by the way spontaneity is active'' synthesizes the already represented material anew and thus produces a concept (422). Reinhold stresses the necessity of combining sensibility and understanding in the process of cognition. The representation on the basis of which the concept was formed becomes a predicate of the object, which means that the way to produce a concept is by means of judgment (cf. 424). The unity that is brought about by judging is called objective unity, and is the form of objects in general (\textit{Gegenstand \"{u}berhaupt}) (\S \S  69{-}70). The next step is the distinction between analytic and synthetic judgment which concerns the relation of the objective unity to intuition (\S  71). In judging synthetically, so Reinhold explains, objective unity is produced (\textit{hervorbringen}) out of an intuition, whereas in judging analytically, an already existing objective unity is connected to intuition (cf. 435{-}436). This entails that any analytic judgment must be preceded by a synthetic judgment (cf. 439).\footnote{ This appears to be close to Kant's statement in \S  15 of the transcendental deduction (B) that analysis presupposes synthesis ``for where the understanding has not previously combined anything, neither can it dissolve anything'' \textit{KrV}, B 130; \textit{CPR}, 246. Since the transcendental deduction in the A{-}edition does not discuss the relation between analysis and synthesis in this way, the terminological correspondence suggests that Reinhold did have a good look at the second edition of Kant's first \textit{Critique} when he wrote the \textit{Versuch}. It must be noted, however, that he speaks of analytic and synthetic judgements, whereas the context of the passage from Kant discusses `synthesis' and `analysis,' rather than judgements. } 

 With the basic terms in place, Reinhold can start the pi\`{e}ce de r\'{e}sistance of the theory of the understanding: the deduction of the categories. He first introduces them in section 72 as ``particular forms under which objects have to be thought,'' which are determined a priori by the different forms of judgment (441). Reinhold's presentation of the forms of judgment and the categories do not concern us here in detail.\footnote{ For more on the details of Reinhold's categories, see Onnasch, `Vor\"{u}berlegungen zur Herleitung der Urteilsformen und Kategorien.' } Let it suffice to note that, in contrast to Kant, Reinhold does not distinguish a metaphysical and a transcendental deduction. He uses the `objective unity' to be brought about by synthetic judgment as a guideline to systematically establish all possible forms of judgment as forms of uniting a subject and a predicate. This should guarantee the completeness of the table of forms of judgment and of the derived table of categories as well.\footnote{Cf. \textit{Beytr\"{a}ge I}, 316. } By applying the categories to the common form of all intuition, time, Reinhold, like Kant, provides the \textit{schemata} (\S  75), or ``forms of cognizability'' (482). 


\subsubsection{Theory of reason}


Reinhold's theory of reason comprises the final twelve sections of his \textit{Versuch} (\S \S  77{-}88) and only matches Kant's Transcendental Dialectics in its outlines, while, for instance, crucial Kantian doctrines such as the antinomies of reason are missing. Before \S  87, however, we find a separate unit of text, entitled `Grundlinien der Theorie des Begehrungsverm\"{o}gens'. The layout suggests that this unit may be a chapter on a par with the theories of sensibility, understanding and reason, yet it does not contain any numbered sections, which suggests that it is something other than a chapter. The last sections of the work continue the structure of the Theory of reason, rather than forming part of the `Grundlinien'.\footnote{ For various interpretations of this curious part of the \textit{Versuch} and its relation to the theory of reason, cf. Lazzari\textit{, Das Eine, was der Menschheit Noth ist}, 89. } As, in fact, it is a part of section 86, that section and the `Grundlinien' will be discussed separately below (2.4.4). First, we will look at the theory of reason, putting the `Grundlinien' in brackets, as it were. 

Similarly to the way in which concepts are formed by the understanding through the synthesis of intuitions, ideas are formed by reason through the synthesis of concepts.\footnote{ \textit{KrV}, A 664/B 672.} This is the starting point for Reinhold's theory of reason. The activity of thus synthesizing concepts is described as syllogistic reasoning, or \textit{schlie\ss{}en}. Unlike the objective unity of the understanding brought about by the categories, the unity of reason, in synthesizing the manifold of the understanding, is independent of sensibility. It cannot, therefore, be the unity of the knowable, but rather is the unity of the thinkable. Whereas the objective unity (of the knowable) established by the concepts of the understanding is ``together with intuition an essential \textit{constitutive} component of experience'' (515), the unconditioned unity of reason, relating to mere concepts cannot be a constitutive component of experience; it is ``a mere law, according to which the objects of experience that are thought can be ordered in a whole of knowledge, in scientific connection'' (515{-}516). It thus systematizes knowledge, through the ``laws of reason for the systematic unity of experience'' \footnote{ Cf. \textit{KrV}, A 228{-}229/B 280{-}281. These laws are: `in mundo non datur hiatus,' `in mundo non datur saltus,' `in mundo non datur casus purus' and `in mundo non datur fatum.' In Kant's \textit{Critique} these laws are related to the third postulate of empirical thinking in general. } and the principles of homogeneity, specification and continuity, according to which reason guides the understanding (cf. 516{-}522).\footnote{ Cf. \textit{KrV}, A 657{-}658/ B 685{-}686. } Thus, in general, the form of syllogistic reasoning is related to the systematicity of our cognition. Like Kant, Reinhold gives a more specific interpretation of the unity brought about by reason, related to three specific forms of syllogistic reasoning. Kant connected the psychological, cosmological and theological ideas of reason to the forms of the categorical, hypothetical and disjunctive syllogisms.\footnote{ Cf. \textit{KrV}, A 405{-}406/B 432{-}433. This passage strongly suggests a link between the forms of syllogistic reasoning and the ideas of reason ordered according to their material. Nikolai Klimmek has argued that in Kant, the ordering in psychological, cosmological and theological ideas comes to override the older ordering based upon the forms of syllogistic reasoning, as the latter cannot be related to the various domains of special metaphysics. Cf. Klimmek, \textit{Kants System der transzendentalen Ideen}, 58. Reinhold, at any rate, saw a close connection between the forms of syllogistic reasoning and the psychological, cosmological and theological ideas. } In Reinhold's terminology, these syllogistic forms are connected to the idea of absolute subject, that of absolute ground or cause,\footnote{ As noted by Lazzari, the terms `\textit{absolute} \textit{Ursache'} and `\textit{absoluter} \textit{Grund}' are equivalent and are used interchangeably by Reinhold. Cf. Lazzari, \textit{Das Eine, was der Menschheit Noth ist}, 99. I have translated both as `absolute cause'. } and that of absolute community respectively (cf. 522). Unlike Kant Reinhold proposes a further specification with regard to these three ideas. They are specified with regard to the difference between outer and inner sense, as these have different kinds of objects (\S  83). The rational unity (\textit{Vernunfteinheit}) of empirical knowledge is thus split up in objective and subjective unity of reason. The first relates to ``the objects outside us,'' the second to the ``representations in us.'' Both of these are brought into complete interconnection (\textit{vollst\"{a}ndiger Zusammenhang}), by the three ideas respectively (526). Thus, the number of ideas comes to six, as the ideas relating to outer sense and those relating to inner sense have ``essentially different objects, to the extent that they relate to either the cognizable of outer sense or that of inner sense'' (526). 

 With this the set{-}up for the presentation of the ideas is complete. Six ideas need to be related to reason, in three pairs. \textit{First}, there is the idea of the absolute subject in relation to the objective unity of reason, which with respect to the objects of outer sense can be specified as that which underlies the phenomena of outer sense, and which has to be thought of as a \textit{noumenon}. With respect to the subjective unity of reason, the idea of the absolute subject \textit{secondly} specifies that which underlies the phenomena of inner sense, or the soul, which has to be thought of as a \textit{noumenon} as well (\S  84). \textit{Thirdly}, the idea of absolute ground or cause in relation to the objective unity of reason yields the idea of a first cause, undetermined and not a member of the knowable series of causes and effects (\S  85). With respect to the subjective unity of reason, \textit{fourthly}, the idea of absolute cause represents a property of the absolute representing subject, namely the way of acting that is peculiar to reason (\S 85). \textit{Fifthly}, the idea of absolute community in relation to the objective unity of reason yields the idea of the physical world (\S 87), whereas, \textit{sixthly}, it yields the idea of the moral world in relation to the subjective unity of reason (\S  87). One would expect Reinhold's presentation of the ideas to be finished at this point, but two more ideas are appended. The first is to be understood as the idea of absolute community applied to the ideas of the physical and the moral world together, which results in the idea of an intelligible world in which both worlds are in harmony (\S  87). Finally the idea of absolute community is applied to predicates, yielding the idea of the most real being, which is to be regarded as the first cause (\S  88).\footnote{ My interpretation of the relation of the different idea to the respective sections differs from Lazzari's, who takes \S  86 to be an integral part of the presentationof the ideas and therefore locates the introduction of the idea of absolute cause relating to the subjective unity of reason in that section. However, this overlooks the fact that the final paragraph of \S  85 does introduce that idea, whereas \S  86 does something else, namely introduc the notion of absolute freedom. Cf. Lazzari, \textit{Das Eine, was der Menschheit Noth ist}, 87{-}88.} These last two ideas correspond to Kant's postulates of practical reason and are not introduced on the basis of the distinction between objective and subjective unity of reason. 

 From the above overview it will be clear that Reinhold's presentation of the ideas differs from Kant's. It is more systematic, streamlining the psychological, cosmological and theological ideas into three objective{-}subjective pairs. This has implications for the location of the ideas of `soul', `world' and `God'. The psychological idea is no longer the sole paradigm for the idea of absolute subject; it must share its title with the idea of absolute subject in the relation to the objects of outer sense. The cosmological idea of the physical world is no longer presented as part of a set of antinomies, but as a form of the idea of absolute community. Kant's discussions of the dialectical illusions that the ideas of reason may produce (paralogisms, antinomies, ideal) have no place in Reinhold's account. His idea of absolute causation yields the ideas of the first cause of phenomena and of free activity of the representing subject. Like the idea of the soul is mirrored by the idea of the absolute subject with regard to the objects of outer sense, the idea of the physical world is mirrored by the idea of the moral world. The idea of God as the ideal of reason, or the highest reality,\footnote{ Cf. \textit{KrV}, A 579{-}580/B 607{-}608. } is not included in the set of six ideas that Reinhold has set out to present. However, it does appear in the last section of the \textit{Versuch}, after the mentioning of the idea of an intelligible world. Both these last ideas are related to the idea of absolute community, and they do not appear to fit in the subjective{-}objective scheme with regard to the unity of reason. The idea of the intelligible world is the idea of absolute community applied to the ideas of the physical and the moral worlds together, while the idea of the most real being as a first cause originates from the application of the idea of absolute community to predicates instead of subjects. Apart from the fact that these ideas do not appear to fit into the intended scheme, they show a striking similarity to Kant's postulates of practical reason. At the end of his theory of reason, therefore, Reinhold is not only providing a presentation of the ideas, but is showing that the practical postulates belong to reason as well. The relation to Kant's practical philosophy is obvious since the final two ideas, resembling the postulates, are first mentioned in the final paragraph of the `Grundlinien', which implicitly discusses Kant's second \textit{Critique}. 

Wie sich aus der n\"{a}heren Bestimmung und weiteren Ausf\"{u}hrung dieser Pr\"{a}missen der \textit{Glaubensgrund }f\"{u}r das \textit{Daseyn einer intelligiblen Welt} (in welcher das h\"{o}chste Gute nur durch eine ins Unendliche fortdaurende Existenz und Personalit\"{a}t des endlichen vern\"{u}nftigen Wesens erreichbar ist) und f\"{u}r das \textit{Daseyn einer von der Natur unterschiedenen und der moralischen Gesinnung gem\"{a}\ss{} wirkenden Ursache der gesamten Natur} ergebe: l\"{a}\ss{}t sich nur in der eigentlichen \textit{Theorie der praktischen Vernunft}, und nach einer v\"{o}llig entwickelten \textit{Theorie des Begehrungsverm\"{o}gens} einleuchtend genug darthun. Die Theorie der \textit{Vernunft \"{u}berhaupt}, in wieferne sie ein Theil der blossen Theorie des \textit{Erkenntni\ss{}verm\"{o}gens \"{u}berhaupt} ist, mu\ss{} sich begn\"{u}gen, die blossen Ideen der \textit{intelligiblen Welt,} und jenes \textit{Urwesens}, in wieferne dieselben in der Form des Vernunftverm\"{o}gens gegr\"{u}ndet sind, aufzustellen. (575) 

[How from the more precise determination and further development of these premises follows the \textit{ground of belief }for the \textit{existence of an intelligible world} (in which the highest good is attainable only through the infinite duration of existence and personality of the finite rational being) and for the \textit{existence of a cause of nature as a whole that differs from nature and works in accordance with a moral disposition}, can only be adequately shown in the proper \textit{theory of practical reason} and after a completely developed \textit{theory of the faculty of desire}. The theory of \textit{reason in general}, insofar as it is part of the mere theory of the \textit{faculty of cognition in general}, must be satisfied with establishing the mere ideas of the \textit{intelligible world} and of that \textit{original being}, in as much as these are grounded in the form of the rational faculty.] 

In accordance with his first mention of these ideas in the final paragraph of the `Grundlinien', Reinhold does not discuss them in their practical capacities. Rather, he relates them to the structure of reason, which suggests that they are both legitimate and inevitable.


\subsubsection{Section 86 and `Grundlinien der Theorie des Begehrungsverm\"{o}gens'}
\label{subsubsection:_Ref228718282}


Although the layout of the `Grundlinien' chapter suggests that it is indeed a chapter on a par with the earlier theories of sensibility, understanding and reason,\footnote{ Cf. for instance Breazeale, `Between Kant and Fichte: Karl Leonhard Reinhold's ``Elementary Philosophy'',' 802. } the fact that the theory of reason continues right after the `Grundlinien' with section 87, presenting the ideas related to absolute community, shows that it is not on a par with the other parts of the Third Book, such as the `theory of sensibility' or the `theory of reason'.\footnote{ According to Klemmt the `Grundlinien' forms an appendix to the theory of reason serving as a transition towards the final two sections. Klemmt, \textit{Karl Leonhard Reinholds Elementarphilosophie}, 117.} It is more like a chapter within the theory of reason. If we look at the contents, however, it clearly does not belong as a constituent part to the theory of reason in which we find it. It deals with reason only from the perspective of the faculty of desire, not from that of the faculty of cognition, in which context the theory of reason is treated. The `Grundlinien' can be regarded as a separate chapter, albeit one that relates to the main structure of the text in another way than the other parts of the Third Book. In fact, it is a chapter \textit{within} section 86, since it supplies argumentation that is missing from the main text of that section.\footnote{\label{footnote:_Ref211748364} Lazzari points out that the text of section 86 following the aphorism does not live up to the expectations of that aphorism, because it does not contain arguments concerning the determination of the faculty of desire by means of reason. Based on this supplementary function, Lazzari describes the `Grundlinien' as ``gewisserma\ss{}en ein Teil'' of section 86. Lazzari, \textit{Das Eine, was der Menschheit Noth ist}, 94.} Taking a closer look at the aphorism of the section will reveal why Reinhold needs additional argumentation. 

Durch die Idee der absoluten Ursache, in wieferne dieselbe auf die Kaussalit\"{a}t der Vernunft bezogen werden mu\ss{}, wird das vorstellende Subjekt als \textit{freye Ursache} vorgestellt; und zwar als \textit{komparativ{-}frey}, in wieferne die Vernunft beym \textit{Denken} gesch\"{a}ftig ist, und das \textit{Begehrungsverm\"{o}gen} \textit{a posteriori} bestimmt; \textit{absolut{-}frey}, in wieferne sie das \textit{Begehrungsverm\"{o}gen a priori} bestimmt. (558) 

[Through the idea of absolute cause, in so far as it must be related to the causality of reason, the representing subject is represented as a \textit{free cause}; that is, as \textit{comparatively free}, in so far as reason is active in \textit{thinking} and determines the \textit{faculty of desire} \textit{a posteriori}, and \textit{absolutely free}, in so far as it determines the capacity for desire \textit{a priori}.]

In the previous section (\S  85) Reinhold has already treated the idea of the absolute cause with respect to both the objective and subjective unity of reason. The discussion of the idea of absolute cause with regard to the causality of reason in section 86 appears to be something extra, apart from the presentation of the six ideas. The whole of section 86, including the `Grundlinien', is to be regarded as an excursion outside the scope defined by the presentation of the six ideas. This is a novel interpretation of the connection of the `Grundlinien' to the rest of the \textit{Versuch}. Although Lazzari has already interpreted the `Grundlinien' as in a way belonging to section 86 (cf. above, footnote \ref{footnote:_Ref211748364}), he still considers section 86 itself as belonging to the basic structure of the theory of reason. This interpretation is illustrated by his table regarding the structure of Reinhold's presentation of the ideas of reason.\footnote{ Lazzari, \textit{Das Eine, was der Menschheit Noth ist}, 87. } This table situates the presentation of the idea of an absolute cause with regard to the subject (idea of a free will) in section 86, parallel to the presentation of the same idea with regard to objects (idea of a first cause) in section 85. It overlooks the circumstance that section 85 in fact already presents the idea of an absolute cause with regard to the subject. Section 86 does not relate the idea of an absolute cause to the subject, but rather to the `causality of reason', which differs from the `absolute representing subject', which is discussed in the final paragraph of section 85, and which is, according to that section to be regarded as an absolute cause. Taking in consideration that the other two pairs of ideas (absolute subject and absolute community) are both presented in a single section (sections 84 and 87), interpreting the whole of section 86 as an anomaly within the structure of the theory of reason makes good sense. This new interpretation is strengthened by the fact that Reinhold introduces two sets of new terminology in this section. First, a specification is made regarding the free causality of reason as comparative and absolute freedom. Secondly, the aphorism contains the first reference to the faculty of desire. Although, in a similar context in section 83, Reinhold had employed a differentiation between a theoretical and a practical faculty of representation, no introduction or explanation of the faculty of desire had been given so far. 

Let us briefly compare its introduction here, in section 86, to the passage of section 83 introducing the practical faculty of representation. There Reinhold had considered the rational unity as the effect of the absolute subject. He had also stated, in a parenthetical remark, that this rational unity in the theoretical faculty of representation determines the systematicity of knowledge, whereas in the practical faculty of representation it determines the morality of acts of will (cf. 537). The similarity of the context of this passage to that of section 86 warrants the claim that the phrases `practical faculty of representation' and `faculty of desire' denote the same faculty. In both cases the subjective idea of an absolute ground is related to a form of causality of reason, and in both cases the activity of reason is specified with regard to a theoretical (thinking/cognizing) and a practical (willing/desiring) activity. Moreover, in both cases no immediate explanation is provided. Since section 86 makes claims concerning the faculty of desire that are part of an argument, Reinhold needs to come forward with some explication of this newly introduced faculty. He first discusses reason as an absolute cause, that is, a free cause, in general and then continues to discuss the comparatively free activity of reason in thinking. At the point where Reinhold would have to start a discussion of the comparative and absolute freedom of reason in determining the faculty of desire, he comes up with the chapter on the `Grundlinien der Theorie des Begehrungsverm\"{o}gens'.

 Within this chapter Reinhold's first move is to relate the faculty of desire to the faculty of representation. The link is made by defining the `representing power' (\textit{vorstellende Kraft}) as the ground of ``that which is actualized through the representing subject'' (560).\footnote{ The use of the term `vorstellende Kraft' may be a bit surprising, given the trouble Reinhold had taken in the Second Book to exclude the representing power from the investigation into the faculty of representation (especially \S  8). His use here, however, does not contradict or ignore the earlier restrictions, as Reinhold does not make any claims about the nature of the representing power, for instance, whether it is material or spiritual. Both the representing power and the things in themselves (cf. \S  28) are part of the external conditions for representation. In section 28 and 29 the reality of ``things outside us'' had already been postulated as being necessary for the ``reality of representation'' (299), so there is nothing worrisome about Reinhold's introduction of the representing power as something needed for the reality of representations at this point. The way in which he uses it to introduce the faculty of desire merely shows that this faculty is not only related to the capacity for representation but rather to its actuality. } As such, the representing power differs from the spontaneity that with receptivity belongs to the grounds for the mere possibility of representation.\footnote{ Cf. \S  20, 268, where Reinhold's claim that the capacity of representation involves spontaneity leads to the claim that this spontaneity is not to be confused with the representing power, in so far as the latter is connected to a representing subject} Abstaining from metaphysical claims about the (material or spiritual) nature of this power, Reinhold stresses that the power must ``express itself in accordance with the capacities given to it,'' and defines `drive' (\textit{Trieb}) accordingly as the relation or the connection ``of the power [\textit{Kraft}] to the capacity [\textit{Verm\"{o}gen}]'' (561). `Drive' is the activity of the principle responsible for the actuality of the representation (the representing power) in accordance with the forms given a priori in the faculty of representation. Building upon his definition of `drive', Reinhold continues to define `desire' and the `faculty of desire' as ``being determined by drive to produce a representation'' and ``the capacity of being determined by drive,'' respectively (561).\footnote{ A parallel to this view on drive and the faculty of desire can be found in the first instalment of the article `Ueber die Natur des Vergn\"{u}gens' (\textit{TM}, October, November 1788, January 1789). In his discussion of the aesthetical theories of Pouilly and Du Bos, he describes `drive' as the `feeling of this need [for the occupation of the faculty of representation]' (63) and relates the representing power to the actuality of representations. } 

The next step is to build up a theory of this faculty of desire along the lines of the theory of representation. Reinhold identifies an empirical drive (ordinary desires relating to empirical objects), a rational{-}sensible drive (drive for happiness) and a purely rational drive (moral drive). With regard to the freedom of reason in determining the faculty of desire our present interest lies with the latter two. Regarding the rational{-}sensible drive the rational component consists of the extension of the empirical drives towards a maximum. It is the drive for happiness. Here Reinhold resumes the discussion on freedom, by stating that reason acts only comparatively free with regard to the drive for happiness. It is free ``in so far as the form of the unconditioned imparted on the drive is the effect of absolute self{-}activity'' (566). Insofar as the drive itself, however, springs from the need to be affected and ``its satisfaction depends upon the objective material's being given,'' it is ``neither free, nor unselfish'' (566). 

With the claim that reason may also determine the faculty of desire in an a priori way, Reinhold proceeds to the argument for the absolute freedom of reason in its determination of the faculty of desire. In contrast to the rational{-}sensible drive Reinhold now introduces the `purely rational' (\textit{rein{-}vern\"{u}nftig}) drive, which ``is only determined by the self{-}activity of reason, and thus has only the exertion of self{-}activity, the mere action of reason as object'' (569). The action of reason, that is, the object of the purely rational drive consists ``in the \textit{realization} of (\ldots ) the \textit{form of reason}, which is only given in the subject as a possibility, but the reality of which outside the subject has to be produced by the subject'' (569). In contrast to the activity of reason with regard to the rational{-}sensible drive, in the purely rational drive, reason ``does not presuppose the sensible drive and sensibility for the realization of its action, and thus acts \textit{a priori}, from the completeness of its self{-}activity'' (570). This is the sense in which reason can be called practical, that is, insofar as it can ``determine itself a priori to an action, which has no other purpose than the reality of the way of acting of reason'' (571). Reinhold is trying to establish an understanding of `practical reason' that does not presuppose anything but reason, so that absolute, not comparative freedom can be assigned to it. 

Thus, Reinhold sides with Kant in stating that pure reason can indeed be practical.\footnote{ Cf. \textit{AA} 5:3. The Preface of the \textit{Critique of Practical Reason }opens with the statement that this work has to show ``\textit{that there is pure practical reason}.'' \textit{PP}, 139. } He also follows Kant in not specifying how reason's practicality can be active in sensibility. Like Kant he addresses the issue by means of the effects of the moral law as incentive. According to Reinhold the necessity of the determination of the will by the moral law is what is called duty (\textit{Pflicht}). With regard to practical reason the `ought' that expresses the necessity is a freely willing of lawfulness, whereas with regard to the faculty of desire, it is a commanding (cf. 574). Thus the subject, acting free through practical reason, can only get its orders followed by forcing its selfish drive to comply with them (cf. 574). Comparing this to what Kant says in the third chapter of the Analytic of the second \textit{Critique}, Reinhold conspicuously fails to mention the feeling of \textit{Achtung} for the moral law. The main point of Kant's chapter is that we cannot know how reason can become an incentive, we can only be aware of its effects when it is.\footnote{ Cf. \textit{AA }5:72. } These effects, called \textit{Achtung }by Kant, are described by Reinhold as an `ought' and a `commanding' by practical reason, directed at sensibility. This is not so different from Kant's claim that the feeling of respect supplies authority to the law. Reinhold does choose a different perspective, however. Instead of emphasizing the feeling effectuated by the moral law, he stresses the manifestation of the moral law in sensibility as an `ought'. 

Thus, `practical reason' as it appears at the end of Reinhold's \textit{Versuch} is in many ways a Kantian conception. Reinhold defends the absolute freedom of the will in following the law prescribed be practical reason, while the rational following of natural inclinations is only comparatively free, because of heteronomy. Reinhold's account differs from Kant's in that, for instance, he understands the moral law as being formulated (\textit{verfa\ss{}t}) by theoretical reason, whereas it is sanctioned as an actual law (\textit{wirkliche} \textit{Gesetz}) ``by the mere self{-}activity of practical [reason], which itself imposes it upon itself'' (572). Thus, the principal characteristic of practical reason from Reinhold's perspective is that it sanctions the law of reason, establishing it as a law with authority. 


\section{Evaluation: Practical reason in the Versuch }


In the \textit{Versuch} the term `practical reason' appears to play a less crucial role than in the `Briefe'. It is only empoyed at the end of former work in a section that is not even a proper part of the order of argumentation. In order to understand the role of practical reason in Reinhold's \textit{Versuch} it is important to distinguish between the role played by the concept `practical reason' in the `Briefe' and his use of the actual term `practical reason'. First of all, we need to establish in what way the role that Reinhold attributed to practical reason in the `Briefe' is still relevant in the context of the \textit{Versuch}. After all, his understanding of practical reason was, as we have seen in Chapter 4, crucial for his public endorsement of the Kantian philosophy. In the `Briefe' Reinhold employs the term `practical reason' without strict reference to Kant's practical philosophy. This is not only because there was no second \textit{Critique }yet and Reinhold did not aspire to go into the technical details of Kant's philosophy anyway, but also and most importantly because he uses the term to call attention to what he believes to be the most salient feature of Kantian philosophy: the unification of our rational and sensible capacities it provides. We have also seen with regard to the `Briefe' that `practical reason' is not the only Kantianizing term that Reinhold employs for that purpose. When discussing the problem of the grounds for the conviction of the immortality of the soul, he uses `pure sensibility' to point out that according to the Kantian theory of cognition, both our rational and our sensible capacities need to be involved in order to have cognition. It is in calling attention to these features of Kant's philosophy that Reinhold's personal contribution to the discussion consists, which has developed out of his own pre{-}Kantian Enlightenment engagement. In the \textit{Versuch }this mediating role between our receptive and spontaneous capacities is fulfilled by the theory of the faculty of representation, without reference to `practical reason' or `pure sensibility'. Reinhold no longer needs the Kantian terms to call attention to the core premise of the Kantian philosophy; he has developed his own vocabulary. In section 3.1 the way in which he expresses his insights into the crucial premise of the Kantian philosophy in his theory of the faculty of representation will be discussed. 

Yet Reinhold did not abandon the use of the term `practical reason'. In the `Grundlinien' it figures in a context that is clearly related to Kant's second \textit{Critique}, which was not available when Reinhold wrote the `Briefe'. On the one hand the circumstances suggest that `practical reason' was of subordinate importance with regard to the main argumentation of the \textit{Versuch}. On the other hand, Reinhold apparently had something to say regarding `practical reason' that was important enough to deviate from his framework, which suggests anything but a subordinate role. Section 3.2 will address Reinhold's reasons for adding a discussion relating to practical reason at this point in the \textit{Versuch}. 


\subsection{`Practical reason' / `pure sensibility' and the theory of the faculty of representation}


In order to assess what role the thought behind Reinhold's use of `practical reason' / `pure sensibility' in the `Briefe' plays in the \textit{Versuch}, a brief recapitulation of the results of Chapter 4 will be useful. In the `Briefe' Reinhold had argued that Kant was able to solve the misunderstanding of reason, brought to light by the pantheism controversy regarding the grounds of the fundamental truths of religion. These convictions, that there is a God and an afterlife, were neither to be grounded by sensibility alone, nor by theoretical reason alone, but rather by morality, that is, by practical reason, which was presented as combining sensibility and reason. The details of this grounding remained unclear, because Reinhold was only interested in making a case for the relevance of the Kantian philosophy. The most important point to him was that Kant had overcome the dichotomy of reason and sensibility and he employed the term `practical reason' to express this feature of Kant's philosophy in the context of the grounds for the conviction that God exists. In the context of the grounds for the conviction that the soul continues to exist after the body has died, he employed the term `pure sensibility' to call attention to the same characteristic, that is, that the essential feature of the human mind is the combination of receptive and spontaneous capacities. Both terms have a Kantian ring, but are not used by Reinhold to deal with Kantian arguments. Rather, they are presented as the expressions of the Kantian discovery that cognition only arises as a composite of the activities of our rational and our sensible capacities. As we have argued in the previous chapter, both terms serve to express the same claim, namely that Kant has been able to solve the fundamental problems of his time by means of the discovery of the necessity and possibility of a combination of man's sensible and rational capacities. We have also seen that Reinhold's position develops during the writing of the `Briefe' as, in his elaborations in the context of his use of the term `pure sensibility', he begins to use formulations that appear to foreshadow the `Satz des Bewu\ss{}tseins'. 

 On the basis of the important role of the terminological complex `practical reason'/ `pure sensibility' in the `Briefe' some points that are relevant for our interpretation of the \textit{Versuch }can already be noted. First of all, Reinhold's claims regarding the nature and importance of the Kantian philosophy are not exclusively related to his use of one particular Kantianizing term, but to two different terms that both carry Kantian associations. Secondly, the claim that the Kantian philosophy represents a higher standpoint because it has discovered the nature of the human faculty of cognition can be considered as Reinhold's personal contribution to the debates regarding the Kantian project. If we consider these points in relation to the \textit{Versuch}, the theory of the faculty of representation may be regarded as a continuation of the `Briefe'. After all, it is Reinhold's own addition to the Kantian philosophy and it lays the foundation for the claim that all cognition consists of an interaction between our rational and sensible capacities. Moreover, it is in the Second Book of the \textit{Versuch} that an early form of the `Satz des Bewu\ss{}tseins' figures, indicating a kinship between the ideas expressed there and the development taking place in the `Briefe', especially in relation to `pure sensibility'. 

 Taking a closer look at Reinhold's argumentation for the need for a theory of the faculty of representation as given in the First Book strengthens the case for interpreting the Second Book as a development from the `Briefe'. The five sections of the First Book contain the five steps of an argument for the necessity of a theory of the faculty of representation. The first step establishes the lack of universally accepted principles and grounds of cognition in the fields of religion, morality and natural right. Secondly, this lack of universally \textit{accepted} (\textit{allgemeingeltend}) principles leads to critical doubt concerning universally \textit{valid} (\textit{allgemeing\"{u}ltig}) principles and grounds of cognition. The third section argues for an investigation into the possibility of such principles. This, however, cannot be done before the human faculty of cognition has been thoroughly investigated in order to establish its boundaries, as Reinhold argues in the fourth section. In the `Briefe' of course, he had already claimed that Kant had successfully undertaken such an investigation and he indeed refers to these articles in support of his argument (cf. 149). In a lengthy paraphrase he argues that the investigation of reason or the faculty of cognition has become inevitable given the current misunderstandings regarding that faculty.\footnote{ It must be noted, however, that the version that is found in the \textit{Versuch} is not taken directly from the first `Brief' as published in the \textit{Merkur}. It appears to be somewhere in between the first \textit{Merkur}{-}`Brief' and the later edition of \textit{Briefe I}, which appeared in 1790. Cf. Reinhold, \textit{Briefe I}, 93{-}103; `Erster Brief', 117{-}123. Reinhold was working on his \textit{Briefe I }more or less alongside his work on the \textit{Versuch} and revised the text several times. Cf. \textit{RK} 2:63, Letter 145, February 26, 1789, to G\"{o}schen. Reinhold states the following: ``Meine Briefe \"{u}ber die Kantische Philosophie sind rein abgeschrieben und warten auf Durchsicht und zum Theil Umarbeitung.'' At the time when Reinhold was working on the First Book of the \textit{Versuch}, he would have had an early version of what would become \textit{Briefe I} available. } The relevance of the Kantian project had been questioned, however, by \textit{Popularphilosophen} claiming that the nature of the faculty of cognition is understood by common sense. Reinhold replies that these philosophers have not understood Kant (cf. 155{-}156). He then argues against the claim that everybody knows what cognition is. The most commonly shared view on reason is that it is the capacity for syllogistic reasoning. The right form of reasoning, however, by no means implies true conclusions, so this may not be the most relevant perspective on `reason' (cf. 160). There is less agreement among philosophers when it comes to reason's metaphysical capacities, that is, the capacity to come to conclusions with regard to supersensible objects, since they disagree about the question where the material for those conclusions would come from (cf. 162{-}163). In order to find out whether or not reason possesses a metaphysical capacity, this faculty must be investigated in its relation to the understanding, which ``first works on the raw materials received by sensibility and hands them to reason'' (173). Understanding, in turn, if it is not to be a mere logical capacity, relies on the material supplied to it by sensibility. It is therefore to sensibility that the investigation must be turned in order to find out the relation between sensibility and understanding. As in the `Briefe', this question is closely related to the question of the nature of the representing subject. Reinhold proposes taking the faculty of cognition as the starting point instead of the material or spiritual nature of the subject (cf. 178). The materialists see sensibility as a foundation of the faculty of cognition as a whole, whereas the spiritualists deny the role of sensibility when it comes to knowledge. Given the different opinions of philosophers on both reason and sensibility it is not surprising that there is no shared opinion regarding the faculty of cognition. Reinhold concludes the section with the claim that the philosophers ``are not in agreement with one another, or with themselves when it comes to the meaning of the term `cognition''' (188). 

 Of course, this is exactly the conclusion he needs to refute the claim that everybody knows what the faculty of cognition is. By the time Reinhold undertook the writing of the \textit{Versuch} it was clear to him that, in spite of his efforts, this analysis of the importance of the Kantian philosophy was not widely shared. Thus, instead of arguing for the importance of Kant, Reinhold, in his final section of the First Book, argues for the need of an investigation of the concept of the faculty of representation. The argumentation consists of the claims that the concept of representation is presupposed by that of cognition and that misunderstandings regarding the faculty of cognition are usually due to differences in the concept of the faculty of representation (cf. 189). This is intuitively plausible, but does not really constitute an argument. It remains to be seen what is to be gained by an investigation of that faculty for the establishment of universally accepted principles, since the philosophers only agree on the fact that there are representations and by no means on the content of the concept `representation'. To be sure, agreement on the existence of representation is indeed a marked advantage in comparison to `cognition', the existence of which is doubted by the skeptical party. However, it is not clear what good it would do to agree that there is such a thing as representation, when it is apparently very hard to agree on the marks of that concept. Reinhold's further argumentation in favor of an investigation of the faculty of representation only states that once its ``essential mark'' has been found, it will be very easy to develop this in a universally valid manner and to find the criterion for `representability'(191). The main work in this section is done by the negative argumentation showing the lack of agreement on the faculty of cognition. Reinhold makes clear that the main area on which the philosophers do not agree is the relation between our rational and sensible capacities. The reason they do not agree is found in their taking metaphysical status of the subject as their starting point, that is, their presumptions on the materiality or spirituality of the subject. Apart from the direct reference and extensive paraphrase from the `Briefe' the matter of the argument is related to the seventh and eighth installments. There, the proper relation of sensibility and rationality was also an issue, with reference to the problem of the continued existence of the soul. There, Reinhold argued that Kant had discovered the proper relation between our sensible and rational capacities. 

For the Critique of Reason explained for the first time sensibility as an essential part of our faculty of cognition that is present in the mind before all sensation and before all receptivity of the organs (\ldots ), and it showed the essential cooperation of sensibility with the understanding in all actual cognition. (\textit{Letters}, 97; Siebenter Brief, 155)

In the \textit{Versuch}, however, Reinhold abstains from referring to Kant's solution, since the very relevance of the Kantian \textit{Critique} had been doubted by the popular philosophers. Instead, he provides his own theory, namely on the faculty of representation. Given the direct and indirect references to the `Briefe', the project of that theory, that is, of the Second Book of the \textit{Versuch}, must be understood as closely related to what in the `Briefe' was described as the Kantian solution. That is to say, the Second Book of the \textit{Versuch} should be read as aiming to prepare an audience that is potentially hostile to Kant for the results that Reinhold ascribed to the Kantian philosophy in the `Briefe'. We may therefore expect it to pave the way for a theory of cognition according to which sensibility and understanding are presented as cooperating throughout the formation of cognition. 

 This is indeed the case. In the introductory sections of the Second Book (\S \S  6{-}14) Reinhold argues that his enquiry concerns `representation' in general, without regard to the special kinds of representation, that is, intuitions, concepts and ideas. From section 15 onwards he starts to develop the two essential characteristics he believes to have discovered by means of the conceptual analysis of `mere representation', `material' and `form'. He bases this conviction on the `Satz des Bewu\ss{}tseins' as expressed in its embryonic form in section seven, stating that subject and object belong to any representation and are, at the same time, distinct from it.\footnote{ Cf. footnote \ref{footnote:_Ref229468623}. } Both subject and object determine a distinct but necessary element of representation. The material relates to the object, while the form relates to the subject (cf. 237). The next step argues that the material of any representation is given from the side of the object, while the form is produced by the subject (\S  18). Reinhold's reasons for stating that they cannot be both given or produced is that that would render the distinction of subject and object, required by consciousness, impossible (cf. 257). This means that on the one hand every representation must have a given material, even ideas of reason. On the other hand, even the most basic representations, intuitions, involve the activity of the subject, producing the representation by giving form to the material. These implications of Reinhold's thoughts on the faculty of representation refute both the empiricist and the rationalist approach to the formation of cognition. It is therefore no wonder that in section 18, Reinhold restates his criticism that Leibniz and Locke presented one{-}sided accounts by exclusively focusing on the production of the form and the givenness of the material respectively (cf. 260{-}261). This is precisely the one{-}sidedness that Reinhold claimed Kant had remedied with his theory of cognition. That theory, however, was by no means securely established since its relevance had been called into question. By situating the combination of man's receptive and spontaneous capacities at the level of consciousness, Reinhold aims to overcome the difficulties surrounding the Kantian philosophy, arising from the lack of consensus regarding the faculty of cognition. Granting his claim that any concept of `cognition' will presuppose the concept of `representation', Reinhold's strategy of introducing the combination of receptivity and spontaneity at the level of representation should imply that, on the level of cognition, there will be a similar unity of elements. 

 In the following we shall see that this framework did not work too well for Reinhold's practical concerns, yet this was not fully clear to him when he finished his \textit{Versuch}. On the contrary, most probably he congratulated himself on his way of arguing for the importance of the discovery made by Kant without presupposing the Kantian philosophy itself. He had already pointed out this discovery in the `Briefe', which unfortunately had not succeeded in convincing the entire philosophical world. This meant that he had to find a way to present the same discovery, that both our receptive and spontaneous capacities must work together to produce cognition, without presupposing the Kantian theory of cognition. With the claim that `representation' is prior to `cognition' Reinhold's theory of the faculty of representation is prior to Kant's theory of the faculty of cognition. The theory is meant to remedy the incomplete conceptions that people have of `representation' and show that the Kantian theory is built upon the proper concept of representation, that is, the concept that entails that in any representation both receptivity and spontaneity have a role to play. The fact that Reinhold's attempt was not entirely successful and that the foundational role he had in mind for `representation' was by no means universally accepted, does not mean that we cannot understand that this is what he set out to achieve. Reinhold's claim, cited earlier in the present chapter, that he had worked on the theory of the faculty of representation for four years,\footnote{ Cf. footnote \ref{footnote:_Ref213408642}.} which is at first sight blatantly untrue, becomes more understandable in light of what we have seen in this final section. After all, the core claim of that theory was exactly what Reinhold had spotted as the most relevant and interesting feature of the Kantian philosophy several years earlier. It is also no wonder that he identified the claim of the necessary combination of rational and sensible capacities at a very early stage of his study of Kant, as it was precisely what he had been looking for throughout his philosophical career up to that time. 


\subsection{Section 86 and Rehberg's review of Kant's second \textit{Critique}}


Now that we have shown how Reinhold transforms his insights into what he considers the crucial feature of the Kantian philosophy, into a theory of the faculty of representation, the question remains what happened to his usage of the term `practical reason'. It has obviously been removed from the context in which it was all{-}important to the Kantian project as a mediator between our receptive and spontaneous capacities, for this role is now fulfilled by the theory of the faculty of representation. Furthermore, the publication of Kant's second \textit{Critique }would have made it hard for Reinhold to maintain his previous use of `practical reason' as if it was in line with Kantian philosophy. As we shall see below, Reinhold's use of the term in the \textit{Versuch} is indeed closely related to the context of the \textit{Critique of practical reason}, especially to Rehberg's review of that work. 

The term only figures in the `Grundlinien', which, being a part of section 86 must be regarded as a sideline to the main argument of the \textit{Versuch}. In order to understand Reinhold's use of the term `practical reason' in this context we must turn our attention to the question why Reinhold would want to make an excursion like that. As we have seen above (\ref{subsubsection:_Ref228718282}) the new terminology introduced in section 86 relates to the question of human freedom, more specifically the causality of reason with respect to the faculty of desire and the question whether this freedom is absolute or comparative. The wider context therefore includes Reinhold's earlier discussion relating to freedom and the subject as absolute cause in section 83. Reinhold's thoughts on freedom as we find them there can be termed the `theory of degrees of spontaneity'.\footnote{ Cf. Lazzari, \textit{Das Eine, was der Menschheit Noth ist}, 95. } He opens with the claim that the absolute subject can only be called an absolute cause with regard to what is produced by mere reason with respect to mere representations and explains this thesis in the following manner (cf. 534). The absolute subject is to be understood as acting, in so far as it is the subject of spontaneity. There are, however, different kinds of representation, namely intuition, concept and idea, in the production of which spontaneity expresses itself in different ``degrees of activity'' (535). In intuitions, the activity of spontaneity consists in synthesizing a given manifold, which is the lowest degree of spontaneity, as the activity is only a forced reaction to the manifold affecting the receptivity of the faculty of representation (cf. 535). In this context it is worthwhile to point out that Reinhold's account here is consistent with his introduction of a low degree of spontaneity in the Second Book. There he attributed spontaneity to the spring of a watch in so far as the ground why it opposes (\textit{entgegenwirken}) the tension of being wound up is internal to the spring (cf. 269). Similarly, the faculty of representation possesses spontaneity in so far as the ground of the activity in the representation cannot be found outside the faculty of representation, as the production of the form of representation must have its ground in the faculty of representation itself (cf. 269{-}270). This understanding of spontaneity is confirmed by the `theory of degrees of spontaneity' in section 83, where the first and lowest degree is also described as a reaction (\textit{entgegenwirken}) to a given manifold. 

 In contrast, the second degree of spontaneity, which is active when the understanding produces a representation, is unforced. The activity of the understanding consists in the synthesis of a manifold that has already been synthesized in intuition, which synthesis is therefore not a reaction to something given from the outside and spontaneity is only determined by itself (cf. 536). However, as the understanding always synthesizes a manifold of the intuition it is bound by the forms of intuition, by means of which the given manifold is represented. Because of this Reinhold says that the understanding can only be regarded as an absolute cause of the production of concepts, not as an absolute cause of their form, because it acts in unison with sensibility (cf. 536).

 Finally, Reinhold discusses the activity of reason, which consists in the synthesis of the manifold that is determined ``in the mere nature of the understanding and by the mere form of concepts,'' that is, it is a synthesis of the ``concepts in so far as they are the mere products of the second degree of spontaneity'' (537). As these concepts that make up the manifold subsequently synthesized by reason are themselves products of spontaneity, in the activity of reason, spontaneity synthesizes a material that is not given to it by sensibility and is therefore no longer bound to the forms of sensibility. This brings Reinhold to the following conclusion. 

In wieferne also das vorstellende Subjekt durch Vernunft handelt, in soferne handelt dasselbe als \textit{absolute Ursache, ungezwungen, ungebunden}, durch nichts als seine Selbstth\"{a}tigkeit bestimmt, das hei\ss{}t, \textit{frey}. (537)

[To the extent that the representing subject thus acts through reason, in so far it acts as \textit{absolute cause, unforced, unbound}, determined by nothing but its own spontaneity, that is, \textit{free}.]

The connection between being an absolute cause and being free is reinforced by identifying being an absolute cause with being a free cause immediately following this passage. Furthermore, freedom can only be attributed to the representing subject in so far as it is ``the subject of \textit{reason}'' (537). 

 From the `theory of the degrees of spontaneity' it appears that Reinhold has succeeded in founding the freedom of the representing subject upon the structure of the faculty of representation. At least, he has identified the absolute subject as an absolute cause, which to him is the same as a free cause. Both the absolute subject and absolute causation, however, are only thinkable, not knowable. This means that the freedom established in this context is ``incomprehensible as to its \textit{real} possibility'' (538). \footnote{ Probably in the same sense as `the existence of God' and `life after death' were termed incomprehensible in the `Briefe'. Cf. \textit{Letters}, 63 (Vierter Brief, 140{-}141); \textit{Letters}, 69 (F\"{u}nfter Brief, 174{-}175). } This is in line with the requirement made in the First Book, that philosophy should establish the \textit{logical} possibility of freedom, that is, it should establish that its concept is non{-}contradictory (cf. 93). There Reinhold assumed that the non{-}philosophical mind would be convinced of the reality of freedom because it is aware of itself (\textit{durch das Selbstgef\"{u}hl bewu\ss{}t}), while the philosophers could not even agree on its possibility. Therefore, in order to secure the \textit{Selbstgef\"{u}hl }of freedom against skepticism, the establishment of the possibility of freedom will count as ``\textit{philosophical ground of cognition} of the fundamental truth of morality'' (93). 

 Having accomplished with respect to freedom what he set out to do, why did Reinhold return to the subject of freedom and introduce the distinction between comparative and absolute freedom in the curious section 86? The simultaneous introduction in section 86 of the faculty of desire and the activity of reason with regard to that faculty points to problems with the application of freedom, not to problems regarding its possibility. As mentioned above, Reinhold had only hinted at the different concrete expressions of freedom vis{-}\`{a}{-}vis the theoretical and practical faculty of representation. 

 In order to understand why Reinhold did return to the matter we must turn to factors beyond the cover of the \textit{Versuch}. The context in which Reinhold was feverishly trying to get his first substantial monograph ready before the book fair was full of debates on human freedom and the possibility of its actuality. More specifically, Kant's moral philosophy, greeted enthusiastically by Reinhold,\footnote{ \textit{RK} 1:312{-}313, Letter 84, January 19, 1788, to Kant. Reinhold writes: ``Wie lieb ist mirs nun da\ss{} ich mich in meinen \textit{Briefen \"{u}ber die kantische Philosophie} bis itzt noch nicht auf die eigentliche Er\"{o}rterung des \textit{moralischen Erkenntni\ss{}grundes der Grundwahrheiten der Religion} eingelassen habe. Ich h\"{a}tte da ein schwaches L\"{a}mpchen aufgesteckt, wo Sie durch die Kr. d. p. V. eine Sonne hervorgerufen haben. Ich mu\ss{} gestehen, da\ss{} mir ein solcher Grad von Evidenz, eine so ganz vollendete Befriedigung, als ich wirklich gefunden habe, unerwartet war.'' } was under fire. Jacobi had published a second edition of his \textit{Ueber die Lehre des Spinoza} in the spring of 1789.\footnote{ Jacobi, \textit{Ueber die Lehre des Spinoza in Briefen an den Herrn Moses Mendelssohn. Neue vermehrte Ausgabe} (Breslau: L\"{o}we 1789). Reinhold had received a copy of this second edition as a gift and was impressed by it. Cf. \textit{RK} 2:172{-}173, Letter 176, October 18, 1789, to Jacobi. } Its introduction contains a small treatise `Ueber die Freyheit des Menschen', in which it is argued that man has no freedom. Jacobi concludes that he has shown how ``moral laws, which are called apodictic laws of practical reason'' come to be and that in this way ``only mechanism and no freedom'' follow.\footnote{ Jacobi, \textit{Ueber die Lehre des Spinoza}, xxxiv. } The reference to Kant is unmistakable and it is suggested that a Kantian moral theory is compatible with a mechanistic view of the origin of moral law in which there is only apparent freedom. The Jena professor Ulrich had, in his book on freedom \textit{Eleutheriologie} (1788), proposed a form of determinism and attacked Kant for not specifying the nature of human freedom.\footnote{ Ulrich, \textit{Eleutheriologie oder \"{u}ber Freyheit und Nothwendigkeit} (Jena: Cr\"{o}ker 1788). The attack on Kant was condemned by Christian Jakob Kraus in his \textit{ALZ} review of the work. Cf. Kraus, review of \textit{Eleutheriologie oder \"{u}ber Freyheit und Nothwendigkeit}, by Ulrich, \textit{ALZ}, April 25 (nr. 100), 1788. } Reinhold wrote to Kant that Ulrich ``abused'' Kant's theory of freedom.\footnote{ \textit{RK} 1:316{-}317, Letter 84, January 19, 1788, to Kant. } These are but a few examples to indicate that Kant's moral philosophy could not count on such enthusiasm such as Reinhold's everywhere. 

 One piece that is of particular interest here is Rehberg's review of Kant's second \textit{Critique}.\footnote{ Rehberg, review of \textit{Kritik der praktischen Vernunft}, by Kant, \textit{ALZ}, August 6 (nr. 188 a and b), 1788. } This review may have played an important role in Reinhold's decision to add section 86 to his \textit{Versuch} and to consider the freedom of reason in its practical application in some detail. First of all, Rehberg had questioned one of the core claims of the second \textit{Critique}, namely the claim that pure reason can be practical. He had argued that if pure reason is to have any effect on our (empirical) decision{-}making, it must become empirical too. Reinhold's section 86, including the `Grundlinien', deals explicitly with the determination of the faculty of desire by reason. Reinhold would have had good cause to react if he believed this criticism to be potentially lethal to Kant's theory of freedom. The overthrow of Kant's theory would be disastrous for Reinhold's own attempts to secure the same theory by means of a theory of the faculty of representation. Furthermore, Reinhold may have thought that his theory of freedom put forward in the `theory of degrees of spontaneity' would solicit the same criticism from Rehberg. Reinhold had only established the absolute causality of the absolute subject as a possibility, whereas Rehberg demanded that the actual practicality of pure reason be established in order for a Kantian theory of freedom to be viable. As he (Rehberg, that is) deemed this impossible, he proposed a `slight modification' of the Kantian theory, which in effect was a form of Rehberg's own Spinozism. If Rehberg raised the same to parts of the \textit{Versuch}, Reinhold's theory might become publicly associated with Spinozism as well. Naturally, he would want to avoid that, especially because the whole point of the exercise was to convince the world that Kant (and of course he himself as well) had in effect overcome the past struggle in which Spinozism was only one of four opposing parties. Moreover, he assumed that Rehberg would actually review his \textit{Versuch}. There is also textual evidence hinting at a role for Rehberg's review. The terminology of comparative and absolute freedom that Reinhold introduces in section 86 is crucial to both Kant's claim that, for morality, absolute freedom is needed and Rehberg's claims that Kant has not sufficiently shown that there is such a thing as absolute freedom and that mere comparative freedom suffices for morality. The fact that Reinhold in section 86 introduces this debated terminology to address an issue that is debated by Rehberg, one of its main polemical users, strongly indicates that Reinhold at this point seeks to address the issues put forward by Rehberg, which he may well have feared could be raised against his own theory as well. By trying to establish absolute freedom on the basis of his own theory of representation, Reinhold could present Rehberg's criticism as stemming from one of the many misunderstandings of Kant that are remedied by the universally acceptable theory of representation. 

  In his \textit{Critique of Practical Reason} Kant had used the term `comparative freedom' several times in relation to the problem of accountability in a world determined by natural causal laws. One who acts according to an inevitable law of nature cannot be held accountable for his actions. Kant believes it is ``a wretched subterfuge to seek to evade this by saying that the \textit{kind} of determining grounds of this causality in accordance with natural law agrees with a \textit{comparative }concept of freedom.''\footnote{ \textit{AA} 5:96; \textit{PP}, 216{-}217.} The reason why it is such a miserable defense is that, for instance, the movement of the hands of a watch is not determined externally, but is still determined in accordance with natural causal laws.\footnote{ This would be a more or less Spinozist approach to the issue of freedom, which Kant clearly rejects. In light of Rehberg's criticism it is worthwhile to note that Rehberg had fewer problems with Spinozism. Cf. di Giovanni, \textit{Freedom and Religion}, 125{-}136. As we have seen above, Reinhold, on the basis of his own theory of the degrees of spontaneity, had no trouble attributing the lowest degree of spontaneity, that of \textit{entgegenwirken} to the activity resulting from the spring of a watch (cf. \textit{Versuch}, 269{-}270). } The agent cannot be held responsible for any action following from this comparative kind of freedom, which is only freedom from one kind of natural causation (external), but not from another (internal) kind of natural causation. On the basis of the second \textit{Critique} C. Ch. E. Schmid describes comparative freedom in his \textit{Dictionary }as follows: ``when only a certain type of cause, e.g. external, mechanical, does not determine the action necessarily.''\footnote{ C. Ch. E. Schmid, \textit{W\"{o}rterbuch zum leichtern Gebrauch der Kantischen Schriften} (Jena: Cr\"{o}ker 1788; 2nd edition), 178: ``Freyheit; \textit{relativ}, comparativ; wenn nur eine gewisse Art von Ursachen z.B. \"{a}ussere, mechanische, die Handlung nicht nothwendig bestimmt.'' } As we shall see, Reinhold's definition of comparative freedom is closely related to Schmid's. A similar definition must have been at the back of Rehberg's mind. 

In his critical assessment of Kant Rehberg used the distinction between absolute and comparative freedom in such a way as to conclude that morality is connected to comparative, not to absolute freedom.\footnote{ For an evaluation of Rehberg's review from a Kantian point of view, cf. Schulz, \textit{Rehbergs Opposition}, 9{-}42; for an evaluation more sympathetic to Rehberg's points of criticism on Kant, cf. di Giovanni, \textit{Freedom and Religion}, 125{-}136. } He deems absolute freedom, which he connects to the practicality of pure reason, to be improvable.\footnote{ Cf. Rehberg's review of \textit{Kritik der praktischen Vernunft}, 357. } His line of argument is as follows. If we are to know that there is such a thing as absolute freedom, we must know that pure reason can indeed be practical, that is, can be the cause of an empirically perceivable action. However, we cannot understand how this might be the case. Therefore, we cannot accept that pure reason is practical and we cannot accept absolute freedom. Rehberg takes Kant's discussion of the moral feeling of \textit{Achtung} as an attempt to provide a transition from pure reason to action in the sensible world and claims that this attempt fails.\footnote{ Rehberg's review of \textit{Kritik der praktischen Vernunft}, 353{-}354. Cf. Schulz, \textit{Rehbergs Opposition}, 16{-}20.} From this failure Rehberg concludes that any attempt to prove the reality of absolute freedom must fail {--} note that Kant never claimed that it could be proven. In order to guarantee the connection between moral reason and action in the world of the senses, reason must depend for its existence on God. The reason why the noumenal subject can be said to be active in the phenomenal world, as Rehberg requires, is that God must be regarded as the creator of both the noumenal subject and the phenomenal world. Because of its dependence on God, however, the subject is not completely independent and hence only comparatively free. At the same time Rehberg appears to agree with Kant that freedom entails freedom from natural causation, as he insists that morality is not endangered by his (Rehberg's) version of comparative freedom, for it is still based on reason. Reason's being part of a ``system of intelligible necessity'' is no issue for Rehberg. We would not be able to get rid of this ``intelligible fatalism'' without at the same time rejecting the idea of an original being.\footnote{ Rehberg's review of \textit{Kritik der praktischen Vernunft}, 357 } Although Rehberg claims his theory is only a slightly moderated version of Kant, it is not. First of all, Kant rejected the idea that God is to be regarded as the creator of the phenomenal world, as well as the noumenal world.\footnote{ Cf. \textit{AA} 5:102. } Further, Kant saw absolute freedom as the only guarantee for the independence of morality from the sensible world and natural inclinations. With regard to the distinction between absolute and comparative freedom, it must be noted that Rehberg does comply with Kant's objections against comparative freedom, which demand that neither external nor internal natural causes are involved. He introduces another kind of comparative freedom, however, consisting in the dependence of reason on God. The related claim is that reason on its own cannot relate to sensibility, that is, that pure reason cannot be practical. 

Reinhold's description of comparative freedom, ``in case of which a certain kind of foreign [\textit{fremde}] cause does not determine the action necessarily'' (558), is very similar to Schmid's. Reinhold describes its opposite, absolute freedom, as follows, ``in case of which no foreign cause at all contributes to the determination of an action'' (558). Since Reinhold adds `foreign' to `a certain kind of cause' it is clear that for him the distinction between comparative and absolute freedom is about the involvement of foreign causes; an action being absolutely free if no foreign causes at all determine it, while comparative freedom only rules out a specific type of foreign causation. Rehberg's comparative freedom would fit this description as well, as dependence on God would entail dependence on a cause foreign to human reason, even if it also entails independence from sensibility, which would be another kind of foreign cause. According to Reinhold, causes foreign to reason can be at work even if there is no empirically given material, thus denying Rehberg's contention that morality would depend upon reason alone, for it would depend on the cause of reason (God) as well. Rehberg's description of reason as dependent upon its cause bears resemblance to Reinhold's own description of the comparatively free activity of reason in thinking. In thinking, Reinhold admits, reason only acts comparatively free, as its material (the manifold of categories) is not created by it, but given in the structure of the faculty of representation (cf. 559). In a way the structure of the faculty of representation is a condition for Reinhold's `reason', just as God is a condition for Rehberg's `reason'. In claiming that this kind of freedom is not sufficient for morality Reinhold denies Rehberg's claim that his version of comparative freedom would save morality, without further argumentation. Rehberg reinterpreted `comparative freedom' so as to entail internal dependence upon non{-}sensible causes. Precisely such a description would apply to Reinhold's account of the free activity of reason as the highest degree of spontaneity as well.\footnote{ Although Lazzari attributes Reinhold's attempt to prove absolute freedom here to problems with the notion of `absolute cause' and the theory of degrees of spontaneity (cf. Lazzari, \textit{Das Eine, was der Menschheit Noth ist}, 95{-}108), I believe that Reinhold has in principle proved enough with this theory. It would suffice to analyse the structure of reason so as to ensure that we need to think of the representing subject as an absolute cause, i.e. a free cause. Lazzari's claim that Reinhold's theory of degrees of spontaneity is not sufficient because it does not prove that the subject is absolutely free, only makes sense in so far as it can be shown that proving absolute, not comparative freedom is Reinhold's aim from the outset. However, I do not believe that this can be shown conclusively, as the distinction between absolute and comparative freedom does not occur, except in section 86. In the First Book, where Reinhold introduces freedom as the basic truth (\textit{Grundwahrheit}) of morality, he claims that common sense is as convinced of the reality of freedom ``as of a fact.'' It is up to the philosophers to prove at least the logical possibility of freedom, that is, to present freedom as something thinkable. Cf. \textit{Versuch}, 91{-}94; Cf. \textit{AA} 5:47.} For, although this activity is independent of sensibility, its material still needs to be given. Assuming that Reinhold took Kant's claim that morality requires absolute, not comparative freedom seriously, he would need to make sure that the activity of reason with regard to morality involves absolute, not comparative freedom. In section 86 Reinhold indeed admits that the activity of reason in thinking entails only comparative freedom. The next step is showing that, contrary to Rehberg's contentions, the activity of reason in morality must be thought of as being absolutely free, as it must be understood as determining the faculty of desire in an a priori way. 

 The `Grundlinien der Theorie des Begehrungsverm\"{o}gens' serves precisely this purpose. As we have seen, Reinhold needs it in order to be able to make claims regarding the faculty of desire, introduced in section 86. In the `Grundlinien' he gives a sketchy account of different kinds of drives, related to the receptive and spontaneous capacities of our faculty of representation in relation to the representing power.\footnote{ This is not to say that Reinhold claims here that the faculty of desire is no more than a species of the faculty of representation, as some commentators have claimed from a Fichtean perspective. Cf. Klemmt, \textit{Karl Leonhard Reinholds Elementarphilosophie}, VII; more recently cf. Lohmann, `Reinholds Philosophie im Spiegel der Kritik von Heydenreich und Fichte,' 93. For the contrary view, i.e. that the faculty of representation becomes a species of the faculty of desire, cf. Beiser, \textit{The Fate of Reason}, 264{-}265. Beiser relates the problems Reinhold encounters in this field to Wolff's problems with his `single{-}faculty' theory of mind. On the relation of the faculty of desire to the faculty of representatition, cf. also Gerten, `Begehren, Vernuft und freier Wille,' 156{-}159. To me it is not obvious that Reinhold himself had a clear idea on the relation between the faculty of representation and the faculty of desire. We have already seen that in \textit{Beytr\"{a}ge I} he reconsidered the structure of his theory, which restructuring appears to be related to the effort to find the proper place for the faculty of desire. In the following chapter we shall see that in the end Reinhold abandoned his efforts to introduce the faculty of desire in terms of the faculty of representation. } In the purely rational, moral drive Reinhold identifies pure activity of reason, as the object of this drive is the realization of the form of reason itself (569). Therefore, it needs nothing foreign to itself, and is an a priori drive. Rehberg would of course object that it cannot be proven that this a priori drive determines the faculty of desire and that we therefore cannot say that pure reason is practical. Like Kant, Reinhold does not attempt to \textit{prove }the reality of absolute freedom and of its influence on the faculty of desire, but addresses the issue through the effects of the moral law as an incentive. In comparison to Kant's account in the third chapter of the Analytic of the second \textit{Critique}, the feeling of \textit{Achtung} for the moral law is missing from Reinhold's picture. This does not mean, however, that Reinhold rejects the thought that the effects of the determination of the will by the moral law are felt. Like Kant, Reinhold does not claim to know how reason can become an incentive, but rather focuses on its effects when it is. These effects, called \textit{Achtung }by Kant, are described by Reinhold as an `ought' directed at sensibility. This `ought' is a ``\textit{commanding}'' with regard to the faculty of desire, although it is a ``\textit{free willing}'' with regard to practical reason itself (574). Instead of emphasizing the feeling that is effected by the moral law, Reinhold stresses the manifestation of the moral law in sensibility as an `ought'. This neatly avoids Rehberg's suggestion that the moral feeling is a mediator. With Kant, Reinhold holds that no mediator is needed; moral feeling is the effect of practical reason being both rational and practical.

 Since Reinhold's main defense against Rehberg at this point consists in reasserting the Kantian point of view in relation to his own theory of the faculty of representation, it is not very likely that Rehberg would have been impressed. He could still claim that Reinhold, like Kant, needs to prove that a pure and absolutely free reason can influence our faculty of desire a priori. This means that the Kantian{-}Reinholdian concept of absolute freedom is still under pressure.\footnote{ In my opinion Lazzari does not sufficiently relate this pressure on Reinhold's thoughts on freedom to Rehberg's review. Cf. Lazzari, \textit{Das Eine, was der Menschheit Noth ist}, 157{-}163.} In the following chapter we shall see how Reinhold, instead of elaborating a further defense, comes to share at least part of the criticism. In the second volume of the \textit{Briefe \"{u}ber die Kantische Philosophie} (1792) he gives up on the Kantian identification of the free will with practical reason. He admits that Rehberg's review of Kant's second \textit{Critique }has taught him that reason cannot be called practical ``as if it would entail the complete ground of an act of will, determined by itself.''\footnote{ \textit{Briefe II}, x. } 

Upon the interpretation provided here section 86 and the `Grundlinien' are to be regarded as a pre{-}emptive strike at Rehberg, who, so Reinhold correctly believed, would probably review his \textit{Versuch}. Given the way Rehberg had reacted to Kant's theory of freedom, Reinhold may well have felt that his views as expressed in the `theory of the degrees of spontaneity' would be vulnerable to a similar reproach, namely, that showing the possibility of freedom of a noumenal subject is not sufficient. Rehberg had demanded proof of the actual existence and activity of this absolute freedom. Thus, the question concerns the relation between Reinhold's theorizing about the nature of the faculty of representation and the activity of an actual (empirical, not absolute) subject. Reinhold's response to this potential criticism is an attempt to relate Kant's thoughts on the matter as provided in the third chapter of the Analytic of the second \textit{Critique} to his own theory of the faculty of representation. In order to do this, however, he needs to break the framework of that theory and appeal to representing power, one of the external conditions of representation. The introduction of the distinction between absolute and comparative freedom does not harmonize with the thoughts on freedom provided by the `theory of degrees of spontaneity', for the kind of freedom established in that context is called `comparative freedom' in section 86. This has led Lazzari to interpret this section including the `Grundlinien' as a revision of that theory, stemming from Reinhold's insight that it would not be sufficient for his purpose, that is, the establishment of absolute freedom. Based on that interpretation, section 86 is a regular part of the structure of the theory of reason, aiming at establishing something that is not to be had within the framework of the \textit{Versuch}, hence the supplementary argumentation in the `Grundlinien'. However, I do not believe that Reinhold needed to introduce the distinction between comparative and absolute freedom, since he had already established the possibility of absolute (that is, free) causality with regard to the absolute subject, which is exactly what he had set as the task for philosophy. The need to introduce this distinction arises instead from the fact that precisely the sufficiency of establishing the possibility of freedom of the absolute subject had been questioned by Rehberg. It is that criticism that Reinhold wants to avoid at a considerably high cost. It means that he needs both to break with the structure of his \textit{Versuch} and to discuss a distinction (between absolute and comparative freedom) that is, to say the least, problematic within the framework of the theory of the faculty of representation. 

