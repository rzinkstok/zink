
\chapter{Conclusion}


The present study has presented the Kantianizing phase of the philosophy of Karl Leonhard Reinhold from the perspective of his pre{-}Kantian concerns. The main premise underlying the project has been that Reinhold's education in Josephite Vienna and his early philosophical concern were of the highest importance for the way he would not only understand the Kantian philosophy, but also set out to present it. As we have seen in the first chapter, Reinhold was exposed to an extraordinary range of influences during his formative period. Apart from being thoroughly familiar with the Baroque Catholicism of late 18\textsuperscript{th} century Austria, he also became acquainted with philosophy within the monastic curriculum. That may not seem very exciting, but the circumstance that he had a philosophy teacher, Pepermann, that introduced him to the English language, literature and philosophy as well proved to be highly significant. It meant that he could and would access a work like Locke's \textit{Essay} in the original. Not only did Pepermann include British philosophy in his lessons, he also went beyond the textbooks, for example by letting his pupil read Malebranche. Because of the relative openness of the Barnabite Order, Reinhold could still socialize with the friends he knew from his time as a Jesuit novice. They introduced him to Enlightenment in the form of Freemasonry and Illuminatism. The Illuminate view of human history, as presented by Weishaupt had a lasting influence on Reinhold's way of understanding and presenting the history of religion and philosophy. Apart from the specific influence of Freemasonry and Illuminatism, Vienna provided Reinhold with experience on a specific form of Enlightenment and its consequences. He was confronted with the abolition of the Jesuit Society when he was just a teenager. As a reviewer for the Viennese \textit{Realzeitung} he dealt with the results of the widened freedom of the press, which resulted in an explosion of publications. From the works he reviewed, it is clear that issues of religion, especially in combination with issues of Enlightenment were heavily discussed in the circles in which Reinhold now moved. All of these factors, together with those (Platner, Herder) he encountered on his way to Weimar, played a role in determining Reinhold's outlook on Enlightenment. 

 In the second chapter, this outlook on Enlightenment as it manifested itself in a range of Reinhold's pre{-}Kantian writings was presented and analyzed. Being one of the first to address the question `what is Enlightenment?', he took into account the course of history and his view on human nature and human reason. The role of history is not only related to particular issues regarding the question whether certain institutions are still adequate given the changes in society. The history of human reason also plays a crucial role in arguing for the inevitability of Enlightenment at a certain point in history. Reinhold's considerations concerning Enlightenment also relied on his views of human nature, in which he distinguishes between capacities that deal with the real world (`heart', `sensibility'), and capacities that deal with and abstract world of thought (`mind', `reason'). True and effective Enlightenment can only occur when the ideas and plans developed by the elite bear the fruit of action, are concretized in the real world. Apart from the fact that from his earliest writings on, Reinhold aims for the synthesis of reason and sensibility, it is also clear that he has a pedagogic vocation that is connected to his membership of Freemasonry and the League of Illuminati. This vocation is elitist in the sense that Reinhold sees himself as a member of the enlightened intellectual elite, who should try and enlighten others as far as possible. The focus of this Enlightenment is on religion, which should be purified from superstition. The pure religion that Reinhold and his fellow{-}Illuminati adhere to appears to differ considerably from the common forms of Christianity. It is based on morality, which is understood as the rationalization of the natural inclination towards one's family. This rationalization allows the inclination to be universalized, so that morality consists in loving one's neighbor as one's brother. In this understanding of morality we also see the combination of heart and mind, of reason and sensibility.

 Reinhold's conviction that religion was to be based on morality instead of vice versa is also of the highest importance for understanding his turn to Kant. We have seen in the third chapter that Reinhold, in his first letter to Kant, claimed that it was Sch\"{u}tz's presentation of Kant's moral justification for the `fundamental truths of religion' in the \textit{Allgemeine Literatur{-}Zeitung} that highly impressed him. The justification of the most fundamental religious convictions in a way that was independent from current rationalist metaphysics was of the highest interest to Reinhold. In his \textit{Herzenserleichterung }he already stated that religion had to be based on morality instead of vice versa. The thought that religion needs to be based on morality implies that it is not to be based on metaphysics. Reinhold's writings on Enlightenment had already distrusted the one{-}sidedness of the Leibnizian{-}Wolffian school metaphysics, claiming that its high level of abstraction entailed that it failed to provide the most basic religious convictions with a foundation that would be relevant for ordinary people. 

This distrust of school metaphysics and the firm conviction that something more is needed are also apparent from Reinhold's virtual encounter with Kant over the latter's review of Herder's \textit{Ideen zur Philosophie der Geschichte der Menschheit}. In his counter review, Reinhold attacked Kant {--} wrongly {--} for being an old fashioned, orthodox metaphysician, who, because of his abstractions, had lost sight of the empirical facts. Kant's mild response may have first drawn Reinhold's attention to Kant. When the opportunity arose to put his studies to good use and secure a job at the University of Jena, Reinhold did not hesitate. His personal situation would certainly benefit from having a regular income. His response to minister Voigt's question regarding the influence of the Kantian philosophy shows that he places this philosophy in the context of Enlightenment and presents it as a means to correct the one{-}sidedness of current metaphysics. From both the skirmish with Kant over Herder's \textit{Ideen} and Reinhold's letter to Voigt it is clear that his interest in Kant was not solely dictated by philosophical considerations. Despite these external factors, or maybe because of them, Reinhold soon related the Kantian project to his own pre{-}Kantian thoughts and saw in it the potential to achieve the unification of sensible and rational capacities he had been looking for in order to justify morality as a foundation for religion. 

 When we consider the state that German philosophy was in at the time that Reinhold started studying the Kantian philosophy, it is no wonder that he thought it might contain the solution to the philosophical problems that were on his mind. The pantheism controversy between Mendelssohn, Jacobi and others centered on the question whether rationalist metaphysics provided an adequate foundation for fundamental religious convictions such as the conviction that there is a God. This question was not new to Reinhold, whose writings on Enlightenment show that he was keenly aware of a possible tension within Enlightenment. Jacobi presented this tension in the form of an inevitable choice: either to develop rational philosophy and be an atheist, or to forsake rationality and have faith. Given his previous concerns, Reinhold would hope to overcome this choice and establish a form of rationality that would reinforce fundamental religious tenets rather than destroy them. The hint that Kant's philosophy might provide the means to achieve this would have come from Sch\"{u}tz's reviews of works by Kant and of works that were central to the pantheism controversy. In reviewing Mendelssohn's \textit{Morgenstunden} and Jacobi's \textit{Ueber die Lehre des Spinoza}, Sch\"{u}tz, who was especially attracted by Kant's moral theology, stressed his conviction that the Kantian philosophy had in effect overcome the opposed positions in the pantheism controversy. This was, as we have seen in our fourth chapter, also Reinhold's starting point for the `Briefe \"{u}ber die Kantische Philosophie'. 

 In this series of articles appearing in 1786 and 1787, Reinhold applied his views on Enlightenment to make a case for the Kantian philosophy. In it he perceived the possibility of a coherent system doing justice to both man's rational and sensible capacities which would be conducive to true Enlightenment. If that was not enough, the Kantian philosophy showed a distinct promise to employ this new view on human cognitive capacities to justify an enlightened form of religion. Reinhold's reading of Kant from this perspective of course only captures some of the aspects of the Kantian philosophy, yet his interpretation became very influential, precisely because he was interested in those elements of Kant that average readers of his time were interested in as well. For that is exactly what the `Briefe' offer: an account of why the Kantian philosophy would be of interest to friends of Enlightenment, to people who are seriously worried by the tension within philosophy that Jacobi appeared to have identified. It is especially in the first two `Briefe' that Reinhold takes this approach, which relies mainly on sketching the situation of German philosophy at the time, Kant's own introduction to the first \textit{Critique }and Sch\"{u}tz's reviews regarding Kant, Mendelssohn and Jacobi. Enlightenment is presented as being deadlocked, which is indicated by the pantheism controversy. If both horns of Jacobi's dilemma are to be avoided, the only option is to show that his conception of reason is flawed. Reinhold employs the Kantian term `practical reason' to present a viable alternative to the rationalist metaphysics of Mendelssohn that does not forsake rationality. In presenting Kantian philosophy as the only way forward out of the crisis of metaphysics, Reinhold follows Kant's own statement of his project in the introduction to the first edition of the \textit{Kritik der reinen Vernunft}. In presenting Kant as a solution to the issue raised by the pantheism controversy, Reinhold follows Sch\"{u}tz. 

It is in his understanding of the pantheism controversy as the crisis of metaphysics, however, that Reinhold added his own perspective. Kant had mainly stressed the opposition between dogmatism and skepticism, which needed to be overcome. Reinhold rather focuses on the philosophy of religion and identifies an opposition between those who believe that rational metaphysics can justify religious convictions and those who deny this. Although these positions are similar to the dogmatism and skepticism presented by Kant, the circumstance that Reinhold is able to apply the opposition to a current issue (the pantheism controversy) as well as to an issue of substantial practical interest (the justification of religious conviction) make an important difference in arguing for the relevance of the Kantian philosophy.

 It is clear that the issues and themes that occupied Reinhold before he became involved in the Kantian philosophy are the main factor in determining his perspective on the new philosophy. That his pre{-}Kantian outlook is more important than the Kantian philosophy itself is shown by the way he employs the Kantianizing terms `practical reason' and `pure sensibility'. We have seen that these terms are used by Reinhold to alert the reader to what, according to him, is the central aspect of the Kantian philosophy, namely the importance of a connection between our sensible and rational capacities. He presents this connection as a fact about the human cognitive capacity that has been established by the Kantian investigation of reason. From there, it is not hard to argue that the previous deadlock in philosophy is the result of the failure to recognize this fact. The further claim is that once the misunderstanding of reason is corrected and practical reason and pure sensibility are recognized as capacities that mediate between man's rational and sensible abilities, beneficial results will follow with regard to the justification of fundamental religious convictions. Reinhold's use of the terminology of `practical reason' and `pure sensibility' is more related to his own thoughts on the task of philosophy in the context of Enlightenment than with any Kantian considerations regarding these terms. Apart from using Kantianizing terms {--} which is important for conveying the message that Kant has done something special {--} Reinhold argues for the relevance of the new philosophy in a way that is similar to the way in which he had earlier argued for the relevance of Enlightenment. The development of reason over time, the development of religion, the importance of the Reformation and the importance of linking philosophy to reality are themes that in the `Briefe' come together in one argument for the relevance of the Kantian philosophy. 

 Although the `Briefe' were a profound success {--} they helped Reinhold qualify for an extraordinary professorship {--} they did not convince Kant's philosophical opponents that the first \textit{Critique} was indeed a `gospel of pure reason'. He attributed the failure of these critics to understand Kant to their attachment to their own systems, which either overemphasized reason or sensibility. For that reason they could not appreciate Kant's efforts, which stressed the need to combine these two elements in cognition. Note that this reproach is similar to the one that Reinhold had once made Kant, in his counter review of Herder's \textit{Ideen}. Now, however, it did not remain just a reproach; in his \textit{Versuch} Reinhold sought to remedy the misunderstanding of Kant by presenting the theory of the faculty of representation as a premise of the Kantian theory of cognition. This was not because he believed that the latter theory was insecure, but rather because he wanted to make sure that everyone was on the same page. His strategy with the \textit{Versuch} does therefore not essentially differ from his strategy in promoting Enlightenment in his `Gedanken \"{u}ber Aufkl\"{a}rung', or from his efforts on behalf of the Kantian philosophy in his \textit{Merkur}{-}`Briefe'. In all three cases, the first step is trying to achieve conceptual clarity. With a proper concept (of Enlightenment, the Kantian philosophy, or cognition) in place, it will not be difficult to convince people that Enlightenment and the Kantian philosophy are good things. In all three cases the proper concept allows for understanding the previous conflicts and misunderstandings from a higher standpoint. 

 It is only in the \textit{Versuch}, however, that Reinhold explicitly credits himself for having found this higher standpoint. His theory of the faculty of representation is designed to provide philosophers of all denominations with a point of access to the central feature of the Kantian philosophy: the way it overcomes the traditional opposition between man's rational, spontaneous capacities and his sensory, receptive capacities. In his `Briefe' Reinhold had already identified this feature and claimed that it was this feature that enabled Kant to provide a rational foundation for the fundamental tenets of religion without being trapped by the fallacies of previous metaphysics. In the \textit{Versuch}, this foundational role is hinted at, rather than made explicit. Reinhold now aims to show that the Kantian theory of cognition actually works in this way, that is, that the Kantian concept of cognition entails a combination of the activity of our receptive and spontaneous capacities. This is done by Reinhold's own theory of the faculty of representation. Since Reinhold analyzes `representation' as consisting of `material', which is received from the object, and `form', which is produced by the representing subject, he can now claim that any representation (and therefore any cognition) involves the activity of both the receptivity and the spontaneity of the human faculty of representation. 

 Having developed his own terminology to point out what he believes to be the key feature of the Kantian philosophy, Reinhold has no further use for the terms `practical reason' and `pure sensibility' in the sense he used them in the `Briefe'. This is also due to the aim of the \textit{Versuch}. Reinhold is no longer trying to show that the Kantian philosophy provides a rational foundation for religious convictions. Rather he is calling attention to what he believes is the key feature of the Kantian philosophy. The consequences of the Kantian theory of cognition when it comes to these convictions are dealt with only negatively. That is to say, Reinhold's discussion of the ideas of reason shows that ideas such as the absolute subject, absolute causation, a moral world and a most real being can legitimately be thought, for they follow from the structure of reason. Instead of claiming that these ideas are grounded in `practical reason', Reinhold now uses the more Kantian strategy of distinguishing between the `thinkable' and the `cognizable' in order to claim that these ideas are legitimate, yet do not allow us to know anything about their objects. 

 In the case of human freedom, however, Reinhold does not appear to be content with establishing its possibility on the basis of the structure of human reason. Section 86 of the \textit{Versuch}, which interrupts the line of argument, seeks to establish the application of the causality of reason in the real world. This involves introducing the faculty of desire and a distinction between comparative and absolute freedom. In the `Grundlinien einer Theorie des Begehrungsverm\"{o}gens', which forms a part of section 86, Reinhold, without a reference, uses Kant's second \textit{Critique} to argue that pure reason really can be practical and that it can determine the faculty of desire. Reinhold's reasons for interrupting his argumentation at this point can be understood as an attempt to counter the criticism he expected on the basis of Rehberg's review of Kant's second \textit{Critique}. Rehberg had doubted that Kant had successfully shown that pure reason can be practical and had presented an alternative that claimed that comparative freedom would be sufficient for morality. Clearly, Rehberg's alternative was not acceptable for Reinhold, who might have feared that his efforts to establish the possibility of human freedom would solicit a similar criticism. Section 86, including the `Grundlinien' are therefore to be regarded as a preemptive strike against Rehberg. 

 Unfortunately for Reinhold, his desire to establish absolute freedom within the framework of his theory of the faculty of representation created tension. After all, the premise of that theory entailed that all activity of that faculty involved both spontaneity and receptivity. Considering the `absolute freedom' that he needed to distinguish from `comparative freedom', Reinhold sought to establish an activity of reason (belonging to the faculty of representation) that would not involve any receptivity. Partly as a result of this tension, his account in the `Grundlinien' is somewhat confused. The confusion is also due, however, to the circumstance that Reinhold mixes a Kantian account of the freedom of the will, based on the lawgiving activity of practical reason with his own account, which appears to situate human freedom in the choice for or against the law. Thus, Reinhold's efforts to present an account of absolute freedom within the framework of the \textit{Versuch} do not yield a satisfactory result. On the one hand, these efforts do not fit well within the framework; on the other hand, Reinhold has no clear thought on how to understand the absolute freedom he is trying to establish. 

 In the essays he produced between the publication of the \textit{Versuch} in 1789 and the second volume of his \textit{Briefe \"{u}ber die Kantische Philosophie}, Reinhold gradually develops a theory of the freedom of the will that is to be understood independently from his theory of the faculty of representation. It not only breaks away from Reinhold's own previous framework, but also from the framework of the second \textit{Critique}, by sharply distinguishing between the will and practical reason. Reinhold combined later versions of these articles with new material to make up \textit{Briefe II}. From the Preface it is clear that this is a unified collection that in a sense does for Kant's practical philosophy what the \textit{Versuch} did for his theoretical philosophy. Both works first identify a misunderstanding among philosophers, which is to be remedied by bringing the Kantian philosophy into play. Reinhold's version of the Kantian philosophy, that is, for in both the \textit{Versuch} and in \textit{Briefe II} he adds his taste for conceptual clarity to the Kantian philosophy to make it work the way he needs it to work. In order to get people to see what he saw in Kant's first \textit{Critique}, Reinhold used the concept of representation. In order to make the second \textit{Critique} fit his thoughts on morality, he needed in the end to make a distinction between practical reason and the will. Saving his understanding of the freedom of the will involved understanding it as a separate fundamental capacity of the human mind, independent from both the demands of sensibility and of reason. Understanding the will as a fundamental capacity, which (like the faculty of representation) announces itself in consciousness, allows Reinhold to put Rehberg's criticism of Kant aside. He admits that pure reason cannot be directly practical. The will mediates. 

 With this, Reinhold's journey from a Viennese monk who wrote on Enlightenment and was a member of a secret society to a University professor involved in one of the most important philosophical debates of his time has come full circle. In his discussions of the freedom of the will Reinhold is once again trying to come to terms with the process of how rationality is put to action. His answers are this time clearly inspired by the context of the Kantian philosophy, but the fundamental thought that there needs to be something in the middle, something that can combine both the abstract demand of reason and the concrete desires of everyday life, is still in place. It remains to be seen whether his solution, the free will as an incomprehensible fundamental capacity of the mind which cannot be grounded philosophically is a viable solution. Yet at least we now understand how this solution came to be from the transformation of Reinhold's ideals of Enlightenment into the context of the Kantian philosophy. 

The above account of Reinhold's philosophical development from his Vienna reviews up to \textit{Briefe II} has shown how strongely his efforts on behalf of the Kantian philosophy are related to his commitment to Enlightenment. It is not just that the religious and moral aspects that he emphasizes in his \textit{Merkur{-}}`Briefe' are related to his pre{-}Kantian work. He also seeks to establish a similar unity, of reason and sensibility, of intellectual ideas and the real world. Moreover, his pre{-}Kantian Illuminatism had given him a sense of belonging to an intellectual elite, whose responsibility it is to improve society. It is no coincidence that he speaks of his decision to write the \textit{Merkur{-}}`Briefe' as a `vocation', or `calling'. Reinhold's self{-}image as an apostle of the `gospel of pure reason' was undoubtedly inspired by the way the Illuminati viewd themselves and their role in society. The \textit{Versuch} and the book edition of the \textit{Briefe} likewise stem from the desire to spread the good news of the Kantian philosophy, rather than from dissatisfaction with its foundations. This important observation of Reinhold's fundamental motives is easily obscured by the fact that his foundational work set the tone for the next generation of philosophers, for whom the problems of the Kantian philosophy indeed formed a starting point. It is not so much the fact that Reinhold was the first to demand a foundation of the Kantian philosophy, but rather the way in which he presented a philosophy that was a curious mix of Kantian terms and Reinholdian themes, that triggered the foundational responses. This study has shown how Reinhold's efforts to find a common ground for reason and sensibility do not stem from a dissatisfaction with the Kantian answer regarding the unity of the two stems of knowledge, but rather from the firm belief that the Kantian philosophy indeed provides the means to identify this unity. His initial situation of this unity in practical reason has less to with the foundational structure of the Kantian philosophy than with Reinhold's pre{-}Kantian interests. There is a continuous line of development through Reinhold's Kantianzing phase, of which the efforts (by Reinhold and others) to ground philosophy as a scientific discipline was an important and fruitful side{-}effect.

backmatter

