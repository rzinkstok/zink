
\chapter{Reinhold's Enlightenment}


Although he had hardly received a formal academic education in philosophy, Reinhold was by no means a philosophical \textit{tabula rasa} when he got to know Kant's philosophy. Not only had he been thoroughly trained in theology and philosophy to become a Barnabite priest, he had also become acquainted with Enlightenment philosophy as embodied in Freemasonry and the ideology of the Illuminates. The present chapter will focus upon the themes that together comprise Reinhold's field of interest during the period in which his first writings were published. This field of interest can be labeled `Enlightenment', with a focus on humanity and religion. The specific issues that are of interest to him can easily be connected with his own situation: Enlightenment, Freemasonry and religion were all part of his own experience and education. His interest in Enlightenment is reflected in his writings on its nature and its relation to established religion, many of which originate from his Masonic/Illuminatist context. It must also be kept in mind that Reinhold's Enlightenment engagement originated from a political environment in which the Austrian government was trying to enhance state power at the expense of ecclesiastical institutions with Enlightenment as its ideology.\footnote{ For the relation between Reinhold's views on Enlightenment and the ideology of Joseph II, cf. Sauer, \textit{\"{O}sterreichische Philosophie}, chapter 3, esp. 64{-}70.} Reinhold's engagement was thus focused on the concrete application of Enlightenment in society and religion, which finds expression in his early writings. At this stage of his career, he clearly did not seek an academic career in philosophy, and expressed no special interest in the theoretical branches of philosophy, such as metaphysics.

The disentangling of Reinhold's pre{-}Kantian field of interest is of crucial importance for the later chapters, since it will provide an insight into his background and preconceptions. This in turn will help to appreciate his reception of Kant in more depth. The current chapter will approach Reinhold's views on Enlightenment from different perspectives. The first section will discuss Reinhold's own use and description of the term `Aufkl\"{a}rung', along with its general implications. Borrowing the phrase coined in the 1790's by Thomas Paine,\footnote{ Cf. Paine, \textit{The Age of Reason} (1794{-}1795). This work aimed to apply reason to religion in general and the Bible in particular. Thomas Paine (1737{-}1809) was an inventor and Enlightenment author, involved in bringing about the American Declaration of Independence (1776). } we can say that Reinhold views Enlightenment as an `Age of Reason'. That is to say, by stressing its historical aspects he presented Enlightenment as an `age' in human history, regarding is the historical period in which `reason' was came into its own. Following this structure, section two will focus on the historical aspect of Reinhold's views of Enlightenment, while section three will elaborate on Reinhold's understanding of human reason in the context of Enlightenment. The fourth section of this chapter will link these ideas of Enlightenment to another favorite theme of his pre{-}Kantian work, the criticism of established religion. Three of his works on this subject will be discussed. The first, `Ueber den Hang zum Wunderbaren' (1784) is the published version of a Masonic speech written and delivered by Reinhold when still in Vienna.\footnote{ Reinhold, `Ueber den Hang zum Wunderbaren,' \textit{JF}, 1784 III, 123{-}138.} The second work concerning religion and Enlightenment, \textit{Herzenserleichterung zweyer Menschenfreunde }(1785) was written to denounce Lavater's religious views.\footnote{ Reinhold [anonymously], \textit{Herzenserleichterung zweyer Menschenfreunde in vertraulichen Briefen \"{u}ber Johann Caspar Lavaters Glaubensbekentnis} (Frankfurt and Leipzig: 1785). Due to the Leipzig censorship, the work was printed in Halle. Cf. \textit{RK} 1:199, Letter 43, March 23, 1787, to Nicolai. } The third of the works to be dealt with, `Skizze einer Theogonie des blinden Glaubens', appeared only two months before the `Briefe \"{u}ber die Kantische Philosophie' and is Reinhold's last publication without any obvious Kantian influence.\footnote{ Reinhold, `Skizze einer Theogonie des blinden Glaubens,' \textit{TM}, June, 1786, 229{-}242. } The chapter will conclude with an evaluation of Reinhold's views on religion and Enlightenment. 


\section{Determining the concept `Enlightenment'}


When trying to understand Reinhold's conception of \textit{Aufkl\"{a}rung}, it is first of all important to look at the closest thing to a definition he gives. Although at the end of the first part of `Gedanken \"{u}ber Aufkl\"{a}rung' (1784) he claims that a theory of Enlightenment is out of the question, he does believe it is possible to determine the concept more precisely.\footnote{ Cf. Reinhold, `Gedanken \"{u}ber Aufkl\"{a}rung,' \textit{TM}, July, 1784, 21. Further references will be in the text.} Reinhold aims at analyzing the concept in a descriptive way and finding the middle ground between the various normative views on Enlightenment. 

Wenn ich den Begriff der, meiner Meynung nach, dem Worte \textbf{Aufkl\"{a}rung} entspricht, auseinanderzusetzen versuchte, so w\"{u}rde ich mich begn\"{u}gen die Wahrheit zwischen den Meynungen derjenigen, welche die Aufkl\"{a}rung f\"{u}r eine sehr leichte, f\"{u}r eine unm\"{o}gliche, f\"{u}r eine gef\"{a}hrliche, u.s.w. Sache halten, mitten inne gefunden zu haben. (21)

[If I were to try to explain the concept which, in my opinion, corresponds to the word `Enlightenment', I would be satisfied to have found the truth in the middle between the opinions of those who think Enlightenment is something `very easy', `impossible', `dangerous' etc.]

Although this ambition appears to be modest, Reinhold does have a normative agenda of his own. He is convinced that Enlightenment is a good thing and hopes that the proper conception of Enlightenment will turn its honest opponents into friends and its false friends into proper enemies, and thus contribute to Enlightenment itself (cf. 21{-}22). Thus, Reinhold's descriptive analysis of the concept `Enlightenment' is intended to further the cause of Enlightenment. 

The second part of the article links up perfectly with this end of the first part, as it opens with Reinhold's more precise determination of `Enlightenment'. ``I think that enlightenment means, in general, the making of rational people out of people who are capable of rationality.''\footnote{ Reinhold, `Gedanken \"{u}ber Aufkl\"{a}rung,' 123. Translation [amended, KJM] Kevin Paul Geiman, `Thoughts on Enlightenment' in \textit{What is Enlightenment?}, ed. James Schmidt, 65. Although the brief introduction gives the wrong year for Reinhold's `Briefe \"{u}ber die Kantische Philosophie' (and also for his birth), the notes are accurate. Only the second and third parts of the tripartite article are translated. Whenever I cite from this translation the page will be indicated in brackets following the original pagination.} In this still very general description Reinhold presents Enlightenment as an action performed on people from the outside, as it were. Reinhold's `making of rational people' does not carry the individualistic overtone of the definition that Kant would present just a couple of months later.\footnote{\label{footnote:_Ref218782785} ``\textit{Enlightenment is the human being's emergence from his self{-}incurred minority}.\textit{ Minority} is inability to make use of one's own understanding without direction from another.'' Kant, `Beantwortung der Frage: Was ist Aufkl\"{a}rung?,' \textit{BM}, December, 1784, 481; \textit{AA} 8:35; \textit{PP}, 17. According to Wahl, Kant's piece reacts to Reinhold's. Wahl, \textit{Geschichte des Teutschen Merkur}, 189.} This difference may be related to the different forms Enlightenment had taken in Austria and Prussia. In Austria, Enlightenment was strongly connected to centralized state{-}building, resulting in a focus on reforming religious institutions top{-}down, whereas Prussia at the time was one of the most modern states in Europe, with a relatively free press and freedom of religion already in place. Reinhold's definition concentrates on the transition of a capacity into an actual state of rationality, whereas Kant's focuses on the act of thinking for oneself and the importance of political freedom.\footnote{ For the difference between Kant's and Reinhold's views on Enlightenment, cf. also Batscha, \textit{Karl Leonhard Reinhold}, 57.} Another possible influence, pointed out by Sauer, is that of Herder, who in his \textit{Ideen zu einer Philosophie der Geschichte der Menschheit}, which will be discussed in more detail in Chapter 3, similarly speaks of making the transition from `Vernunftf\"{a}higkeit' to `Vernunft'.\footnote{ Cf. Sauer, \textit{\"{O}sterreichische Philosophie}, 100, n. 54. According to Sauer, this does not entail that Reinhold could be called a Herderian at this point. Ibid., 100{-}101, n. 58. Cf. Bondeli, `Von Herder zu Kant,' 204, stating that Reinhold was fully `under the spell' of Herder's philosophy of nature and history.}

Reinhold specifies his definition in a classical way reminiscent of the Wolffian school philosophy. He starts with describing the capacity for becoming rational as rooted in human physiology (again reminiscent of Herder), an approach that would lead to a very general conception of Enlightenment indeed. In a narrower sense, the capacity for becoming rational is the condition of the mind when it is fit for having distinct concepts (cf. 123), leading to the description of Enlightenment in a narrow sense as ``the application of the means that lie in nature to elucidate confused concepts into distinct ones'' (123 [65{-}66]). Reinhold adds the following remarks to his basic definition. First of all it is important to note that not everyone who happens to have some arbitrary distinct concepts is to be called `enlightened'. This term is reserved for those who have achieved a sufficient degree of distinction among the concepts that are of importance for human happiness (cf. 124{-}125). Further, the term can also be used for nations, but only for those that are already civilized and are intent on developing rationality at a higher level (cf. 125{-}126). Finally, Reinhold sets out to explain how people obtain distinct concepts, by distinguishing an artificial distinctness of concepts from a natural distinctness of concepts, the former belonging to philosophers, whereas the latter is available to all (cf. 127). Reinhold clearly distinguishes between these two ways of having distinct concepts.

Die F\"{a}higkeit des P\"{o}bels zu deutlichen Begriffen ist mehr leidend als wirkend; die des Philosophen mehr wirkend als leidend; der Philosophy lehrt; der P\"{o}bel lernt; der Philosoph zergliedert den Begriff; der P\"{o}bel fasset den zergliederten auf. (128)

[The capacity of the masses for distinct concepts is more passive than active; that of the philosopher is more active than passive. The philosopher teaches; the masses learn. The philosopher analyzes the concept; the masses apprehend that which has been analyzed.] (67)

Thus, the philosopher forms his own distinct concepts, whereas the common man can only receive them from the outside, for instance, from the philosopher. Reinhold uses an example to clarify the process of communicating distinct concepts to the masses. It concerns the concept of God's justice, which is commonly seen as a reason to resent God rather than to love Him, but the philosopher, who has a distinct concept of God, knows better. He can communicate his distinct concept to the common man by introducing the mediating concept of a good and wise father. 

Gott ist ein weiser Vater, wird der Philosoph anfangen; und der P\"{o}bel wird es ihm sogleich einr\"{a}umen; der Philosoph wird fortfahren: ein weiser Vater strafet nur aus G\"{u}te; und der P\"{o}bel wird ihm mit dem Schlusse zuvorkommen: Gott strafet nur aus G\"{u}te. Es wird ihm nun sehr begreiflich seyn, wenn ihm sein Lehrer die Gerechtigkeit als eine Eigenschaft Gottes vorstellt, die eben so liebensw\"{u}rdig ist als die G\"{u}te selbst. (131) 

God is a wise father, the philosopher begins, and the masses immediately grant him that. [The philosopher proceeds: a wise father punishes only out of mercy; and the masses anticipate him with the inference: God punishes only out of mercy. It now will be easily understood, if the masses are taught that this justice is a quality of God, which is as lovable as mercy itself.] (69)

It is important to note that the mediating concept here is one that is common to mankind as such, as everyone has a father. This allows philosophers, who are, after all, human beings, to communicate their distinct concepts to others, who are not able to create their own artificial distinctness. 

In the second part of `Gedanken \"{u}ber Aufkl\"{a}rung' Reinhold developed his more or less determinate concept of Enlightenment. The most important features of this concept are his emphasis on education of the common man and the use of common human concepts to achieve this. In this, Reinhold's concept of Enlightenment is strongly reminiscent of the Illuminatist conception of Christianity as it surfaces in a speech written by the founder of the Illuminati, Adam Weishaupt, introduced in more detail in the previous chapter. This speech was to be read to members of the rank of \textit{Illuminatus dirigens}, or leading Illuminate. Whether or not Reinhold actually had risen to that rank at the time, the similarities between his thoughts on the history of mankind and this speech strongly suggest that he was familiar with the text. Among other things, it presents Christianity as a way in which morality could be accessed by the masses, as it uses the metaphor of family to clarify moral concepts in a way that made it easier for people to act upon them.\footnote{ Cf. `Anrede an die neu aufzunemenden Illuminatos dirigentes von A. Weishaupt. 1782' in \textit{Der Geheimbund der Illuminaten}, ed. van D\"{u}lmen, 185{-}187. } Reinhold's conception of Enlightenment as laid down here is not only inspired by the general form Enlightenment had taken in Austria, but also by the more specific Illuminatist views on history and Christianity.\footnote{ Cf. Fuchs, \textit{Karl Leonhard Reinhold}, 21{-}23.} 

The article on \textit{Aufkl\"{a}rung} is, however, not the only place where Reinhold comes close to giving a definition of Enlightenment. The description given in `M\"{o}nchthum und Maurerey'\footnote{ Reinhold, `M\"{o}nchthum und Maurerey. Eine Rede von Br. R\textasteriskcentered \textasteriskcentered ,' \textit{JF}, 1784 IV, 167{-}188.} is different in its formulation, but has a similar focus on the education of mankind. 

Indessen wird uns eine nur sehr kurze Betrachtung sowohl in dem einen als in der anderen [monasticism and Masonry] zwo der wichtigsten Anstalten gewahr werden lassen, deren sich die Vorsehung zur Erziehung eines betr\"{a}chtlichen Theiles des menschlichen Geschlechtes bedienen wollte, Anstalten, bey welchen sie nichts geringeres als unmittelbare Entwickelung unsrer h\"{o}heren Geisteskr\"{a}fte, vollkommner Bearbeitung unsrer n\"{o}thigsten Kenntnisse, allgemeinere und schnellere Verbreitung der uns angelegensten Wahrheiten, mit einem Worte, \textbf{Aufkl\"{a}rung} zum Zwecke hatte.\footnote{ Reinhold, `M\"{o}nchthum und Maurerey,' 169.}

[In the meantime, a very brief consideration will allow us to perceive in both monasticism and Masonry two of the most important institutions that providence wanted to employ for the education of a considerable part of mankind. The purpose of these institutions was nothing less than the immediate development of our higher mental powers, the more perfect processing of the cognitions that are needed most, the more general and quicker dissemination of the truths that matter most to us, in one word `Enlightenment'.]

As in `Gedanken \"{u}ber Aufkl\"{a}rung', this description focuses on the making of rational people. Reinhold here speaks of the ``education of a considerable part of humankind.'' In both articles the development of reason is not related to any object or to theoretical speculation, but rather to the things that matter most to us as human beings. According to `Gedanken \"{u}ber Aufkl\"{a}rung' the term `enlightened' was only to be applied to someone who has ``brought sufficient distinctness to those concepts which have a considerable influence on human happiness.''\footnote{ Reinhold, `Gedanken \"{u}ber Aufkl\"{a}rung', 124{-}125; Schmidt, 66.} In the passage from the article on monasticism and Masonry cited above Reinhold refers to the ``cognitions that are needed most'' and the ``truths that matter most.'' Thus, from Reinhold's descriptions of Enlightenment it is already clear that he does not view rationality as something purely intellectual and detached from mundane interests. Rather, from the start his conception of Enlightenment includes a regard for human nature with its interests and needs in the world. In keeping with Reinhold's thought that it will be possible to enlighten the masses by means of concepts that are common to mankind as such, he appears to assume that the readers will understand what those concepts and cognitions and truths are that are of such high interest to us. 

 Even his earliest works, the reviews for the Viennese \textit{Realzeitung}, show that he is not enthusiastic about a rationality that focuses on breaking down the old structures, without building up new and better ones.\footnote{ Reinhold, review of \textit{Lehr{-} und Gebethbuch f\"{u}r die unm\"{u}ndige Jugend}, by Seibt, \textit{RZ}, December 3, 1782 [128]. ``Eine Aufkl\"{a}rung, die uns anstatt grauer Fr\"{o}mmlinge, unb\"{a}rtige Sp\"{o}tter, statt furchtsamer Heuchler, zaumlose Schwelger, statt abergl\"{a}ubiger B\"{u}rger, ungl\"{a}ubige Wilde g\"{a}be, w\"{u}rde uns f\"{u}r die zuf\"{a}lligen Vortheile der heiligen Dummheit unm\"{o}glich schadlos halten.''} Nevertheless, some demolishing has to be done, which explains the resistance to Enlightenment. It will take considerable didactical skills to convince those who cling to a misunderstood religion and therefore fear a misunderstood Enlightenment, but Reinhold is optimistic. 

 Seine Religion ist ihm der Fels, der ihn wider die Bedr\"{a}ngnisse dieses und des k\"{u}nftigen Lebens in Schutz nimmt {--} wie mu\ss{} sich der Arme \"{a}ngstigen, wenn er manche Tr\"{u}mmer davon losreissen sieht, die er f\"{u}r St\"{u}cke seines Felsens bisher gehalten hat? Man zeige ihm, da\ss{} es lockere Erde {--} Aberglauben ist {--} Koth {--} den der Strom der Zeiten, da er noch vor kurzem so dicht und tr\"{u}be daherzog, daran nothwendig zur\"{u}ckelassen mu\ss{}te, und den nun die helleren, und reineren Fluthen hinwegsp\"{u}len; und er wirds mit Dank erkennen, da\ss{} ihm und manchem seiner Br\"{u}der der Weg dahin ges\"{a}ubert und gesichert ist.\footnote{ Reinhold, review of \textit{Glaubensbekentni\ss{} und Lehre der \"{a}chtdenkende Katholiken (\ldots )}, by Koch, \textit{RZ}, November 12, 1782 [117].}

[His religion is like a rock to him, protecting him from the difficulties of this life and the life to come {--} how anxious must the poor man be, when he sees the debris being torn off which he had believed to be pieces of his rock? When it is shown to him that it is only loose earth {--} superstition {--} rubbish which was necessarily left behind by the stream of time, when, only recently, it flowed densely and darkly and which is now washed away by brighter and purer waters, he will acknowledge thankfully that the way is now clean and safe for him and many of his brothers.]

It is only natural, according to Reinhold, for people to cling to the religion they have been brought up with {--} as he himself may have done. The cause of Enlightenment can only be furthered when it becomes clear that this religion consists of very different parts; there is rock,\footnote{ Possibly a reference on Matthew 16:18. ``And I say also unto thee, That thou art Peter, and upon this rock I will build my church; and the gates of hell shall not prevail against it.'' In a modernised version of Luther's German translation: ``Und ich sage dir auch: Du bist Petrus, und auf diesen Felsen will ich bauen meine Gemeinde, und die Pforten der H\"{o}lle sollen sie nicht \"{u}berw\"{a}ltigen.''} and there is rubbish. The removal of the rubbish is a purification of religion, rather than its demolition. As seen in the later article on \textit{Aufkl\"{a}rung}, Reinhold has high hopes of being able to provide a more precisely determined concept of Enlightenment that will provide insight into the cleansing nature of Enlightenment. The following sections will shed more light on the way in which Enlightenment is to undertake this purification of religion. 

It is clear from Reinhold's statements on Enlightenment that he considers himself among its friends, devoting himself to its dissemination. As we have just seen, in the Vienna reviews as in the later article on \textit{Aufkl\"{a}rung}, he believes that a properly clarified concept of Enlightenment is needed. An important part of his exertions consists in determining the concept of Enlightenment more precisely, with the aim of showing that Enlightenment is not to be feared. In order to argue this successfully, Reinhold has to show that instead of demolishing cherished beliefs, Enlightenment purifies and thus strengthens the core of these beliefs. From the outset this means that the rationality propagated and achieved by Enlightenment is not the abstract theoretical rationality of the philosophers, but rather a rationality that incorporates and clarifies basic human beliefs, most importantly religious beliefs. The rationality of the philosophers is no goal in itself; it is needed to clarify the chaotic, irrational beliefs of the masses. From the example regarding God's justice cited above, it is apparent that Reinhold envisages the education of the masses as taking place through mediating concepts that are common to both the masses and the philosophers, such as the concept `father'. Apart from being a common concept (as everyone has a father) it is also appealing, as it is close to the heart. In the third section of the current chapter we will investigate the relation between mind and heart further. 


\section{History and Enlightenment}


Reinhold did not merely rely on conceptual clarity in his efforts to disseminate Enlightenment. Part of his strategy is to present Enlightenment as the final stage in the necessary development of human rationality in general, thus adding a dimension of historical inevitability to his advocacy of Enlightenment. This historical dimension has two aspects that are connected by the thought that human reason in general develops over time, a thought encountered earlier (Chapter 1) in Weishaupt's description of human history. The first aspect concerns the concrete consequences of the development of human reason: institutions that were rational or met the needs of humankind adequately at a given time in history may lose their rationality or fail to meet the needs of humanity at a later stage. This historical aspect of Reinhold's early works on Enlightenment is especially clear in his Vienna reviews and his Masonic articles. Secondly, there is the more general question as to the actual development reason has undergone throughout human history. From this point of view Reinhold can present Enlightenment as a necessary result of historical developments and thus argue with more force in favor of Enlightenment. The first part of this section (2.1) will discuss the first historical aspect of Reinhold's earliest works. The second part (2.2) will relate his discussion of the development of reason to his advocacy for Enlightenment


\subsection{The rationality of institutions over time}


As the origin of Freemasonry was not very clear, Masons generally took a keen interest in understanding their own history and the origins of the secrets in their society.\footnote{ Alexander Piatigorsky calls Freemasonry ``obsessed with its own history,'' while at the same time being anti{-}historical, because ``almost unaffected by the history of mankind in general.'' Piatigorsky, \textit{Who's Afraid of Freemasons?}, xiii. } Reinhold contributed to this interest by publishing an article on the `\textit{kabirische Mysterien}' for the \textit{Journal f\"{u}r Freymaurer} which contains a detailed overview of what was known about the Kabeiroi, their religion and their mysteries.\footnote{ Reinhold, `Ueber die kabirischen Mysterien. von Br. R\textasteriskcentered \textasteriskcentered ,' \textit{JF}, 1785 III, 5{-}48. The text shows awareness of the Masonic historical tradition, for instance, it mentions James Anderson, \textit{Constitutions of freemasonry} (1723), the first part of which discusses the history of Masonry, starting with Adam.} It is, however, unclear to what extent this article is Reinhold's own work, as it is an adaptation of an adaptation.\footnote{ Cf. \textit{RK} 1:16{-}17, Letter 2, April 19, 1784, from Ignaz von Born. } Although the way the article deals with history does thus not necessarily exactly correspond to Reinhold's view, it must have been sufficiently close to his own convictions to have it published as his work. The origins of Freemasonry are sought among the mysteries of the ancient inhabitants of Greece, the so{-}called Kabeiroi. This results in a search for similarities `proving' the ancient origin of Freemasonry, thus presumably contributing to the respectability of this secret society. Many striking similarities are noted, but the author is also critical of too much enthusiasm for the ancient mysteries. 

Allein man darf bey der Vergleichung der kabirischen Mysterien mit den unsrigen nie vergessen, da\ss{} wir in aufgekl\"{a}rteren Zeiten zu leben das Gl\"{u}ck haben.\footnote{ Reinhold, `Ueber die kabirischen Mysterien,' 48.}

[In comparing the Kabirian mysteries with ours, one must, however, never forget that we have the fortune of living in more enlightened times.]

A remark like the above hints that Enlightenment affects our understanding of `mysteries'. The way in which the Kabeiroi interpreted their mysteries is outdated, no longer adequate to the current understanding. In this, the ancient mysteries may be like the practices of worship of the Kabeiroi, which were called `superstitious'.\footnote{ Reinhold, `Ueber die kabirischen Mysterien,' 48. } In more Enlightened times, mysteries should not lead to superstition anymore. Hence, comparing Masonic mysteries to their ancient counterparts should be done with caution. This is clearly a point Illuminates like Reinhold would support, for they intended to reform Freemasonry and fought superstition within it as they fought it in society as a whole.\footnote{ Cf. Weishaupt, `Anrede,' 191.} What was once good and useful may now be deemed superstitious and should be shunned. 

This general attitude towards traditional institutions is also applied outside the Masonic context, in Reinhold's reviews for the Vienna \textit{Realzeitung}. Again, Reinhold need not have invented the arguments from history himself, as they may well have been taken from the books he reviewed. In that case he at least cites them with approval. Moreover, arguments from history offered for the abolition of monastic orders resemble the argumentation used in government circles. Reinhold's approval of them shows his support for Emperor Joseph's policy. The general line of these arguments from history is as follows. In the early days of Christianity a certain practice or institution, for instance monasticism, was absent. Therefore, this practice or institution does not belong to the core of Christianity and has been introduced at a particular time in history to fit the needs of that time. When nowadays the practice can be proven useless or even harmful, there is every reason to abolish it. Another example is the use of Latin in public Mass, which has, according to Reinhold, ``lost its reasonable cause, but not yet its existence.''\footnote{ Reinhold, review of \textit{Abhandlung von der Einf\"{u}hrung der Volkssprache in den \"{o}ffentlichen Gottesdienst} \textit{(\ldots )}, \textit{RZ}, 1783, March 4 [176]. } Showing that the introduction of a certain practice is closely connected to a historical situation also involves interpreting parts of the Bible in their historical context. The command of Jesus that his followers leave their homes can easily be cited in favor of monasticism, but given the historical context of the missionary works of the apostles, there appears to be no reason to believe that this command was to be followed by ordinary Christians and result in a life of seclusion.\footnote{ Reinhold, review of \textit{Abhandlung \"{u}ber den Werth der M\"{o}nchsprofessen (\ldots )}, transl. from French by Stambach, \textit{RZ}, 1783, March 11 [187]. } 

Apart from placing the origin of traditional institutions in a determinate historical context,\footnote{ Reinhold also does this elaborately in \textit{\-Die Hebr\"{a}ischen Mysterien oder die \"{a}lteste religi\"{o}se Freymaurerey}, which originally appeared in 1786 as two articles in the \textit{Journal f\"{u}r Freymaurer} and was published in 1788 as a book with Reinhold's Illuminatist pseudonym Br. Decius on the title page. In it, Reinhold discusses the Egyptian origins of both the substance and the rituals of Judaism. } historical argumentation for the abolition of these institutions also involves showing that they have become useless or harmful in contemporary society. Regarding monasticism and religious intolerance one can argue that these practices are harmful to the state, because, for instance, they prevent marriages and thus population growth.\footnote{ Reinhold, review of \textit{Rede von dem erlaubten und n\"{o}thigen Bande der freyen Religionsduldung mit der Freyheit der Handlung}, by Faber, \textit{RZ}, 1783, April 8 [209].} With arguments like these the Austrian Chancellor, F\"{u}rst Kaunitz, had been working on the reduction of the monastic orders and their members from the late 1760's onwards. Interestingly, his arguments are close to those used over a decade later in the books and pamphlets reviewed by Reinhold. In a report Kaunitz used both historical and economic arguments. Apart from the fact that entering a monastery made young people useless for the state's economy, their capital is lost to society as well, for they produce no heirs. Moreover, the monasteries and the Church are generally less burdened with taxes than laymen, with the result that a considerable part of the country's wealth does not contribute to the wealth of the state. The argument from history is put forward with force. Early Christianity, which was still pure and perfect, did fine without monasticism. Therefore, the reduction of monasticism will not harm this core of Christianity.\footnote{ `Gutachten des F\"{u}rsten Kaunitz vom 21. Juni 1770 \"{u}ber die Notwendigkeit, den Ordensklerus zu reformieren' in Maa\ss{}, \textit{Der Josephinismus}, 140{-}141.} 

It is very clear that the advocates of this kind of argument measure by different standards when it comes to the contextuality of the past. Although, for instance, the supremacy of the Pope, monasticism and the excessive veneration of saints or the Sacred Heart are condemned as at least outdated, because they belong to another historical context, Christianity itself is never considered subject to the historical context of its origin. The arguments used in late eighteenth{-}century Vienna are based on the assumption that there is an eternal and indestructible core to Christianity, corresponding to the early Church, and it has to be found again by peeling off the layers of later additions that have been determined by their respective historical contexts. Again, there appears to be a strong relation here to Illuminatist thought. From Weishaupt's `Anrede' it is clear that he regarded the Illuminate Order as the heir to the secret core doctrine of Christianity. This core doctrine boiled down to the idea that, as children of God, all humans are to be regarded as brothers and sisters.\footnote{ Cf. Weishaupt, `Anrede,' 191{-}192.} As we have seen earlier, Reinhold believes that this core of Christianity would be compatible with Enlightenment and that only the rubbish that has been introduced in the course of history would have to be washed away.


\subsection{The development of human reason}


A less concrete aspect of Reinhold's historical orientation in his early writings is his presentation of Enlightenment as a necessary stage of the development of human reason. He explores this theme mainly in two essays published almost simultaneously in the \textit{Merkur}, `Gedanken \"{u}ber Aufkl\"{a}rung', which we have encountered earlier, and `Die Wissenschaften vor und nach ihrer Sekularisation. Ein historisches Gem\"{a}hlde'. Both articles present a historical exposition on the development of the sciences. It must be noted that Reinhold does not conceive of `the sciences' primarily as the natural sciences. He uses the term `Wissenschaften' to denote philosophy, or the totality of knowledge. In the following the term `sciences' is used to denote this understanding of `Wissenschaften'. In both articles the point of departure is the condition of human knowledge in the Middle Ages. The hegemony of the Church over philosophy resulted in types of knowledge that were both useless and incomprehensible.\footnote{ Reinhold, `Die Wissenschaften,' 35.} The first part of `Gedanken \"{u}ber Aufkl\"{a}rung' shows us the harmful consequences of this hegemony for the condition of reason.

Die Religion k\"{u}ndigte der Vernunft ein Verderbni\ss{} an, das der Mensch eben darum f\"{u}r unheilbar ansehen mu\ss{}te, weil es ein unwiderrufliches Urtheil der g\"{o}ttlichen Gerechtigkeit seyn sollte; sie n\"{o}thigte der Vernunft fast nichts als unbegreifliche S\"{a}tze auf, deren Anzahl sich beynahe ins Unendliche vervielf\"{a}ltigte, und gew\"{o}hnte sie nicht nur durch die unaufh\"{o}rlichen Unterwerfungen an ein sclavisches Stillschweigen, sondern erhob die Unterdr\"{u}ckung aller ihrer Wirksamkeit unter dem Namen des Glaubens zum ersten und beynahe einzigem Bedingnisse des Seelenheils.\footnote{ Reinhold, `Gedanken \"{u}ber Aufkl\"{a}rung,' 7.}

[Religion announced to reason a damnation which man had to believe to be incurable, because it supposedly was an irrevocable judgment of divine justice. Religion forced nothing but incomprehensible doctrines on reason, the number of which was multiplied almost to infinity, and not only did it accustom reason to servile silence by means of incessant subjugations, but even, under the name of faith, elevated the suppression of all its activity to the first and almost only condition of salvation.]

This oppression of reason was not beneficial to the sciences in any direct way. However, in the essay on the sciences Reinhold claims that the hegemony of the Church of Rome was preparing its own downfall, by taking the oppression to the point where reason had no other choice but to fight back. 

Der Druck des Elendes zwang endlich die Layen mit Gewalt zum Selbstdenken, und so ergriffen sie nach und nach dies einzige Mittel ihrer Rettung.\footnote{ Reinhold, `Die Wissenschaften,' 39. } 

[The pressure of misery finally violently forced the laymen to think for themselves, and thus they slowly but surely got hold of this only way to save themselves.]

In this way, the oppression of reason generates it own dynamic and the worst circumstances carry the seeds of the end of oppression.\footnote{ This dialectic view of the dynamic of human history was made famous through Hegel's dialectics and, via Hegel, in Karl Marx's (1818{-}1883) theories on the \textit{Verelendung }of the proletariat, which provides the dynamic for a revolution. Reinhold showed a taste for dialectics already when he interpreted the abolition of the Jesuit order as the preliminary to its glorious return. Cf. Batscha, \textit{Karl Leonhard Reinhold}, 14{-}15.} The thought that `the darkest hour is just before dawn' is not uncommon in Reinhold's early work. In his `M\"{o}nchthum und Maurerey', he assessed the negative contribution of monasticism to Enlightenment, namely by suppressing reason, which made the need to liberate reason all the more pressing. This view is again close to the Illuminatist way of viewing history.\footnote{ Cf. Weishaupt, `Anrede,' 177. } In the article on the sciences the means to the mental salvation of the laymen is situated in the discovery of the ancient classics. It is not entirely obvious which authors Reinhold has in mind here. He speaks of the `Greek sciences' and claims that its teachers taught ``only things that were understandable'' and ``convinced people that there was also happiness for them on this side of the grave.''\footnote{ Reinhold, `Die Wissenschaften,' 39{-}40.} Later, in his `Briefe \"{u}ber die Kantische Philosophie', Reinhold would distinguish four main currents in antiquity: Platonism, Aristotelianism, stoicism and Epicureanism.\footnote{ Reinhold, `Briefe \"{u}ber die Kantische Philosophie,' especially the seventh and eight `Briefe' as they appeared in \textit{Der Teutsche Merkur}, on the history of the rational psychology of the Greeks, passim. } The term `ancients' in general probably refers to the main exponents of these systems. According to Reinhold, this Renaissance discovery of the classics broke the monopoly of the monks, which resulted in sciences that were based on common sense; a philosophy of the laymen.\footnote{ Reinhold, `Die Wissenschaften,' 40{-}41.} Reinhold very quickly and with broad strokes links the rediscovery of the Classics to the rise of the sciences and does not provide a historical account in the modern sense of the word. Yet it must be remembered that Reinhold is not aiming to be either neutral or detailed; his essay on the sciences consists of only eight pages. According to his picture the Renaissance is followed smoothly by the Reformation, when reason came to its rights even in matters of religion, and the sciences became free as a result of this freedom.\footnote{ Reinhold, `Die Wissenschaften,' 42.} Again, this is Reinhold's interpretation of the importance and the effect of the Reformation, colored by the significance he attaches to the independent use of reason in Enlightenment. The picture he sketches is more related to neological currents in late eighteenth{-}century protestant theology than to the early Reformers themselves. The freedom of the sciences that Reinhold presents as the beneficial effect of the Reformation resulted in sciences that are truly useful to all kinds of people and to society as a whole.\footnote{ Reinhold, `Die Wissenschaften,' 42{-}43.} Thus, Reinhold's essay on the sciences presents a straight line of progress from the Middle Ages through the Renaissance and Reformation into his own time. In situating the root of the freedom of the sciences and of Enlightenment in the circumstance that laymen started to think for themselves, Reinhold's view of Enlightenment is, in this respect, related to that of Kant, who, as we have seen, advocated thinking for oneself as one of the major characteristics of Enlightenment.\footnote{ In his `Was ist Aufkl\"{a}rung?'; cf. footnote \ref{footnote:_Ref218782785}.}

The positive and uncomplicated perspective on the progress of the sciences from the Renaissance onwards as sketched above is given a nuance in the first part of `Gedanken \"{u}ber Aufkl\"{a}rung', appearing in the same month as the essay on the sciences. The article opens with a similar account of the oppressed position of the sciences in the Middle Ages. The subsequently established freedom of reason resulted in elaborate doctrinal structures (\textit{Lehrgeb\"{a}ude}). Unfortunately, this was not all that was needed. The \textit{Lehrgeb\"{a}ude} turned out to be no more than the scaffolding, so that the apparently finished building had in fact hardly been begun. 

Die gelehrten Baumeister konnten gar nicht begreiffen, wie so mancher Einwohner ihrer Geb\"{a}ude durch den Fu\ss{}boden, den sie mit allem Flei\ss{}e gezimmert hatten, durchfallen; wie sich so mancher \"{u}ber die Ungem\"{a}chlichkeiten der Witterung beklagen konnte, der er von allen Seiten ausgesetzet blieb. Man kann diese Herren, die sich ihrer M\"{u}he und Geschicklichkeit bewu\ss{}t waren, auch nicht verdenken, da\ss{} sie das h\"{o}chlich Wunder nahm; denn sie h\"{a}tten sich ehe alles in der Welt einfallen lassen, als da\ss{} sie von dem Geb\"{a}ude, welches die Menschen gegen Aberglauben und Despotismus in Schutz nehmen sollte, weiter nichts als das \textbf{Ger\"{u}ste} fertig h\"{a}tten.\footnote{ Reinhold, `Gedanken \"{u}ber Aufkl\"{a}rung,' 4{-}5.} 

[The learned architects could not understand at all why so many inhabitants of their buildings fell through the floors that they had built with great care; why so many complained of the inconveniences of enduring the weather to which they were exposed from all sides. These gentlemen, being aware of their industry and their talents, cannot be blamed for being surprised, for they would have thought of anything in the world before they would understand that they had only finished the \textbf{scaffolding} of the building that was supposed to protect people from superstition and despotism.]

It is again obvious that Reinhold is by no means a neutral commentator on the story of the \textit{progress} of the sciences. His account is based on a specific view of their \textit{purpose}. In this particular instance, philosophers are compared to construction workers that are building a structure that will protect the people from the threats of superstition and despotism. Naturally, the `building' metaphor would appeal to any Freemason. The point of the citation is that the structure thus built is not fit to protect the people, as it is only the scaffolding. Now, the question is why philosophy does not succeed in fulfilling its aim and protecting the people from superstition and despotism. Reinhold blames the abstractions carried out by the philosophers, resulting in a science that has come to be disconnected from human reality.\footnote{ In this criticism of the Wolffian philosophy of the schools, Reinhold manifests himself as a \textit{Popularphilosoph}. We shall see in later chapters how he turns against this development in German Enlightenment, disavowing his own origins, so to speak. Cf. B\"{o}hr, \textit{Philosophie f\"{u}r die Welt}; di Giovanni, `Die \textit{Verhandlungen \"{u}ber die Grundbegriffe und Grunds\"{a}tze der Moralit\"{a}t} von 1798.'} 

Der Gelehrte der seine Analyse wissenschaftlicher Begriffe immer weiter hinauf verfolgte, verlohr in eben dem Verh\"{a}ltnisse jeden andern der noch nicht wissenschaftlich war aus dem Gesichte. Er fand die Ideen die er mit dem gemeinen Manne gemein hatte, zu unbedeutend, um sie seiner Bearbeitung zu w\"{u}rdigen. Indessen wurden selbst die allgemeinen Notionen f\"{u}rs menschliche Leben in eben dem Grade unbrauchbar, als sie sich von den individuellen Empfindungen, den Triebfedern aller Th\"{a}tigkeit des Menschen, entfernten, und mit der ganzen \"{u}brigen Ideenmasse ungleichartig wurden.\footnote{ Reinhold, `Gedanken \"{u}ber Aufkl\"{a}rung,' 5. } 

[The scholar who carried his analysis of scientific concepts ever further, to that extent lost sight of any other concepts that were not yet scientific. He considered those ideas that he had in common with common people not worthy of his attention. In the meantime even the general notions lost their use for human life to such an extent that they became distanced from individual impressions, the incentives of all human activity, and dissimilar to all the rest of the ideas.]

In the first section of this chapter it has been shown that the ideas that philosophers and common men have in common are central to Reinhold's conception of Enlightenment, functioning as mediating concepts that are needed in order to communicate the distinct concepts of the philosophers to the rest of society. The lack of attention for this kind of ideas is precisely the cause of the problem that Enlightenment is supposed to solve. Philosophy has lost touch with the concrete world in which human beings live, with sensations, with incentives. Therefore, philosophers are unable to eliminate the threats that real humans face, because they are not properly related to the reality in which these threats exist. Enlightenment is presented as the solution to the problem by reconnecting the scientific progress to the world of the common man. 

Man stieg nun vom Allgemeinen immer mehr aufs Besondere herab, und berichtigte nun auch Begriffe, die von kleinerem Umfange, aber desto gr\"{o}\ss{}erer Wichtigkeit waren, denn sie wurden die Br\"{u}cke zwischen Spekulation und Handlung. Diese neuere wissenschaftliche Begriffe pa\ss{}ten nun den wirklichen Gegenst\"{a}nden in der Welt viel besser an, n\"{a}herten sich den gemeinern F\"{a}higkeiten, leuchteten den Wirkungskreisen aller St\"{a}nde, und wirkten jene Revolution, die als sie anfieng merklicher zu werden, mit dem Namen \textbf{Aufkl\"{a}rung} bezeichnet wurde.\footnote{ Reinhold, `Gedanken \"{u}ber Aufkl\"{a}rung,' 6.} 

[They descended more and more from the general to the particular and also corrected those concepts that had a smaller extension, yet were all the more important, for they became the bridges between speculation and action. These newer scientific concepts were much better suited to the real objects in the world, they approached more common capacities, illuminated the spheres of action of all classes and effected that revolution that was to be called `Enlightenment' when it began to be noted] 

Only the last stage of the development that can be traced back to the end of medieval science can properly be called Enlightenment, when the fruits of reason are reconnected to human reality. Reinhold's historical account of Enlightenment shows that there are at least two conditions that need to be fulfilled, as far as he is concerned. First of all, reason needs to be released from the limitations set by the Church. Secondly, it must be (re)connected to human reality, so that it can influence man's actions and thus be of practical use. In this way, the historical presentation of Enlightenment contributes to the more precise determination of the concept that was discussed in the first section of this chapter. It explains what is needed to make rational people out of those who are merely capable of being rational. Moreover, it shows how the situation in which some people require assistance from philosophers in order to become rational is a result of the inevitable development of human reason throughout history. It is becoming clearer in what way Reinhold's conception of Enlightenment differs from Kant's cited earlier, now that we have seen that Reinhold insists on the importance of thinking for oneself in the first stage of the emancipation of reason. He would therefore probably agree with Kant that thinking for oneself is a crucial ingredient for Enlightenment. However, to him it is not an unproblematic ingredient, since he believes that the newly found freedom of reason will lead to an overenthusiastic use of reason, overlooking the vital importance of the real world in which reason is actually employed. It may well be for this reason that Reinhold initially was to think of Kant as an old{-}fashioned metaphysician, as will be shown in more detail in Chapter 3. 


\section{Human reason }


From Reinhold's own description of Enlightenment and his use of the term discussed in the first section it has become clear that already in his early writings he had a specific understanding of what reason is and what it is not. One aspect of Reinhold's thoughts on reason has been discussed in the previous section, namely the historical development of human reason in general. The present section focuses on another characteristic of Reinhold's conception of human reason, namely his insistence that in its proper use reason is not severed from sensibility and that the philosophical use of reason is not to be separated from its use in daily life. Starting from the final point of the previous section, that the connection of reason to the daily reality of human life is a crucial component of the Enlightenment as propagated by Reinhold, the current section first investigates Reinhold's views on the relation between reason and sensibility (3.1). Secondly, it discusses the intimate connection that Reinhold sees between human nature and human reason (3.2).


\subsection{Reason and sensibility in matters of religion and superstition}


Although it may appear as if Reinhold's preoccupation with the gap between the abstract thinking of the philosophers and the needs of the masses first occurs in his `Gedanken \"{u}ber Aufkl\"{a}rung', it actually dates back to his Vienna reviews and is implied in Reinhold's view of Enlightenment as \textit{Volksaufkl\"{a}rung}. Indeed, the abstractness of the philosophers' concepts is only problematic insofar as it hinders the newly established clarity from setting in where it is most needed, in the unclear and superstitious thinking of the masses. What is called for is a rationalization of religion that does not carry its abstractions to the extreme. In Reinhold's reviews for the \textit{Realzeitung}, the main focus is on the first part of that requirement, the rationalization of religion. After his move to the Protestant North, Reinhold started showing more interest in the potential difficulties of rationalizing too much. In his new, Protestant environment, he appears to have found a new enemy in the philosophers of the schools, who rationalize to the extreme, and he consequently stresses the need for reconnecting the abstract systems to the concrete, real world.\footnote{ Herder's dislike of metaphysics, which we shall encounter in the third chapter, may have influenced Reinhold's choice of perspective as well. } However, even in some of the Vienna reviews Reinhold already shows some awareness of the needed connection between abstract speculation and concrete action that was to become so prominent in his `Gedanken \"{u}ber Aufkl\"{a}rung'. 

We find a prime example of this awareness in Reinhold's review of a work on the veneration of the sacred heart of Christ (January 1783). The nature of the venerated object was, and still is, subject to debate: is it the actual heart of Christ, the heart as a symbol for His love, or this love itself, through the symbol of the heart? In the book under review, the author criticizes even symbolic veneration for its danger to the masses, that is, the possible lapse into superstition. The veneration of sensible objects is part of the unenlightened religion that Reinhold opposes. Yet he does not advocate a religion without any sensible objects. He agrees that even symbolic veneration would be superfluous for the enlightened believer, and adds that an enlightened believer's need for sensation is already satisfied by the sensible objects presented to him by the general Church.\footnote{ Cf. Reinhold, review of \textit{Die Herzjesuandacht nach theologischen und historischen Gr\"{u}nden gepr\"{u}ft}, by Huber, \textit{RZ}, 1783, January 1 [143].} Enlightened believers do need sensible objects in their religion, but they do not need to venerate sensible objects. The Church as it is fulfils their need for sensation in religion. In the opening paragraph of the review Reinhold had already discussed the importance of sensibility in a more general context. This passage develops thoughts that foreshadow Reinhold's later philosophical themes.

Der kennt die Menschen nicht, der den Gebrauch der Bilder und Symbolen aus dem Gebiethe der Religion verbannt wissen will, und nicht begreifen kann, wieviel die Wahrheit durch die sinnliche H\"{u}lle gewann, in welche sie von den ersten Gesetzgebern und V\"{o}lkerlehrern eingekleidet wurde. Auch in dem hellsten Kopfe eines Weisen m\"{u}ssen sich die abgezogenen Begriffe in sinnliche Bilder zusammendr\"{a}ngen, bevor sein erhabenster Gedanke zur That reift; er braucht sichtbare Merkst\"{a}be sogar in seiner Ideenwelt, braucht Erinnerungen von aussen an Gegenst\"{a}nde, die in der Reihe seiner Begriffe von innen nicht so oft vorkommen w\"{u}rden, als es ihres wohlth\"{a}tigen Einflusses halber zu w\"{u}nschen w\"{a}re.\footnote{ Reinhold, review of \textit{Die Herzjesuandacht }[139].}

[He does not understand people, who wants to banish the use of images and symbols from the domain of religion and who cannot understand how much truth has gained from the sensible cover in which it was dressed by the first legislators and instructors of the nations. Even in the brightest mind of a wise man the abstracted concepts must be compressed into sensible images before his most elevated thought bears the fruit of action. Even in his world of ideas he needs visible marks, needs external reminiscences of objects that do not occur in his train of thought as often as it would be wished for the sake of their beneficial influence.]

Even the abstract concepts of the wise need to be translated or converted into sensible images in order to have any effect upon their actions. The concepts can only have a beneficent influence when they are connected to sensibility. Here, as in `Gedanken \"{u}ber Aufkl\"{a}rung' we find theoretical speculation on the one hand and action on the other. The thought that reason must be connected to sensibility in order for its activity to have any concrete result in action is also encountered in this early review. The ultimate goal is neither action nor reason on their own, but rather action guided by reason. The abstract thinking of the wise is associated with an inner world, while sensible objects are located in the outside world. It is clear that abstract speculation needs to be linked with this outside world of the senses in order to have any beneficial effects in the real world. Without this connection, the wise man will be a stoic sage, detached from the world. The remainder of the opening paragraph, not cited here, focuses on the other side of the problem. Not only does reason need to be connected to sensibility in order to be effective, but sensibility also needs to be connected to reason in order to be properly guided. Excessive attention to sensation leads to misunderstanding and misdirected veneration, in other words, idolatry and superstition.\footnote{ Reinhold, review of \textit{Die Herzjesuandacht }[139{-}140].} Although the early instructors of the Church needed to dress the truth for the senses, the dress should not be the primary focus of religious attention. The sensible objects presented by the general Church (\textit{allgemeine} \textit{Kirche}, with the original meaning of `katholikos' in mind), are likely to be the `sensible cover', instituted at an early stage of Christianity. They are the concrete manifestation of religion in daily life, with buildings, prayers and services. As these are deemed sufficient for the enlightened believer, Reinhold's view on religion as expressed here again shows that he regards early Christianity as an ideal to which the contemporary Church should return. His criticism is directed at overemphasizing the sensible objects associated with religion to the point where the veneration of these objects takes the place of religion itself. 

 We can take another example to show that Reinhold's interest in combining reason and sensibility dates from his early days in Vienna. The theme was not only expressed in reaction to the established religion, but also with regard to Freemasonry, in one of his earliest extant works, a speech, read in the Lodge \textit{Zur wahren Eintracht}. In his speech `Der Wehrt einer Gesellschaft h\"{a}ngt von der Beschaffenheit ihrer Glieder ab' (September 1783), he portrays true Freemasonry as a society of men of both healthy minds and good hearts. 

Die einzigen Merkmale, durch die der Freym\"{a}urer den Charakter seines Ordens ank\"{u}ndigen darf und mu\ss{}, sind der Adel und die G\"{u}te seines Geistes und Herzens, Merkmale die eben darum den Thoren, und den B\"{o}sewicht aus unsrem Kreise ausschliessen, weil sie ihnen Geheimnisse bleiben, den w\"{u}rdigen Mann aber desto gewisser zu uns einladen, weil sie ihm so ganz verst\"{a}ndlich sind.\footnote{ A edition of the manuscript of this speech will be published in the collection of papers presented at the 2007 Reinhold conference in Montr\'{e}al. Karianne Marx and Ernst{-}Otto Onnasch, `Zwei Wiener Reden Reinholds. Ein Beitrag zu Reinholds Fr\"{u}hphilosophie' in George di Giovanni ed., \textit{Reinhold and the Enlightenment} (Dordrecht: Springer forthcoming). The manuscript is kept in the Viennese Haus{-}, Hof{-} und Staatsarchiv, Vertrauliche Akten, Kart. 73, fol. 64{-}68. The citation is taken from fol. 65v. }

[The only characteristics through which the Mason may and must announce the nature of his order are the nobility and goodness of his mind and heart; characteristics that precisely for this reason exclude the fool and the villain from our circle, because these things remain secrets to them. They do invite the worthy man all the more to us, because they can be fully understood by him.]

Neither fools nor villains have a place in Freemasonry, for they have no access to the nobility of mind and heart that is at its centre. They will view these characteristics as the Masonic secrets. For the worthy man, however, the character of this secret society is anything but hidden, as he has insight into the most important traits of the Order. A proper understanding of Freemasonry and especially the status of its secrets was of the highest importance to Reinhold, because of his Illuminatist views. The Illuminates not only worked toward social and political reform throughout society, they were also aiming at reforming the Masonic society from within. Regarding Freemasonry, Reinhold warns against both superstition and non{-}belief, as he did with regard to established religion. In the speech we also find a passage similar to the introduction to the review cited above, focusing on the need for reason to be connected to the senses (here \textit{Empfindung}) in order to have a beneficial influence on action. 

Allein kan ich de\ss{}wegen die f\"{u}r die Menschheit so traurige Thatsache l\"{a}ugnen, da\ss{} eben dasselbe Licht der Aufkl\"{a}rung, welches den Verstand eines Mannes erleuchtet, sein Herz kalt l\"{a}\ss{}t, und da\ss{} die Menschheit durch das Herz manches ihrer S\"{o}hne vielleicht mehr einb\"{u}\ss{}te, als sie durch seinen Geist gewann. Die Frucht der deutlichsten und m\"{u}hsamsten \"{U}berzeugung der herrlichste Gedanken eines grossen Geistes verungl\"{u}cket nur gar zu oft auf dem Wege zur That den er durch die Empfindung nehmen mu\ss{}. Da erwartet ihn eine Menge ungleichartiger Vorstellungen der Sinne, und droht ihn im Gedr\"{a}nge zu ersticken. Nichts kan ihn retten als eine besondere Lebhaftigkeit, die er nicht so viel der angebohrene Gr\"{o}sse des Geistes, als einer durch anhaltende \"{U}bung erlangten Fertigkeit zu danken hat. Diese Fertigkeit ist Tugend, G\"{u}te des Herzens, Adel der Seele.\footnote{ Reinhold, `Der Wehrt einer Gesellschaft,' 67v{-}68r.}

[This is no reason for me to deny the sad fact for mankind, that the same light of Enlightenment that illuminates the understanding of a man leaves his heart cold and that humanity may have lost more by the heart of some of its sons than it gained by their minds. The fruit of the most distinct and most difficult conviction of the most elevated thought perishes too often on the way to deeds that it has to take through sensation. There, a manifold of dissimilar sense representations awaits and threatens to stifle it in the jostling. Nothing can save it but a particular liveliness, for which he does not have to thank an innate magnanimity, but rather a skill acquired by constant exercise. This skill is virtue, goodness of the heart, nobility of the soul.]

One of the most interesting elements of this passage is the connection assumed between the senses and the heart. At first, there appears to be an opposition between the heart and the mind (\textit{Verstand/Geist}), as apparently the condition of the heart can prevent clarity of mind from spreading the beneficial influence of Enlightenment throughout humanity. However, it soon becomes clear that the real problem is the incapability of the mind to concert its thoughts into action without virtue, or a good heart. It is not very clear what Reinhold means with this `goodness of the heart', he obviously situates it in a proper relation between elevated thoughts and sensibility, so that those thoughts can bear the fruit of action. 

 These two examples cited from Reinhold's earliest works show that his preoccupation with the connection between abstract concepts and concrete action predates `Gedanken \"{u}ber Aufkl\"{a}rung', in which it became the prominent issue of Enlightenment. From the start of his commitment to Enlightenment, Reinhold is clear on the point that it has to be a \textit{Volksaufkl\"{a}rung}, which means that action will be needed alongside thought, since the intellectual elite is called on to correct some of the evils in society and religion. On several occasions he expresses concern for the process of the connection of thoughts to action. If this process fails, even the most elevated and potentially beneficial thoughts or ideas do not lead to the proper actions in the sensible world, because the motivational system, the heart, is not in it. Rational action, needed for Enlightenment, can only take place when mind and heart work together. We can compare the process through which the heart helps the thoughts become active in the sensible world of action to the process through which the philosopher leads the common man to clear and distinct concepts, as described in the first section of this chapter. The latter process may also be seen as one in which the heart plays a decisive role, as the philosopher uses a concept that is close to the heart, that of a wise and loving father, to exemplify, or make concrete and sensible, the distinct concept of God's justice. 

 Although the primary focus of the current section has been Reinhold's conviction that reason needs to connect to the concrete world of sensibility and the heart, he was just as convinced that in action and sensibility rationality must play a role to prevent excesses like idolatry and superstition. This means that already in his earliest works Reinhold stresses the complexity of human nature, consisting of several elements that need one another to function properly, although they are not consistently distinguished and analyzed at this stage of his philosophical development. 


\subsection{Human reason and human nature }


Above we have seen that Reinhold, at this early stage of his development, distinguishes several human capacities such as reason, sensibility, thought, mind and heart but at the same time stresses the importance of connecting these elements in the context of Enlightenment in order to realize its goals in real life. As there is not yet a single account of the precise relations between these elements, it will be worthwhile to look at his views on human nature in general and the role of reason in human nature.  

As already stated, according to Reinhold, man's functioning properly depends on his ability to connect his mind to his heart, his theories to his actions. For Reinhold, this theme was closely related to Freemasonry as is even clearer from another speech, `M\"{o}nchthum und Maurerey', cited earlier in this chapter, about the contrast between the institutions of monasticism and Freemasonry. According to Reinhold the essence of monasticism consists in disciplining human nature by debasing all (natural) inclinations and potentials of humanity.

Der (\ldots ) Endzweck des M\"{o}nchthums ist {--} \textbf{Z\"{u}chtigung} der menschlichen Natur. Sie haben dieser seit Adams Apfelbisse abgesagte Feindinn Gottes eine ewige Fehde zugeschworen, und glauben den heiligsten Willen des Menschenvaters zu erf\"{u}llen, wenn sie die Menschheit in allen ihren Trieben, Neigungen und W\"{u}nschen \textbf{qu\"{a}lten}, und in allen ihren Vorz\"{u}gen, Kr\"{a}ften und F\"{a}higkeiten \textbf{erniedrigten}.\footnote{ Reinhold, `M\"{o}nchthum und Maurerey,' 171. } 

[The final goal of monasticism is \textit{disciplining} human nature. They have sworn an eternal feud with this refused enemy of God since Adam ate the apple, and they believe they fulfill the most holy will of the Father of Man when they \textit{torment} mankind in all its drives, inclinations and wishes and \textit{humiliate} it in all its merits, powers and capacities.]

This degradation of human nature manifests itself on the one hand in the way of living of the monks themselves, who, because of their vows cannot take part in common human activities: they cannot own property, they cannot procreate and they have sworn obedience to their order. As Reinhold puts it in one of his reviews this is a violation of human rights. For why should God have given man certain abilities if not using them at all is considered superior to using them wisely?\footnote{ ``(\ldots ) da\ss{} so viele unserer Gottesgelehrten Gott und sein edelstes Werk, den Menschen, noch immer so wenig kennen, da\ss{} sie daf\"{u}r halten, der Urheber der Natur k\"{o}nne in den Menschen dringende Triebe geleget haben, deren g\"{a}nzliche Vernachl\"{a}ssigung an sich selbst besser und gottesgef\"{a}lliger w\"{a}re, als ein vern\"{u}nftiger Gebrauch davon (\ldots ).'' Reinhold, review of \textit{Hat dich keiner verdammt?}, by Elexia, \textit{RZ}, 1783, May 6 [235]. } Although the monks can only place such restrictions on themselves, not on others, they have sought to make the whole of Christianity abstain from the use of reason, by bringing it under the yoke of faith (cf. 172). 

It will come as no surprise that Reinhold conceives of Freemasonry as the exact opposite of monasticism. Masons value human nature and especially its most particular characteristic, reason, ``the noblest of human distinctions'' (177). The rationality of human nature is highly appreciated by Masons, whereas it is evaluated negatively by monks. Next, Reinhold discusses the respective constitutions of monasticism and Freemasonry. By depriving its members of the exercise of fundamental human rights, monasticism severs them from humanity, thereby thwarting their human nature.

Eine Zunft von Menschen, welche die s\"{u}ssesten, st\"{a}rksten, und heiligsten Bande, die den Menschen an die Menschheit kn\"{u}pfen, zerrissen haben, kann \textbf{die} vom allgemeinen Menschensinne \textbf{mehr} beherrschet werden als jedes vom menschlichen K\"{o}rper abgetrenntes Glied von der Seele? Was soll die Vernunft denen n\"{u}tzen, die da aufgeh\"{o}ret haben Menschen zu seyn? (180) 

[A kind of people who have torn the sweetest, strongest and holiest ties that bind human beings together, can \textit{such} a kind of people be \textit{any} \textit{more} controlled by common human sense than any limb torn from the human body could be controlled by the soul? What use is reason for those who have stopped being human?]

Freemasonry advocates the exact opposite. Not only are members allowed to remain connected to their worldly goods and their familiar and civil attachments, this is even stimulated, for to be a true Mason one must have learned to be a good human, that is, a good husband and citizen.\footnote{ In line with social practice at the time, Reinhold does not discuss the rational development of women in relation to their role in society (or in Freemasonry). This does not make him a misogynist, but rather serves as a reminder for the modern reader that the Enlightenment as it actually took place is not the same as the Enlightenment as it lives in our minds.}

Verm\"{o}ge ihrer Verfassung l\"{a}\ss{}t sie [Freemasonry] nicht nur jedem von uns seine Freyheit, seine G\"{u}ter, seinen Rang, sein Weib, und seine Kinder, \"{u}berhaupt alle seine nat\"{u}rliche und b\"{u}rgerliche Anspr\"{u}che; sondern sie lehret, erleichtert, und sch\"{a}rfet jedem den vollkommensten Gebrauch davon ein, und erkennet keinen f\"{u}r ihren \"{a}chten Sohn, der nicht ein guter Gatte, B\"{u}rger, und Unterthan seyn gelernet hat. (183) 

[Because of its constitution, Masonry not only allows each of us his freedom, goods, rank, wife and children, and all of his natural and civil claims in general; moreover, it teaches, facilitates and impresses upon each the most perfect use of these and does not acknowledge anyone as its true son who has not learned to be a good husband, citizen and subject.]

Whereas the monkish vows of obedience, poverty and chastity sever humans from humanity by depriving them of the need to use reason, Freemasonry stimulates its initiates to be active members of human society. It is in the field of human relations that people develop the kind of reason that Masons value. Thus, this Masonic speech shows a particular conception of the use of reason that is closely related to being truly human, connected to others, practicing precisely those natural human activities that monastic life has sought to discredit. In this sense Reinhold's understanding of human reason is practical, since human nature, epitomized by human reason is closely connected to the social and civil aspects of human activities. This is, of course, not only associated with Reinhold's Masonic, but also with his Illuminatist engagement, as the latter organization sought to establish a social and political reform on the basis of morality. 

 The connection of reason with human relations brings to mind the first two of Reinhold's extant letters, to his father and to Blumauer, both cited in Chapter 1. There, we saw that the connection to the rest of humanity and especially to his family was important to Reinhold, as he felt cut off from them being a monk. As a Mason Reinhold understood human relations as belonging to human nature and as something to be cherished by human reason, and thus placed them in a rational and positive context. The unpublished speech on the `Wehrt einer Gesellschaft' cited earlier, in the first part of this section, also uses family relations to exemplify the first stage of moral behavior. 

Die niedrigste Menschenseele ist immer diejenige, welche nicht \"{u}ber die Spanne hinaussieht, die ihr K\"{o}rper auf dem Erdboden einnihmt, und ihre eingeschr\"{a}nkte Kraft geht ganz f\"{u}r thierische Bed\"{u}rfnisse darauf. \"{U}ber sie raget die Geisteskraft hinaus die wenigstens nahe umherstehende Gegenst\"{a}nde erreichet, und dem gew\"{o}hnlichen Menschen das Wohl seiner Familie wichtig genug finden l\"{a}\ss{}t. Von einen viel gr\"{o}sseren H\"{o}he wirkt der n\"{u}tzliche B\"{u}rger aufs Wohl vieler tausend Famillien herab, und ist Wohlth\"{a}ter seines Vaterlandes. Gebet seinem Geiste noch st\"{a}rkere Fl\"{u}gel, und er wird sich zu einer H\"{o}he emporschwingen, von welcher er den Erdenkreis umfasset. Nichts geringeres als das Wohl der Menschheit wird nun die Besch\"{a}ftigung seyn die seiner Thatkraft angemessen ist.\footnote{ Reinhold, `Der Wehrt einer Gesellschaft,' 66r{-}66v.}

[The lowest human soul is always that one that does not see beyond the space its body occupies on earth and whose limited power is entirely used for animal needs. Higher than this rises the strength of mind that stretches at least to objects that are close at hand, and which makes the well{-}being of his family sufficiently important for ordinary people. A useful citizen works from a greater height for the well{-}being of many thousands of families and becomes a benefactor of his country. If his spirit has even stronger wings, it will climb to the height at which its effects encompass the whole world. Nothing less than the well{-}being of mankind will be the occupation that is fit for his powers.] 

Thus, human relations to others and the natural feelings involved in these relations are crucial to human nature. Ordinary people will direct their will to do good to their next of kin. The more strength of mind (\textit{Geisteskraft}) someone has, the wider the circle of people to which he can extend his ability to do good. In a way, those at the highest stages of this moral ladder have the abilities to treat all their fellow citizens or even the whole of mankind as an ordinary man would treat his family. Therefore, it can be no surprise that the example of the philosopher helping the common man to have a distinct concept of God's justice uses the idea `father', because this idea is common to both of them, as they are both human beings.\footnote{ Cf. section 1 of the current chapter. } Moreover, it is ordinarily connected to a set of positive feelings that the philosopher would like the common man to extend to God. By presenting God as the father of mankind there is indeed an opening to regard the rest of humanity as family. Reinhold's inspiration for this view may well be related to his Illuminatist engagement.\footnote{ Cf. Weishaupt, `Anrede,' 191{-}192.}


\section{Reinhold's Enlightenment versus blind faith: `Ueber den Hang zum Wunderbaren', Herzenserleichterung zweyer Menschenfreunden and `Skizze einer Theogonie des blinden Glaubens'}


In the previous three sections different aspects of Reinhold's views on Enlightenment have been discussed, on the basis of a representative cross section of his early writings. We saw that Reinhold sought to further the dissemination of Enlightenment by clarifying the term `Enlightenment' and to explain the phenomenon `Enlightenment' by referring to history. The history he described is the history of the development of human reason. It is of crucial importance to understand that Reinhold's view on human reason is closely connected to his view on human nature, which entails that the ideal or complete use of reason includes a connection to sensibility. On the one hand this is to ensure that the abstract concepts of reason are still meaningfully connected to concrete experience. Without such a connection good ideas can not be turned into good actions. On the other hand, sensibility needs to be guided by reason so that it will be less prone to superstition. 

All of the above{-}mentioned elements of Reinhold's views on Enlightenment come together in his criticism of certain features of established religion. This criticism also applies to certain tendencies within Freemasonry, for that matter. It is closely related to the attitude of the Illuminates, who tried to infiltrate Masonic Lodges and reform them for their own purposes. The main point of Reinhold's criticism concerns the tendency within Freemasonry to allow superstition and mysticism to take root in the organization and is expressed in the speech `Ueber den Hang zum Wunderbaren', held in the Vienna Lodge \textit{Zur wahren Eintracht}, just before his move to Leipzig in 1783. Published in 1784, this Masonic piece is thematically close to his final pre{-}Kantian work published in \textit{Der Teutsche Merkur}, `Skizze einer Theogonie des blinden Glaubens' (June 1786), which is of a more general nature, dealing with the development of religion. Since the `Skizze' is thematically related to Reinhold's earlier work, while in time it is close to the beginning of his Kantianizing period, discussing it will be helpful in assessing Reinhold's way of thinking at the time when Kant became an important philosophical influence. It must be noted that this essay also marks the continuity in Reinhold's thought, as it was recycled, with minor changes, as a part of the `Zw\"{o}lfter Brief' in the first volume of Reinhold's \textit{Briefe \"{u}ber die Kantische Philosophie}.\footnote{ Reinhold, \textit{Briefe \"{u}ber die Kantische Philosophie}, `Zw\"{o}lfter Brief,' 358{-}371. Cf. \textit{Briefe I}, Bondeli ed., 339, n. 401. } 

Although the circumstances in which and the audience for which Reinhold wrote `Ueber den Hang' and `Skizze' are quite different, the essays have similar subjects and show a similar structure. Both offer an explanatory account of the origin of blind faith or superstition, by looking at the original state of humanity, and especially the relation of man to nature. In `Ueber den Hang zum Wunderbaren' Reinhold sets out by describing the `tendency for the miraculous' against which he wants to warn as an ``illness of the soul'' that has wrought havoc in both the Christian and the Masonic traditions.\footnote{ Reinhold, `Ueber den Hang,' 125. } Reinhold's diagnosis that the Masonic world is not immune to this illness shows his Illuminatist allegiance. The view on human history presented in both `Ueber den Hang' and `Skizze' appears to be very close to the view expressed by the founder of the Illuminati, Adam Weishaupt in his `Anrede'. In `Ueber den Hang' Reinhold describes the problem in the following manner. 

Sie [this illness of the soul] \"{a}ussert sich in dem Geschmacke am Uebernat\"{u}rlichen, das sie \"{u}berall an die Stelle des Ausserordentlichen setzet, sie wittert allenthalben ausserordentliche Einfl\"{u}sse, vermenget das Unbekannte mit dem Unbegreiflichen, schaft sich Geisterwelten, und glaubet dem Urheber der Natur viel Ehre zu erweisen, wenn sie ihn \"{u}berall unmittelbar wirken l\"{a}\ss{}t, und ihn alle Augenblicke herausfordert, sein vollkommenes Werk zu verbessern.\footnote{ Reinhold, `Ueber den Hang,' 125{-}126.}

[This illness of the soul expresses itself in a taste for the supernatural, which it inserts wherever it finds something extraordinary; it senses extraordinary influences everywhere; mixes the unknown with the incomprehensible; creates worlds of spirits for itself and believes that the author of the universe is highly honored when it lets Him immediately work everywhere and summons Him every second to improve His perfect creation.] 

The dangerous tendency thus consists in preferring supernatural explanations to natural ones combined with an image of the direct and repeated divine influence upon creation. In the `Skizze' Reinhold attributes a similar tendency to see the invisible hand of God everywhere to the only partial development of reason in the earlier days of man.\footnote{ Cf. Reinhold, `Skizze,' 236} In `Ueber den Hang' he describes the psychology of man in this early stage of development in more detail. When reason had hardly developed, man's mind was fully occupied with sensations. Therefore, his image of nature entirely consisted of ``his experiences up to that moment heaped together.''\footnote{ Reinhold, `Ueber den Hang,' 126.} The occurrence of extraordinary events was not understood and their causes were sought outside nature, in a supernatural world. This, according to Reinhold, was all excusable in the case of early humanity,\footnote{ Cf. Reinhold, `Ueber den Hang,' 126; Cf. Weishaupt, `Anrede,' 169, saying that humanity as a whole has to go through the stages of childhood, adolescence, maturity. } which implies that he believes that people should know better now, since reason has progressed as well as our knowledge of nature. However, due to the activities of those who claim to speak on behalf of the divine power, this progress has been significantly hindered.\footnote{ Cf. Reinhold, `Ueber den Hang,' 128; Cf. Weishaupt, `Anrede,' 172.} The class of priests has sought everywhere to preserve its position and serve its own interests by stimulating superstition among the masses. As in `M\"{o}nchthum und Maurerey' Reinhold blames the monks for presenting nature as God's enemy.\footnote{ Cf. Reinhold, `Ueber den Hang,' 130.} The `Skizze' shows a similar diagnosis: through the influence of the representatives of established religion, God and nature are being understood as opposites. People began to attribute everything ``extraordinary, terrifying and big'' to God, while nature was only credited with the small, ordinary things and the normal order of events.\footnote{ Reinhold, `Skizze,' 237.} The thought that man would not be able to fulfill God's will with his natural powers and was always in need of supernatural assistance, was just around the corner. From there, everything that went easy or was pleasant was thought of as only natural and therefore against God's will and it was thought that painful sacrifice was needed in order to do God's will.\footnote{ Cf. Reinhold, `Skizze,' 241; Cf. Reinhold, `Ueber den Hang,' 131.} 

 Even if in Reinhold's days the progress of the natural sciences had become clearly marked and the room for supernatural explanations accordingly diminished, the danger of superstition and blind faith still remains, so both essays warn the audience. In `Ueber den Hang' this persistent danger is attributed to the natural circumstance that miraculous stories are often more engaging for the mind,\footnote{ In his tripartite article `Ueber die Natur des Vergn\"{u}gens' Reinhold would describe different ways of approaching aesthetics. The theory that looks at pleasure from a subjective perspective centres on the notion of `need' (\textit{Bed\"{u}rfnis}) in a way that appears to be related to the thought expressed here, for Reinhold claims that all needs are derived from the need for the faculty of representation to be occupied. The central role assigned to the faculty of representation is striking in light of his \textit{Versuch}. Cf. Reinhold, `Ueber die Natur des Vergn\"{u}gens,' \textit{TM}, October, 1788, 63. For the importance of this article with regard to Reinhold's later systembuilding, see Lazzari, `K.L. Reinhold: Die Natur des Vergn\"{u}gens und die Grundlegung des Systems.'} because they are more appealing to the senses.\footnote{ Cf. Reinhold, `Ueber den Hang,' 132.} The remnants of the taste for the miraculous in culture and society continue to spell danger, even for those who claim that they are the representatives of the Enlightenment. Therefore the Masons in particular must be on their guard, lest they should seek the same miracles in Masonry as they have freed themselves from when rejecting monkish superstition.\footnote{ Cf. Reinhold, `Ueber den Hang,' 133; Cf. Weishaupt, `Anrede', 191.} In the speech on the `Wehrt einer Gesellschaft' discussed earlier in this Chapter, Reinhold proposed that the mysteries of Freemasonry are to serve as a test. 

Es geh\"{o}rt unstreitig \textit{Geisteskraft} dazu um durch die manigfaltigen H\"{u}llen der Sinnbilder bis zu jenen Wahrheiten hindurchzudringen, welche die vorsichtige Klugheit erleuchteter Vorfahren als eine Hinterlage f\"{u}r scharfsichtige Enkel zur\"{u}ckelie\ss{}. \footnote{ Reinhold, `Der Wehrt einer Gesellschaft,' 67r.}

[Strength of mind is undoubtedly needed in order to penetrate the manifold covers of sensible images to those truths that the cautious prudence of inspired ancestors left as a deposit for sharp{-}sighted descendants.]

Those who take these mysteries to be either merely a game or a reference to supernatural hidden truths have no place in true Masonry, that is, Illuminatism. In `Ueber den Hang' Reinhold, arguing that Masonry and superstition are incompatible, exhorts his audience to stick to nature, which sufficiently occupies reason {--} not unlike the general Church mentioned in the review discussed above in section 3.1. There is no need at all to go beyond.\footnote{ Cf. Reinhold, `Ueber den Hang,' 137.} In his attempt to explain man's taste for the miraculous and the supernatural by means of natural psychological mechanisms and in his recommendation to stick to nature, Reinhold can be called a naturalist, especially with regard to the supernatural aspects of religion. 

 Outside of the direct Masonic context, the `Skizze' takes a dim view of the current situation of mankind as a result of blind faith. Notwithstanding the progress that has been made in the sciences, which has dealt severe blows to the importance of the supernatural in explanations, the opposition between God and nature and as a consequence between God and reason endures with regard to morality.\footnote{ Cf. Reinhold, `Skizze,' 242. } The only consolation offered by the `Skizze' is the larger picture of human history, which learns that `blind faith' is but a passing phase in the course of human development, which will only last as long as the mental powers (\textit{Geisteskr\"{a}fte}) of man are not yet fully developed. 

Es war eine Zeit, da die Geisteskr\"{a}fte der Menschen zu schwach, {--} und es k\"{o}mmt eine Zeit, da sie zu stark seyn werden {--} das Joch des \textbf{blinden Glaubens} zu ertragen.\footnote{ Reinhold, `Skizze,' 229; Cf. Weishaupt, `Anrede,' 168.}

[There was a time when the mental powers of man were too weak {--} and there will be a time when they will be too strong {--} to bear the yoke of \textit{blind faith}.] 

The period of blind faith is the transitional period in which man's mental powers have developed to a certain extent, but not fully. Reinhold pays special attention to the period that went before and the one that will come after the phase of blind faith. In the early period, before reason had begun to develop, the relation between God and mankind was immediate and uncomplicated. Man could not ``distinguish God from nature, the only organ through which He could become visible to him.''\footnote{ Reinhold, `Skizze,' 229. } The original relation between man and God was thus a harmonious one, and divine law was followed spontaneously. Here, the problem of a discrepancy between the mind and the heart does not arise. As this situation is characterized as `innocent', it can be regarded as the philosophical version of Eden. This paradise was lost because of the development of reason in answer to population growth.\footnote{ Cf. Weishaupt, `Anrede,' 170. } The incomplete development of reason leads to blind faith, but once reason will be fully developed, man will learn ``to will from \textit{free} \textit{choice}'' the same things he had spontaneously wanted in his original state of innocence.\footnote{ Reinhold, `Skizze,' 230.} Enthusiastically, Reinhold sketches man's situation in those times to come. 

Die unendliche Macht, Weisheit und G\"{u}te, die durch ihre vereinigte W\"{u}rkung sein kindliches \textbf{Herz} gelenkt hatte, tritt dann als reines \textbf{Ideal der h\"{o}chsten Vollkommenheit} aus dem Chaos seiner verworrenen Begriffe hervor, und wird der h\"{o}chste leitende \textbf{Grundsatz seiner Vernunft}, so wie die vornehmste \textbf{Triebfeder seines Herzens}.\footnote{ Reinhold, `Skizze,' 230{-}231.}

[The infinite power, wisdom and goodness that had steered his childish \textit{heart} with their unified activity then appear as pure \textit{ideal of highest perfection} out of the chaos of his confused concepts and this becomes the highest leading \textit{principle of his reason} as well as the most important \textit{incentive of his heart}.] 

The vision sketched here entails the return of the original harmony between God and man, but on another, higher level, understood by reason. This makes the period of blind faith and superstition only a transitional phase in the history of the relation between God and humanity. The use of the term `Triebfeder' here places this article firmly in the context of Reinhold's pre{-}Kantian Enlightenment writings. In section 3.1 of the present chapter we saw that he, both in his Masonic speech and in a review, attached great importance to the link between intellectual efforts and our motivational system. In the `Skizze' the highest stage of human development is described in terms of an ideal that is at once the highest principle of reason and the most important incentive for action. Another passage, comparing the original harmony with the expected outcome of the development of human reason, can be called emblematic for Reinhold's views on Enlightenment. 

Es geh\"{o}ren Jahrtausende von vielf\"{a}ltigen und meistens traurigen Erfahrungen dazu, bis der \textbf{freygelassene} Z\"{o}gling der Natur mit \textbf{freyer Wahl} wollen lernt, was er \textbf{im Stande der Unschuld }aus \textbf{unvermeidlichem Triebe} gewollt hatte, und in der Gesellschaft als \textbf{B\"{u}rger} durch \textbf{Vernunft} findet, was ihm als \textbf{blossen Menschen} in seinem engen Familienkreise durch \textbf{Instinkt} eingegeben worden war.\footnote{ Reinhold, `Skizze,' 230.}

[It takes millennia of various and mostly sad experience until the \textit{released} pupil of nature learns to want by means of his \textit{free choice} that which he, \textit{in the state of innocence}, had wanted from \textit{inescapable drive}, until he finds as a \textit{citizen} in society by means of \textit{reason} that which, in the small circle of his family, had been given to him as a \textit{mere human being} by means of \textit{instinct}.]

All key elements of Reinhold's views on Enlightenment are contained in this passage. First of all, it presents man as a natural being who develops reason over time; that is, human beings are historical beings, whose reason develops throughout history. Secondly, the passage points to a parallel between moral behavior in society on the one hand and natural human behavior in smaller social units such as the family on the other. Moral behavior is presented as the rational extrapolation of the behavior that comes naturally to human beings as members of a family. Finally, the development of reason that makes this extrapolation possible is by no means a straightforward and easy process; things go wrong along the way. 

 In both `Ueber den Hang' and `Skizze' the presumed state of innocence has not endured because a misguided form of religion has obscured the understanding of the relation between God and nature, between God and mankind. The opposition between God and nature, forged by the defenders of blind faith, is understandable in light of the development of reason, but it hinders a proper understanding of nature and a proper religion. A new, rational state of harmony can be achieved through knowledge of and reverence for nature.  

As it is the task of Enlightenment to oppose blind faith and bring humanity closer to the more elevated state in which God and nature will be reconciled, it is worthwhile to take a closer look at Reinhold's \textit{Herzenserleichterung zweyer Menschenfreunde}. This booklet is not so much a philosophical as a polemical work. It warns against the religious views of Johann Caspar Lavater, which may seem compatible with Enlightenment, but according to Reinhold are not. In a wider sense, the book deals with blind faith, the origins of which are described in the works discussed previously, in the context of the progress of Enlightenment. Published anonymously in 1785, it was written as a fictitious exchange of letters between Lichtfreund and Wahrmund, the latter of whom probably represents Reinhold's own views most closely. Reinhard Lauth has placed this work in an Illuminatist context, by pointing out Reinhold's contacts at this time with Joachim Christoph Bode, one of the main Illuminates in Saxony. Lauth claims Bode instructed Reinhold to write the \textit{Herzenserleichterung} to warn against certain mystic tendencies in (French) Freemasonry and the crypto{-}Catholicism of for instance Lavater, which was also thought to be stimulated by secret societies, other than the Illuminates.\footnote{ Lauth, `Nouvelles R\'{e}cherches,' 614.} Unfortunately, Lauth provides no evidence that Bode indeed inspired or instructed Reinhold to write this work. Nevertheless, the aim Lauth claims it served is clearly present in the text, which evolves as a discussion on the progress of Enlightenment. Lichtfreund represents a very optimistic outlook on the way Enlightenment has progressed thus far and has high hopes for the time to come, whereas Wahrmund presents a more realistic outlook, warning that there is still a lot of work to be done before victory can be claimed. 

 It will not be necessary here to go into details concerning the precise arguments of the discussion, in order to understand Reinhold's views on the struggle between blind faith and Enlightenment. For this it will suffice to look at a few telling passages. Lichtfreund's optimistic view on the progress that is being made is obvious from the following passage, claiming that opposite systems may peacefully coexist in one person.

Ja, mein Freund, Orthodoxie und Aufkl\"{a}rung, die festeste Anh\"{a}ngigkeit an einem Systeme mit der sanftesten Schonung aller \"{u}brigen, der feurigste Bekehrungseifer mit der uneingeschr\"{a}nktesten Duldung, der entschiedentste Wunderglauben mit der bed\"{a}chtlichsten Ueberzeugung, die verworrensten Begriffe von \"{u}bernat\"{u}rlichen Gnadenwirkungen mit den hellsten psychologischen Einsichten, und \textit{theologischer} Ha\ss{} mit \textit{philosophischer} Liebe der Natur!\footnote{ Reinhold, \textit{Herzenserleichterung}, 31. }

[Indeed, my friend, orthodoxy and Enlightenment, the strongest attachment to one system with the gentlest treatment of all others, the most passionate zeal to convert with the most unlimited tolerance, the most resolute belief in miracles with the most carefully considered convictions, the most confused concepts of the effects of supernatural grace with the clearest insights of psychology and \textit{theological} hatred with \textit{philosophical} love of nature!]

Such is Lichtfreund's optimistic observation of Lavater, in whose views he sees a sign that even in those who clearly belong to the world of orthodoxy, new and enlightened ideas find some foothold. Note that, as in the essays discussed above, the world of blind faith is linked to a religious hatred of nature, while the enlightened point of view entails the love of it.\footnote{ Elsewhere in the \textit{Herzenserleichterung} the origins of this view of nature are located in the ignorance of man, which is a thing of the past. Cf. Reinhold, \textit{Herzenserleichterung}, 12{-}13. } Notwithstanding Lichtfreund's enthusiasm, Wahrmund cannot share the optimism regarding Lavater. By analyzing Lavater's views on Christianity and tolerance he shows that Lavater endangers Enlightenment, rather than promoting it. In the end, the friends conclude that Lavater ``would turn out to be an instrument of Enlightenment, in the sense in which the popes and monks were such for Germany, because they went too far for our ancestors!''\footnote{ Reinhold, \textit{Herzenserleichterung}, 110; Cf. Weishaupt, `Anrede,' 177{-}178. } We have encountered a similar attitude to the mechanisms of history in Reinhold's essay on the sciences, presented in section 2.2 of the present chapter. There, he presented the downfall of medieval science as precipitated by its own oppressive character. The way in which the \textit{Herzenserleichterung }presents Lavater's work as a contribution to Enlightenment in a negative sense is also strongly reminiscent of Reinhold's evaluation of monasticism in `M\"{o}nchthum und Maurerey' which was discussed in section 3.2. 

 Looking at Reinhold's pre{-}Kantian works regarding the history of religion, it is obvious that his views in this field are strongly connected to his allegiance to Enlightenment, especially in its Illuminatist version. He opposes blind faith and superstition not only in society in general, but also in Masonic circles. His views on the origins of these shortcomings of the religion of his day are clearly inspired by Illuminatist writings such as Weishaupt's `Anrede'. Although a substantial part of Reinhold's criticism of established religion was presented in the relatively private context of Freemasonry, the present section has shown that the themes he discussed in his Masonic writings did not fail to find their way into his works aimed at a more general public. The way in which Reinhold criticizes tendencies in established religion and in Freemasonry from a historical, developmental point of view is definitely linked to his more general thoughts on the development of human reason and the place of Enlightenment in it. 


\section{Evaluation }


As shown in the first section, Reinhold's efforts on behalf of Enlightenment partly concentrated on providing the proper conception of Enlightenment, as he believed this to be of crucial importance for its dissemination. Naturally, this theme is most clearly present in Reinhold's works concerning Enlightenment itself, such as `Gedanken \"{u}ber Aufkl\"{a}rung'. We have also seen that his efforts on behalf of Enlightenment were closely connected to his Masonic engagement, for instance in `M\"{o}nchthum und Maurerey'. Clearly, his Enlightenment thought concentrated on discussing the practical means of Enlightenment as well as on conceptual clarity. Regarding the contents of Reinhold's conception of Enlightenment, it has become clear in the second and third sections of this chapter that it is closely related to his views on the historical development of humanity and on human nature. This was exemplified in the fourth section, where Reinhold's criticism of the `blind faith' opposed to Enlightenment was taken as a prime example to show how these general characteristics of his conception of Enlightenment are expressed when it comes to a subject that was very important to him, religion and its abuses. He was critical of blind faith and superstition in society and Freemasonry alike. Moreover, the three works discussed in the previous section provide a rough idea of the ultimate goal after which Reinhold's Masonic/Illuminatist Enlightenment strives, and show how this goal is connected to his ideas on history and human nature discussed earlier.

 In all three works discussed in section four the opposition of God and nature, as presented by blind faith, is supposed to be remedied by Enlightenment. Nature is not something to hate, but is rather to be loved and investigated, as man is a part of it. In undoing the evils of blind faith, Enlightenment will rid religion of all things mystic and supernatural. This is clearly a rather radical conception of the goals of Enlightenment, based on a naturalistic view of religion. Based on Reinhold's own later statements, there was a time in which he was committed to materialism, which according to Lauth is strongly connected to his Illuminatist ideas.\footnote{ Reinhold, \textit{Versuch}, 12, 15; Lauth, `Nouvelles R\'{e}cherches,' 612.} Especially the emphasis he puts on the abolition of the opposition of God and nature and the view that nature is the organ of God,\footnote{ Reinhold, `Skizze,' 229.} point in the direction of identification of God and nature, associated with a form of Spinozism. A naturalistic outlook on religion is easily taken to entail atheism or materialism. Based on Reinhold's early works on religion it is justifiable to say that he went through a phase of materialism/atheism. His naturalism links him on the one hand, as we have seen, to the Illuminatist views on the history of man, and on the other to Herder, in whose \textit{Ideen zur Philosophie der Geschichte der Menschheit }(1784) he started taking an interest after he had moved to Weimar, as we shall see in the following chapter.

 There appears to be some tension, however between this naturalism and Reinhold's own description of Enlightenment discussed in the first section of the present chapter. His clarification of the concept of Enlightenment was undertaken to make people understand that it is a good thing, that true Enlightenment does not harm true religion.\footnote{ Batscha understands these claims as being naive, showing that Reinhold was not yet aware of the possible tension between faith and reason. Cf. Batscha, \textit{Karl Leonhard Reinhold}, 49.} Those who fear that Enlightenment will destroy religion are portrayed as having either a mistaken conception of Enlightenment or of religion and quite possibly of both. Accordingly, Reinhold claims that the aim of Enlightenment with regard to religion is purification, instead of destruction. This involves an image of religion as consisting of several parts, of which some belong to the core, and others can easily be taken away without damage to the core of religion, which in Reinhold's thought corresponded to early Christianity. This way of thinking about Enlightenment and religion conveys the impression of a moderate Enlightenment, which does not attack religion, but rather aims to improve it by getting rid of the bad habits that have developed over time. 

In the first section of this chapter we also saw that Reinhold, in his article on Enlightenment, stresses the importance of enlightening not just the intellectual elite, but also the common people, which were more likely to cling to the traditional religion that they had been brought up with. In order to reach them, Reinhold suggested that the abstract concepts of the philosophers should be explained in more concrete terms, using concepts that are familiar (literally) to everyone, such as `father'. The process of a \textit{Volksaufkl\"{a}rung} will have to take the intellectual capacities of the common people as a guideline. If they tend to cling to traditional religion, Enlightenment will have to take these religious sentiments seriously in order to deliver the message. All in all, Reinhold's view on Enlightenment as expressed in `Gedanken \"{u}ber Aufkl\"{a}rung' is very considerate towards religion as such, reassuring the public that Enlightenment will not endanger religion, but will rather fortify the important core, as it gets rid of the now superfluous rubbish. 

 So, how can the view that Enlightenment and established religion are compatible be combined with Reinhold's more naturalistic views on religion, which leave little room for the revelation and the supernatural. If the chronology of the different pieces is considered, it turns out that the more radical and the more moderate view of Enlightenment coexist peacefully. We can, however, clearly distinguish a private and a public view. Many of the more radical thoughts are found in the more or less private context of Reinhold's Masonic and Illuminatist writings, whereas the more moderate view is predominantly present in reviews and essays intended for a more general public. All of Reinhold's statements that suggest an identification of God and nature are found in his Masonic works. From the works discussed in the fourth section `Ueber den Hang' especially addresses a Masonic audience, while \textit{Herzenserleichterung} probably has Illuminatist origins and was published anonymously. The `Skizze', being the most public of the texts, is less explicitly naturalistic. 

In combination with Reinhold's enthusiasm for a \textit{Volksaufkl\"{a}rung} and his education in Josephite Vienna, we can understand his idea of Enlightenment as elitist and looking for a top{-}down approach.\footnote{ Cf. Sauer, \textit{\"{O}sterreichische Philosophie}, 68.} The members of the intellectual elite discuss radical ideas with far{-}reaching consequences for society amongst themselves, as Reinhold does in his Masonic works. In the meantime, they prepare the rest of society, which lags behind, for Enlightenment. The ideas of the intellectual elite, however, cannot be introduced out of the blue, as the rest of society is not sufficiently educated to understand them. Instead, many people feel comfortable with traditional religion. Reinhold's suggestion for communicating enlightened concepts by means of natural concepts that everyone is familiar with, such as `father', points in the direction of educating the masses in a way that is especially tailored to their mindset. So even though the ultimate goal of the Illuminates was an egalitarian society without a ruler, the means of achieving this were quite autocratic, not involving any conscious consent of those educated in this way. This education is not meant to introduce radical ideas, but rather to take preliminary steps towards a new social order and to protect the common people from superstition. 

 Reinhold would have found support for such an elitist approach to the Enlightenment of the masses in the Illuminatist ideology. In the speech `Der Wehrt einer Gesellschaft', cited in section 3.2 of the present chapter, he presented a moral hierarchy according to which those who work towards the general good of humanity (read: the true Masons or Illuminates) cannot but be society's best hearts and minds. We have also seen that Reinhold viewed their `mysteries' as an important instrument to make sure that their work is not hindered by those of lesser capacities. Clearly, the Illuminates were confident that they knew what society needed and how to fulfill those needs. Their ideas on Enlightenment and religion were more radical than could be shared with a general public, which brings us to the question of the status of Reinhold's more moderate statements about the compatibility of Enlightenment and religion. It is important to keep in mind that the crucial element of the claim that Enlightenment and religion are compatible is Reinhold's stress on determining the proper concept of Enlightenment. He is not claiming that any version of Enlightenment is compatible with any version of religion, only that Enlightenment properly understood is compatible with true religion. At the end of section 2.1 it was shown that this true religion was widely identified as the core of Christianity, historically located with the first Christians. Now, from Weishaupt's `Anrede', which presents a history of man as well as a historical justification for the existence and the aims of the Illuminates, we know that they saw their society as the keeper of this all{-}important core, which in the Church had become corrupted over the ages. As we have seen in the previous section, Reinhold's accounts of the history of this corruption in `Ueber den Hang' and `Skizze' are clearly inspired by an Illuminatist perspective. The phases of the history of the human religion as presented in the `Skizze' neatly correspond to Weishaupt's views as presented in the `Anrede'. The true religion that Reinhold speaks about may thus well be the naturalistic religion of the Illuminates; in this case his more moderate pieces would be read differently by his fellow Illuminates than by the general public. To Reinhold and other Illuminates, however, the difference between their naturalistic conception of religion and the religion of the common people may not have appeared as big as it may appear to us. In the end, they do have room for some sort of religion in their system, but it appears to be based on morality rather than on supernatural events. They believe the germs of morality are natural to mankind. It can be developed by building upon the natural loving relation between members of a family.\footnote{ This thought may also be behind the moral hierarchy presented in Reinhold's speech, where the strength of mind determines the width of his moral action. Ordinarily this circle of action is limited to one's own family, but with sufficient strength of mind, it might be extended to encompass all of mankind. Cf. section 3.2 of the present chapter.} Although there is no need to doubt that the Illuminates considered their strategy to contribute to the good of all mankind and to be justified by their perspective on human nature and human history, we still may regard it as a strategy of popular deception. 

 The above exposition shows why Reinhold put such emphasis on the historical aspects of Enlightenment (see the second section), that is, its place in the inevitable development of human reason. Both his moderate and his more radical statements are clearly inspired by the Illuminatist view on human history and the place of their organization in it. The difference in tone between the more private Masonic writings and his more public essays and reviews can be easily explained by this view of human history, which entailed an elitist view of Enlightenment and a sharp distinction between esoteric and exoteric doctrine. Reinhold's work on the Hebrew mysteries exemplifies the uses of such an elitist approach.\footnote{ Sabine Roehr has brought the connection between Reinhold's portrayal of the Hebrew religion in this work and his attitude towards Masonry to the fore in her `Reinholds \textit{Hebr\"{a}ische Mysterien oder die \"{a}lteste religi\"{o}se Freymaurerey}: Eine Apologie des Freimaurertums.'} The premise of that work is that the Egyptian mysteries contained a monotheistic esoteric doctrine with pantheistic features (the one God is described as `all that is'). Initiation into this doctrine involves the revelation of the falsity of the polytheism that is taught at lower levels (lesser mysteries). The legislation of Moses made this monotheism the \textit{exoteric} doctrine of the Hebrews, by means of a wholesale initiation of the Hebrew people into the falsity of polytheism. However, revealing a doctrine to a whole people that was originally reserved for the happy few is a risky business. Reinhold accordingly interprets the extensive rites and regulations of the Hebrews as remnants of their customs in Egypt, adapted in a way that would satisfy the religious appetites of the masses to the extent that they would not return to polytheism.\footnote{ Cf. Reinhold, \textit{\-Die Hebr\"{a}ischen Mysterien}. } We can see a parallel here with the situation of the Illuminates, who need to adapt their views in a way that does not alienate the general public, while trying to keep it from superstition. We have seen that their ultimate goal involved radical ideas not socially acceptable at the time, such as a naturalistic explanation of religion, which may be close to the pantheistic monotheism that Reinhold describes as the esoteric doctrine of the Egyptian mysteries that in time became the source of Judaism and of Christianity. 

 Apart from their radical tendencies, Reinhold's ideas regarding the final stage of the development of human reason give us valuable insights into his pre{-}Kantian views on the constitution of man's mental capacities. In several of his works discussed throughout this chapter we have seen that the final stage of human development, identified as Enlightenment, is characterized as a period in which sensibility/feeling/heart is successfully combined with reason/understanding/mind. The first complex of concepts is associated with the concrete world of everyday life, whereas the second set relates to abstract and universal cognition. Since Reinhold is not yet presenting a \textit{system} of his own, his discussion of this theme does not involve a set of clearly defined concepts, but rather a complex of related terms placed in contradistinction to one another. In accordance with such a description of the final stage, the earlier phases in the history of humanity are characterized as periods in which one or the other of these two had the upper hand. In the `Skizze' we have witnessed how the original state of humanity is associated with the spontaneous or instinctive use of sensibility, in harmony with God's will, without interference from reason. Since the development of reason, there has been a struggle to recover that harmony, which is only to be instituted with the completion of the development of reason, including a balanced relation to sensibility. We have also seen that the incomplete development of reason is attributed to the hindrances instituted by blind faith in `Ueber den Hang' and in the essay on the sciences.\footnote{ Cf. Reinhold, `Ueber den Hang,' 128; Reinhold, `Die Wissenschaften,' 39.} Elsewhere, it has become clear that the full development of human reason not only involves its freedom from blind faith, but also entails a reconnection to the sensible world. In section 3.1 this has been shown by means of two very early works, the review on the veneration of the Sacred Heart and the speech on `Der Wehrt einer Gesellschaft'. It is also clear from the example Reinhold gives in `Gedanken \"{u}ber Aufkl\"{a}rung', discussed in the first section of this chapter. This view on the completion of the development of human reason was also apparent from the `Skizze', where it was also explicitly claimed that, by means of reason and sensibility, man would come to regain what had been lost when reason had only partially developed.\footnote{ Reinhold, `Skizze,' 231.} The combination of head and heart is central to Reinhold's conception of Enlightenment, both in its public and in its more private expression. In \textit{Herzenserleichterung} Lichtfreund expresses his optimism regarding the progress of Enlightenment accordingly, apparently convinced that the final dawn has already broken. 

Das Licht, das den Verstand erleuchtet, erw\"{a}rmet nach und nach auch die Herzen; unsre besten K\"{o}pfe werden zugleich unsre besten Menschen, und auf diese Art sowohl willig als geschickt, von ihren Einsichten nicht immer den gl\"{a}nzendsten, sondern den gemeinn\"{u}tzigsten Gebrauch zu machen.\footnote{ Reinhold, \textit{Herzenserleichterung}, 16{-}17. } 

[The light that illuminates the understanding, starts to warm the heart slowly but surely; our best minds at the same time become our best men, who are in this way both willing and able to employ their insights not in the most splendid way, but in the way that is most useful to everyone.]

As has been clear throughout the present chapter, the emphasis on the combination of intellectual and sensible capacities of man is one of the main characteristics of Reinhold's view on Enlightenment. The above citation confirms that he was by no means only a theorist of Enlightenment, but rather sought to bring about Enlightenment by participating in the activities of the intellectual elite and by producing reviews and articles to guide a wider audience. Through the works regarding blind faith discussed in section 4 we have seen that this characteristic is strongly connected to Reinhold's Masonic and Illuminatist background. This background also helps us to understand how he reconciles a radical, almost Spinozist view on God and nature with the publicly expressed view that Enlightenment poses no threat to religion at all. 

 The key to this reconciliation is to be found in Reinhold's conception of reason. As we have seen, he believes reason to be an elevated part of human nature, which has developed throughout human history. It is not primarily associated with abstract speculation, but rather with the rational direction of human social enterprise in the real and concrete world. In its full development, it will not harm religion because it will have come to understand religion properly. This is possible because the supernatural religion as it is practiced has only developed because human reason was not fully developed yet. Once reason is fully developed, the religion it prescribes will be the fulfillment of the natural and original harmony between God and man. It is important to keep these characteristics of Reinhold's conception of Enlightenment in mind, for we shall see in the following chapters that they were crucial for his reception of Kant. 

