
\chapter{Acknowledgements}


When I decided to apply for the PhD{-}position that resulted in the present study, I knew little more about Reinhold than Reinhold knew about Kant when he decided to write against Kant on Herder's behalf in 1785. Like Reinhold, I was tempted by a job opportunity to engage with the writings of a philosopher I was not yet familiar with. And like Reinhold, I fancy that I have been able to make an interesting contribution to the field because of my outsider perspective. (Although of course I do not expect it to have the far{-}reaching impact that Reinhold's work on Kant has had.) 

During the past four and a half years I not only learnt a lot about Reinhold and some of his contemporaries. I also learnt how to process heaps of source material and how to develop and present my own views about it. Since I could not have done this without the help and support of others, it is only right that some should be thanked here. 

First of all, I would like to thank my supervisor, Ernst{-}Otto Onnasch, and \textit{promotor}, Wim de Jong for trusting me to bring this project to fruition. Both have, in their own specific ways provided both encouragement and criticism when it was needed. Their different styles of commenting on my writings complemented each other. I would like to thank Ernst{-}Otto for the inspiring and confusing discussions and for sensitizing me to the philosophical topics of late eighteenth{-}century Germany. I thank Wim for always being alert to the little slip{-}ups in my presentation and for keeping a watchful eye over the terminology and structure of my writings. 

Although writing a dissertation is usually a lonely job, I was glad not to be the only \textit{promovendus} in this project. Hein van den Berg, with whom I shared an office, supervisor and \textit{promotor}, proved to be a great help and support. Always ready to comment on what I was writing, providing criticism and encouragement. In turn I have learnt a lot from reading his work and discuss whatever he happened to be dealing with at any given time. There was also coffee, lunch and philosophically totally irrelevant conversation, for which I am grateful as well. Without Hein, writing this dissertation would really have been a lonely job. 

Someone else who was always ready to comment on what I had written was Job Zinkstok, whose attention for detail is admirable. Discussing Kant's first \textit{Critique} over dinner at his place with him and Hein was both useful and truly enjoyable. I would further like to thank Job and his brother Roel for making the finished product look this good. 

Since the project was carried out at the Department of Philosophy of the Vrije Universiteit, I thank its board for its hospitality, which allowed me to develop my teaching and organizing skills. The institutional setting also provided further comments and support from participants in the research group, which convened in different forms. Apart from the people mentioned above, I thank Reinier Munk, Ludwig Geijsen and Karel Mom for their contribution in that context. I am also grateful for the opportunities I had to present my work at conferences in Houghton (NY), Montr\'{e}al and Helsinki. It was encouraging to find attentive listeners and commentators in these settings. 

 I have not only received support and help from the people at the workplace. A project like this definitely benefits from being supported by family and friends. Therefore I would also like to thank my parents for always having encouraged me to make my choices on the basis of what would make me happy. Over the years (not just the last four) they have helped me get myself together when I was confused and shown sincere and deep interest in whatever I was involved in. A special word of thanks for my mum, whose comments helped improve my English in so many ways. She is a true hero for reading through this whole dissertation to check it. I take full responsibility, of course, for whatever errors remain. 

 Since writing a dissertation is rarely a smooth process, it is not always easy on the home front. Beer Meijlink, my love, has no real taste for philosophy, yet he lovingly tries to understand what I am up to and reassures me whenever I am plagued by self{-}doubt. For that I am deeply grateful. Further, he is always happy to distract me from work when it is needed. My life would be less joyful without him. 

This study is the result of research undertaken within the VIDI{-}project `The Quest for the System in the Transcendental Philosophy of Immanuel Kant', funded by NWO (Nederlandse organisatie voor Wetenschappelijk Onderzoek), and directed by Ernst{-}Otto Onnasch. 

