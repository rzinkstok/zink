
\chapter{ `Practical reason' from Versuch einer neuen Theorie des menschlichen Vorstellungsverm\"{o}gens to the second volume of Briefe \"{u}ber die Kantische Philosophie}


Although Reinhold had written his \textit{Versuch einer neuen Theorie des menschlichen Vorstellungsverm\"{o}gens }with the aim of creating unanimous support for the Kantian philosophy, the work failed to achieve that object. Instead of securing support for the Kantian project, the project of providing the premises of Kant's theory of cognition solicited criticism from Kantians and anti{-}Kantians alike.\footnote{ Among those who did not react positively to the \textit{Versuch} in review, for instance, were Johann Friedrich Flatt (1759{-}1821) (anti{-}Kantian) and Karl Heinrich Heydenreich (1764{-}1801) (Kantian). These and other reviews regarding Reinhold's \textit{Versuch} and later \textit{Elementarphilosophie} have been collected. See Fabbianelli ed., \textit{Die zeitgen\"{o}ssischen Rezensionen der Elementarphilosophie K.L. Reinholds}. } For Reinhold the years following the publication of the \textit{Versuch} were filled with responding to his critics and revising the \textit{Elementarphilosophie}, until, in 1797, he admitted ``that by means of [Fichte's] \textit{Wissenschaftslehre} the scientific foundation that was sought by the \textit{Elementarphilosophie }(\ldots ) has actually been found.''\footnote{ Reinhold, \textit{Auswahl vermischter Schriften: Zweyter Theil}, xi. } The changes in Reinhold's thought taking place in the early 1790s were not limited to his \textit{Elementarphilosophie}, that is, the attempt to base a scientific philosophy upon one single fundamental principle that developed during the discussions on his \textit{Versuch}, but his practical philosophy underwent significant development as well.\footnote{ These developments are not independent of one another. According to Lazzari, the final attempt to save elementary philosophy must be understood as the result of developments in Reinhold's practical philosophy. Cf. Lazzari, \textit{Das Eine, was der Menschheit Noth ist}, 226. Fabbianelli has likewise pointed out the importance of Reinhold's practical philosophy in this phase of his development, enabling Reinhold to reconsider his \textit{Elementarphilosophie}. Cf. Fabbianelli, Introduction to \textit{Beitrage II}, XVII. } In the \textit{Versuch} practical philosophy had only appeared at the very end of the work and in a polemical context, as we have seen in the previous chapter. The practical benefits of the Kantian philosophy were presented as consequences, rather than as subject matter of philosophy. Although at the end of the `Grundlinien' Reinhold hinted that he would be publishing a theory of the faculty of desire, this promise was not fulfilled. However, in the years following the publication of the \textit{Versuch} he did develop a moral philosophy on the basis of Kant's. This development can be traced in several articles published in \textit{Der neue Teutsche Merkur}, some of which reappeared in an adapted form in the second volume of Reinhold's \textit{Briefe \"{u}ber die Kantische Philosophie} (1792). This is, however, only a preliminary end of the development that was continued in the fourth essay of the second volume of \textit{Beytr\"{a}ge zur Berichtigung bisheriger Mi\ss{}verst\"{a}ndnisse der Philosophen} (1794)\textit{,} entitled `Ueber das vollst\"{a}ndige Fundament der Moral', and in his later \textit{Verhandlungen \"{u}ber die Grundbegriffe und Grunds\"{a}tze der Moralit\"{a}t }(1798).\footnote{ Reinhold, \textit{Beytr\"{a}ge zur Berichtigung bisheriger Mi\ss{}verst\"{a}ndnisse der Philosophen. Zweyter Band, die Fundamente des philosophischen Wissens, der Metaphysik, Moral, moralischen Religion und Geschmackslehre betreffend }(1794); Reinhold, \textit{Verhandlungen \"{u}ber die Grundbegriffe und Grunds\"{a}tze der Moralit\"{a}t aus dem Gesichtspunkte des gemeinen und gesunden Verstandes, zum Behuf der Beurtheilung der sittlichen, rechtlichen, politischen und religi\"{o}sen Angelegenheiten} (L\"{u}beck and Leipzig: Bohn, 1798).}

The aim of this final chapter is to present and analyze the development in Reinhold's practical philosophy from the \textit{Versuch} until the second volume of the \textit{Briefe \"{u}ber die Kantische Philosophie} (\textit{Briefe} \textit{II}). Because of its focus on Reinhold's practical philosophy the contemporaneous development of his more theoretically oriented \textit{Elementarphilosophie} falls outside the scope of the current chapter.\footnote{ The development of Reinhold's \textit{Elementarphilosphie }is mainly documented in two volumes of \textit{Beytr\"{a}ge zur Berichtigung bisheriger Mi\ss{}verstandnisse der Philosophen} (1790 and 1794) and \textit{Ueber das Fundament des philosophischen Wissens nebst einigen Erl\"{a}uterungen \"{u}ber die Theorie des Vorstellungsverm\"{o}gens }(1791). For an extensive overview and analysis of Reinhold's struggle with the deduction of the whole of philosophy from a single fundamental principle, cf. Bondeli, \textit{Das Anfangsproblem bei Karl Leonhard Reinhold}. } Instead we will focus on some of the essays he produced for \textit{Der neue Teutsche Merkur }(\textit{NTM}) after the publication of the \textit{Versuch}.\footnote{ \textit{Der neue Teutsche Merkur} was the new name of \textit{Der Teutsche Merkur} as of 1790, reflecting Wieland's desire to start his magazine anew. Nothing much changed, however, apart from the fact that, in the early 1790s, Wieland himself contributed many pieces reflecting his current opinions on the situation in post{-}revolutionary France. Cf. Wahl, \textit{Geschichte des Teutschen Merkur}, 203{-}209.} Since they were later adapted to be included in the second volume of \textit{Briefe \"{u}ber die Kantische Philosophie} they provide us with very useful material regarding the development of Reinhold's moral philosophy. Taking the `Grundlinien'{-}chapter of the \textit{Versuch} as a starting point and \textit{Briefe II} as the end of this development, the focus will be on the relation of practical reason and the will in these essays. My analysis of this development leans on the important work done by Lazzari in establishing the chronology and circumstances of Reinhold's development towards distinguishing between practical reason and the will.\footnote{ Cf. Lazzari, \textit{Das Eine, was der Menschheit Noth ist}, chapter 5. } On the basis of development in the longer term of Reinhold's conception of practical reason as presented in the earlier chapters of this study it will be possible to supplement Lazzari's account by showing the strong continuities among all those changes. 

The focus of the current chapter is not so much the development itself, a detailed account of which can be found in the work of Lazzari, but rather Reinhold's understanding of `practical reason' in \textit{Briefe II} as the final point of this development. The reasons for zooming in on \textit{Briefe II} and not, for instance, on Reinhold's later works on the foundation of morality are firstly that \textit{Briefe} \textit{II} represents the end of the development towards sharply distinguishing between practical reason and the will, which can be traced quite well as the essays included in the collection are adapted versions of earlier articles. Secondly, since \textit{Briefe II} is presented by Reinhold as part of essentially the same project as the original `Briefe' in \textit{Der Teutsche Merkur} marking Reinhold's first reception of the Kantian philosophy,\footnote{ That Reinhold thought of the publication of the volumes of \textit{Briefe }as one coherent project is clear from his remarks in the first volume of \textit{Beytr\"{a}ge}, which appeared when he had already adapted the \textit{Merkur}{-}`Briefe' and published them as a the first volume of \textit{Briefe \"{u}ber die Kantische Philosophie}. Cf. Reinhold, \textit{Beytr\"{a}ge I}, IV, where Reinhold speaks of the \textit{Briefe }as dealing with the ``effects, application and influence'' of the critical philosophy, whereas the \textit{Beytr\"{a}ge} deal with its ``grounds, elements and principles.''} focusing on \textit{Briefe II} paves the way for an evaluation of his thoughts on practical reason in relation to his original plans with the Kantian philosophy. The later works do not relate to the Kantian philosophy to the same extent.\footnote{ Although Beytr\"{a}ge \textit{II} thematizes practical philosophy, it does so from the context of \textit{Elementarphilosophie}, that is, with regard to the place of practical philosophy in the system of philosophy as a whole, even if Reinhold has had to abandon the thought of basing philosophy on a single principle. Within this context the results of \textit{Briefe II }regarding practical reason and the will are defended. Cf. Fabbianelli, introduction to \textit{Beytr\"{a}ge II}, LI. } Therefore, looking at \textit{Briefe} \textit{II} presents us with an excellent opportunity to round off our investigations on Reinhold's understanding of the Kantian term `practical reason'. Finally, the position on the freedom of the will that Reinhold defended in \textit{Briefe II} significantly influenced the debates concerning the Kantian perspective on freedom.\footnote{ For the reception of \textit{Briefe II}, see Bondeli, introduction to \textit{Briefe II}, section 3. For the reactions of Fichte and Schelling, see Pich\'{e}, `Fichtes Auseinandersetzung mit Reinhold im Jahre 1793. Die Trieblehre und das Problem der Freiheit'  and Stolzenberg, `Die Freiheit des Willens. Schellings Reinhold{-}Kritik in der \textit{Allgemeinen \"{U}bersicht der neuesten philosophischen Literatur}.'} That this position was no longer Kant's position became painfully clear when the latter, in his \textit{Metaphysics of Morals }(1797) distanced himself from Reinhold, by stating that according to him ``freedom of choice cannot be defined {--} as some have tried to define it {--} as the ability to make a choice for or against the law.''\footnote{ \textit{AA} 6:226; \textit{PP}, 380. } Although Kant does not mention him by name, Reinhold appears to be the target of this remark, since Kant continues by providing considerations regarding definitions and expositions and Reinhold had claimed that Kant had not defined the concept of the will properly. Although Kant's objections against Reinhold may not have been entirely fair, Reinhold was disappointed by this reception.\footnote{ Bondeli, introduction to \textit{Briefe II}, LII. } 

The first section of this chapter will concentrate on the development of Reinhold's practical philosophy leading up to \textit{Briefe} \textit{II}, with special attention for the relation between the will and practical reason. This section will include a brief recapitulation of the `Grundlinien' from the \textit{Versuch}, as well as an analysis of relevant passages from some of Reinhold's essays written for \textit{Der neue Teutsche Merkur}.\footnote{\label{footnote:_Ref232222730} Reinhold, `Ueber die Grundwahrheit der Moralit\"{a}t und ihr[ ] Verh\"{a}ltni\ss{} zur Grundwahrheit der Religion' \textit{NTM}, March, 1791; Reinhold, `Ehrenrettung des Naturrechts', \textit{NTM}, April, 1791; Reinhold, `Ehrenrettung des positiven Rechtes', \textit{NTM}, September, November, 1791; Reinhold, `Beytrag zur genaueren Bestimmung der Grundbegriffe der Moral und des Naturrechts,' \textit{NTM}, June, 1792.} The second section contains a presentation of \textit{Briefe II} itself, starting from its aims and structure and including a brief analysis of the contents of this collection. The third and final section of the final chapter will evaluate the development of Reinhold's practical philosophy from the \textit{Versuch} to \textit{Briefe} \textit{II}. This evaluation is intended to supplement Lazzari's analysis of the changes in Reinhold's practical philosophy by paying attention to a number of important continuities. 


\section{Practical reason and the will up to Briefe II}


In order to understand the development of Reinhold's practical philosophy after the publication of the \textit{Versuch}, this development must first be described. The structure of the \textit{Versuch} clearly shows that, although Reinhold promised beneficial practical consequences, this is a work of theoretical philosophy. We have seen in the previous chapter that the attention to practical philosophy in the form of the `Grundlinien der Theorie des Begehrungsverm\"{o}gens' can be understood as following from a polemical context involving Rehberg. It was obviously composed in haste and its relation to the theory of the faculty of representation or the theory of the faculty of cognition was not worked out sufficiently. The explicit promise of a full theory of the faculty of desire implies that Reinhold was intending to work more extensively on practical philosophy; yet the promise was not fulfilled. Instead, he produced several articles in which he related current issues in practical philosophy, especially the philosophy of right to the promise of the Kantian project to solve the problems of philosophy. These essays enable us to see how Reinhold's practical philosophy took shape in the years after the publication of his \textit{Versuch}. 

 This section first returns to the `Grundlinien', pointing out the issues that Reinhold had to address in developing his practical philosophy. Its general line of argument and the reasons for its occurrence have already been discussed in the previous capter (section 2.4.4. and 3.2). Here the focus will be on the problems Reinhold needed to address in his writings on practical philosophy, following the publication of the \textit{Versuch}. Secondly, it analyzes the relevant passages from articles dating from the spring and the autumn of 1791, showing his initial ways of dealing with these issues. These essays would later be adapted to be part of \textit{Briefe II}. Finally, we will consider a 1792 essay in which Reinhold explicitly starts to distinguish between the will and practical reason. In \textit{Briefe II} Reinhold would consolidate that approach. 


\subsection{`Grundlinien': implicit identification of practical reason and pure will}


Because of the role of the `Grundlinien' in supplementing the argumentation for section 86, its starting point is that freedom is associated with reason as an absolute cause.\footnote{ Cf. Chapter 5, section 2.4.4.} We have seen in Chapter 5 that Reinhold departs from his original scheme establishing the idea of the representing \textit{subject} as an absolute cause, and, in section 86, adds that the causality of \textit{reason} must be understood as absolute causation as well, consisting of three different kinds of activities. First, there is the activity of reason in thinking, that is, the production of ideas. Secondly, reason is active in determining the faculty of desire a posteriori. Thirdly, it can also determine this faculty a priori (cf. \textit{Versuch}, 558{-}559). To complicate matters, only the a priori determination of the faculty of desire is considered to be absolutely free, whereas the others are presented as comparatively free expressions of the causality of reason. Reinhold's introduction of the differentiation between comparative and absolute freedom can be understood as a pre{-}emptive reaction to the criticism he feared his work would receive from Rehberg.\footnote{ Cf. Chapter 5, section 3.2.} The introduction of the faculty of desire occasions the `Grundlinien'{-}chapter. 

 The core concept of the `Grundlinien' is `drive' (\textit{Trieb}), the ``relation of the representing power to the possibility of representation that is a priori determined in its capacity'' (561). The faculty of desire, or the capacity of being determined by drive to produce a representation, is thus instrumental in bringing about representations. In accordance with the basic characteristics of representations, namely that they consist of material and form, `drive' is likewise understood as being basically twofold, a drive for the material of representation and a drive for form. The drive for form is the need for the spontaneity of the representing subject to express itself in accordance with the categories. This drive, thus determined by the understanding, can be further modified by reason, which ``extends the selfish drive that is conditioned by sensibility and determined by the understanding to the unconditioned'' (564). Since reason's activity here is bound to the forms of sensibility, it only acts comparatively free (cf. 566).

 At this point in the `Grundlinien' Reinhold introduces the will, which differs from desire, as the will is a ``being determined by reason, an act of spontaneity'' (567). When the subject determines itself to an action considered a means of satisfying the drive for happiness, this determination is an action of the empirical will. The causality of reason, expressed as empirical will is only comparatively free, since it determines a given sensible drive by prescribing a rule for it ``that is only sanctioned by pleasure, through a drive that essentially differs from reason'' (568). Reason acts comparatively free in two different ways. First, it acts comparatively free in generating the drive for happiness, by extending the sensible drive to the unconditioned. Secondly, it acts comparatively free as empirical will, by being the self{-}determination of the representing subject to an action that is considered a means to this happiness. The circumstance that the latter case entails self{-}determination of the representing subject, rather than of reason can be understood by referring to Reinhold's theory of degrees of spontaneity according to which the spontaneity of the representing subject expresses itself to the highest degree in the acts of reason. Hence, the self{-}determination of the representing subject by its spontaneity is the same as a determination by reason. 

 Next to the comparatively free drive, there is the drive that is determined ``by nothing but the spontaneity of reason,'' the purely rational drive (569). This has as its object ``exercising spontaneity, the mere act of reason,'' which act is the ``realization of the way of acting of reason'' (569). The object of this purely rational drive is called morality (cf. 570). As with the sensible drive, Reinhold proceeds by linking the activity of reason with regard to the purely rational drive to the will. The determination to an action ``that has no other purpose than the actuality of the way of acting of reason'' is carried out by practical reason (571). `Pure will' is then described as the capacity of the representing subject\footnote{ The text reads ``Verm\"{o}gen des vorstellenden Objektes'' which makes no sense. Lazzari, who in his citations of the passage, tacitly corrects the error, interprets making a mistake like this as a definite sign that Reinhold had no time for a thorough correction of the final sheets. See Lazzari, \textit{Das Eine, was der Menschheit Noth ist}, 154{-}155. The new edition of the \textit{Versuch}, by Onnasch also corrects the error. } to ``determine itself to an action by means of the spontaneity of the purely rational drive'' (571). Reinhold's descriptions of practical reason as presented here amount to the claim that the realization of morality, that is, moral action is tantamount to acting for the sake of realizing the form of reason. Practical reason appears to be closely related to the pure will, yet the relation is not specified. The matter is only clarified slightly in the remainder of the paragraph, in which the will as a whole is described as ``self{-}determination to an action'' (571). In what Reinhold terms `empirical will' reason considers the action as a means to happiness, that is, it acts for the sake of sensibility. The pure will is now described as self{-}determination to an action that is ``determined by the object of the purely rational drive'' and that consists in ``the solely intended realization of the way of acting of reason'' (571). The pureness of the will is described as the independence from sensibility, since it ``realizes the form of reason that was determined according to its possibility only'' (571). Although Reinhold does not explicitly identify the pure will with practical reason, the following is suggested. Reason is considered practical insofar as it works as pure will, that is, insofar as it determines the representing subject to an action that solely aims for the realization of the form of reason, that is, a moral action. 

 The identification of practical reason and the will is confirmed by the next paragraph, concerning the freedom of the will. Earlier, in section 86, different kinds or degrees of freedom {--} comparative and absolute {--} were attributed to different expressions of the causality of reason. Reinhold's attention to the freedom of the will in the `Grundlinien' shows that he understood the will as causality of reason. He makes a distinction between the will as being free and as acting free. The will \textit{is }free because of its independence from coercion by the senses and because it is not bound to ``either the law of the unselfish drive or that of the selfish drive'' (571). The independence from the senses parallels the activity of reason in thinking (cf. section 86), employing material that itself is the product of spontaneity. The second of the determinations, that is, the freedom freedom from the unselfish drive, is new, but relates to Reinhold's thoughts on freedom as expressed in the First Book of the \textit{Versuch}. There he attributed the freedom needed for morality to the actor, who has a free choice ``either to determine his decision himself by means of reason, or to let it be determined by the objects of sensibility'' (90). This rules out being forced by laws of reason as well as being forced by sensible desire (cf. 91). Because of its relation to the First Book the understanding of `freedom of the will' as independence from both the selfish and the unselfish drive is more related to Reinhold's preconception of moral freedom than to the previous argument in the `Grundlinien'. It appears to be at odds with a strong relation between the will and practical reason, since it implies that the spontaneity of the will in its capacity to decide differs from the spontaneity of reason that only strives for the realization of its way of acting. The former spontaneity must include the freedom to determine oneself to be determined by the sensible drive, which appears incompatible with the latter kind of spontaneity, which solely strives for the realization of the form of reason. It is precisely this tension that Reinhold would try to resolve in the years to come, resulting in an explicit distinction between the will and practical reason. 

The determination of the freedom of the will as freedom from both the law of the selfish drive and the law of the unselfish drive also opens the way for the different kinds or degrees of freedom {--} comparative and absolute {--} associated with the different ways in which the will \textit{acts}. It can act comparatively free, when it ``subjects itself to the law of the selfish drive'' which is ``foreign'' to it (571{-}572). This on the one hand echoes the understanding of freedom as presented in the First Book, the freedom to let oneself be determined. On the other hand, it echoes section 86, where `comparative freedom' was described as ``when only a certain kind of foreign causes does not determine the action necessarily'' (558). Being free, the will may choose to subject itself to the foreign laws of sensibility, thus acting comparatively free. It acts absolutely free, however, ``when it follows the law of the unselfish drive'' (572). This law, being established by theoretical reason, ``is sanctioned as an actual law only by the mere spontaneity of practical reason'' (572). This distinction indicates a differentiation between the spontaneity of reason involved in drafting the moral law (ascribed to theoretical reason) and the spontaneity of reason involved in sanctioning this law (ascribed to practical reason) which ``itself imposes it [the moral law] upon itself'' (572). Apart from this distinction there appears to be another form of spontaneity still, involved in deciding which law to follow. 

 Summarizing, Reinhold uses the terms `freedom' and `spontaneity' in different senses in the `Grundlinien'. First, there is the comparatively free action of reason in extending the sensible drive to the unconditional. Secondly, we have the comparatively free action of the (empirical) will in choosing to be determined by the laws of sensible desire. Although this action is conceived as an action of the will, it is associated with the activity of reason as the highest degree of the representing subject's spontaneity. The will is also said to be free in a third sense, namely in the sense of being independent of both the laws of selfish and unselfish drive. This latter drive is understood as the result of the absolutely free activity of (practical) reason, striving to realize its own way of acting, independently of any empirical considerations. This is Reinhold's fourth understanding of `freedom'. Linked to this is, fifthly, the absolutely free way of acting of the will in determining itself to follow this law rather than empirical laws. Reinhold, at this point, has no clear position on exactly where to locate human freedom. There is a strong relation between the will and reason and especially between the pure will and practical reason, yet the kinds of freedom ascribed to each differ. There is a tension between, on the one hand, considering the faculty of desire as relating to the realization of representations and, on the other hand, the realization of the way of acting of reason outside the mind. There is further a strong intuition, already expressed in the First Book of the \textit{Versuch}, that freedom in morality requires the ability of the subject to decide either to determine himself or to let himself be determined. This appears to be the freedom of the will, yet it is unclear why this should be an ability of the \textit{representing} subject. Reinhold does not succeed in presenting a coherent and convincing picture of the relation between the freedom of the will and the faculty of representation. These characteristics of the `Grundlinien' can be sufficiently understood from Reinhold's desire to pre{-}emptively defend his work against Rehberg, by restating or at least trying to restate the Kantian position in terms of the theory of the faculty of representation.\footnote{ The importance of the faculty of representation in this context may be due to the influence of Platner, who links the will to the faculty of representation. Lazzari, \textit{Das Eine, was der Menschheit Noth ist}, 119 n. 3. }


\subsection{\textit{NTM}{-}essays from 1791: explicit identification of practical reason and the will}


By the end of 1791 Reinhold started to make a sharp distinction between practical reason and the will.\footnote{\label{footnote:_Ref232226383} According to Lazzari's detailed account of the origins of Reinhold's distinction between will and practical reason, Reinhold started to work on his `Beytrag zur genaueren Bestimmung der Grundbegriffe der Moral und des Naturrechts' in November 1791, which was to be published June 1792 and in which he for the first time explicitly distinguishes between the will and practical reason. Cf. Lazzari, \textit{Das Eine, was der Menschheit Noth ist}, 195{-}196. This essay will be discussed in section 1.3 of the current chapter. } In order to understand how he came to this, we must take a look at some of the articles he published during 1791, starting in March, when he published `Ueber die Grundwahrheit der Moralit\"{a}t und ihr[ ] Verh\"{a}ltni\ss{} zur Grundwahrheit der Religion'.\footnote{ See footnote \ref{footnote:_Ref232222730}. An adapted version of this essay would figure as `Zehnter Brief. Ueber die Unvertr\"{a}glichkeit zwischen den bisherigen philosophischen Ueberzeugungsgr\"{u}nden vom Daseyn Gottes und den richtigen Begriffen von der Freyheit und dem Gesetze des Willens' in \textit{Briefe} \textit{II}. References will be in the text.} As in the \textit{Versuch}, we find a listing of moral theories that is inspired by Kant's Table of Practical Material Determining Grounds in the second \textit{Critique}.\footnote{\label{footnote:_Ref232237843} \textit{AA} 5.40; cf. \textit{Versuch}, 102{-}114. } Reinhold himself describes morality as ``the free and unselfish willing of that which is lawful'' (229). This willing is described as depending on ``the spontaneity of the mind,'' with which we are familiar through the ``facts of consciousness'' (229). In contrast to the `Grundlinien', in which Reinhold tried to link the capacities of desire and will to the faculty of representation, here `personhood' is the central notion, announcing itself in self{-}consciousness as independence and the spontaneity of our actions. This spontaneity, so Reinhold explains, ``is limited to a willing, and even only to that willing that is determined by reason and that consists in acting through reason'' (230). Since `willing' is described in terms of `acting through reason', Reinhold does not appear to think of the will and reason as separate and independent capacities. The power of the will is described as ``steering'' our responses to the needs we have as living beings, rendering us rational animals (231). Reinhold distinguishes between desiring and willing on the basis of the involvement of reason. 

Das \textbf{Begehren} hei\ss{}t nur dann und in soferne ein \textbf{Wollen}, wenn und in wieferne es durch Vernunft bestimmt wird, wenn und in wieferne der Grund desselben in unsrer \textbf{Selbstth\"{a}tigkeit} liegt; wenn und in wieferne dasjenige, was mich zur \textbf{Person} macht (\ldots ) dabey wirksam ist. (239)

[\textit{Desiring} is only called \textit{willing} when and only in so far as it is determined by reason, when and insofar as its ground lies in our \textit{spontaneity}; when and in so far that which makes me a \textit{person} (\ldots ) is active in it.] 

Although willing is clearly associated here with reason, reason in turn is not associated with the representing subject, but rather with personhood. `Spontaneity' is no longer understood in terms of the degrees of spontaneity involved in different kinds of representations, but, in a literal translation, as `self{-}activity', which is understood in terms of ``that which makes us a person.'' The introduction of personhood and the kind of spontaneity it involves in order to distinguish `willing' from `desiring' has, of course, a distinct Kantian ring to it. For Kant personality is intimately connected to morality, since it singles human beings out as ends in themselves.\footnote{ Cf. \textit{Grundlegung zur Metaphysik der Sitten}, \textit{AA }4:428;\textit{ Critik der praktischen Vernunft}, \textit{AA }5:87.} In the second \textit{Critique}, personality is described as ``freedom and independence from the mechanism of the whole of nature.''\footnote{ \textit{AA }5:87; \textit{PP}, 210. } It is therefore no wonder that Reinhold would turn to `personality' in this context, when reason is no longer exclusively understood in terms of the theory of the faculty of representation. He continues by discussing the relation of will and reason in the following manner. 

Die Regel der Handlung wird daher bey jedem Wollen durch \textbf{Vernunft} gegeben; und der Entschlu\ss{} des Willens besteht in der Handlung nach einer sich selbst gegebenen Vorschrift. (239)

[Thus, the rule of action in any willing is given by \textit{reason}; and the decision of the will consists in the action according to a prescription that it has given to itself.] 

From the continuation of the passage it is clear that ``spontaneous reason'' both gives and follows the prescription, either ``for the sake of its own interest'' or ``for the sake of a foreign interest'' (239). The former giving and following of the prescription is called `moral', and the latter `amoral' ``but not, because of that, immoral'' (239). The only instance of immoral action is when reason follows a foreign law, the prescription of which contradicts the moral law (cf. 241). In line with Reinhold's statements on the foundation of morality in the First Book of the \textit{Versuch}, the freedom of the will is presented as the fundamental condition of morality, expressing itself morally or immorally (cf. 241{-}242). Practical reason, identified here with the activity of reason as will, must have its ground in freedom, which is associated in this case with the independence from sense{-}impression (cf. 242). The remainder of the article focuses on the relation of the conviction that the will is free and the grounds for the conviction that God exists.\footnote{ This issue appears to be related to Rehberg's review of Kant's second \textit{Critique}, since Reinhold explicitly states that the human will needs to be followed for its own sake which he considers to be impossible if it depends on a divine will the existence of which is established independently of morality. Rehberg had claimed that the will can only be comparatively free, because it must always depend on God, who, by creating both the noumenal and the phenomenal world, guarantees the correspondence between both. Cf. Chapter 5. } In March 1791 Reinhold explicitly identifies practical reason and the will, as he considers reason to be practical insofar as it is a free will choosing between the different laws available for prescribing action. The one law, the law of practical reason itself, is independent of empirical input, while the other law prescribes on the basis of empirical input, and is thus foreign to the subject's spontaneity. 

 In another article he produced for the \textit{Merkur}, in April 1791, entitled `Ehrenrettung des Naturrechts', Reinhold considers the fundamentals of morality in relation to the concept of natural right.\footnote{ See footnote \ref{footnote:_Ref232222730}. References will be in the text. The theme of natural right was not new to Reinhold: in the First Book of the \textit{Versuch} he had already called attention to the lack of universally accepted principles in that field. Cf. Chapter 5, section 2.2. } In it he first establishes the lack of a distinct concept of `natural right' and then briefly considers the previous systems of morality, again following the Kantian table.\footnote{ See footnote \ref{footnote:_Ref232237843}.} All attempts to give morality a foundation are presented as attempts to find the grounds of moral feeling. Reinhold believes that a proper explanation of these grounds will follow from the Kantian philosophy. 

Ich suche die \textbf{wirkende Ursache} alles \textbf{moralischen Gef\"{u}hls} (\ldots ) in derjenigen \textbf{Selbth\"{a}tigkeit} des menschlichen Geistes auf, durch welche sich unsre \textbf{Selbstst\"{a}ndigkeit}, die sich nur durch Unabh\"{a}ngigkeit im Handeln denken l\"{a}\ss{}t, in dem \textbf{Bewu\ss{}tseyn unsrer Personalit\"{a}t}, ank\"{u}ndiget. Diese Selbstth\"{a}tigkeit, die, in wie ferne sie beym Erkennen gesch\"{a}ftig ist, \textbf{theoretische}, und beym \textbf{Begehren} {--} \textbf{praktische} \textbf{Vernunft} hei\ss{}t, ist das \textbf{Verm\"{o}gen der Gesetze} \"{u}berhaupt, die als solche (\ldots ) nur in der \textbf{Handlungsweise} der blo\ss{}en Vernunft, und nicht au\ss{}er derselben in der Natur der Dinge an sich ihren Grund haben. Die praktische Vernunft handelt als \textbf{reiner} (\textbf{moralischer}) \textbf{Willen}, in wiefern sie die Gesetzm\"{a}\ss{}igkeit des Begehrens blo\ss{} um der Gesetzm\"{a}\ss{}igkeit willen realisirt; und folglich ihr eigenes Gesetz aus einer und eben derselben F\"{u}lle ihrer Kraft sich selbst giebt, und befolgt. (361{-}362)

[I look for the \textit{effective cause} of all \textit{moral feeling} (\ldots ) in that \textit{spontaneity} of the human mind through which our \textit{independence}, which can only be thought as independence in acting, announces itself in the \textit{consciousness of our personhood}. This spontaneity, which is called \textit{theoretical} reason in so far as it is active in cognizing and \textit{practical reason} in so far as it is active in \textit{desiring}, is the \textit{capacity for laws in general}, which as such have their ground only in the \textit{way of acting} of reason and not outside it, in the nature of things in themselves. Practical reason acts as \textit{pure} (\textit{moral}) \textit{willing}, insofar as it realizes the lawfulness of desiring merely for the sake of lawfulness, and thus from one and the same fullness of power gives its own law to itself and follows it.]

It is clear that Reinhold at this point identifies practical reason and the will to a certain extent and situates its activity in the realization of lawfulness for its own sake. Practical reason, acting as moral will, is presented as both the giver and the follower of laws, some of which are fully related to its own spontaneity, while others depend on something given. Moral feeling has its ground in this capacity since it is nothing more or less than the feeling of joy or pain at the consciousness of the agreement or disagreement of an action with the moral law. The remainder of the article reassesses the spurious grounds of morality introduced earlier. Since this is not relevant to the main argument of this chapter, it will not be discussed any further. 

 Reinhold's preference for the philosophy of right is further expressed in the autumn of 1791, with his `Ehrenrettung des positiven Rechtes' (September and November 1791).\footnote{ See footnote \ref{footnote:_Ref232222730}. } The first part of this essay, published in September, mainly concerns the disputes between philosophers and jurists regarding the status of positive jurisprudence. As usual, the dispute is due to the lack of a determined concept, in this case of `right', needed to distinguish properly between natural and positive right. It is only in the second part of the article, published in November, that Reinhold presents his thoughts on how the concept of `right' should be determined. He opens with the bold claim that the Kantian investigation of the capacities of the human mind has established the reality of the `unselfish drive', without being acknowledged by the majority of philosophers (cf. 278).\footnote{ Cf. Kant's explicit aim in the \textit{Critique of Practical Reason} as expressed in the Preface. ``It has merely to show \textit{that there is pure practical reason.}'' \textit{AA} 5:3; \textit{PP}, 139. } Therefore, Reinhold assumes this result as a hypothesis, not as an established fact in his efforts to establish the concept of `right' in order to distinguish properly between natural and positive right without losing sight of what these two have in common (cf. 279). He first distinguishes between `right' and `useful' and claims that human reason is employed in different ways when judging each of these. What is right is related to the unselfish drive, independently of ``needs and impressions,'' while the useful relates to the selfish drive, which ``as an instinct'' depends on those two things (280). Reinhold describes the different ways of acting of reason with respect to judgments of right and of utility in the following manner. 

Im ersten Falle [that of right] wirkt sie [reason] als die \textbf{f\"{u}r und durch sich selbst wollende}, folglich als handelnde (praktische) {--} im zweyten hingegen nur als die f\"{u}r und durch den eigenn\"{u}tzigen Trieb wollende, folglich an sich nur als \textbf{denkende} (theoretische) Vernunft. Beym blo\ss{}en \textbf{Rechte} ist ihre Vorschrift ein \textbf{Gesetz}, dem sich der eigenn\"{u}tzige Trieb unterwerfen soll; beym blossen \textbf{Nutzen} ist ihre Vorschrift eine \textbf{Regel}, die nur durch den eigenn\"{u}tzigen Trieb die Sanktion ihrer Nothwendigkeit erh\"{a}lt. (281)

[In the first case reason works as willing \textit{for and through itself}, that is, as acting (practical) reason; in the second case, however, only as willing for and through the selfish drive, that is, in itself only as \textit{thinking} (theoretical) reason. With regard to mere \textit{right} its prescription is a \textit{law} to which the selfish drive should subject itself; with regard to mere \textit{usefulness} its prescription is a \textit{rule}, the necessity of which is only sanctioned by the selfish drive.]

When deciding that something is right, reason only depends on itself and wills through its own demand, that is, the unselfish drive. We here find the same idea as in the `Grundlinien', namely that the prescription that only involves reason also sanctions itself, whereas the prescription that relates to something given needs external sanction, from the selfish drive. The distinction here between practical and theoretical reason is twofold. On the one hand Reinhold distinguishes them in terms of `acting' and `thinking' reason, respectively. On the other hand the identification of reason (in general) and the will is obvious from his distinction between practical reason as ``willing for and through itself'' and theoretical reason as ``willing for and through the selfish drive.'' Again, there appear to be four distinguishable activities of reason. First, (practical) reason establishes and sanctions the moral law. Secondly, it acts for and through itself in following this law. Thirdly, (theoretical) reason in thinking considers what is useful and finally, it acts for the selfish drive in following its demands. It is clear that reason is on the one hand involved in establishing the rules or law while on the other carrying them out. Similarly, reason is involved in carrying out the demands of both the moral law and the selfish drive. 

 Reinhold then turns to establishing `right' and `duty', deriving them from the relation between the selfish and the unselfish drive in order to be able to distinguish properly between morality and natural law. It is considered a given that ``the relation of the unselfish drive to the selfish drive can only consist in a limitation (\textit{Beschr\"{a}nkung}) of the latter in favor of the former'' (285). Using this as his starting point, Reinhold describes `duty' as ``a necessary limitation of the selfish drive, determined by the law of the unselfish drive'' (285). When the law of the unselfish drive necessitates the limitation of the selfish drive, this means that the satisfaction of the latter drive is morally impossible (even though it may very well be physically possible) (cf. 285). Parallel to this description of `duty' Reinhold describes `right' as ``the possibility of a limitation of the selfish drive'' on the basis of the law of the unselfish drive, entailing the non{-}impossibility of its opposition, that is, the moral possibility of satisfying the selfish drive (285). Because of the interplay of the selfish and the unselfish drive, right and duty are not possible when the selfish drive cannot be limited by the unselfish drive, for instance, when we act on instinct, in animals and in beings which lack a selfish drive (cf. 286).\footnote{\label{footnote:_Ref225931208} Lazzari interprets the passages as stating that there is already a limitation of the selfish drive because of the mere presence of the unselfish drive. Cf. Lazzari, \textit{Das Eine, was der Menschheit Noth ist}, 209. Although it is true that Reinhold claims that only a being with two drives, so that one can be limited by the other, can act from duty, he does not actually claim that the selfish drive needs to be \textit{limited} by the \textit{presence }of the unselfish drive, but rather that they both at least need to be present in order for the one to limit the other. Since Lazzari uses his understanding of the passage to claim that Reinhold equivocates on the term `Beschr\"{a}nkung' in this context, it is useful to point out that the latter claim is more related to Lazzari's use of the term than to what Reinhold actually says. Cf. Lazzari, \textit{Das Eine, was der Menschheit Noth ist}, 209{-}211. } Reinhold makes clear that the modalities involved in describing `right' and `duty' are moral, that is to say, `necessity' means an obligation, whereas `possibility' means a permission. Yet he also speaks of the possibility of a limitation of the selfish drive in another, more ontological sense, when he claims that in order for a limitation to be possible, both the selfish and the unselfish drive need to be present in a being. The first, moral, sense of the modality refers to the content of the law given by the unselfish drive with regard to the selfish drive. The second, metaphysical use of the term `possible' refers to the nature of a being which makes it ontologically possible or impossible that the unselfish drive effectuates such law with regard to the selfish drive. This sufficiently solves the equivocation that Lazzari attributes\footnote{ Cf. Lazzari, \textit{Das Eine, was der Menschheit Noth ist}, 209{-}211; cf. footnote \ref{footnote:_Ref225931208}. } to Reinhold's use of the term `Beschr\"{a}nkung', which according to the interpretation provided above does not arise from Reinhold's use of `Beschr\"{a}nkung' but rather from his use of different notions of possibility. There is no confusion of terminology on Reinhold's part and the lack of clarity in his terminology does not point to a problem for his theory here. Reinhold's definition of `duty' and `right' as modalities of the limitation of the selfish drive by the unselfish drive is complicated, for it must be noted that the limitation, when it actually takes place, is carried out differently in both cases. In the case of duty, the actual limitation is carried out by the unselfish drive or practical reason, which, in the case of right, only declares that a limitation is possible and thereby leaves the question of the actual limitation for theoretical reason to decide, which may or may not actually limit the selfish drive for prudential reasons.\footnote{ Cf. Lazzari, \textit{Das Eine, was der Menschheit Noth ist}, 212{-}213. } Such an action is called ``in conformity with right [\textit{rechtm\"{a}\ss{}ig}]'' in so far as it is carried out ``with regard to its possibility by the law, in so far as it merely does not contradict the moral law, it is called `permitted''' (288). Later on in the article, Reinhold speaks of `strict right' and `strict duty', where `strict right' is described as ``the morally possible, the impossibility of which (\ldots ) contradicts the law of that [unselfish] drive totally and always'' (292). This confirms that Reinhold's descriptions of duty and right are exclusively related to moral possibility and necessity. The question as to how a possible or necessary limitation is actually carried out appears to be of little interest to him in this context. 


\subsection{`Beytrag'{-}essay: distinction between practical reason and the will}


Almost immediately upon finishing the work on the second part of his `Ehrenrettung des positiven Rechts' discussed above, Reinhold started reworking its material.\footnote{ See footnote \ref{footnote:_Ref232226383}. } Apparently he was not fully satisfied with his distinction between natural and positive right on the basis of the concepts of `right' and `duty', which in turn were defined in terms of the relation between the selfish and the unselfish drive. We have seen above that the question regarding the activity of reason when it allows itself to be determined by the selfish drive was left unanswered. In his `Beytrag zur genaueren Bestimmung der Grundbegriffe der Moral und des Naturrechts', which was presented as an addition to a dialogue that he had published earlier,\footnote{\label{footnote:_Ref234654472} Originally, the piece was not intended as a part of that dialogue, but as a separate article for the \textit{Berlinische Monatsschrift}. Hence the only link with the previous dialogue in the \textit{Merkur} is the introduction `Frank an Horst,' referring to the characters of the earlier parts of the dialogue. Reinhold, `Die drey St\"{a}nde. Ein Dialog,' \textit{NTM}, March, 1792; Reinhold, `Die Weltb\"{u}rger. Zur Fortsetzung des Dialogs, die drey St\"{a}nde, im vorigen Monatsst\"{u}ck,' \textit{NTM}, April, 1792. Cf. Lazzari, \textit{Das Eine, was der Menschheit Noth ist}, 188{-}190, nn. 19 and 20. References to the `Beytrag'{-}essay will be in the text. } Reinhold addressed the themes of defining right and duty in terms of the relation between the selfish and the unselfish drive anew, by presenting an overview of the things he believes are evident on the basis of the awareness of these dispositions of the human mind. 

 First of all, he believes that it is evident that ``the human faculty of desire (\ldots ) consists of two original (\ldots ) drives,'' one of which is grounded in sensibility, having pleasure as its object (109). The other is ``grounded in personal spontaneity,'' having as its object the prescription that a person gives herself through reason (109). The unselfish drive is now situated in the person, rather than in reason, although reason is the means through which this spontaneity expresses itself. This shift from reason to the person as the seat of spontaneity shows that the faculty of desire is now fully emancipated from the faculty of representation. Reinhold no longer needs to link the faculty of desire to that of representation, since the spontaneity that is needed is no longer thought of as the spontaneity of reason, as in the theory of the degrees of spontaneity. It is the spontaneity of the person as a whole, not just the spontaneity of the representing subject. Reinhold had already begun this process of emancipation in his article on natural right discussed above. After identifying the first drive as the selfish and the second as the unselfish drive, Reinhold continues by developing a number of further concepts that are necessary in order to derive the proper concepts of right and duty. This is remarkable, since we have seen above that in the article on positive right Reinhold built his concepts of right and duty directly upon the relation between the unselfish and selfish drives. The first step is the introduction of the will, which is defined as ``the capacity of a person to determine herself to satisfaction or non{-}satisfaction of a demand of the selfish drive'' (110). Again we see the growing importance of the concept of personhood, replacing some of the many meanings of the spontaneity of reason. The importance of the person is confirmed by Reinhold himself, who, in his explanation, states 

Ich sage das Verm\"{o}gen der Person, und nicht der Vernunft. Diese ist das Verm\"{o}gen Vorschriften zu geben (Regeln hervorzubringen). Sie geh\"{o}rt zum Willen, inwieferne die Person nur durch eine Vorschrift, die sie sich zur Befriedigung oder Nichtbefriedigung des eigenn\"{u}tzigen Triebes giebt, sich selber bestimmen kann. Allein sie ist nicht der Wille selbst. (111)

[I say, capacity of a person, not of reason. The latter is the capacity to give prescriptions (produce rules). It belongs to the will, insofar as a person can only determine herself by means of a prescription regarding the satisfaction or non{-}satisfaction of the selfish drive. But it is not the will itself.]

The authority prescribing rules for itself is now the person instead of reason. This giving and following of prescriptions, however, is accomplished by two distinct capacities of the person, reason and the will. It is clear that at this point Reinhold no longer identifies the two. In a note appended to this explanation, he explicitly states that the will ``cannot be defined as causality of reason, for then it would be confused with the power to think'' (111). In the main text, Reinhold further points out that not all prescriptions of reason are acts of will, but ``only those through which the subject determines itself to the satisfaction or non{-}satisfaction of the selfish drive'' (111{-}112). He subsequently stresses that the selfish drive itself is necessary and involuntary, only the satisfaction or non{-}satisfaction of the demands of that drive is voluntary in so far as it depends on the will. Further, he introduces the term `pure willing' (\textit{reines Wollen}) for ``the mere demand of the unselfish drive, in so far as it becomes an incentive [\textit{Triebfeder}] for the satisfaction or non{-}satisfaction of a demand of the selfish drive'' (112). Conversely, the willing is empirical or impure, in so far as ``the prescription that a person gives herself, is given on the impulse [\textit{auf Antrieb}] of the selfish drive and for its sake only, and is therefore in no way the only incentive'' (113). 

 Having explicated the two different drives, Reinhold distinguishes between different meanings of the term `freedom'. He first defines `natural freedom' as the freedom to determine oneself with regard to the demands of desire ``either in accordance with or against the demand of the unselfish drive'' (113). This freedom is a capacity of a person and as such it is to be distinguished from the freedom of reason, which ``consists in the independence from external impressions and from the structure of sensibility'' (114). The freedom of the will, which appears to be equivalent to `natural freedom' and thus differs from the freedom of reason, can only be thought as a ``fact occurring in self{-}consciousness'' (114).\footnote{\label{footnote:_Ref232953324} In contrast, Kant's `fact of reason' {--} the announcement of reason as lawgiving {--} appears to be related to what Reinhold calls the `freedom of reason' rather than `natural freedom'. Cf. \textit{AA} 5:31. Reinhold's formulation here, referring to self{-}consciousness, appears to be related to his description in the \textit{Versuch}, according to which freedom is a fact of which one is conscious through one's \textit{Selbstgef\"{u}hl }(\textit{Versuch}, 92). As we shall see in the following (section 2.2.3 of the present chapter), Reinhold would later (in \textit{Briefe II}) describe human freedom as something we can only be aware of as a `fact of consciousness,' which cannot be explained any further. } This means that no further explanation of it can be given and that only its non{-}impossibility can be established. Because of the freedom of the will it is equally impossible to give reasons why, in any given case, the will has decided to go this way or that. It uses the unselfish drive and selfish drive as occasioning grounds (\textit{veranlassende Gr\"{u}nde}), yet is its own determining ground, so that it is not determined by either sensible impulses or the law of reason. 

  The difference between this account and the ones discussed in sections 1.1 and 1.2 is obvious. In the `Beytrag'{-}essay Reinhold clearly and explicitly distinguishes the will from reason, by means of the distinction introduced above between natural freedom and the freedom of reason. The spontaneity that expresses itself as the free will is different from the spontaneity of reason expressing itself in establishing the moral law. By explicitly presenting the will as a capacity that is separate from both sensible desire and the law of reason, its spontaneity can be properly distinguished from the activity of reason acting involuntarily. This solves the previous lack of clarity regarding the various activities of reason while at the same time completing the emancipation of the spontaneity of the will from the faculty of representation. This may seem far away from Reinhold's position in the \textit{Versuch}, where there was no clear distinction between the spontaneity of the will and the spontaneity of reason and where Reinhold sought to relate his thoughts on the faculty of desire to his theory of the faculty of representation. Although the account in the `Beytrag'{-}essay is indeed very different from the account in Reinhold's `Grundlinien', it must be remembered that it is rather close to the starting point of the \textit{Versuch}, as expressed in the First Book. We noted earlier that there Reinhold already emphasized the independence from both sensibility and reason as a necessary condition of the kind of freedom required by morality. He also claimed that the actuality of this freedom is accepted as a fact by common sense, while the philosophers seek to demonstrate its possibility. We have seen that the demand for the double independence of the will returned in the `Grundlinien'. Reinhold's development from the `Grundlinien' to the `Beytrag'{-}essay can thus be seen as a growing awareness that in order to realize the demand intuitively made with regard to morally relevant freedom, the spontaneity grounding this freedom must be different from the spontaneity of reason that is related to the faculty of representation. 


\section{Briefe II}


Now that we have established the main line in the development of Reinhold's practical philosophy it is time to turn to the second volume of the \textit{Briefe \"{u}ber die Kantische Philosophie}. In 1790 Reinhold had published the first volume, about half a year after he had finished the \textit{Versuch}.\footnote{ In the Preface to \textit{Briefe I} Reinhold announced that the next volume would deal with ``morality, freedom and instinct'' and the results of the critical philosophy regarding these subjects. \textit{Briefe I}, x.} He must have been eager to deliver the expanded version of the \textit{Merkur}{-}`Briefe' as a collection in order to compete with the two pirate editions of the series that had appeared earlier.\footnote{ \textit{Briefe \"{u}ber die Kantische Philosophie von Hn. Karl Leonhard Reinhold Rath, und Professor der Philosophie zu Jena. Zum Gebrauch und Nuzen f\"{u}r Freunde der Kantischen Philosophie gesammelt} (Mannheim: Bender 1789);\textit{ Auswahl der besten Aufs\"{a}zze \"{u}ber die Kantische Philosophie} (Frankfurt and Leipzig; de facto Marburg: Krieger 1790).} Although this first volume appeared after the \textit{Versuch }and indeed contains new material compared to the \textit{Merkur}{-}`Briefe',\footnote{ Reinhold was working on his \textit{Briefe I }more or less at the same time as writing the \textit{Versuch} and revised the text several times. In a letter to G\"{o}schen, he states the following: ``Meine Briefe \"{u}ber die Kantische Philosophie sind rein abgeschrieben und warten auf Durchsicht und zum Theil Umarbeitung.'' Cf. \textit{RK} 2:63, Letter 145, February 26, 1789, to G\"{o}schen. At the time when Reinhold was working on the First Book of the \textit{Versuch}, he would have had an early version of what would become \textit{Briefe I} available.} it will not be taken into consideration in the present chapter. The reason for this is that Reinhold did not use the occasion of rewriting and expanding his `Briefe' into the first volume of \textit{Briefe }to explore the issues of practical philosophy from the point at which the `Grundlinien' had left them. \textit{Briefe I} does discuss the fundamental truth of morality (that is, human freedom), but only in its more abstract form as `the fundamental truth of morality', not as the freedom of the will. Nor does Reinhold discuss the distinction introduced in the `Grundlinien' between the selfish and unselfish drives. This means that the first volume of \textit{Briefe }is not particularly relevant for our purpose in this chapter, that is, tracing the development of Reinhold's practical philosophy from the \textit{Versuch} to \textit{Briefe II}. In the following I will first analyze the aims and structure of the second volume of \textit{Briefe} and then briefly present the contents of this work in order to have a basis for our evaluation of Reinhold's understanding of `practical reason' at that time. 


\subsection{Structure and aims of \textit{Briefe II}}


In the Preface Reinhold presents the second volume of \textit{Briefe} as an attempt to assist a friend in studying Kant's second \textit{Critique}. He clearly presents its contents as a unified collection, although many `Briefe' contained in it are actually adaptations from separately published articles in \textit{Der neue Teutsche Merkur}.\footnote{ Some of these have only been changed stylistically, others have undergone strong revision. Cf. Bondeli, introduction to \textit{Briefe II}, XIII{-}XIV. This new edition notes all relevant changes between the original articles and the texts as they are found in \textit{Briefe II}.} Based on the table of contents and Reinhold's presentation of the individual articles in the Preface we can say that \textit{Briefe II} can be divided into five main parts. The first, consisting of the first article, introduces the subject of the work in a rather wide sense. It presents a conflict in current philosophy regarding the status of metaphysics vis{-}\`{a}{-}vis applied philosophy. This conflict is related to debates regarding the usefulness of the Kantian philosophy. The second main part, consisting of the second, third, fourth and fifth articles, introduces more specific problems, namely the lack of accepted principles in the fields of natural and positive right. It further prepares for the solution of these problems by means of the Kantian philosophy. That is, Kantian philosophy properly understood, for Reinhold is keenly aware of the fact that this philosophy was by no means fully accepted in the philosophical world and tries to remedy the misunderstandings by making Kant's premises explicit. This explication follows in the sixth article and is further developed in the seventh and eighth, which together form the third main part of \textit{Briefe II}. It aims at a proper determination of the concept of the will. Having established this concept and developed its characteristics, Reinhold fourthly turns to the consequences of the newly determined concept of the will in the ninth, tenth and eleventh articles. Finally, the twelfth article concludes the collection by presenting a dialogue on the likelihood of the reform in philosophy actually coming about.

 From the above account of the structure of the second volume of \textit{Briefe} it is clear that there are certain similarities to the first volume. The first volume, however, responded to `external grounds' by referring the reader to Kant's first \textit{Critique}, which supposedly solved all the problems introduced. This strategy is no longer available to Reinhold in the second volume of \textit{Briefe}. As the title of the first article, `Ueber einige Vorurtheile gegen die Kantische Philosophie',\footnote{\label{footnote:_Ref233181041} This first `Brief' is a revised version of an earlier article in \textit{Der neue Teutsche Merkur}, January, 1791, 81{-}112 entitled `Ehrenrettung der neuesten Philosophie.' } indicates, the Kantian philosophy itself, now the subject of debate has become part of the problem in a way that precludes a simple `Read Kant!' as an advice to the reader. 

Notwithstanding this changed status of the Kantian philosophy, the structure of the first article in \textit{Briefe II} is remarkably similar to that of the very first two `Briefe' in the \textit{Merkur} of August 1786. Reinhold starts by recounting the pessimistic outlook of his correspondent on the current philosophical landscape in order to reinterpret the phenomena in more positive terms. His starting point is a complaint made in \textit{Neues Deutsches Museum} that Germany suffers from ``metaphysical influenza.''\footnote{\textit{ Briefe II}, 6. Further references to \textit{Briefe II} will be cited in the text. Reinhold refers to the following article: Marcard, `Ist die Deutsche Nation die erste Nation des Erdbodens?' \textit{Neues Deutsches Museum}, October, 1790, 1030. Reinhold cites this source both in the \textit{NTM}{-}article and in the second volume of \textit{Briefe}, but uses the term in a way that is independent of the article by Marcard. } The metaphor of an illness is of great use to Reinhold, who reinterprets the symptoms of the disease as the crisis announcing the recovery from a previous ailment (cf.\textit{ Briefe II}, 7). The symptoms consist of the retreat of philosophy from the domain of experience (cf. 8). The consequences of metaphysical speculation are deemed harmful, and the anarchy in France is presented as resulting from claims in the name of such abstract concepts as `humanity' and `cosmopolitanism' (cf. 12). 

 Reinhold's reply starts by pointing out that, if there is indeed a `metaphysical influenza' it is not of epidemic proportions since the number of Kantianizing writings is still relatively small (cf. 15). As for the remark that `experience' is neglected in the current philosophical debates, Reinhold claims that the parties involved misunderstand Kant's views on experience, because Kant has first given the proper determination of the boundaries of experience (cf. 18). Those who are used to working with their own indeterminate concepts of experience naturally misunderstand Kant's efforts. This way of reinterpreting the controversy to the advantage of the new, critical philosophy is reminiscent of the \textit{Bisherige Schicksale}, functioning as the Preface to the \textit{Versuch}. The controversy helps to uncover the weaknesses of the old systems, which will in the end collapse and give way to the only possible system to be built upon determinate fundamental concepts (cf. 21). The only cure for the `metaphysical influenza', that is, the ongoing debates concerning the principles of philosophy, is, therefore, the establishment of pure and scientific principles (cf. 37). 

 Although Reinhold abstains from explicitly claiming that either the Kantian philosophy or his own \textit{Elementarphilosophie} has firmly established these principles it is certainly implied that the Kantianizing philosophy deemed responsible for the metaphysical influenza, in fact contains the only possible cure. In its presentation of a current chaos in philosophy, which has to be solved by a proper understanding of its presuppositions and principles, this first article is related to the introductory articles of the first series of `Briefe'.\footnote{ `Erster Brief' and `Zweyter Brief' in \textit{Der Teutsche Merkur}, August 1786; first four `Briefe' of \textit{Briefe I}. } In dealing with the misunderstanding of the Kantian philosophy and attacking the \textit{Popularphilosophen}, it is also akin to the Preface and the First Book of the \textit{Versuch}. The most striking difference with all of these previous works is that Reinhold no longer claims that it is the Kantian philosophy that, although almost universally misunderstood, contains the solution. In the Preface to \textit{Briefe II }the Kantian philosophy appears as the necessary condition for the solution rather than the solution itself. When introducing his plan to present the inner premises of morality by means of proper characteristics of the will, Reinhold remarks the following with regard to the work done by Kant. 

Zu Folge des analytischen Ganges an welchen die philosophierende Vernunft bey der fortschreitenden Entwickelung der Grundverm\"{o}gen des Gem\"{u}thes gebunden ist, konnten jene Merkmale nur erst \textbf{nach} dem vorl\"{a}ufig bestimmten Begriffe von dem eigenth\"{u}mlichen \textbf{Gesetze} des Willens, welcher durch \textbf{Kant} zuerst aufgestellt worden ist, gefunden werden. Sie sind in der \textbf{Kritik der praktischen Vernunft} sowohl als in der \textbf{Grundlegung zur Metaphysik der Sitten} zwar nicht unrichtig, aber v\"{o}llig unentwickelt vorausgesetzt, und die Aufstellung ihrer \textbf{bestimmten} Begriffe ist durch diese Werke zwar erst m\"{o}glich, aber eben so wenig leicht als entbehrlich gemacht worden. (vii)

[As a result of the analytical path to which philosophizing reason is bound by the proceeding development of the fundamental capacities of the mind, those characteristics could only be found \textit{after} the provisionally determined concept of the proper \textit{law} of the will, which has first been established by Kant. These characteristics have been presupposed in the \textit{Kritik der praktischen Vernunft} and in the \textit{Grundlegung zur Metaphysik der Sitten}, not incorrectly, but totally undeveloped. Through these works the establishment of their \textit{determinate} concepts has first become possible, but it has not become easy or dispensable.]

The characteristics of the will can only be determined after a provisional concept of the law of the will has been established. Reinhold secures his own place in the history of philosophy, by claiming that the course of development of the human mind makes it necessary that Kant's work is a \textit{conditio sine qua non} but nevertheless only a condition. A similar way of arguing is found in the Preface to the \textit{Versuch}, where Reinhold also referred to the ``analytical way of proceeding'' of the human mind as responsible for the circumstance that ``proper premises of a science are only found after that science itself'' (\textit{Versuch}, 67). The concept of `representation' was presupposed rather than determined when Kant set out to determine the concept of `cognition'. According to Reinhold this inevitable circumstance explains why Kant's first \textit{Critique }had been widely misunderstood. The argument in the Preface of \textit{Briefe} \textit{II} amounts to the same, although Reinhold appears to be more confident in proclaiming that Kant only provided the preparation for the final stage of philosophy, which consists in making Kant's premises explicit. 

 It is clear that Reinhold's aim in \textit{Briefe II} is closely related to the aim of the \textit{Versuch}. He intends to bring philosophy a step closer to its final, scientific form by making the premises of the Kantian philosophy explicit. In the \textit{Versuch} this was done with regard to the first \textit{Critique}, or the theoretical philosophy of Kant. Three years later, in the second volume of the \textit{Briefe}, he is ready to do the same for Kant's practical philosophy or the second \textit{Critique}. Both the \textit{Versuch} and \textit{Briefe II} are related to the project of the first set of `Briefe' in the \textit{Merkur} in that all of these projects aim at solving current philosophical conflicts by finding a common starting point. The main difference between the \textit{Merkur}{-}`Briefe' on the one hand and the \textit{Versuch }and \textit{Briefe II} on the other is, as we have seen, that the first project presents the Kantian philosophy as the solution to current debates in philosophy, whereas the latter two start from the debates surrounding the critical philosophy itself. The second volume of \textit{Briefe} in turn differs from the earlier two projects in that it explicitly relates to the premises of Kant's practical philosophy. 


\subsection{Contents of \textit{Briefe II}}


The current section analyses the contents of \textit{Briefe II} in accordance with the division in parts presented earlier. Since the first part, consisting only of the first `Brief' has already been discussed in the preceding section, section 2.2.1 discusses Reinhold's introduction of the problem and his preparation for the solution; section 2.2.2 deals with that solution itself; section 2.2.3 presents the consequences as Reinhold sketches them; and section 2.2.4 considers the twelfth, concluding article of the collection.


\subsubsection{The problem and the preparation of the solution: second to fifth `Briefe'}


These four preparatory articles sketching the problems of the current debates on the sources of duties and rights and the relation of natural and positive right had been published in 1791 as two articles in \textit{Der neue Teutsche Merkur}. The second and third `Briefe' are an adaptation of `Ehrenrettung des Naturrechts', briefly discussed in section 1.2 of the present chapter. The fourth and fifth pieces in \textit{Briefe II} are based on the essay `Ehrenrettung des positiven Rechts', which was discussed in section 1.2 as well. In the following I will first pay attention to Reinhold's two sketches of the problem, that is, to the second and fourth `Briefe', and then turn to his preparations of the solution in the `Briefe' three and five.

In the `Zweyter Brief', entitled `Von der bisherigen Uneinigkeit der philosophierenden Vernunft mit sich selbst \"{u}ber die Quelle der Pflicht und des Rechts', Reinhold takes the current debate on the situation in France as the starting point for his discussion.\footnote{ This in part reflects the political outlook of \textit{Der neue Teutsche Merkur}, in which Wieland actively commented on current developments in France. } According to Reinhold the discussions regarding the principles of the new political system in France clearly show the lack of determination and clarity concerning the main terms, such as `natural right' (cf. 38). The \textit{Selbstdenker} are divided between themselves regarding the desirability of basing a constitution on natural right, while failing to investigate the concept of `natural right' itself, because they take its meaning for granted (cf. 39{-}40). Reinhold writes that the ground of our natural rights and duties is at the same time ``obvious and mysterious'' (43). On the one hand ``the existence of these rights and duties announces itself through feeling'' (43). The effective cause of these feelings, on the other hand, has only become available to philosophizing reason since the progress made by Kant. The reality of our feelings shows that natural right is a real subject matter, which might be treated scientifically although we lack as yet the proper concept and hence the proper scientific form. According to Reinhold the disagreement on the status of natural right is due to a confusion of the reality of the subject matter with that of the scientific form. Thus the defenders, focusing on the existence of the thing, claim that the concept has been sufficiently established, whereas their adversaries, focusing on the lack of a determinate concept cast doubt on the existence of the thing (cf. 44{-}45). 

 The remainder of the second `Brief' builds on the historical approach applied in the first series of `Briefe', describing how the disagreement on the status of natural right necessarily arose from the combination of a clear feeling with the lack of a determinate concept. First, Reinhold argues that it is necessary that the awareness of right and duty is first installed in mankind through feeling (cf. 47). All philosophical attempts to establish a determinate concept of natural right or morality must be understood as attempts to identify the cause of this feeling, whereas the feeling itself is independent of any speculation (cf. 48). Reinhold links the question regarding the cause of moral feeling to the question of the ground of moral obligation (\textit{Verbindlichkeit}) and concludes, as in the \textit{Versuch}, that up till now philosophical theories have not been able to provide a satisfactory explanation (cf. 49{-}57). The true explanation of moral feeling must combine the true parts of all the other explanations, while dispensing with the falsities contained in them (cf. 62). This higher standpoint must therefore on the one hand establish the ``reality of natural right,'' that is, establish the unselfishness, necessity and universality of our sense of justice. On the other hand it must explain why natural right had not yet acquired a scientific form (63). The formulation of the subsequent claim that this new explanation ``follows from the principles of the Kantian philosophy'' expresses both Reinhold's loyalty to Kant and his conviction that the Kantian philosophy is not the final stage of philosophy yet (63). 

 Reinhold's analysis of the problem is continued in the fourth article in the collection, entitled `Ueber die bisherige Mi\ss{}helligkeit zwischen der moralischen und politischen Gesetzgebung, und zwischen der nat\"{u}rlichen und der positiven Rechtswissenschaft'. Here he investigates the animosity between the philosophers on the one hand and the ``practitioners of the positive sciences,'' especially jurists on the other, concerning the status of positive right (98). Philosophers claim that jurists overestimate its importance, while the latter blame philosophers for underestimating it (cf. 100). In the background, the conflict concerns the status of natural right in relation to positive right, since the latter is the domain of jurists, while philosophers are more concerned with the former. Both parties claim that the cause of their disagreement is the lack of a determined concept of positive right on the part of their adversaries (cf. 111{-}112).\footnote{ The motive of two parties reproaching one another for misunderstanding the central concept of their controversy already figures in Reinhold's \textit{Merkur}{-}`Briefe.' Cf. `Erster Brief,' 122. } Again, the question cannot be solved without a determinate concept of positive right (cf. 113). According to Reinhold, the jurists pay too much attention to empirical studies of actual legal documents and only consider the historical sides of a claim to justice (118). Further, they think that the concept of right has been sufficiently established and that they need not address the problem because they confuse the feeling of right with its concept (cf. 122). Essentially, this is the same problem as the one identified in the second `Brief', only with regard to positive right rather than natural right. The problem is the lack of a properly determined concept of the subject matter of these disciplines, even if the subject matter is related to clear feelings. It is up to philosophy to provide positive jurisprudence with its `propaedeutic', a preparatory science in which the main concepts are determined and the domain and its relation to natural right are established and presented (127). However, the philosophers have not yet succeeded in providing this proper concept of `right' (cf. 129). 

Summarizing, the second and fourth `Briefe' clearly sketch similar problems in addressing the conflicts that result from the lack of determinate concepts in the field of theories of right. Both theories of natural right and positive jurisprudence lack a determinate concept of their subject matter, `natural right' and `positive right', respectively. In both cases Reinhold refers to a clear feeling of right that serves to prove the reality of right, which has become confused with the as yet unclear concept of right that must be the foundation of any scientific treatment of the subject. Only with Kant it has become possible to determine this concept, for which the third and fifth `Briefe' pave the way. 

In the third `Brief', entitled `Von dem k\"{u}nftigen Einverst\"{a}ndnisse der philosophierenden Vernunft mit sich selbst \"{u}ber die Quelle der Pflicht und des Rechts', Reinhold puts forward a series of claims as hypotheses, since he is not aiming at this point to show how they are related to their Kantian grounds (cf. 64). The strategy of claiming that the Kantian philosophy has produced such and such results, without giving the grounds for these results is of course similar to the strategy of the first series of `Briefe', only now the results relate to Kant's second \textit{Critique}. First, he claims that the second \textit{Critique }has shown that the source of morality, ``and therefore also the efficient cause of moral feeling'' can in no way be found in the ``receptivity for pleasure and pain'' (64).\footnote{ Cf. \textit{AA} 5:58.} From the explication, Reinhold here appears to understand this receptivity so broad as to include all possible material determining grounds of the will according to Kant's table.\footnote{ See footnote \ref{footnote:_Ref232237843}. } Secondly, Kant is credited with having shown that the moral law is ``a prescription that contains the ground of its necessity in itself,'' that is to say that it is a law that does not need any foreign sanction; it can therefore ``only be followed for its own sake'' (65). Finally, Kant is reported to have shown that the ``source of this law is only to be found in the spontaneous nature of reason'' (65). In its capacity of giving this law reason is called `practical reason'. With the basic claims in place, Reinhold is quick to admit that this novel concept `practical reason' must seem very unclear to anyone who has not yet studied the \textit{Critique of Practical Reason}.\footnote{ Note that this also applies to his own understanding of the term at the time of writing the first series of `Briefe'. Despite the fact that the second \textit{Critique} was not even written at that time, Reinhold's \textit{Merkur}{-}`Briefe' claim that `practical reason' is of crucial importance, as we have seen in Chapter 4. Reinhold's remark here could be interpreted as an admission that he now, with the benefit of hindsight, can see that at the time he was not in a position at all to understand what Kant meant by `practical reason.' } Instead of providing an exegesis of Kant's understanding of the term, he presents his own thoughts on the subject. 

 He starts by describing `reason' as the capacity of a person to give herself prescriptions (\textit{Vorschriften}) for any of the effects that are possible through her other capacities. When reason, in order to make a prescription, needs an external ground, it is called theoretical reason. When no external ground is needed and the ground for the prescription is to be found in the self{-}activity of reason, reason acts as practical reason (cf. 66). The laws that are established by theoretical reason are called `laws of nature', whereas the law made by practical reason, as a law that is grounded in the self{-}activity of reason is a `law of freedom' (cf. 68). Since this law of practical reason is solely based on reason's capacity to prescribe a rule to itself, its only prescription consists of a rule that is valid through itself, ``that needs no sanction, because it contains it in itself.'' This prescribing a rule for its own sake is called the ``autonomy of reason'' (68). Given this autonomous nature of practical reason, it can only prescribe for those actions that depend on the person as a person, and not for behavior that follows from instinct. That is to say, the practical law is a law of willing. Not the regulation of desire is subject to the practical law, but rather the self{-}determination of a person either to satisfy the demand made by desire or to refuse it. This self{-}determination is what Reinhold calls `willing' (\textit{Wollen}) (cf. 69). 

 This willing must, according to Reinhold, be carefully distinguished from the activity of practical reason itself. The action of practical reason is to establish (\textit{aufstellen}) the moral law, whereas the action of the free will is self{-}determination to satisfy a given desire or not.\footnote{ In the original article these two actions were identified. Cf. section 1.2 of the present chapter. } When this action is done ``for the sake of the practical law,'' it is called `pure willing', which hence also differs from practical reason (70). Next the definitions of duty, right and injustice are related to the law. Duty is that which is necessary through the law, right is what is possible through the law, while injustice is that which is impossible through the law (cf. 70). Moral feeling is then described as the (dis)pleasure following from the (dis)agreement of an act of willing with the law (cf. 71). Hence, we can now identify practical reason as ``the effective cause of moral feeling'' (71). 

Next Reinhold sets out to criticize previous attempts to identify the cause of moral feeling from his newly acquired insights. Since duty and right at first manifest themselves as feelings rather than as concepts it is very understandable that some have sought the effective cause of these feelings in a special moral sense. According to Reinhold, moral sense theorists confuse the ``effect of the moral incentive'' with the incentive itself (82). Likewise those who, like Rousseau, sought the source of right and duty in benevolence are criticized, because \textit{moral} benevolence can only follow from the moral law and does not ground its obligatory character (cf. 85). Those who believe that the determining ground of right and duty can be found in the drive for happiness are also corrected. In a remark that may be directed at Rehberg, Reinhold states

Als Vergn\"{u}gen geh\"{o}rt das moralische Gef\"{u}hl unter die \textbf{Bestandtheile} der \textbf{Gl\"{u}ckseligkeit}, und als Objekte dieses Vergn\"{u}gens geh\"{o}ren Pflicht und Recht unter die Objekte des Triebes nach \textbf{Gl\"{u}ckseligkeit}. Allein, daraus, da\ss{} die Sittlichkeit auch eine der unmittelbaren Befriedigungen dieses Triebes ist, folgt doch keineswegs, da\ss{} sie \textbf{nichts anderes} sey. (86)

[As a pleasure, moral feeling belongs to the \textit{elements} of \textit{happiness} and as objects of this pleasure, duty and right are among the objects of \textit{happiness}. However, just because morality is one of the direct fulfillments of this drive, it does not follow that it is \textit{nothing else}.]

Rehberg had attacked Kant's conception of `respect for the moral law' on the basis that it could not provide a bridge between the pure law and the empirical action, since according to him this respect itself was a feeling of pleasure, hence could only be empirical.\footnote{ In his review of Kant's second \textit{Critique}. See Chapter 5, section 3.2. } According to Reinhold, the relation between morality and happiness is to be understood in the following manner. Although there is a strong connection between morality and happiness, the role of morality is not restricted to being involved in achieving happiness. Identifying morality and happiness means bringing in theoretical reason, since happiness cannot be achieved without a certain degree of prudence (cf. 89). Reinhold again appears to target Rehberg in stating that philosophers who have made that mistake have also confused the application of the moral law with the law itself (cf. 90). Rehberg had required that the application of pure practical reason be demonstrated.\footnote{ See Chapter 5, section 3.2. } Finally, it is clear why perfection cannot be the source of moral obligation, even if the moral law is a perfection. Since not all perfection is moral perfection, however, perfection cannot serve as an explanation for the obligatory nature of moral perfection (cf. 90{-}91). 

 In the third `Brief' Reinhold indicated how Kantian practical philosophy may be able to solve the problems regarding the status of natural right; in the fifth he tries to accomplish the same with regard to positive jurisprudence. It is fittingly titled `Ueber die k\"{u}nftige Einhelligkeit zwischen der moralischen und politischen Gesetzgebung und zwischen der nat\"{u}rlichen und positiven Rechtswissenschaft'. As in the third `Brief', Reinhold begins by putting forward a `Kantian result' as hypothesis. In this case he posits the unselfish drive in order to establish the proper relation between natural and positive jurisprudence (cf.137).

 Earlier, by limiting natural right to ``duties of force'' (\textit{Zwangspflichten}) and morality to ``duties of conscience'' (\textit{Gewissenspflichten}), philosophers had overlooked the source of ``duty in general'' (142). Reinhold seeks support from the common understanding of the term `natural right'. The term `right' is commonly understood to refer to something that is ``morally possible'' rather than ``physically possible'' (146). This basic understanding is founded on moral feeling and provides the connection between the concepts of morality, natural right and positive jurisprudence. A proper understanding of the connection of the different concepts relating to `right' first depends on a proper concept (not feeling) of `right' and of morality. Morality is situated in a ``relation of an act of will to the law of practical reason,'' which enables one to follow or break the law (146{-}147). This ability is called ``natural freedom'' and is distinguished from the ``moral capacity'' to do something that is in agreement with the moral law (147).\footnote{ The distinction made here between moral and natural freedom appears to be related to the distinction made in the `Grundlinien' between the will as being free and as acting free, which was hinted at, rather than drawn in an explicit way. Cf. section 1.1 above. } Both in natural and positive right and in morality, the concept `right' refers not to the physical capacity of freedom, but rather to ``the moral freedom of the will,'' that is, the possibility of following the moral law (148). Secondly, it is very important to have a proper, determinate ``concept of the distinction between the objects of these three sciences,'' that is, of natural right, positive right and morality (148). Again Reinhold relies on the common usage. It is relevant to note that in common usage `natural right' does not have `natural duty' as its counterpart, whereas morality and positive jurisprudence acknowledge both rights and duties. Morality is accordingly defined as the ``science of moral legislation,'' while positive jurisprudence is understood as the ``science of positive legislation'' (149). Natural right on the other hand is ``a science of mere rights'' (150). This is elaborated in the remainder of the fifth `Brief', but it need not be discussed in any detail here, since it has no bearing on our argument. 


\subsubsection{The solution: sixth to eighth `Briefe}


After the preparations presented in the third and fifth article, Reinhold in his sixth `Brief' aims for a `new presentation of the fundamental concepts and principles of morality and natural right'.\footnote{ The article is an adaptation of `Beytrag zur genaueren Bestimmung der Grundbegriffe der Moral und des Naturrechts. Als Beylage zu dem Dialog der Weltb\"{u}rger,' discussed in section 1.3. Cf. footnote \ref{footnote:_Ref232226383}. } The title is reminiscent of the rigor of the \textit{Versuch} and indeed we find numbered propositions with explanations. Like the \textit{Versuch}, this article aims at convincing the reader/correspondent that Kant's innovations make a thorough determination of fundamental concepts and principles possible, in this case of morality (cf. 175). While the pre{-}Kantian era is described as progress towards science, with conflicting philosophical currents that did not succeed in establishing universally acceptable principles, there will be an endless progress within the scientific philosophy established by the \textit{Elementarphilosophie}. The period of transition from the one kind of progress to the other is the period of ``Kantian or critical philosophy'' (178). The foundations of the new scientific philosophy are not complete, yet the previous philosophies have been discredited. Many defenders of Kant think that his system will be the new philosophy. They get the same reproach Reinhold had once made the Wolffian metaphysicians, namely that they mistake the scaffolding for a finished building (179).\footnote{ Reinhold had already used a similar metaphor in his `Gedanken \"{u}ber Aufkl\"{a}rung' and his \textit{Merkur}{-}`Briefe.' Cf. `Gedanken \"{u}ber Aufkl\"{a}rung,' 4{-}5; \textit{Letters}, 43; Dritter Brief, 29. .} As in the case of the metaphysicians, ``mediating concepts'' are needed for a complete building, in this case, concepts that mediate between the \textit{Critique of Practical Reason} and a future ``system of pure morality and natural right'' (179).\footnote{ Kant was only to put forward a system of morality in his \textit{Metaphysik der Sitten} (1798).} Reinhold puts forward as evident his own fundamental concepts of morality and natural right as an attempt to establish those mediating concepts. 

 First of all, Reinhold thinks it is evident that the human faculty of desire consists of two original drives, which are both ``essentially different and essentially unified.'' One is the drive for pleasure, while the other is practical reason, which is called a drive because its activity is involuntary (181). The drive for pleasure (\textit{Vergn\"{u}gen}), activated by pleasure (\textit{Lust}) and pain (\textit{Unlust}) is described as a selfish drive, while the drive that is practical reason is called unselfish (cf. 182{-}183). The will is defined as ``the capacity of a person to determine herself to actual satisfaction or non{-}satisfaction of a demand of the selfish drive'' (183). Reinhold stuck to the text he used for the `Beytrag'{-}essay, discussed above (section 1.3). Both in the `Beytrag'{-}essay and in this sixth `Brief', the action of the selfish drive is described as ``desire in the narrow sense,'' whereas the demands of the unselfish drive, considered as incentive in willing, is called ``pure willing'' (184). In line with the fifth article Reinhold defines the natural freedom of the will as the ``capacity of a person to determine herself (\ldots ) either in line with or against the demand of the unselfish drive'' (185) and he expressly distinguishes this kind of freedom from the self{-}activity of reason. 

 With the basic definitions in place, Reinhold continues with a closer look at `morality', which in the widest sense refers to ``the relation between the demands of the selfish and unselfish drives.'' In the narrower sense of morally good, this relation is the ``subordination of the satisfaction of the selfish drive under the demand of the unselfish drive'' (186). The subordination of the selfish drive under the demand of the unselfish drive requires that the satisfaction or non{-}satisfaction of the demands of the former be undertaken for the sake of the lawfulness of the latter (cf. 188). This subordination, however, has its limits, and the selfish drive must not be destroyed or extinguished by the moral law. Reinhold refers to incorrect concepts of moral action, according to which a moral action is ``merely the action of practical reason'' and of freedom, according to which freedom is only to be found ``in the mere spontaneity of this [i.e. practical] reason'' (191). This misunderstanding leads to an understanding of morality in which the source of moral action is freedom, whereas the immorality can only be understood as a limitation of freedom.\footnote{ This is the interpretation of Kant that Schmid had put forward. Cf. Schmid, \textit{Versuch einer Moralphilosophie}, \S  255, pp. 209{-}210. } 

 The remainder of this sixth article is dedicated to defining right, duty and related concepts like perfect and imperfect right and duty, duties against oneself and against others and more. First of all the genus `right' is defined as that ``which is possible through the moral law by means of freedom'' (193). Two species reside under this genus, namely right in the narrower sense {--} that which is merely possible {--} and duty {--} that which is necessary according to the moral law (cf. 193). Reinhold starts by taking a closer look at duty, distinguishing between perfect and imperfect duty. Perfect duty, according to him, follows ``immediately from the moral law,'' whereas imperfect duty follows only from the moral law under extra assumptions (195). The only thing that immediately follows from the moral law is the impossibility of its contradiction, that is, the impossibility of ``the voluntary subordination of the most general demand of the unselfish drive under the demand of the selfish drive'' (196). Forgoing a more precise determination of imperfect duty, Reinhold continues with the definition of right in the narrower sense. Like perfect duty, perfect right follows ``immediately from the moral law,'' whereas imperfect right only follows from the law under certain assumptions (202). Again, Reinhold attacks some `friends of the Kantian philosophy', that is, Schmid,\footnote{ In his introduction to \textit{Briefe II} Bondeli also mentions Ludwig Heinrich Jakob (1759{-}1827) and Johann Heinrich Abicht (1762{-}1816), and also Salomon Maimon (1753{-}1800). Cf. Bondeli, introduction to \textit{Briefe II}, XXXIV. } who confuse the causality of reason with the freedom of the will (204{-}205), whose understanding of the will as a causality of reason is, according to him, not fit for deriving the proper concepts of right and duty.\footnote{ The disagreement between Reinhold's and Schmid's interpretation of Kant's thoughts on freedom has been extensively discussed in recent literature. See for instance Z\"{o}ller, `Von Reinhold zu Kant. Zur Grundlegung der Moralphilosophie zwischen Vernunft und Willk\"{u}r'; di Giovanni, `Rehberg, Reinhold und C. C. E. Schmid \"{u}ber Kant und die moralische Freiheit'; Fabbianelli, `Die Theorie der Willensfreiheit in den \quotedblbase Briefen \"{u}ber die Kantische Philosophie`` (1790{-}1792) von Karl Leonhard Reinhold';  Goubet, `Der Streit zwischen Reinhold und Schmid \"{u}ber die Moral' and Lazzari, \textit{Das Eine, was der Menschheit Noth ist}, section 5.1. } He believes that these can only be established ``from the relation of the unselfish drive to the selfish drive, not from one of them considered on its own'' (204). 

 In the seventh `Brief', `Ueber den bisher verkannten Unterschied zwischen dem uneigenn\"{u}tzigen und dem eigenn\"{u}tzigen Triebe, und zwischen diesen beyden Trieben und dem Willen', Reinhold returns to the first definitions of the previous article, in order to elaborate on the concepts used. It is necessary to specify the differences between both drives and the will, since philosophers do not agree on the nature of these differences because they share the misunderstanding: 

da\ss{} Lust und Unlust die Triebfeder nicht nur des unwillk\"{u}hrlichen Begehrens, sondern auch des willk\"{u}hrlichen, oder des \textbf{Wollens} seyen und seyn m\"{u}\ss{}ten. (223)

[that pleasure and pain were not only the incentives for involuntary desire, but were and had to be also the incentives for voluntary desire, or \textit{willing}.]

Hence, willing was understood as rational desire and it was assumed that the will could only be moved by pleasure, not by ``grounds of reason on and for itself'' (224). Since it was not questioned whether all drives were selfish (based on pleasure), the debate regarding the selfishness (\textit{Eigenn\"{u}tzigkeit}) or unselfishness (\textit{Uneigenn\"{u}tzigkeit}) of a certain drive often centered on the question how the term `use' (\textit{Nutzen}) is employed (cf. 227). 

 Not only the terms `selfish' and `unselfish' have proved to be problematic; the understanding of the term `pleasure' (\textit{Vergn\"{u}gen}) has also shown a marked lack of clarity (cf. 231). Since the source of pleasure has often been identified as an agreeableness on the part of the object, Reinhold investigates whether morality could involve pleasure, based on the agreeableness of a moral action. The agreeableness of moral action may be situated in the pleasure that follows from the awareness of having done the right thing, even if the action itself is not particularly pleasant or involves forsaking some other pleasure. However, if this anticipated pleasure were a factor in the judgment that an action is to be taken, moral action would inevitably follow (cf. 238). This implies moral determinism, which abolishes responsibility and the distinction between moral and amoral action (\textit{sittliche und nichtsittliche Handlungen}) (238). Any system that only allows pleasure as determining ground for the will and thus seeks to describe the moral will as being determined by unselfish pleasure, abuses the term `unselfish' (cf. 241). A pleasure can only be called `unselfish' ``when and insofar it could only be thought as the consequence, not as the ground of moral action'' (241{-}242). Not surprisingly, Reinhold credits Kant's moral philosophy with having established the proper concept of `unselfishness' (cf. 243).\footnote{ Given Reinhold's apparent fascination with Kant's Table of Material Determining Grounds it seems likely that he presents Kant as having established the proper concept of `unselfishness' in because the latter claimed that the truly moral determining ground of the will would have to be formal rather than material. } As usual Reinhold does not provide any argumentation but instead explains why this proper concept has not been recognized as the foundation of morality before. One of these reasons is the hitherto only partially understood conception of the will (cf. 243). Willing has been understood as the ``drive for pleasure, guided by reason'' (244). Although it is true that desire for pleasure takes place in willing, this is only part of the story. There is another thing involved that is specific for willing, namely a decision (\textit{Entschlu\ss{}}), which concerns the satisfaction of the demands of involuntary desire (245). This determination to satisfy the demand of involuntary desire or not is active and voluntary. The other element of willing, reason, has likewise been misunderstood in so far as it was not acknowledged that in willing a person could ``act against the verdicts of reason, abuse reason'' (248). This is, according to Reinhold, a fact of consciousness, the denial of which leads to the abolition of the distinction between amoral and immoral actions. The combination of the characteristics of both desire and reason in the will leads to the following description: 

wirklich besteht das ganz Eigenth\"{u}mliche der Willenshandlung, der Entschlu\ss{}, in nichts anderm als in der Vorschrift die sich die Person zur Wirklichkeit der Befriedigung oder Nichtbefriedigung des eigenn\"{u}tzigen Triebes giebt. (251)

[actually the proper characteristic of the act of will, the \textit{decision}, consists in nothing else but the \textit{prescription that a person gives herself for the actualization of the satisfaction or the non{-}satisfaction of the selfish drive}.]

This kind of prescribing something for oneself must be carefully distinguished from several other kinds of prescriptions. First of all, there are the rules of involuntary desire itself, which are natural laws. These are not established by the person, but rather by ``the selfish drive by means of reason'' (252). Secondly, the prescription that a person gives herself in willing must be distinguished from the ``prescription that a person gives herself by means of practical reason'' (252). The prescription given in willing is a maxim for action that determines whether or not to satisfy a given demand of desire in light of the moral law, prescribed by practical reason. Both the demand of desire and the demand of the moral law are only ``incentives of the will'' in so far as the will decides to accept one of them as determination for the satisfaction or non{-}satisfaction of the desire (cf. 254{-}255). 

 In the eighth `Brief', `Er\"{o}rterung des Begriffes von der Freyheit des Willens', Reinhold starts from the premise that the concept of the freedom of the will needs further explanation, because all philosophical systems before Kant contradict the proper concept of freedom, whereas the Kantian philosophy has only given an indication and ``by no means established it with those characteristics that distinguish its object from all other objects'' (263). Rather, Kant has only made a full definition of the concept of the freedom of the will possible. Reinhold intends to provide this concept by identifying the elements that are part of the proper concept of the freedom of the will, but have hitherto been interpreted as the full concept. First he uses his description of the will as the capacity of a person to self{-}determine with regard to the satisfaction of a desire to point out that this entails the ``independence of the person from the coercion of that demand'' (264). This is granted by philosophers like Schmid, who agree that freedom means freedom from the force of instinct, but allow the will to be necessarily determined by reason, and are thus determinists. According to Reinhold, the ``limitation of instinct that inevitably follows from the power to think'' on which these determinists rely, is not necessarily voluntary (265). The determinists further believe that reason's determination is grounded in pleasure. Thus the difference between an action from instinct and an action based on reason is only that the first immediately depends on coercion by pleasure and pain, whereas the second depends on it ``mediately, through the power of thought'' (265). Finally, as they think of reason as the capacity to know the connection of things in themselves, `being determined by reason' must mean ``being determined by the connection of things in themselves that is completely independent of the person, that is, depending by reason on the inevitable necessity of nature'' (266). 

 Secondly, on the basis of the description of the will as the capacity of a person to self{-}determine with regard to the satisfaction of a desire in accordance with the moral law, Reinhold points out that reason, in its practical law, must be independent of the drive for pleasure (cf. 267). According to him, this characteristic of the freedom of the will has been accepted as the only one by certain `friends of the Kantian philosophy', that is, Schmid. The prevalence of this characteristic entails a confusion of the ``spontaneous, yet by no means free, action of practical reason'' in establishing the moral law, with the action of the will (267). This leads to the claim that ``the will is only free in moral action'' (268). We have seen earlier that Schmid defended such a claim in his \textit{Versuch einer Moralphilosophie}. Reinhold claims that Kant's expressions regarding the nature of the will must be understood as expositions, rather than full definitions (cf. 268{-}269). The Kantian identification of the will with `causality of reason' does not sufficiently distinguish the will from other capacities, such as rational thought (cf. 269). Schmid's understanding of `empirical will' in his \textit{W\"{o}rterbuch} is cited as an example of such a poor Kant interpretation, for it leads to the conclusion that the empirical will is not free. Freedom is situated in ``the dependence of the will on reason that determines it immediately'' (271, citing Schmid\footnote{ Schmid, \textit{W\"{o}rterbuch zum leichtern Gebrauch der Kantischen Schriften} (2nd edition), s.v. `practische Freyheit,' 178.}). According to Reinhold, however, the dependence on reason is ``posited by freedom, which can follow or break the practical law'' (271). 

 Thirdly, on the basis of the definition of the will as the capacity of a person to self{-}determine with respect to the satisfaction of a desire either in accordance with or against the practical law, Reinhold points out that the freedom of the will must include the ``independence from the coercion of practical reason itself'' (272). The point Reinhold makes here is in fact a continuation of the previous point against Schmid, since he aims to show that both pure and impure, or empirical, willing are based in the freedom of the will (cf. 272{-}273). 

Ohne das praktische Gesetz w\"{u}rde er [the will] von dem blo\ss{}en Naturgesetze des Begehrens abh\"{a}ngen, und nicht nur nicht frey, sondern nicht einmal eine \textbf{Wille}, sondern ein unwillk\"{u}hrliches Begehren seyn, und ohne die Naturgesetze des Begehrens w\"{u}rde er von dem blo\ss{}en praktischen Gesetze abh\"{a}ngen, die blo\ss{}e praktische Vernunft selbst, und folglich zwar selbstth\"{a}tig, aber nicht frey, und kein \textbf{Wille}, kein Verm\"{o}gen sich zur Befriedigung oder Nichtbefriedigung eines Begehrens zu bestimmen seyn. (275{-}276)

[Without the practical law the will would depend on the mere natural laws of desire, and not only would it not be free, but it would not even be a \textit{will}, but rather an involuntary desire; and without the natural laws of desire it would depend on the mere practical law, mere practical reason itself, and therefore it would indeed be spontaneous, but not free, and not a \textit{will}, not a capacity to determine oneself to the satisfaction or non{-}satisfaction of a desire.]

Since, according to Reinhold, both natural law and practical law are required for the freedom of the will, he understands Kant's claim that the consciousness of the moral law yields cognition of freedom\footnote{ Cf. \textit{AA} 5:28{-}30.} as a partial exposition of the freedom of the will. The consciousness of the reality of freedom depends on the consciousness of the demands of both the moral law and of desire, but also on ``the consciousness of the capacity to self{-}determine the satisfaction or non{-}satisfaction of the selfish drive either through or against the demand of the unselfish drive'' (276). 

 One philosophical position has taken the independence of the will from all determining grounds seriously, namely that of the equilibrists.\footnote{ Reinhold also considered the equilibrist position in \textit{Versuch}, 97. } Reinhold, at least, interprets the equilibrist claims regarding the indifference of the will towards its incentives and the balance of these incentives as expressing the total independence of the will from all kinds of incentives. The equilibrist account, however, lacks a proper understanding of the important role of maxims in moral action. According to Reinhold, the proper understanding of freedom rests on the following understanding of maxims.

Die Maxime ist ein Resultat der Willk\"{u}hr und der Vernunft, eine Vorschrift unter der Sanktion der Willk\"{u}hr, durch die entweder das praktische Gesetz, oder die demselben entgegen gesetzte Reitze der Lust oder Unlust in den Willen aufgenommen, und aus blo\ss{} veranlassenden zu bestimmenden Gr\"{u}nden der Handlung gemacht werden. (279)

[The maxim is a result of will and reason, a prescription sanctioned by will, through which either the practical law or the opposite stimulus of pleasure and pain is incorporated in the will and is made from an occasioning ground into a determining ground of action.]

This understanding of maxims is missing from the account of the equilibrists, who have correctly given the negative description of freedom, namely the independence from both the unselfish and the selfish drive, but have neglected the positive side of freedom, namely the capacity for self{-}determination, that is, ``the capacity to elevate one of the occasioning grounds to be a determining ground'' (280). Since the spontaneity of the will consists in the independence from the objective grounds of reason and desire, there is no answer to the question as to the objective ground determining the will. Nevertheless, free action is not groundless; ``its ground is freedom itself'' (282). This freedom is the first cause of the action, which is legitimized for common sense with reference to self{-}consciousness, ``through which the action of this capacity announces itself as a fact'' (283). Philosophical reason cannot give a further justification since it cannot understand the possibility of freedom any more than common sense can (cf. 283). It can, however, explain why further justification is impossible; the free will is a ``fundamental capacity'' (\textit{Grundverm\"{o}gen}) of the human mind, that is, a capacity ``that cannot be understood or explained on the basis of another capacity'' (284). Only the effects of such a capacity may be understood or explained, yet the source of these effects cannot be understood. 

 Having established his thoughts on the proper concept of the freedom of the will and having indicated the way in which previous philosophers have failed to grasp this concept, Reinhold dedicates the remainder of the eighth `Brief' to the issue of Schmid's interpretation of Kant, according to which the will is ``nothing but the causality of reason with regard to desire'' (285). Reinhold first takes the claim from the second \textit{Critique} that reason, in moral legislation, is practical\footnote{ Cf. \textit{AA} 5:31{-}32.} and interprets it as the claim ``that the will cannot be free without the practicality of reason (by no means through it alone)'' (286). He further claims that Kant calls reason practical ``not in so far as it acts as will itself (\ldots ), but because and in so far it gives the will a prescription only through itself, for the mere sake of prescribing'' (288). This practicality of reason is the important thing that has been established by Kant. It is important, because it is the only way in which moral necessity can be united with natural freedom (cf. 290). If the only prescriptions regarding the satisfaction or non{-}satisfaction of desire were those that were sanctioned by pleasure and pain themselves, a person would be bound by ``natural laws of desire'' and could not will, only desire (290). When, however, there is another prescription, sanctioned only by itself, independent of pleasure and pain, the person is dealing with ``two equally involuntary, opposed demands'' that can only be unified by subordinating one to the other (291). Some of the `friends of the Kantian philosophy' have, however, confused this practicality of reason with the will (293). Seeking to unite moral necessity with the freedom of the act of will, they have interpreted the independence of instinct as coercion on the part of reason. Like Rehberg, they call a moral action free in so far ``as it was forced by reason and not by sensibility'' (295). Consequently, they cannot ascribe freedom to immoral action. 

 Instead of making the will the slave of the passions, the friends of the Kantian philosophy have made the will ``the slave of practical reason,'' or rather, they let practical reason act in its place (295{-}296). The problem here is that the same spontaneity is ascribed to the foundation of the moral law and to the action in accordance with that law (cf. 297). Reinhold describes moral and natural necessity and freedom as united in the following manner. 

In der sittlichen Handlung ist \textbf{absolute praktische Nothwendigkeit} und \textbf{Freyheit} in so ferne vereinigt, als das absolute nothwendige Gesetz, die Wirkung der praktischen Vernunft, durch Willk\"{u}hr in einem gegebenen Falle ausgef\"{u}hrt, und in so ferne zur Wirkung der Freyheit gemacht ist. In der unsittlichen Handlung ist die \textbf{Naturnothwendigkeit} und die \textbf{Freyheit} in so ferne vereinigt, als die blo\ss{} dem Naturgesetz des Begehrens gem\"{a}\ss{}e, aber dem praktischen Gesetze widersprechende Forderung des eigenn\"{u}tzigen Triebes durch Willk\"{u}hr ausgef\"{u}hrt, und in so ferne zur Wirkung der Freyheit erhoben ist. (297{-}298)

[In moral action \textit{absolute practical necessity} and \textit{freedom} are united in so far as the absolutely necessary law, the effect of practical reason is carried out in a given case by will and to this extent is made the effect of freedom. In immoral action \textit{natural necessity} and \textit{freedom} are united insofar as the demand of the selfish drive that merely accords with the natural law of desire, but contradicts the practical law is carried out by will and insofar is elevated to an effect of freedom.] 

It is important for Reinhold to clearly distance himself from the position that would make the will the slave of practical reason. We have seen that in the `Grundlinien', he had not properly distinguished between the spontaneity of reason in establishing the moral law and the spontaneity of the will in following or disobeying it. This is not to say that he, at that point, was proposing the form of moral determinism that he now condemns {--} after all, his starting point was that moral freedom must entail freedom from both sensible desire and reason. Nevertheless, his thoughts as expressed in the `Grundlinien' and in the essays on justice discussed in section 1.2 are very confused with regard to the different activities of reason. In criticizing the position of Schmid, Reinhold also distances himself from his own previous position. Given the remark on the importance of Rehberg's review of the second \textit{Critique}, it may have been this review that made him realize that his own position might result in moral determinism because of the identification of the will and practical reason.\footnote{ For more on the Reinhold{-}Rehberg{-}Schmid triangle cf. di Giovanni, `Rehberg, Reinhold und C. C. E. Schmid \"{u}ber Kant und die moralische Freiheit.'} This is clearly something that Reinhold did not yet realize when he reacted to Rehberg's review in the `Grundlinien', reasserting a Kantian line on the practicality of pure reason. As we have seen in section 1.1 of the present chapter, however, Reinhold was quite confused at the time regarding the distinction between will and practical reason. He did not explicitly identify the two, yet gives us nothing to distinguish them by. An implicit identification of will and practical reason is at odds with the `pre{-}philosophical' thoughts on freedom as expressed in the First Book of the \textit{Versuch}. Reinhold's account there appears to require a will that can act independently from both sensibility and reason, thus mediating the demands of both. Apart from this tension between his defense of Kant in the `Grundlinien' and his common{-}sense requirements for the free will, he had one more reason to revisit his initial reaction to Rehberg's review. In section 3.2 of the previous chapter, it became clear that there is also a tension between the framework of the theory of the faculty of representation and establishing an absolutely free will. Since Reinhold's initial reaction to Rehberg is characterized by the attempt to relate a `Kantian' theory of morality to the theory of the faculty of representation, he may not have realized that this created tensions. The `Grundlinien' are no more than a temporary solution. It was clear to Reinhold that more work needed to be done. When he returns to the subject he first liberates his practical philosophy from the framework of the theory of the faculty of representation and then starts to distinguish sharply between the will and practical reason.\footnote{ The combination of these adaptations amount to allow  for several foundational principles, rather than one, which is, as Lazzari has show, very relevant with respect to Reinhold's \textit{Elementarphilosophie}. Lazzari, \textit{Das Eine, was der Menschheit Noth ist}, 322{-}324.} 


\subsubsection{The beneficial consequences: ninth to eleventh `Briefe'}


The final four articles in \textit{Briefe} II are dedicated to presenting the consequences of the proper understanding of the freedom of the will. One would expect those beneficial consequences to be situated in the fields of natural and positive right, yet Reinhold first returns to the themes of the first series of `Briefe', namely the conviction of the existence of God and of an afterlife. In that first series he had boldly claimed that Kant's investigation of human reason had yielded important results. The newly found insight into the nature and structure of human reason made the solution of the philosophical debates regarding the rational grounds for the convictions of the existence of God and of an afterlife possible. Especially with regard to the conviction that God exists Reinhold had claimed that the Kantian investigation of reason had revealed a special role for practical reason as providing a rational ground for this conviction. The fourth chapter has shown that Reinhold's understanding of `practical reason' at that time was more related to his Enlightenment ideals than to the context in which he presented it, that of Kant's first \textit{Critique}. It is not surprising that in his second volume of \textit{Briefe}, having presented a view on practical reason and the will that is much more related to the second \textit{Critique} than to the first, Reinhold would want to address the issue again. 

 The ninth article, `Ueber die Unvertr\"{a}glichkeit aller bisherigen philosophischen Begriffe von der Seele mit dem richtigen Begriffe von der Freyheit des Willens', presents Reinhold's previous explanation of the concept of the freedom of the will as ``a justification of the conviction of common and healthy understanding'' (308). The relation between philosophical reasoning and common understanding is then described in the following manner. 

Der gemeine Verstand und die philosophierende Vernunft sind an \textbf{eben dieselben Grundverm\"{o}gen} des menschlichen Geistes gebunden, die sich in dem gemeinen Verstand durch unwiderstehliche und unfehlbare \textbf{Gef\"{u}hle} ank\"{u}ndigen und durch sie die Ueberzeugungen bewirken, \"{u}ber welche die philosophierende Vernunft, welche die \textbf{Gr\"{u}nde} jener Gef\"{u}hle aufsucht, so lange mit sich selbst uneinig bleiben mu\ss{}, als es ihr noch nicht gelungen ist, deutliche und bestimmte Begriffe der Grundverm\"{o}gen aufzustellen. (309)

[Common understanding and philosophizing reason are bound to the \textit{same fundamental capacities} of the human mind. These capacities announce themselves in common understanding through irresistible and infallible \textit{feelings} through which they establish those convictions, about which philosophizing reason, seeking the \textit{grounds} of those feelings, must remain in discord with itself as long as it has not succeeded in establishing distinct and determinate concepts of the fundamental capacities.]

It is clear that the voice of common understanding carries substantial weight, whereas philosophical reason cannot at first achieve its aim to justify the convictions of common understanding. With regard to freedom this means that common understanding is convinced of its actuality through feeling, while philosophical reason seeks to establish its possibility. Since the understanding of freedom as a fundamental capacity of the mind has only become possible with the Kantian philosophy, the concept of the soul, ``the subject of the fundamental capacities of the mind'' was likewise not sufficiently determined (310). 

 Reinhold starts from scratch by establishing that there must be a metaphysical conception of the soul that is compatible with the concept of the freedom of the will. 

Da man den Willen nur als ein \textbf{Pr\"{a}dikat der Seele} denken kann, so fordert der Begriff vom Willen einen Begriff von der Seele, als dem \textbf{Subjekte} desselben, und da dieser letztere seiner Natur nach \textbf{metaphysisch} ist, so mu\ss{} freylich auch ein metaphysischer Begriff von der Seele m\"{o}glich seyn, aus dem sich zwar die Freyheit nicht \textbf{ableiten} l\"{a}\ss{}t, mit dem sich aber der Begriff von derselben vertr\"{a}gt. (317)

[Since the will can only be thought as a \textit{predicate of the soul}, the concept `will' presumes the concept `soul' as its subject. Since the latter is \textit{metaphysical} in nature, there must be possible a metaphysical concept of the soul, from which we may not \textit{infer} freedom, but which is compatible with the concept of freedom.]

As we have seen in the earlier citation, the task of philosophy is to establish the conceptual framework for the convictions of common sense. Guided by feeling, we are convinced that we are free; philosophy, although unable demonstrate that we are free, must show that the concept of this freedom contains no contradiction. From the above passage it is clear that this entails that philosophy must show that it is possible to think of the subject of the will, the soul, in a way that is compatible with the freedom of this will. From this starting point Reinhold discusses the strengths and weaknesses of the previous metaphysical systems. He praises philosophical supernaturalism for rejecting material and naturalistic explanations for morality and ascribing the moral convictions of common sense to revelation (cf. 319{-}320). Yet the philosophical supernaturalist\footnote{ Reinhold aims at describing philosophical systems rather than referring to actual philosophers. In this case he may have had in mind someone like Jacobi. In the \textit{Versuch} he mentions Pascal (\textit{Versuch}, 11), Jacobi and Schlosser (\textit{Versuch}, 86) as supernaturalists.} is wrong in seeking the cause of moral feeling outside reason, because it is indeed grounded in practical reason (cf. 320). Lacking a determinate concept of the role of practical reason, the supernaturalist has no determinate concept of his moral feeling either, and the convictions established by moral feeling remain mysterious to him (cf. 322{-}323). Thus he assumes contradictions without noticing them. Thinking of the soul as a thing in itself, as the supernaturalist does, cannot be done ``without attributing marks to it with which the freedom of the will is entirely incompatible'' (324). 

 In his discussion of the naturalistic ways of conceptualizing the soul, Reinhold distinguishes between ``dogmatically skeptical'' and ``dogmatically metaphysical'' convictions regarding the nature of the soul (325). As in the \textit{Versuch}, Reinhold discusses different forms of skepticism at length with the aim of disqualifying skepticism of a non{-}philosophical kind.\footnote{ Cf. \textit{Versuch}, 120{-}141; Chapter 5, section 2.2.} The dogmatic skeptic who claims that all metaphysical concepts of the soul are untenable, at the same time discredits the determinations ``under which common understanding must think the soul'' as well (328). Hence, common understanding does not benefit at all from skepticism regarding the concept of the soul. With regard to the dogmatically metaphysical naturalists, Reinhold distinguishes between the materialistic and the spiritualistic variety. According to materialists the soul is ``nothing but the human organization itself,'' that is, the organic human body (331). The materialist ascribes the feeling of freedom to ``the ignorance of the actual incentive [\textit{Triebfeder}] of our actions'' (332). According to Reinhold, these materialistic conceptions of the soul and freedom are not only discredited by the Kantian philosophy, but also by the evidence of science, which he believed to show that mechanical and chemical laws ``are limited to unorganized matter'' and therefore do not apply to phenomena in so far as they are organized (333).\footnote{ Reinhold here appears to touch upon themes that are central to Kant's third \textit{Critique}, which was published in 1790. According to Bondeli, introduction to \textit{Briefe II}, XXVII, Reinhold was definitely influenced by the \textit{Critik der Urtheilskraft}, especially by Kant's consideration concering aesthetic judgment and \textit{sensus communis.} } The spiritualist philosophical conceptions of the soul are closest to the verdicts of common sense, yet, thanks to the activities of \textit{Popularphilosophen} it finds itself in a miserable position (cf. 335). According to Reinhold, the spiritualist systems (Descartes, Leibniz, Berkeley) all need a ``system of assistance'' to account for the representations of (seemingly) material objects (338). Popular philosophy, however, has sought to sever the Leibnizian conception of the soul from the theory of pre{-}established harmony. This theory is detrimental to the freedom of the will, and it is only the ``unstable and confused conceptions'' of the popular philosophers that allow them to think that the Leibnizian concept of the soul is compatible with the common conception of the freedom of the will (341).

 Having thus established the incompatibility of the previous conceptions of the soul with the proper conception of the freedom of the will, Reinhold sets out to show that the Kantian idea of the soul does not suffer similar problems. He does this by invoking his `short argument' to idealism that in this context runs as follows.\footnote{ Cf. Ameriks, \textit{Kant and the Fate of Autonomy}, 125{-}135; \textit{Versuch}, 254; Chapter 5, section 2.3.} Kant has countered the realist claim that reason knows things in themselves by pointing out that the forms of representation are only predicates of the represented objects in so far ``as they are represented'' (343). This means that they do not apply to the things as they are in themselves, which, as such, can therefore not be known. There are no laws that are given from the outside; it is only in sensibility that a dependence is found, namely on ``the material of sensible representation'' determined by things outside us (344). From this description of transcendental idealism in combination with the proper conception of the will, Reinhold draws the following conclusions. First, in willing we only depend on things outside us in so far as the ``involuntary and sensible faculty of desire is involved'' (344). Secondly, we are not determined by things outside us insofar as reason is involved. Thirdly, the demand of the selfish drive is determined partly by the things outside us, partly by our own theoretical reason. Fourthly, the demand of the unselfish drive is determined ``by mere reason and definitely not by things outside us'' (345). Fifthly, all determination in willing only concerns the involuntary demands of the selfish and unselfish drives. Finally, the voluntary element in willing, the self{-}determination to satisfaction or non{-}satisfaction of those demands, cannot be thought as a ``being determined, either by the things outside us, or by reason'' (345). It is self{-}determination of the will.

 According to Reinhold the above claims summarize the proper conception of the will and the premises that led to these results also lead to the only concept of the soul that is compatible with the results concerning freedom of the will presented above. It follows that the soul ``cannot be cognized in the capacity of a thing in itself'' (345{-}346). The capacities of the human mind do not follow from the concept of the soul as a substance, but rather ``they announce themselves through different states of consciousness'' (346). The will and its freedom are among the fundamental capacities and reveal themselves ``through facts of consciousness'' (347).\footnote{ Note again that Reinhold's description of the freedom of the will as a fact of consciousness clearly differs from Kant's `fact of reason', which refers to the lawgiving capacity of reason. Cf. footnote \ref{footnote:_Ref232953324}; AA 5: 31.}

In the tenth `Brief', `Ueber die Unvertr\"{a}glichkeit zwischen den bisherigen philosophischen Ueberzeugungsgr\"{u}nden vom Daseyn Gottes und den richtigen Begriffen von der Freyheit und dem Gesetze des Willens', Reinhold revisits the other major theme of his first series of `Briefe', namely the grounds for the conviction that there is a God.\footnote{ The article is an adapted version of Reinhold, `Ueber die Grundwahrheit der Moralit\"{a}t und ihr[ ] Verh\"{a}ltni\ss{} zur Grundwahrheit der Religion'. Cf. footnote \ref{footnote:_Ref232222730}. } The starting point of the article is the claim that morality ``could not exist, if there would be a proof for the existence of God that is independent of it and therefore theoretical'' (352). This strong claim requires argumentation. The first step is to make the relevant understanding of morality explicit. 

Nach unsern Begriffen ist die Sittlichkeit ein v\"{o}llig freyes und ganz uneigenn\"{u}tziges Wollen des Gesetzm\"{a}\ss{}igen um seiner Selbst willen, und die sittliche Handlung so wie die Unsittliche (\ldots ) die eigenth\"{u}mliche Aeu\ss{}erung der Freyheit unsres Willens. (354{-}355)

[According to our conception, \textit{morality is a completely free and unselfish willing of that which is lawful, for its own sake}, and the \textit{moral action} and the \textit{immoral action} alike are characteristic expressions of the freedom of our will.] 

It is only because of the ``independent capacity of the self{-}determining will'' that we understand ourselves as moral beings, as persons (356{-}357). Although this freedom is a fact of consciousness for us, it may be doubted in a society with a certain level of scientific and cultural development (cf. 358). This doubt can only be overcome by means of the ``secret of the complete harmony between thinking and acting reason, grounded in the nature of the human mind'' (360). This `secret' is to be found in the proper concept of the freedom of the will, revealing the relation between practical reason, depending only on itself to prescribe, and theoretical reason, the prescriptions of which depend on given pleasure and pain (cf. 361). Anything contradicting the proper conception of freedom at the same time destroys morality. The next step is to show that any theoretical proof for God's existence is at odds with the proper conception of the freedom of the will. If the existence of God can be cognized prior to and independently of the moral law, that law is no longer ``the law, the following or breaking of which depends on our choice'' but rather the will of a sovereign enforcing it by infinite punishments or rewards (362). If, however, the conviction of God's existence is determined by the independent conviction of the moral law, this fundamental truth of religion provides ``an external ground that supports the demands of the unselfish drive upon the selfish drive'' (363). 

 Without morality as the ground of the conviction that God exists, God's moral nature cannot be acknowledged. The atheist understands God as the primordial power (\textit{Urkraft}) of nature, which is incompatible with the freedom of the will (cf. 365{-}366). The supernaturalist likewise thinks of God as a primordial power, endowed with an incomprehensible will, which is also destructive for morality (cf. 366). Both misunderstandings of the characteristics of God derive from a one{-}sided understanding of the moral feeling.

Der Supernaturalist ahndete die Unzertrennlichkeit zwischen Religion und Moral, der Naturalist die Unabh\"{a}ngigkeit der Moral von Religion. (367)

[The supernaturalist sensed \textit{the inseparability of religion and morality}; the naturalist sensed \textit{the independence of morality from religion}.] 

The reason for these one{-}sided approaches is that both parties misunderstand practical reason: the supernaturalist confuses it with God, while the naturalist confuses it with nature (cf. 369). According to Reinhold we can only distinguish between God and nature by considering the ways of acting that we attribute to them. Unlike nature, the deity acts out of absolute spontaneity ``which we know from no other source, than through the self{-}consciousness in which our own spontaneity reveals itself as free will'' (373). This confirms the connection between the conviction of God's existence and morality. 

 Reinhold continues his overview of the beneficial consequences of his concept of the freedom of the will in the eleventh article, `Grundlinien zur Geschichte der bisherigen Moralphilosophie \"{u}berhaupt, und insbesondere der stoischen und epikurischen'.\footnote{ It was not uncommon in late eighteenth{-}century Germany to employ `stoicism' and `epicureanism' as two extremes of moral philosophy. Platner had done so in his 1776 article entitled `Versuch \"{u}ber die Einseitigkeit des Stoischen und Epikurischen Systems in der Erkl\"{a}rung vom Ursprunge des Vergn\"{u}gens,' \textit{Neue Bibliothek der sch\"{o}nen Wissenschaften und der freyen K\"{u}nste}. In the second \textit{Critique} Kant had also used the scheme of opposing stoics and epicureans with the aim of presenting his own moral philosophy as the means of overcoming this opposition. Cf. \textit{AA} 5:111{-}113. Cf. Bondeli, introduction to \textit{Briefe II}, LXXXI{-}LXXXIII.} He claims that ``the most important misunderstanding'' between these systems can be overcome by the proper concept of the freedom of the will (383). From this concept it follows ``that moral actions are neither produced only by reason nor by striving for pleasure, neither by the selfish, nor by the unselfish drives'' (384). The proper philosophical concept of morality requires a proper concept of the freedom of the will, which in turn requires the distinction between ``the involuntary demand of the selfish drive,'' ``the involuntary demand of the unselfish drive'' and ``the voluntary act of decision'' (384). These ``facts of consciousness'' can only be philosophically distinguished by understanding them as manifestations of different fundamental capacities of the mind, namely theoretical reason, practical reason and the will (cf. 384{-}385). As long as there is no determinate concept of the will as one of the fundamental capacities, philosophers will confuse the will either with the unselfish drive, and adhere to a form of stoicism, or with the selfish drive, and adhere to a form of Epicureanism (cf. 387). Since both the stoics and the epicureans focus on only one of our drives, they cannot have a proper concept of the ``complete object of the moral will, the whole good of man'' consisting in the ``satisfaction of both drives of human nature'' (390). Thus they both confuse morality and happiness, either by replacing happiness with morality (stoics), or by replacing morality with happiness (epicureans) (cf. 392{-}393). Both systems fail to understand happiness and morality properly, since they define them from their own standpoints; thus the epicurean is more right in his conception of happiness and the stoic in his conception of morality (cf. 405). Morality and happiness cannot be though apart from one another, however. In the complete good for man, the ``aim of the united drives of human nature'' morality and happiness come together (407). Notwithstanding the flaws of these opposed systems, the stoics and epicureans themselves had less problems with moral practice, since their faulty concepts were corrected by their feelings of common sense (cf. 408). When the Greek and Roman civilizations of Antiquity declined, the stoic and epicurean systems lost touch with the correction of feeling and degenerated; stoicism became monasticism, while Epicureanism became libertinism (cf. 411). These degraded forms of stoicism and Epicureanism are, according to Reinhold, currently connected to supernaturalism and naturalism, respectively (cf. 413). The supernaturalist stoic differs from his ancient counterpart in taking the source of the moral law to be divine reason instead of human reason (cf. 414{-}415). Current epicureans, naturalists, confuse the law of desiring with the law of willing, while it depends on the details of their metaphysics what they take to be the object of desire (cf. 415). 


\subsubsection{Philosophy and society: twelfth `Brief'}


The twelfth and final article is entitled `Ueber die \"{a}u\ss{}ere M\"{o}glichkeit des k\"{u}nftigen Einverst\"{a}ndnisses der Selbstdenker \"{u}ber die Principien der Moralphilosophie'.\footnote{ This article consists of two pieces published earlier in \textit{Der neue Teutsche Merkur}. Reinhold, `Die drey St\"{a}nde. Ein Dialog'; Reinhold, `Die Weltb\"{u}rger. Zur Fortsetzung des Dialogs, die drey St\"{a}nde, im vorigen Monatsst\"{u}ck.' Cf. footnote \ref{footnote:_Ref234654472}. } With this `Brief' Reinhold concludes the collection by considering the likelihood that the revolution initiated by the Kantian philosophy will indeed change moral philosophy for good. He presents the issue in the form of a dialogue between himself and his fictitious correspondent, who is skeptical and opens by doubting whether freedom is actually possible, since it always appears to degenerate into either despotism or anarchy (cf. 421). Reinhold admits that true freedom can only exist once the proper concept of freedom has been developed (cf. 421). However, he believes that a certain amount of freedom is guaranteed by the existence of extremes limiting one another. The situation in France may be viewed as a negative example for the rest of Europe, making clear that the two extremes of freedom, despotism and anarchy, lead to disaster (cf. 424). His opponent doubts whether the constitutions resulting from this example and expressing the growing political prudence of both subjects and sovereigns will be stable enough, being the result of coincidence (cf. 425). Reinhold admits that in many cases the higher classes (the nobility and (higher) clergy, that is) in society have assumed power at the expense of the third estate, yet also points out that the lack of power of that estate is more due to ``its natural immaturity, than to the national constitution'' (427). Moreover, historically speaking, aristocratic and clerical rule have been ``the instruments of natural necessity, or rather of Providence reigning through natural necessity, in the education of the middle class,'' as initiating the state of majority in Europe (432). With `middle class' Reinhold refers to the higher social strata of the third estate, that is, to those members of the third estate that are relatively well{-}off and relatively educated (cf. 427). In the following Reinhold uses both `dritter Stand' and `Mittelstand' to refer to this class of people. With a brief historical excursion Reinhold seeks to defend his claim that aristocratic and clerical rule was not at first unjust. Rather, it only became unjust after the novelties like the printing press, gunpowder and the discovery of the new world had \textit{de facto} shifted the balance of power between the three estates (cf. 429).\footnote{ Note that Reinhold still uses the same kind of historical argumentation as in his pre{-}Kantian writings, arguing that certain practices may have been very rational at the time of their institution, yet may loose their rationality and legitimacy when historical circumstances change. Cf. Chapter 2, section 2.1.} The third estate, however, can only successfully claim its role in the constitution of the state when it has reached a state of majority, that is, when it is able to have its freedom without oppressing the other classes in turn (cf. 433). The next question is whether the middle class can actually reach this state from its current state of minority. The education provided by the two privileged classes is no longer adequate to the current state of society (cf. 435). Reinhold's answer to this is that as long as major parts of the third estate are by no means sufficiently mature, the third estate as a whole is rightfully limited in its actions by nobility and clergy (cf. 437).\footnote{ Here we encounter Reinhold's elitism, which may very well stem from his Masonic and Illuminatist background. Cf. Chapter 2, section 5.} It is only from the opposition of the extremes (old{-}fashioned prejudices of clergy and nobility versus the new lust for wealth and \textit{Freigeisterei}) that the true middle road becomes possible (cf. 439). The coexistence of opposing standpoints will force those who think for themselves (\textit{Selbstdenker}) to search for the grounds of the competing views, and thus develop their own point of view (cf. 441). Reinhold's opponent is worried that in this way the freedom for the middle class will be limited to those who think for themselves. Although the third estate will always include those who cannot really think for themselves, Reinhold replies, ``at a certain level of scientific and moral culture'' they will follow the cosmopolitan ideas of their intellectual leadership so that all will `come of age' (442). Notwithstanding the further doubts of his opponent regarding the feasibility of this task for the \textit{Selbstdenker} Reinhold remains optimistic and says they must keep trying. The fact that philosophers up to this point have not been able to come up with properly determinate concepts of morality is due to their lack of ``determinate concepts of reason and its relation to sensibility'' (469). The harmony and unity brought forward once the scientific philosophy has been established is like the unanimity regarding logic or mathematics, not varying with character, education or culture (cf. 472{-}473). Although this philosophical paradise may not be a reality yet, Reinhold concludes the second volume of his \textit{Briefe \"{u}ber die Kantische Philosophie} with the hope that one day it will. 


\section{Evaluation}


Now that we have seen how Reinhold's practical philosophy started to take shape after the publication of the \textit{Versuch}, in separate articles and later in the second volume of \textit{Briefe \"{u}ber die Kantische Philosophie}, we are in a position to evaluate the development of his use of the term `practical reason' in these works. It is clear that Reinhold's understanding of this term underwent significant changes in the first few years after the publication of the \textit{Versuch}. In the `Grundlinien', which we took as our starting point for this chapter, Reinhold implicitly identified `practical reason' with the pure will in his somewhat messy attempt to reassert the Kantian line on the practicality of pure reason in opposition to Rehberg. As indicated above, there is some tension between this attempt and Reinhold's statements on human freedom in the First Book of the \textit{Versuch}. There, he stressed a feature of freedom, namely the possibility to choose either to submit to the law of desire or to follow the law of reason, that does not easily fit with the thought expressed in the `Grundlinien' that absolute freedom is to be located in the spontaneity of reason in giving and following its own law. By attributing the spontaneity to give the moral law to the same capacity as the spontaneity that realizes moral action, that is, practical reason, Reinhold fails to present a clear picture regarding his location of human freedom. The picture appears to be something like the following. Reason is free, in both its theoretical and practical expressions, which are independent from the forms of sensibility. Yet in its theoretical expression reason does depend on something outside reason, namely the forms of the understanding, whereas practical reason depends only on itself. Since the forms of the understanding are themselves the products of spontaneity, reason's dependence on them does not abolish its freedom, it merely qualifies it. The dependence of theoretical reason also applies to the expression of reason with regard to happiness; although happiness is a non{-}sensible ideal because it entails infinity, it depends on the modifications of drive through, again, the forms of the understanding. The activity of practical reason is independent and consists in realizing the rational form of lawfulness. From these different expressions of reason arise different laws for action. Practical reason supplies the moral law, sanctioned by itself alone, while the prudential laws of theoretical reason are sanctioned by happiness or pleasure, that is by an external source. Within this structure Reinhold's implicit identification of practical reason and the (pure) will leads to tensions. In line with his demand made in the First Book of the \textit{Versuch}, Reinhold in the `Grundlinien' claims that the will (in general) is free in so far as it chooses between the laws for action that are available to it. It then acts absolutely free when it chooses to follow the moral law, for this is a law that is of its own provenance, while it acts only comparatively free when it chooses freely to subject itself to the external law of desire. In the first case it acts as pure will, in the second as empirical will. The distinction between these two kinds of will implies that there are different authorities making the decision depending on the outcome. When the decision is a moral one, practical reason has been at work; when it is immoral, subordinating the moral law under the law of desire, practical reason, always striving to realize the moral law has apparently failed. It is totally unclear how this situation is compatible with Reinhold's starting point that freedom consists in the capacity to choose freely between two competing laws, especially when this capacity is situated in the will, which in this context does not appear to be a united capacity at all. 

 The source of the confusion is Reinhold's effort in the `Grundlinien' to strongly link his preliminary thoughts on the faculty of desire and the will to his theory of the faculty of representation. This means that he can only present the highest degree of spontaneity (needed for moral freedom) as a form of reason. Remember that section 86, of which the `Grundlinien' are a part, concerns the causality of reason as an absolute causality. This in turn implies that he cannot differentiate between different forms of spontaneity, for instance, between independently establishing a law and choosing freely between laws. The focus on reason means that the will is basically a form of reason, yet Reinhold's starting point as expressed in the First Book of the \textit{Versuch} demands that there is a capacity to choose between laws for action. By attributing this capacity to `the will' Reinhold suggests that the will is an independent capacity, but the set{-}up of the `Grundlinien', in which the will is a form of reason, does not leave much room for such a move. Within the framework of the `Grundlinien' the spontaneity expressing itself as reason is the spontaneity of the faculty of representation, that is, the spontaneity that unifies a given matter to a representation with a specific form. Since the framework requires that form and material go together, Reinhold's effort to understand practical reason as the realization of the pure form of reason is unfeasible within that framework. The text of section 86 preceding the `Grundlinien' makes it clear that dependence on any material, even if it is the result of the spontaneity of the faculty of representation itself {--} as is the case with the activity of reason in thinking {--} will render the activity of reason only comparatively free. It is clear that for the establishment of absolute freedom the framework of the faculty of representation is a hindrance. 

 The development of Reinhold's practical philosophy from the \textit{Versuch }onwards amounts to a gradual solving of the tension described above, that is to say, Reinhold is looking for ways to do justice to his starting point in a way that the `Grundlinien' did not allow. The first and arguably most important step taken in this process is the emancipation of the issue from the theory of the faculty of representation, where it did not appear to belong in the first place. The theory of the faculty of representation considers the premises for the Kantian theory of cognition. One of the results of this theory was that the absolute subject, about which nothing can be known, must be thought as an absolute cause, that is, as free. Reinhold's preliminary thoughts on practical philosophy as he expressed them in de `Grundlinien' are not related to this core of the \textit{Versuch}. We have established above that his attempt to forge a strong link between the theory of the faculty of representation and practical philosophy complicates the issue rather than solving it. In the essays that Reinhold published in 1791, he clearly chose not to link his thoughts on practical philosophy to the faculty of representation. Instead of reason as the seat of spontaneity, we find personhood as a central notion. In his essay `Ueber die Grundwahrheit der Moralit\"{a}t und ihr[ ] Verh\"{a}ltni\ss{} zur Grundwahrheit der Religion', discussed in section 1.2, Reinhold takes the first steps in that direction. Given the subject of this article, the relation between the existence of God and the freedom of the will, Reinhold may still be working from the context of dealing with Rehberg's review of Kant's second \textit{Critique}, yet his account differs from the `Grundlinien' in that he describes the will as depending on the spontaneity of the mind, which in turn is that upon which our personhood depends (cf. section 1.2). This may still be compatible with an attempt to link practical philosophy to the theory of the faculty of representation, since it is not clear how the spontaneity here relates to the spontaneity identified in the \textit{Versuch}. Reinhold's identification of this spontaneity with reason indicates that he is indeed still working within that framework. Yet he no longer presents an explicit connection between the faculty of representation on the one hand and the faculty of desire on the other. What he does present explicitly, however, is the identification of practical reason and willing. The distinction between theoretical and practical reason is made by describing theoretical reason as thinking reason and practical reason as willing reason, that is, reason involved in action. This involvement is in turn described as prescribing a law for itself, which it can either follow for its own sake or for the sake of something else. It is clear that this account is close to Reinhold's preliminary thoughts on the subject in the \textit{Versuch}. The lack of an explicit link with the faculty of representation does not solve the problems that arise from the combination of identifying the will and practical reason and claiming freedom to willingly choose between two different laws. As in the \textit{Versuch}, the will is not an independent capacity, but rather a form of reason. 

 This is also the case in the other essay from the spring of 1791 discussed in section 1.2, on the concept of natural right. Again Reinhold identifies the activity of reason in regulating desire with practical reason, which gives and follows its own law. This time, the discussion is undertaken in an attempt to solve the philosophical controversies regarding the concept of natural right. The misunderstandings and lack of clarity regarding this concept lead to disputes among philosophers that can be solved by means of the Kantian philosophy. Reinhold aims to determine the foundation of morality enabling him to establish natural right as well. This context is no longer reminiscent of the theory of the faculty of representation at all, but rather of the `Briefe \"{u}ber die Kantische Philosophie' in the \textit{Merkur}. Instead of linking his thoughts on practical philosophy to his theory of the faculty of representation, Reinhold appears to start anew and tries to establish practical philosophy upon external, rather than internal grounds. The same strategy is followed in the later essay on positive right, which parallels that on natural right, in that it starts from the disputes among philosophers and jurists on the status of positive right. The disputes are presented as the result of a common misunderstanding of the structure of the human mind, which can be solved by the Kantian philosophy. Still, in the autumn of 1791, Reinhold clearly identifies practical reason with willing and, as we have seen, still generates a substantive amount of confusion by identifying as activities of reason both the giving and the following of laws of both practical and theoretical reason. The issue of freedom appears to have shifted to the background to make place for the determination of the concepts of right and duty. We have seen that the question as to which capacity carries out which limitation of the selfish drive does not have Reinhold's attention at this time. Thus, there is less tension between claiming a freedom of the will and failing to have a unified picture of this will, but the confusion regarding the many activities of reason remains. 

 Yet, as Lazzari has shown in his reconstruction of the production of the essay `Beytrag zur genaueren Bestimmung der Grundbegriffe der Moral und des Naturrechts' Reinhold almost immediately upon finishing the essay on positive right started to rework the material into what would become the `Beytrag'{-}essay.\footnote{ Cf. Lazzari, \textit{Das Eine, was der Menschheit Noth ist}, Chapter 5, section 2, 187{-}198.} The shift away from the theory of the faculty of representation is here completed, since the unselfish drive, presenting the demand of the moral law is related to the spontaneity of the person, instead of reason. Reinhold's letting go of reason as \textit{the} seat of human spontaneity allows him to differentiate between various spontaneous capacities with more clarity. Now the will can be introduced as an independent capacity and as a necessary ingredient of the determination of the concepts of right and duty. The following and giving of prescriptions for action is no longer the sole task of reason, since the decision to follow the rules of reason or not is located in the will, expressly distinguished from reason. On the basis of this distinction in activity Reinhold can also distinguish between different kinds of freedom. The freedom of complying with the moral law or not is situated in the will, while the freedom that consists in total independence from external factors is attributed to practical reason, as it establishes the moral law. In this manner Reinhold can finally do justice to his starting point as presented in the First Book of the \textit{Versuch}. The thought that there must be freedom to either follow the moral law or not, requires acknowledging the will as a capacity that independently decides which prescriptions it follows in a given case. 

 This approach was elaborated and consolidated in the second volume of Reinhold's \textit{Briefe \"{u}ber die Kantische Philosophie}. His reworking of the earlier articles on natural and positive right, with their focus on correcting the current misunderstandings regarding the philosophy of right, matches the outlook of the \textit{Briefe} project in general. It is telling in this respect that only the second halves of those articles were changed significantly, that is, the parts having a direct bearing on the issue of the activities of reason and the will. The first halves, introducing the subject in relation to current philosophical debates remained essentially the same. This means that, even though Reinhold's thoughts on the core subject changed significantly, the format of presenting the Kantian philosophy as the solution to current debates remains applicable. The first `Brief' affirms that the strategy of the first series of `Briefe' was not only of use to introduce a wider audience to the Kantian philosophy, but is also instrumental in solving debates occasioned by the Kantian philosophy. With regard to the aim and structure of \textit{Briefe} \textit{II} it is clear that Reinhold no longer looks at practical philosophy on a Kantian basis as something to be appended to the theory of the faculty of representation. Instead, the second volume of \textit{Briefe} is structured in a way that parallels the \textit{Versuch}. First, it is claimed that the Kantian philosophy, although universally discredited, in fact contains the solution to the current crisis of philosophy, which parallels the \textit{Bisherige Schicksale}. The following four `Briefe', introducing the problem, match the First Book of the \textit{Versuch}, claiming that the solution to the problems of philosophy has only become possible with Kant, but that, inevitably, the concepts he used were not yet thoroughly determined; in the case of \textit{Briefe II} this concerns the concept of the will, in the case of the \textit{Versuch} it concerns the concept `representation'. The core of \textit{Briefe} \textit{II} is formed by the sixth, seventh and eighth \textbf{`}Briefe' in which Reinhold seeks to determine the undeveloped Kantian premise, the concept of the will. Although this is done with somewhat less rigor than the determination of `faculty of representation' in the \textit{Versuch}, the starting point is still somewhat similar. In both cases the starting point is a fact of consciousness, although the \textit{Versuch} does not employ that term for the insight into the nature of representation. In the \textit{Briefe}, the freedom that is involved in morality, the same freedom to choose for or against the moral law is presented as a fact of consciousness. As in the \textit{Versuch}, parts of Kantian doctrine, such as the categorical imperative, are established upon this fact of consciousness. The structure of \textit{Briefe II} differs from that of the \textit{Versuch}, because in the later `Briefe' Reinhold expressly addresses the beneficial consequences of his newly established doctrine at length: the \textit{Versuch}, however, does contain some passages discussing the differences between, for instance, the conception of the absolute subject as Reinhold had established it and that of other major philosophical parties.\footnote{ Cf. \textit{Versuch}, 546{-}556. } 

 With this parallel treatment in \textit{Briefe II }the practical philosophy is definitely emancipated from the theory of the faculty of representation. There are, however, some more points on which the second volume of \textit{Briefe} can be regarded as the conclusion of a development. Although Reinhold had already separated the will from practical reason in the `Beytrag'{-}essay he was still consolidating this position as he set himself to adapting the earlier articles for inclusion in \textit{Briefe II}.\footnote{ That is, articles that would become the first through sixth `Briefe', the tenth `Brief' and the twelfth `Brief' in \textit{Briefe II}. See footnotes \ref{footnote:_Ref232222730}, \ref{footnote:_Ref232226383}\ref{footnote:_Ref234654472}, \ref{footnote:_Ref233181041}.} In the third `Brief' he added a fresh account of the nature of reason and the specific tasks of practical and theoretical reason in order to prepare the reader for his moral theory. We have seen that he now presents reason in general as a capacity of giving prescriptions and that practical and theoretical reason differ because theoretical reason depends on something given and practical reason does not. 

In section 86 of the \textit{Versuch} including the `Grundlinien', Reinhold stressed the independence from receptivity for reason in general, which made it harder for him to make a clear distinction between theoretical and practical reason and the kind of spontaneity involved in them. Thus, he ended up stating that theoretical reason was comparatively free; free, because it was not bound to the forms of sensibility, yet only comparatively, because it was still bound to the forms of the understanding, given in the faculty of representation. With regard to human action, theoretical reason was also regarded as connected to those forms, since they modified the drive that was extended by reason into infinity, yielding the drive for happiness. Practical reason, on the other hand, only focused on realizing its own form, although Reinhold did not explain whether this realization consisted in the establishment of the moral law, or rather in moral action following the moral law. Although the main distinction between theoretical reason as still dependent on something given and practical reason as totally independent appears to remain in place, there are important shifts. First of all, in the \textit{Versuch}, reason in general is associated with rising above nature, the world of given, sensible material. In \textit{Briefe II} the laws of prudence modified by the forms of the understanding are expressly understood as natural laws. This means that Reinhold's understanding of the term natural has changed in order to distinguish more clearly between theoretical and practical reason. The loss of the importance of spontaneity is related to the gradual abandonment of the framework of the theory of the faculty of representation. The description of freedom as the causality of reason, employed by Reinhold in section 86 of the \textit{Versuch} and criticized by him as too wide in \textit{Briefe}, illustrates the growing need for him to differentiate between theoretical and practical reason more clearly. Both this differentiation and the separation of the will and practical reason reflect Reinhold's growing awareness that just one spontaneous capacity will not suffice to ground his starting point for moral philosophy, namely that humans must be free in a way that allows them to choose freely between opposing laws. By distinguishing more clearly between practical and theoretical reason, downplaying the importance of the spontaneity of the latter, Reinhold can more easily oppose the different laws. By distinguishing more clearly between the will and practical reason, making the will a separate fundamental capacity of the human mind, he can explain why the result of our highest form of spontaneity, the moral law, does not invariably bring about moral action. As a different capacity, independent of the demands of both the selfish and the unselfish drives, the will is the determining ground of action. Here we see the main addition in \textit{Briefe II}, if we compare it to the `Beytrag'{-}essay in which Reinhold first separated the will and practical reason. In the \textit{Briefe} he presents a fuller picture of the structure of the human mind that enables us to see better how, after the `Grundlinien', his views of reason have changed together with his practical philosophy. 

 Another issue on which Reinhold's account in the \textit{Briefe} differs from the `Beytrag'{-}essay is his view on the distinction between pure will and empirical will. The essay still used it to differentiate between the will determined by the moral law and the will determined by desire. In the \textit{Briefe}, however, Reinhold rejects the distinction, with reference to Schmid's understanding of it. Instead, he stresses the involvement of both sensibility and reason in every decision of the will, since it is the independence of the will from both that enables to choose any occasioning ground as a determining ground. Schmid's understanding of the earlier formulation would, according to Reinhold amount to moral determinism. 

 Regarding the role of practical reason in the second volume of the \textit{Briefe}, it is clear that the way Reinhold conceives of practical reason here is the result of a process starting in the `Grundlinien' and for the time being ending here. At the beginning of this process, Reinhold, on the basis of his theory of the faculty of representation, counted on the spontaneous nature of reason as such to establish human freedom. We have seen in the `Grundlinien' that at first Reinhold had no clear idea where exactly to situate this freedom. In the 1791 articles he explicitly located it in the will, which was identified with practical reason. His account was still confused with regard to the different activities of reason and how the will related to them. A clear separation of the spontaneity exercised by reason and the spontaneity exercised by the will was introduced in the `Beytrag'{-}essay, where the will was established as a separate fundamental capacity of the human mind, irreducible to other capacities. This finally enabled Reinhold in the second volume of his \textit{Briefe \"{u}ber die Kantische Philosophie} to treat the will, given as a fact of consciousness, in a similar fashion as `representation' in the \textit{Versuch}.\footnote{ Bondeli, introduction to \textit{Briefe II}, LXX{-}LXXI.} Between the `Grundlinien', in which practical reason was both the giver and executor of the moral law and the supreme seat of human spontaneity, and \textit{Briefe} \textit{II}, in which role of practical reason was reduced to that of involuntary giver of the moral law, lies a big difference. Reinhold started out to reassert the Kantian line on the causality of reason, yet this did not square with his starting point regarding human freedom, which required the independence from both reason and sensibility. Reinhold's development after the `Grundlinien' is to be regarded as the effort to ground this starting point in a Kantianizing framework. It is no coincidence that the final solution consists in reducing the role of practical reason to establishing the moral law, while the will takes prominence as a fundamental capacity that rises above and unifies man's spontaneous and receptive capacities. This was Reinhold's ultimate aim in his works on Enlightenment, his first `Briefe' and the \textit{Versuch }as well. 

