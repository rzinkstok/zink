
\chapter{Bibliography}



\section{Works by Karl Leonhard Reinhold}


Chronologically 

Review of \textit{Glaubensbekentni\ss{} und Lehre der \"{a}chtdenkende Katholiken, den Herrn Aberglauben und Mi\ss{}br\"{a}uche Vertheidigern: Merz, Weissenbach, Jost, Fast, Mazzioli, Pochlin, Uhazzi etc. gewidmet, }by Johann Friedrich Koch. \textit{Realzeitung}, November 12, 1782.

Review of \textit{Lehr{-} und Gebethbuch f\"{u}r die unm\"{u}ndige Jugend}, ed. K.H. Seibt. \textit{Realzeitung}, December 3, 1782.

Review of \textit{Die Herzjesuandacht nach theologischen und historischen Gr\"{u}nden gepr\"{u}ft}, by Karl Joseph Huber. \textit{Realzeitung}, January 1, 1783. 

Review of \textit{Abhandlung von der Einf\"{u}hrung der Volkssprache in den \"{o}ffentlichen Gottesdienst}, by J. N. J. Pehem. \textit{Realzeitung}, March 4, 1783.

Review of \textit{Abhandlung \"{u}ber den Werth der M\"{o}nchsprofessen, zu Gunsten der Vernunft wider das Vorurtheil, samt noch einer kurzen Abhandlung \"{u}ber die ehemahligen Gebr\"{a}uche der Juden}, transl. from French by Joh. F\"{u}rchtegott Stambach. \textit{Realzeitung}, March 11, 1783.

Review of \textit{Rede von dem erlaubten und n\"{o}thigen Bande der freyen Religionsduldung mit der Freyheit der Handlung}, by Ignaz von Faber. \textit{Realzeitung}, April 8, 1783.

Review of \textit{Hat dich keiner verdammt? eine frische Anfrage f\"{u}r die Langeweile bey den Klostergel\"{u}bden, und dem C\"{o}libate, zu gleich ein Antwort auf die kurze Erinnerung eines Ungenannten wegen der Ordensgel\"{u}bde}, by Karl Elexia. \textit{Realzeitung}, May 6, 1783.

`\"{U}ber die Kunst des Lebens zu gen\"{u}ssen' (June 1783). Ms. Haus{-}, Hof{-}, und Staatsarchiv Wien, Vertrauliche Akten, Kart. 73, fol. 74{-}75.

Review of \textit{Vollkommene Abfertigung des Freund Werklins mit seinem vollkommenen Widerlegungsschreiben}, by Gottlieb Geisttrich. \textit{Realzeitung}, August 12, 1783.

`Der Wehrt einer Gesellschaft h\"{a}ngt von der Beschaffenheit ihrer Glieder ab' (September 1783). Ms. Haus{-}, Hof{-} und Staatsarchiv Wien, Vertrauliche Akten, Kart. 73, fol. 64{-}68.

Review of \textit{Ideen zur Philosophie der Geschichte der Menschheit}, by Johann Gottfried Herder. \textit{Anzeiger des Teutschen Merkur}, June 1784, LXXXI{-}LXXXIX.

Review of \textit{Iohannis Physiophili Specimen Monachologiae}, by Ignaz von Born. \textit{Anzeiger des Teutschen Merkur}, June 1784, XCII{-}XCIV.

`Die Wissenschaften vor und nach ihrer Sekularisation. Ein historisches Gem\"{a}hlde.' \textit{Der Teutsche Merkur}, July 1784, 35{-}43.

`Gedanken \"{u}ber Aufkl\"{a}rung.' \textit{Der Teutsche Merkur}, July 1784, 3{-}22; `Fortsetzung.' August 1784, 122{-}133; `Fortsetzung.' September 1784, 232{-}245.

Review of \textit{Ueber die Einsamkeit}, by J.G. Zimmermann. \textit{Anzeiger des Teutschen Merkur}, August 1784, CXIII.

`Ueber die neuesten patriotischen Lieblingstr\"{a}ume in Teutschland. Aus Veranlassung des 3. und 4. Bandes von Hrn. Nicolai's Reisebeschreibung.' \textit{Der Teutsche Merkur}, August 1784, 171{-}186; `Fortsetzung.' September 1784, 246{-}264.

`Ueber den Hang zum Wunderbaren.' \textit{Journal f\"{u}r Freymaurer} 1784 III: 123{-}138.

`M\"{o}nchthum und Maurerey. Eine Rede von Br. R\textasteriskcentered \textasteriskcentered .' \textit{Journal f\"{u}r Freymaurer} 1784 IV: 167{-}188.

Review of \textit{Briefe \"{u}ber die Schweiz, 1. und 2. Theil}, by C. Meiners. \textit{Anzeiger des Teutschen Merkur}, December 1784, CLXXVII.

Review of \textit{Oeuvres de Valentin Jamerai Duval, prec\'{e}d\'{e}s des Memoires sur sa Vie}. \textit{Anzeiger des Teutschen Merkur}, December 1784, CLXXVIII.

`Schreiben des Pfarrers zu \textasteriskcentered \textasteriskcentered \textasteriskcentered  an den H. des T. M. Ueber eine Recension von Herders Ideen zur Philosophie der Menschheit.' \textit{Der Teutsche Merkur}, February 1785, 148{-}174.

`Berichtigungen und Anmerkungen \"{u}ber eine Stelle aus der Brosch\"{u}re Faustin, oder das philosophischen Jahrhundert. Zweytes B\"{a}ndchen. S. 83.' \textit{Der Teutsche Merkur}, March 1785, 267{-}277.

\textit{Herzenserleichterung zweyer Menschenfreunde, in vertraulichen Briefen \"{u}ber Johann Caspar Lavaters Glaubensbekenntnis}. Frankfurt and Leipzig: 1785.

`Ueber die kabirischen Mysterien. von Br. R\textasteriskcentered \textasteriskcentered .' \textit{Journal f\"{u}r Freymaurer} 1785 III: 5{-}48.

`Ueber die wissenschaftliche Maurerey.' \textit{Journal f\"{u}r Freymaurer} 1785 III: 49{-}78.

`Ueber die Mysterien der alten Hebr\"{a}er.' \textit{Journal f\"{u}r Freymaurer} 1786 I: 5{-}79.

`Ehrenrettung der Reformation gegen zwey Kapitel in des k.k. Hofraths und Archivars Hrn. M.I. Schmidts Geschichte des Teutschen.' \textit{Der Teutsche Merkur}, February 1786, 116{-}142; `Fortsetzung.' March 1786, 193{-}228; `Beschlu\ss{}.' April 1786, 42{-}80.

`Revision des Buches: Enth\"{u}llung des Systemes der Weltb\"{u}rger{-}Republik.' \textit{Der Teutsche Merkur}, May 1786, 176{-}190.

`Skizze einer Theogonie des blinden Glaubens.' \textit{Der Teutsche Merkur}, June 1786, 229{-}242.

`Auszug einiger neuern Thatsachen aus H. Nikolais Untersuchung der Beschuldigungen des Hrrn. Prof. Garve u.s.w.' \textit{Der Teutsche Merkur}, June 1786, 270{-}280. 

`Ueber die gr\"{o}\ss{}ern Mysterien der Hebr\"{a}er.' \textit{Journal f\"{u}r Freymaurer} 1786 III: 5{-}98.

`Briefe \"{u}ber die Kantische Philosophie: Erster Brief: Bed\"{u}rfni\ss{} einer Kritik der Vernunft.' \textit{Der Teutsche Merkur}, August 1786, 99{-}127.

`Zweyter Brief: Das Resultat der Kantische Philosophie, \"{u}ber die Frage vom Daseyn Gottes.' \textit{Der Teutsche Merkur}, August 1786, 127{-}141.

`Briefe \"{u}ber die Kantische Philosophie: Dritter Brief: Das Resultat der Kritik der Vernunft \"{u}ber den nothwendigen Zusammenhang zwischen Moral und Religion.'\textit{Der Teutsche Merkur}, January 1787, 3{-}39.

`Briefe \"{u}ber die Kantische Philosophie: Vierter Brief: Ueber die Elemente, und den bisherigen Gang der Ueberzeugung von den Grundwahrheiten der Religion.'\textit{Der Teutsche Merkur}, February 1787, 117{-}142. 

`Briefe \"{u}ber die Kantische Philosophie: F\"{u}nfter Brief: Das Resultat der Kritik der Vernunft \"{u}ber das zuk\"{u}nftige Leben.'\textit{Der Teutsche Merkur}, May 1787, 167{-}185.

`Briefe \"{u}ber die Kantische Philosophie: Sechster Brief: Fortsetzung des vorigen. Vereinigtes Interesse der Religion und Moral bey der Hinwegr\"{a}umung des metaphysischen Erkentni\ss{}grundes f\"{u}r das zuk\"{u}nftige Leben.' \textit{Der Teutsche Merkur}, July 1787, 67{-}88.

`Briefe \"{u}ber die Kantische Philosophie: Siebenter Brief: Skizze einer Geschichte der phychologischen [sic] Vernunftbegriffes der einfachen denkenden Substanz.' \textit{Der Teutsche Merkur}, August 1787, 142{-}165.

`Briefe \"{u}ber die Kantische Philosophie: Achter Brief: Fortsetzung des vorigen. {--} Hauptschl\"{u}ssel zur rationalen Psychologie der Griechen.' \textit{Der Teutsche Merkur}, September 1787, 247{-}278. 

\textit{Die Hebr\"{a}ischen Mysterien oder die \"{a}lteste religi\"{o}se Freymaurerey}. Leipzig: G\"{o}schen, 1788.

Review of \textit{Ueber das Verh\"{a}ltni\ss{} der Metaphysik zu der Religion}, by August Wilhelm Rehberg. \textit{Allgemeine Literatur{-}Zeitung}, June 26 (Nr 153b), 1788, 689{-}696.

`Neue Entdeck.' in \textit{Allgemeine Literatur{-}Zeitung}, September 25 (nr. 231a), 1788, 831{-}832.

`Ueber die Natur des Vergn\"{u}gens.' \textit{Der Teutsche Merkur} October 1788, 61{-}79; `Fortgesetzt von S. 79 des vor. Monatsst\"{u}cks.' November 1788, 144{-}167; `S. T. Merkur 1788 November S. 144.' January 1789, 37{-}52. 

`Ueber das bisherige Schicksal der Kantischen Philosophie.' \textit{Der Teutsche Merkur}, April 1789, 3{-} 37; `Beschlu\ss{}.' May 1789, 113{-} 135.

\textit{Ueber die bisherigen Schicksale der Kantischen Philosophie}. Jena: Mauke, 1789.

`Allgemeiner Gesichtspunkt einer bevorstehenden Reformation in der Philosophie.' \textit{Der Teutsche Merkur}, June 1789, 243{-}274; `Fortsetzung und Beschlu\ss{}.' July 1789, 75{-}99.

`Von welchem Skeptizismus l\"{a}\ss{}t sich eine Reformation der Philosophie hoffen?' \textit{Berlinische Monatsschrift}, July 1789, 49{-}73.

`Wie ist Reformazion der Philosophie m\"{o}glich?' \textit{Neues Deutsches Museum} July 1789, 31{-}47; `Fortsezung.' August 1789, 204{-}226; `Beschlu\ss{}.' September 1789, 284{-}304.

 `Fragmente \"{u}ber das bisher allgemein verkannte Vorstellungs{-}verm\"{o}gen.'\textit{Der Teutsche Merkur}, October 1789, 3{-}22.

\textit{Versuch einer neuen Theorie des menschlichen Vorstellungsverm\"{o}gen}. Prague and Jena: Widtmann and Mauke, 1789. [Reprint: Darmstadt: Wissenschaftliche Buchgesellschaft, 1963]

`Erkl\"{a}rung' in `Intelligenzblatt' \textit{Allgemeine Literatur{-}Zeitung}, December 12 (Nr. 137), 1789, 1138{-}1140.

\textit{Briefe \"{u}ber die Kantische Philosophie von Hn. Karl Leonhard Reinhold Rath, und Professor der Philosophie zu Jena. Zum Gebrauch und Nuzen f\"{u}r Freunde der Kantischen Philosophie gesammelt}. Mannheim: Bender, 1789. [Pirate edition]

\textit{Auswahl der besten Aufs\"{a}zze \"{u}ber die Kantische Philosophie}. Frankfurt and Leipzig; de facto Marburg: Krieger, 1790). [Pirate edition]

\textit{Briefe \"{u}ber die Kantische Philosophie. Erster Band}. Leipzig: G\"{o}schen, 1790. 

\textit{Beytr\"{a}ge zur Berichtigung bisheriger Mi\ss{}verst\"{a}ndnisse der Philosophen. Erster Band, Das Fundament der Elementarphilosophie betreffend}. Jena: Mauke, 1790.

\textit{Ueber das Fundament des philosophischen Wissens nebst einigen Erl\"{a}uterungen \"{u}ber die Theorie des Vorstellungsverm\"{o}gens}. Jena: Mauke, 1791.

`Ehrenrettung der neuesten Philosophie.' \textit{Der neue Teutsche Merkur}, January 1791, 81{-}112.

`Ueber die Grundwahrheit der Moralit\"{a}t und ihr[ ] Verh\"{a}ltni\ss{} zur Grundwahrheit der Religion.'\textit{ Der neue Teutsche Merkur}, March 1791, 225{-}262.

`Ehrenrettung des Naturrechts.' \textit{Der neue Teutsche Merkur}, April 1791, 337{-}382.

`Ehrenrettung des positiven Rechtes.' \textit{Der neue Teutsche Merkur}, September 1791, 3{-}40; `Fortsetzung und Beschlu\ss{}.' November 1791, 278{-}311. 

`Die drey St\"{a}nde: Ein Dialog.' \textit{Der neue Teutsche Merkur}, March 1792, 217{-}241.

`Die Weltb\"{u}rger: Zur Fortsetzung des Dialogs, die drey St\"{a}nde, im vorigen Monatsst\"{u}ck.' \textit{Der neue Teutsche Merkur}, April 1792, 340{-}379.

`Beytrag zur genaueren Bestimmung der Grundbegriffe der Moral und des Naturrechts: Als Beylage zu dem Dialog der Weltb\"{u}rger.' \textit{Der neue Teutsche Merkur}, June 1792, 105{-}139.

\textit{Briefe \"{u}ber die Kantische Philosophie. Zweyter Band}. Leipzig: G\"{o}schen, 1792. 

`Ueber den philosophischen Skepticismus' in \textit{David Humes Untersuchung \"{u}ber den menschlichen Verstand} neu \"{u}bersetzt von M. W. G. Tennemann, i{-}lii. Jena: Verlag der akademischen Buchhandlung, 1793.

\textit{Beytr\"{a}ge zur Berichtigung bisheriger Mi\ss{}verst\"{a}ndnisse der Philosophen. Zweyter Band. die Fundamente des philosophischen Wissens, der Metaphysik, Moral, moralischen Religion und Geschmackslehre betreffend}. Jena: Maukem, 1797. 

\textit{Auswahl vermischter Schriften: Zweyter Theil}. Jena: Mauke, 1797.

\textit{Verhandlungen \"{u}ber die Grundbegriffe und Grunds\"{a}tze der Moralit\"{a}t aus dem Gesichtspunkte des gemeinen und gesunden Verstandes, zum Behuf der Beurtheilung der sittlichen, rechtlichen, politischen und religi\"{o}sen Angelegenheiten}. L\"{u}beck and Leipzig: Bohn, 1798.


\section{Modern editions and translations of works by Reinhold}


Alphabetically by title

\textit{Beitr\"{a}ge zur Berichtigung bisheriger Missverst\"{a}ndnisse der Philosophen. Erster Band, das Fundament der Elementarphilosophie betreffend}. Edited by Faustino Fabbianelli. Hamburg: Meiner, 2003.

\textit{Beitr\"{a}ge zur Berichtigung bisheriger Missverst\"{a}ndnisse der Philosophen. Zweiter Band, die Fundamente des philosophischen Wissens, der Metaphysik, Moral, moralischen Religion und Geschmackslehre betreffend}. Edited by Faustino Fabbianelli. Hamburg: Meiner, 2004.

\textit{Briefe \"{u}ber die Kantische Philosophie von Carl Leonhard Reinhold}. Edited by Raymund Schmidt. Leipzig: Reclam, 1923. 

\textit{Briefe \"{u}ber die Kantische Philosophie.} \textit{Erster} \textit{Band}. Edited by Martin Bondeli. Basel: Schwabe, 2007. Volume 2/1 of Karl Leonhard Reinhold, \textit{Gesammelte Schriften}.

\textit{Briefe \"{u}ber die Kantische Philosophie.} \textit{Zweiter Band}. Edited by Martin Bondeli. Basel: Schwabe, 2008. Volume 2/2 of Karl Leonhard Reinhold, \textit{Gesammelte Schriften}.

`The Foundation of Philosophical Knowledge.' Translated by George di Giovanni. In, \textit{Between Kant and Hegel: Texts in the Development of Post{-}Kantian Idealism}, edited by George di Giovanni and H. S. Harris, 51{-}103. New York: State University of New York, 1985; revised edition: Indianpolis: Hackett, 2000. [Translation of a substantial excerpt from\textit{ Ueber das Fundament des philosophischen Wissens}]

`The Fundamental Concepts and Principles of Ethics: Deliberations of Sound Common Sense, for the Purpose of Evaluating Moral, Rightful, Political and Religious Matters.'  Translated by Sabine Roehr. In Sabine Roehr, \textit{A Primer on German Enlightenment}, 157{-}251. Columbia, University of Missouri Press, 1995. [Translation of the first part of \textit{Verhandlungen \"{u}ber die Grundbegriffe und Grunds\"{a}tze der Moralit\"{a}t}]

\textit{Die Hebr\"{a}ischen Mysterien oder die \"{a}lteste religi\"{o}se Freymaurerey}. Edited and annotated by Jan Assmann. Neckargem\"{u}nd: Edition Mnemosyne, 2001.

\textit{Letters on the Kantian Philosophy}. Edited by Karl Ameriks, translated by James Hebbeler. Cambridge: CUP, 2005. [Translation of the `Briefe \"{u}ber die Kantische Philosophie' as published in \textit{Der Teutsche Merkur} (1786{-}87); including translations of the main additions to these `Briefe' in \textit{Briefe I} (1790)]

`Thoughts on Enlightenment.' Translated by Kevin Paul Geiman. In \textit{What is Enlightenment? Eighteenth{-}Century Answers and Twentieth{-}Century Questions}, edited by James Schmidt, 65{-}77. Berkeley and Los Angeles: University of California Press, 1996. [Translation of the second and third parts of `Gedanken \"{u}ber Aufkl\"{a}rung']

\textit{\"{U}ber das Fundament des philosophischen Wissens/\"{U}ber die M\"{o}glichkeit der Philosophie als strenge Wissenschaft}. Edited by Wolfgang Schrader. Hamburg: Meiner, 1978.

\textit{Versuch einer neuen Theorie des menschlichen Vorstellungsverm\"{o}gens}. Edited by Ernst{-}Otto Onnasch Hamburg: Meiner, forthcoming.


\section{Works by Reinhold's contemporaries}


\ul{Including editions and collections of sources}

\ul{Alphabetically }

Anonymous review of \textit{Ueber die bisherigen Schicksale der kantische Philosophie}, by Karl Leonhard Reinhold. \textit{Allgemeine Literatur{-}Zeitung}, June 23 (Nr. 186), 1789, 273{-}276.

Dobbek, Wilhelm, and G\"{u}nter Arnold, ed. \textit{Johann Gottfried Herder. Briefe. F\"{u}nfter Band, September 1783{-}August 1788}. Weimar: Hermann B\"{o}hlaus Nachfolger, 1979.

Fabbianelli, Faustino ed., \textit{Die zeitgen\"{o}ssischen Rezensionen der Elementarphilosophie K.L. Reinholds}. Hildesheim: Georg Olms Verlag, 2003.

Fabbianelli, Faustino, Eberhard Heller, Kurt Hiller, Reinhard Lauth, Ives Radrizzani, Wolfgang Schrader, ed. \textit{Karl Leonhard Reinhold Korrespondenzausgabe der \"{O}sterreichischen Akademie der Wissenschaften}. Stuttgart{-}Bad Cannstatt: Frommann Holzboog, 1983 {--}. Vol. 1 (1773{-}1788), 1983; Vol. 2 (1788{-}1790), 2008.

di Giovanni, George, and H. S. Harris, ed. \textit{Between Kant and Hegel: Texts in the Development of Post{-}Kantian Idealism} (New York: State University of New York, 1985; revised edition: Indianpolis: Hackett, 2000).

Herder, Johann Gottfried. \textit{Ideen zur Philosophie der Geschichte der Menschheit} I. Riga and Leipzig: Hartknoch, 1784. 

Irmscher, Hans Dietrich, ed. \textit{Immanuel Kant. Aus den Vorlesungen der Jahre 1762 bis 1764. Auf Grund der Nachschriften Johann Gottfried Herders}. Kantstudien Erg\"{a}nzungsheft 88. Cologne: K\"{o}lner Universit\"{a}ts{-}Verlag, 1964. 

Jacobi, Friedrich Heinrich. \textit{\"{U}ber die Lehre des Spinoza in Briefen an den Herrn Moses Mendelssohn}. Breslau: L\"{o}we, 1785.

Kant, Immanuel. `Beantwortung der Frage: Was ist Aufkl\"{a}rung?' \textit{Berlinische Monatsschrift}, December 1784, 481{-}494.

{---}{---}{---}. \textit{Critique of Pure Reason}. Translated and edited by Paul Guyer and Allen W. Wood. Cambridge: CUP, 1998. Series: The Cambridge Edition of the Works of Immanuel Kant. 

{---}{---}{---}. `Erinnerungen des Recensenten des Herderschen Ideen zu einer Philosophie der Geschichte der Menschheit (Nro. 4) und Beil. der Allg. Lit.{-}Zeit.) \"{u}ber ein im Februar des Teutschen Merkur gegen diese Recension gerichtetes Schreiben.' \textit{Allgemeine Literatur{-}Zeitung}, March 1785 (Appendix). 

{---}{---}{---}. \textit{Gesammelte Schriften}. Edited by the Royal Prussian (later German) Academy of Sciences. Berlin: Reimer (later Walter de Gruyter), 1900{-}.

{---}{---}{---}. `Idee zu einer allgemeinen Geschichte in weltb\"{u}rgerlicher Absicht.' \textit{Berlinische Monatsschrift}, November 1784, 385{-}411.

{---}{---}{---}. \textit{Kritik der praktischen Vernunft}. Riga: Hartknoch, 1788 [1787].

{---}{---}{---}. \textit{Kritik der reinen Vernunft}. Riga: Hartknoch, 1781.

{---}{---}{---}. \textit{Practical Philosophy}. Translated and edited by Mary J. Gregor. Cambridge: CUP, 1996. Series: The Cambridge Edition of the Works of Immanuel Kant.

{---}{---}{---}. Review of \textit{Ideen zur Philosophie der Geschichte der Menschheit }by Johann Gottfried Herder. \textit{Allgemeine Literatur{-}Zeitung}, January 6 (nr 4 and supplement), 1785, 17{-}20 and 21{-}22.

{---}{---}{---}. Reviews of J. G. Herder's Ideas for the Philosophy of the History of Humanity, Parts 1 and 2. Translated by Allen W. Wood. In \textit{Anthropology, History and Education}. Edited by G\"{u}nther Z\"{o}ller and Robert B. Louden. Cambrideg: CUP, 2007. Series: The Cambridge Edition of the Works of Immanuel Kant.

{---}{---}{---}. `Was hei\ss{}t: sich im Denken orientiren?' \textit{Berlinische Monatsschrift}, October 1786, 304{-}330.

Kraus, Christian Jakob. Review of \textit{Eleutheriologie }oder\textit{ \"{u}ber Freyheit und Nothwendigkeit}, by Johann August Heinrich Ulrich. \textit{Allgemeine Literatur{-}Zeitung}, April 25 (nr. 100), 1788, 177{-}184.

Landau, Albert ed. \textit{Rezensionen zur Kantischen Philosophie 1781{-}87}. Bebra: Albert Landau Verlag, 1991.

Locke, John. \textit{An Essay concerning Human Understanding}. Edited by Peter. H. Nidditch. Oxford: Clarendon 1975.

Maa\ss{}, Ferdinand. \textit{Der Josephinismus. Quellen zu seiner Geschichte in \"{O}sterreich 1760{-}1790. II. Band. Entfaltung und Krise des Josephinismus 1770{-}1790}. Vienna: Herold Verlag, 1953.

Marcard, Hinrich Matthias. `Ist die Deutsche Nation die erste Nation des Erdbodens?' \textit{Neues Deutsches Museum}, October, 1790, 1015{-}1047.

Malter, Rudolf ed. \textit{Immanuel Kant in Rede und Gespr\"{a}ch}. Hamburg: Felix Meiner Verlag, 1990.

Mendelssohn, Moses. \textit{An die Freunde Lessings. Ein Anhang zu Herrn Jacobis Briefwechsel \"{u}ber die Lehre des Spinoza}. Berlin: Vo\ss{}, 1786.

{---}{---}{---}.\textit{ Morgenstunden oder Vorlesungen \"{u}ber das Daseyn Gottes}. Berlin: Vo\ss{}, 1785.

{---}{---}{---}. `Ueber die Frage: was hei\ss{}t aufkl\"{a}ren?' \textit{Berlinische Monatsschrift}, September 1784, 193{-}200.

Pistorius, Hermann Andreas. Review of \textit{Versuch einer neuen Theorie des menschlichen Vorstellungsverm\"{o}gens}, by Karl Leonhard Reinhold. \textit{Allgemeine Deutsche Bibliothek} 1791 (101), nr. 2, 295{-}318.

Platner, Ernst. \textit{Philosophische Aphorismen nebst einigen Anleitungen zur philosophischen Geschichte}. Leipzig: Schwickert, 1776.

{---}{---}{---}.\textit{ Philosophische Aphorismen nebst einigen Anleitungen zur philosophischen Geschichte. Anderer Theil}. Leipzig: Schwickert, 1782.

{---}{---}{---}. `Versuch \"{u}ber die Einseitigkeit des Stoischen und Epikurischen Systems in der Erkl\"{a}rung vom Ursprunge des Vergn\"{u}gens.' \textit{Neue Bibliothek der sch\"{o}nen Wissenschaften und der freyen K\"{u}nste}, vol. 19, issue 1: 5{-}30.

Rehberg, August Wilhelm. `Antwort' in `Intelligenzblatt', \textit{Allgemeine Literatur{-}Zeitung} January 30 (nr. 15), 1790,118{-}120.

{---}{---}{---}. `Erl\"{a}uterung einiger Schwierigkeiten der nat\"{u}rlichen Theologie' \textit{Der Teutsche Merkur}, September 1788, 215{-}233.

{---}{---}{---}. Review of \textit{Kritik der praktischen Vernunft}, by Immanuel Kant. \textit{Allgemeine Literatur{-}Zeitung} August 6 (nr. 188 a and b), 1788, 345{-}360.

{---}{---}{---}. Review of \textit{Versuch einer neuen Theorie des menschlichen Vorstellungsverm\"{o}gens}, by Karl Leonhard Reinhold.\textit{ Allgemeine Literatur{-}Zeitung}, November 19 (nr. 357), 1789, 414{-}424; continued November 20 (Nr. 358) 425{-}429.

Reinhold, Ernst. \textit{Karl Leonhard Reinhold's Leben und litterarisches Wirken nebst einer Auswahl von Briefen Kant's, Fichte's, Jacobi's und andrer philosophirender Zeitgenossen an ihn}. Jena: Friedrich Frommann, 1825.

Schmid, Carl Christian Erhard. \textit{W\"{o}rterbuch zum leichtern Gebrauch der Kantischen Schriften}. Jena: Cr\"{o}ker, 1788; 2nd edition.

{---}{---}{---}. \textit{Versuch einer Moralphilosophie}. Jena: Cr\"{o}ker, 1790.

Schmidt, James ed. \textit{What is Enlightenment? Eighteenth{-}Century Answers and Twentieth{-}Century Questions}. Berkeley: University of California Press, 1996.

Sch\"{u}tz, Christian Gottfried. Review of \textit{Erl\"{a}uterungen \"{u}ber des Herrn Professor Kant Critik der reinen Vernunft}, by Johann Schultz. \textit{Allgemeine Literatur{-}Zeitung}, July 12 (nr. 162), 1785, 41{-}44; July 14 (nr. 164), 1785, 53{-}56; July 30 (nrs. 178 and 179 and supplement), 1785, 117{-}118, 121{-}128.

{---}{---}{---}. Review of \textit{Morgenstunden oder Vorlesungen \"{u}ber das Daseyn Gottes}, by Moses Mendelssohn. \textit{Allgemeine Literatur{-}Zeitung}, January 2 (nr.1), 1786, 1{-}6; January 9 (nr. 7), 1786, 49{-}56.

{---}{---}{---}. Review of \textit{Grundlegung zur Metaphysik der Sitten}, by Immanuel Kant. \textit{Allgemeine Literatur{-}Zeitung}, April 7 (nr. 80), 1785, 21{-}23.

[Sch\"{u}tz, Christian Gottfried?] Review of \textit{Die Resultate der Jacobischer und Mendelssohnischer Philosophie von einem Freywilligen}, by Thomas Wizenmann. \textit{Allgemeine Literatur{-}Zeitung}, May 26 (nr. 125), 1786, 377{-}384; May 27 (nr. 126), 1786, 385{-}392.

[Sch\"{u}tz, Christian Gottfried?] Review of \textit{\"{U}ber die Lehre des Spinoza in Briefen an den Herrn Moses Mendelssohn}, by Friedrich Heinrich Jacobi. \textit{Allgemeine Literatur{-}Zeitung}, February 11 (nr. 36), 1786, 292{-}296.

Schultz, Johann. \textit{Erl\"{a}uterungen \"{u}ber des Herrn Professor Kant Critik der reinen Vernunft}. K\"{o}nigsberg: Dengel, 1784. 

Ulrich, Johann August Heinrich.\textit{ Eleutheriologie oder \"{u}ber Freyheit und Nothwendigkeit}. Jena: Cr\"{o}ker, 1788.

Weishaupt, Adam. `Anrede an die neu aufzunehmenden Illuminatos dirigentes'1782. Reprinted in Richard van D\"{u}lmen, \textit{Der Geheimbund der Illuminaten. Darstellung. Analyse. Dokumentation}. Stuttgart{-}Bad Cannstatt: Frommann Holzboog, 1975, 166{-}194.

Wizenmann, Thomas.\textit{ Die Resultate der Jacobischer und Mendelssohnischer Philosophie von einem Freywilligen}, Leipzig: G\"{o}schen ,1786.

Z\"{o}llner, Johann Friedrich. `Ist es rathsam, das Eheb\"{u}ndnis nicht ferner durch die Religion zu sanciren?' \textit{Berlinische Monatsschrift}, December 1783, 508{-}517.


\section{Secondary Literature}


\ul{Alphabetically}

Adam, Herbert. \textit{Carl Leonard Reinholds philosophischer Systemwechsel}.Heidelberg: Carl Winters Universit\"{a}tsbuchhandlung, 1930.

Agethen, Manfred. \textit{Geheimbund und Utopie. Illuminaten, Freimaurer und deutsche Sp\"{a}taufkl\"{a}rung}. Munich: R. Oldenbourg, 1987.

Ameriks, Karl. \textit{Kant and the Fate of Autonomy: Problems in the Appropriation of the Critical Philosophy}. Cambridge: CUP 2000.

{---}{---}{---}. \textit{Kant and the Historical Turn: Philosophy as Critical Interpretation}. Oxford: OUP 2006.

{---}{---}{---}. `Reinhold's \textit{first }Letters on Kant.' In \textit{K.L. Reinhold. Am Vorhof des Idealismus}, 13{-}33. Edited by Pierluigi Valenza. Pisa: Istituti Editoriale Poligrafici Internationali, 2006.

{---}{---}{---}. `Reinhold \"{u}ber Systematik, Popularit\"{a}t und die `historische Wende'.' In\textit{ Philosophie ohne Beynamen. System, Freiheit und Geschichte im Denken Karl Leonhard Reinholds}, 303{-}333. Edited by Martin Bondeli and Alessandro Lazzari. Basel: Schwabe, 2004.

Batscha, Zwi. \textit{Karl Leonhard Reinhold. Schriften zur Religionskritik und Aufkl\"{a}rung 1782{-}1784}. Bremen/Wolfenb\"{u}ttel: Jacobi Verlag, 1977.

Beiser, Frederick C. \textit{The Fate of Reason. German Philosophy from Kant to Fichte}. Cambridge, Massachusetts: Harvard University Press, 1987.

B\"{o}hr, Christoph. \textit{Philosophie f\"{u}r die Welt. Die Popularphilosphie der deutschen Sp\"{a}taufkl\"{a}rung im Zeitalter Kants}. Stuttgart{-}Bad Cannstatt: Fromman{-}Holzboog, 2003.

Bondeli, Martin. \textit{Das Anfangsproblem bei Karl Leonhard Reinhold. Eine systematische und entwicklungsgeschichtliche Untersuchung zur Philosophie Reinholds in der Zeit von 1789 bis 1803}. Frankfurt am Main: Klostermann, 1995.

{---}{---}{---}. `Von Herder zu Kant, zwischen Kant und Herder, mit Herder gegen Kant {--} Karl Leonhard Reinhold.' In \textit{Herder und die Philosophie des Deutschen Idealismus}. Edited by Marion Heinz, 203{-}234. Amsterdam: Rodopi, 1998.

Bondeli, Martin, and Alessandro Lazzari ed., \textit{Philosophie ohne Beynamen: System, Freiheit und Geschichte im Denken Karl Leonhard Reinholds}. Basel: Schwabe, 2004.

Bondeli, Martin, and Wolfgang Schrader ed. \textit{Die Philosophie Karl Leonhard Reinholds}. Amsterdam: Rodopi, 2003.

Breazeale, Daniel. `Between Kant and Fichte: Karl Leonhard Reinhold's ``Elementary Philosophy.''' \textit{Review of Metaphysics} 35 (1982): 785{-}821.

van D\"{u}lmen, Richard. \textit{Der Geheimbund der Illuminaten: Darstellung. Analyse. Dokumentation}. Stuttgart{-}Bad Cannstatt: Frommann Holzboog, 1975.

Fabbianelli, Faustino. `Die Theorie der Willensfreiheit in den \quotedblbase Briefen \"{u}ber die Kantische Philosophie`` (1790{-}1792) von Karl Leonhard Reinhold.' \textit{Philosophisches Jahrbuch} 107: 428{-}443. 

Frank, Manfred. \textit{Unendliche Ann\"{a}herung: Die Anf\"{a}nge der philosophischen Fr\"{u}hromantik}. Frankfurt am Main: Suhrkamp, 1997.

Fuchs, Gerhard.\textit{ Karl Leonhard Reinhold {--} Illuminat und Philosoph: Eine Studie \"{u}ber den Zusammenhang seines Engagements als Freimaurer und Illuminat mit seinem Leben und philosophischen Wirken}. Frankfurt am Main: Lang, 1994.

Gerten, Michael `Begehren, Vernunft und freier Wille: Systematische Stellung und Ansatz der praktischen Philosophie bei K. L. Reinhold.' In \textit{Die Philosophie Karl Leonhard Reinholds}, 153{-}189. Edited by Martin Bondeli and Wolfgang Schrader. Amsterdam: Rodopi, 2003. 

di Giovanni, George.\textit{ Freedom and Religion in Kant and His Immediate Successors: The Vocation of Humankind, 1774{-}1800}. Cambridge: CUP, 2005.

di Giovanni, George ed. \textit{Karl Leonhard Reinhold and the Enlightenment}. Dordrecht: Springer forthcoming.

{---}{---}{---}. `Die \textit{Verhandlungen \"{u}ber die Grundbegriffe und Grunds\"{a}tze der Moralit\"{a}t} von 1798 oder Reinhold als Philosoph des gemeinen Verstandes.' In \textit{Philosophie ohne Beynamen: System, Freiheit und Geschichte im Denken Karl Leonhard Reinholds}, 373{-}392. Edited by Martin Bondeli and Alessandro Lazzari. Basel: Schwabe, 2004.

{---}{---}{---}. `Rehberg, Reinhold und C. C. E. Schmid \"{u}ber Kant und die moralische Freiheit.' In \textit{Vernunftkritik und Aufkl\"{a}rung: Studien zu Philosophie Kants und seines Jahrhunderts}, 93{-}113. Edited by Michael Oberhausen. Stuttgart{-}Bad Cannstatt: Frommann{-}Holzboog, 2001. 

Gliwitzky, Hans. `Carl Leonhard Reinholds erster Standpunktwechsel.' In \textit{Philosophie aus einem Prinzip: Karl Leonhard Reinhold. Sieben Beitr\"{a}ge nebst einem Briefekatalog aus Anla\ss{} seines 150. Todestages}, 10{-}85. Edited by Reinhard Lauth.Bonn: Bouvier Verlag Herbert Grundmann, 1974.

Goubet, Jean{-}Fran\c{c}ois. `Der Streit zwischen Reinhold und Schmid \"{u}ber die Moral.' In \textit{Philosophie ohne Beynamen: System, Freiheit und Geschichte im Denken Karl Leonhard Reinholds}, 239{-}250. Edited by Martin Bondeli and Alessandro Lazzari. Basel: Schwabe, 2004.

Heinz, Marion. `Untersuchungen zum Verh\"{a}ltnis von Geschichte und System der Philosophie in Reinholds \textit{Fundamentschrift}.' In \textit{Philosophie ohne Beynamen: System, Freiheit und Geschichte im Denken Karl Leonhard Reinholds}, 334{-}346. Edited by Martin Bondeli and Alessandro Lazzari. Basel: Schwabe, 2004.

Henrich, Dieter. \textit{Grundlegung aus dem Ich: Untersuchunge zur Vorgeschichte des Idealismus. T\"{u}bingen{-}Jena (1790{-}1794)}. Frankfurt am Main: Suhrkamp, 2004.

Jacob, Margaret C. \textit{Living the Enlightenment: Freemasonry and Politics in Eighteenth{-}Century Europe}. Oxford: OUP, 1991. 

Klemmt, Alfred. \textit{Karl Leonhard Reinholds Elementarphilosophie. Eine Studie \"{u}ber den Ursprung des spekulativen deutschen Idealismus}. Hamburg: Meiner, 1958.

Klimmek, Nikolai. \textit{Kants System der transzendentalen Ideen}. Berlin: Walter de Gruyter, 2005. Kantstudien Erganzungsheft 147.

\textit{K.L. Reinhold: Alle soglie dell'Idealismo}. Archivio di Filosofia, volume 73 (2005), issue 1{-}3. Also published as Valenza, Pierluigi ed.\textit{ K.L. Reinhold. Am Vorhof des Idealismus}. Pisa: Istituti Editoriale Poligrafici Internationali, 2006.

Lauth, Reinhard. `Nouvelles R\'{e}cherches sur Reinhold et l'Aufklaerung.' \textit{Archives de Philosophie} 42 (1979): 593{-}629.

{---}{---}{---}, ed. \textit{Philosophie aus einem Prinzip: Karl Leonhard Reinhold. Sieben Beitr\"{a}ge nebst einem Briefekatalog aus Anla\ss{} seines 150. Todestages}. Bonn: Bouvier Verlag Herbert Grundmann, 1974.

Lazzari, Alessandro. \textit{Das Eine, was der Menschheit Noth ist: Einheit und Freiheit in der Philosophie Karl Leonhard Reinholds (1789{-}1792)}. Stuttgart{-}Bad Cannstatt: Frommann{-}Holzboog, 2004.

{---}{---}{---}. `K. L. Reinholds Behandlung der Freiheitsthematik zwischen 1789 und 1792.' In \textit{Die Philosophie Karl Leonhard Reinholds}, 191{-}215. Edited by Martin Bondeli and Wolfgang Schrader. Amsterdam: Rodopi, 2003.

{---}{---}{---}. `K.L. Reinhold: Die Natur des Vergn\"{u}gens und die Grundlegung des Systems.' In \textit{K.L. Reinhold. Am Vorhof des Idealismus}, 213{-}221. Edited by Pierluigi Valenza. Pisa: Istituti Editoriale Poligrafici Internationali, 2006.

{---}{---}{---}. `Zur Genese von Reinholds ``Satz des Bewusstseins''.' In \textit{Philosophie ohne Beynamen: System, Freiheit und Geschichte im Denken Karl Leonhard Reinholds}, 21{-}38. Edited by Martin Bondeli and Alessandro Lazzari. Basel: Schwabe, 2004.

Lohmann, Petra. `Reinholds Philosophie im Spiegel der Kritik von Heydenreich und Fichte.' In \textit{Philosophie ohne Beynamen: System, Freiheit und Geschichte im Denken Karl Leonhard Reinholds}, 82{-}103. Edited by Martin Bondeli and Alessandro Lazzari. Basel: Schwabe, 2004.

Marx, Karianne and Ernst{-}Otto Onnasch, `Zwei Wiener Reden Reinholds. Ein Beitrag zu Reinholds Fr\"{u}hphilosophie.' In \textit{Reinhold and the Enlightenment}. Edited by George di Giovanni. Springer, forthcoming.

Neuper, Horst, ed. \textit{Das Vorlesungsangebot an der Universit\"{a}t Jena von 1749 bis 1854}. Weimar: Verlag und Datenbank f\"{u}r Geisteswissenschaften, 2003.

Oittinen, Vesa. `Ein nordischer Bewu\ss{}tseinsphilosoph: ``Reinholdianische'' Themen bei G. I. Hartman.' In \textit{Die Philosophie Karl Leonhard Reinholds}, 55{-}75. Edited by Martin Bondeli and Wolfgang Schrader. Amsterdam: Rodopi, 2003.

{---}{---}{---}. `\"{U}ber einige ph\"{a}nomenologische Motive in Reinhold's Philosophie.' In \textit{K.L. Reinhold. Am Vorhof des Idealismus}, 115{-}128. Edited by Pierluigi Valenza. Pisa: Istituti Editoriale Poligrafici Internationali, 2006.

Onnasch, Ernst{-}Otto. `Hegel zwischen Fichte und der T\"{u}binger Fichte{-}Kritik.' In \textit{Hegel und die Geschichte der Philosophie}, 171{-}190. Edited by Dietmar Heidemann and Christian Krijnen. Darmstadt: Wissenschaftliche Buchgesellschaft, 2007. 

{---}{---}{---}. `Vor\"{u}berlegungen zur Herleitung der Urteilsformen und Kategorien in Reinholds Theorie des Vorstellungsverm\"{o}gens.' In \textit{K.L. Reinhold. Am Vorhof des Idealismus}, 93{-}113. Edited by Pierluigi Valenza. Pisa: Istituti Editoriale Poligrafici Internationali, 2006. 

Pich\'{e}, Claude. `Fichtes Auseinandersetzung mit Reinhold im Jahre 1793. Die Trieblehre und das Problem der Freiheit.' In \textit{Philosophie ohne Beynamen: System, Freiheit und Geschichte im Denken Karl Leonhard Reinholds}, 251{-}271. Edited by Martin Bondeli and Alessandro Lazzari. Basel: Schwabe, 2004.

Piatigorsky, Alexander. \textit{Who's Afraid of Freemasons? The Phenomenon of Freemasonry}. London: The Harvill Press, 1997.

Roehr, Sabine. \textit{A Primer on German Enlightenment} (Columbia: University of Missouri Press, 1995).

{---}{---}{---}. `Reinholds \textit{Hebr\"{a}ische Mysterien oder die \"{a}lteste religi\"{o}se Freymaurerey}: Eine Apologie des Freimaurertums.' In \textit{Philosophie ohne Beynamen: System, Freiheit und Geschichte im Denken Karl Leonhard Reinholds}, 147{-}165. Edited by Martin Bondeli and Alessandro Lazzari. Basel: Schwabe, 2004.

{---}{---}{---}. `Zum Einflu\ss{} K. L. Reinholds auf Schillers Kant{-}Rezeption.' In \textit{Die Philosophie Karl Leonhard Reinholds}, 105{-}121. Edited by Martin Bondeli and Wolfgang Schrader. Amsterdam: Rodopi, 2003.

Rosenstrauch{-}K\"{o}nigsberg, Edith. \textit{Freimaurerei im josephinischen Wien: Aloys Blumauers Weg vom Jesuiten zum Jakobiner}. Vienna: Wilhelm Braum\"{u}ller, 1975.

R\"{o}ttgers, Kurt. `Die Kritik der reinen Vernunft und K.L. Reinhold. Fallstudie zur Schulbildungsprozessen.' In \textit{Akten des 4. internationalen Kant{-}Kongresses}, Teil II.2, 789{-}804. Edited by Gerhard Funke und Joachim Kopper. Berlin: Walter de Gruyter, 1974.

Sauer, Werner. \textit{\"{O}sterreichische Philosophie},\textit{ Philosophie zwischen Aufkl\"{a}rung und Restauration. Beitr\"{a}ge zur Geschichte des Fr\"{u}hkantianismus in der Donaumonarchie}. Amsterdam: Rodopi 1982.

Selling, Magnus.\textit{ Studien zur Geschichte der Transzendentalphilosophie. I: Karl Leonhard Reinholds Elementarphilosophie in ihrem philosophiegeschichtlichen Zusammenhang. Mit Beilagen Fichte's Entwicklung betreffend}. Uppsala: Lundequistska Bokhandeln, 1938.

von Sch\"{o}nborn, Alexander. \textit{Karl Leonhard Reinhold: Eine annotierte Bibliographie}. Stuttgart{-}Bad Cannstatt: Frommann{-}Holzboog, 1991.

{---}{---}{---}. `Reinholds letztes Werk: Anfang im Ende.' In \textit{Die Philosophie Karl Leonhard Reinholds}, 303{-}321. Edited by Martin Bondeli and Wolfgang Schrader. Amsterdam: Rodopi, 2003.

Schr\"{o}pfer, Horst. \textit{Kants Weg in die \"{O}ffentlichkeit: Christian Gottfried Sch\"{u}tz als Wegbereiter der kritischen Philosophie}. Stuttgart{-}Bad Cannstatt: Frommann{-}Holzboog,2003

Schulz, Eberhard G\"{u}nther. \textit{Rehbergs Opposition gegen Kants Ethik }(Cologne: B\"{o}hlau Verlag, 1975.

Stolzenberg, J\"{u}rgen. `Die Freiheit des Willens. Schellings Reinhold{-}Kritik in der \textit{Allgemeinen \"{U}bersicht der neuesten philosophischen Literatur}.' In \textit{Philosophie ohne Beynamen: System, Freiheit und Geschichte im Denken Karl Leonhard Reinholds}, 272{-}289. Edited by Martin Bondeli and Alessandro Lazzari. Basel: Schwabe, 2004.

Valenza, Pierluigi ed.\textit{ K.L. Reinhold. Am Vorhof des Idealismus}. Pisa: Istituti Editoriale Poligrafici Internationali, 2006. Also published as \textit{K.L. Reinhold: Alle soglie dell'Idealismo}. Archivio di Filosofia, volume 73 (2005), issue 1{-}3.

Wahl, Hans. \textit{Geschichte des Teutschen Merkur: Ein Beitrag zur Geschichte des Journalismus im achtzehnten Jahrhundert}. Berlin: Mayer \& M\"{u}ller 1914. 

Z\"{o}ller, G\"{u}nter. `Von Reinhold zu Kant. Zur Grundlegung der Moralphilosophie zwischen Vernunft und Willk\"{u}r.' In \textit{K.L. Reinhold. Am Vorhof des Idealismus}, 73{-}91. Edited by Pierluigi Valenza. Pisa: Istituti Editoriale Poligrafici Internationali, 2006.

